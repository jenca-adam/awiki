\documentclass[12pt]{article}
\usepackage{geometry}
\usepackage{amsfonts}
\geometry{total={210mm,290mm},
 left=23mm,right=23mm,%
 bindingoffset=0mm, top=20mm,bottom=20mm}





\begin{document}
\begin{center}
{\large  P. Lipka-Bartosik, C. T. Chubb, J. M. Renes, et al.: Quantum dichotomies and coherent thermodynamics beyond first-order asymptotics}

\end{center}
\medskip

\centerline{Referee report}

\bigskip



This paper concerns the problem of exact or approximate transformation of quantum dichotomies in the asymptotic
regime. In the case when the target dichotomy is commutative, explicit expressions for second-order
transformation rates are given under different assumptions on the behaviour of the
 transformation errors, in the general case only the optimality part is proved. The
 results are based on the characterization of approximate transformations of
 classical dichotomies in terms of hypothesis testing errors, together with the properties
 of the pinching map and monotonicity of various relative entropy quantities. These  results are applied to obtain second-order rates for thermodynamic
 transformations and for LOCC transformations between pure bipartite entangled states.

\medskip

\noindent
\textbf{Overall evaluation}

\medskip
 The results of the paper are timely and important, since the problem of conversions
 between sets of states under various regimes is fundamental in many resource theories.
 The present paper gives answers for dichotomies (that is, pairs of states) in the
 second-order asymptotic limit and provides significant extensions of the existing
 results. Moreover, in the case of commuting targets, it explores and succesfully utilises 
 connections of convertibility  problems to asymptotic hypothesis testing.


 However, the paper in the present form is not very well written. While the basic  ideas are
 clearly explained  and the main results are well described, the actual proofs are quite difficult to read. 
 This is partly due to the fact that the arguments are sometimes given in a confusing and
 descriptive way, and partly due to a relatively high number of mistakes. These are mostly
 just typos, but since most of the proofs consist of complicated manipulations, this makes
 reading exceedingly difficult. 

 Some of these issues are listed  in the specific comments below. I suggest that the authors
  carefully read the manuscript, identifying also other places where some confusion may occur.
  Some further suggestions are added at the end, that the authors might find useful.

\medskip

\noindent
\textbf{Some specific comments}

\begin{enumerate}
\item p. 15, eq. (70a):  $\epsilon$ should be $\epsilon -2\delta$ (?)
\item p. 15, eq. (73): the assumption $\lambda>0$ is needed
\item p. 15, paragraph above eq. (74): $\gamma_\lambda(\rho\|\sigma)$: $\lambda$ or $x$?
\item p.16-17, Lemma 17: In the results in [106,107] and [108,109] used in the proof, the
role of type-I  and type-II errors are exchanged, this should reflect in the exchange
$\rho\leftrightarrow \sigma$ in the expressions of the lemma (the bounds for $\lambda$
seem to be given correctly).
\item p. 17, Eq. (86): $D(\sigma,\rho)$ should be $D(\sigma\|\rho)$
\item p.17: the arguments of the proof between the last paragraph of the left column,
finishing the proof for $\lambda\in \mathcal R_R$ are quite unclear and confusing, in
particular see the three points below.
\item  Are the functions in eq. (87) invertible? It is better to explain, prove or give a
reference for this.

\item p. 17, line under eq. (89): the swapping of $\rho$ and $\sigma$ should also affect
the bound in $\mathcal R_L$
\item p.17, looking at eq. (91), I would say that the inequalities are opposite...

\item p. 19, Lemma 19: As far as I can see, the limit values $\check{D}_{\pm\infty}$ were not
specified. (In case I missed it: it is better to point to the place in the
text where it is done)

\item p.19, proof of Lemma 19: this is a bit confusing. $Q_t$ were defined to be projections,
but the test given in eq. (106) is not a projection in general (unless $\Pi$ is rank one).

\item p. 33, paragraph below eq. (A8): $A\le \Phi$ or $A\ge \Phi$?
\item p.34, line below eq. (A16): $y\in (0,\epsilon)$ or $y\in (\epsilon,1)$?
\item p. 35, line below eq. (A21): what is ''our expression for $\epsilon$''?
\end{enumerate}
\medskip

\noindent
\textbf{Some additional remarks and suggestions}


\begin{enumerate}
\item p.3, last line above Sec. II.B: There are some conditions that characterize  transformations between general
quantum dichotomies (and more generally families of states), given in terms of expressions
related to state discrimination, in the spirit of classical  Blackwell and Le Cam
theorems,  e.g. [74], Prop. 5 (exact case) and  
(A. Jenčová, Comparison of quantum channels and quantum statistical experiments, 2016 IEEE
International Symposium on Information Theory (ISIT), 2249 - 2253, IEEE Conference
Publications, 2016), Thm. 4 (approximate case). Usefulness is not clear, however.


\item In the proof of Lemma 25, to show that the limit in eq. (B16) exists, one can use
the following argument: by Lemma 23, there is some $C_0>$ such that the sequence
$y_n:=nC_0+D_\alpha(\mathcal P_{\sigma^{\otimes n}}(\rho^{\otimes n})\|\sigma^{\otimes
n})$ is nonnegative and eq. (B19) shows that it is subadditive: $y_{n+m}\le y_n+y_m$. By
the Fekete Lemma, this implies that $\lim_n n^{-1}y_n=\inf_n n^{-1}y_n$ exists.





\end{enumerate}








\end{document}

