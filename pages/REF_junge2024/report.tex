\documentclass[12pt]{article}
\usepackage{geometry}
\usepackage{amsfonts}
\geometry{total={210mm,290mm},
 left=23mm,right=23mm,%
 bindingoffset=0mm, top=20mm,bottom=20mm}





\begin{document}
\begin{center}
{\large M. Junge, N. LaRacuente, Strong converse rate for state discrimination in
general von Neumann algebras }

\end{center}
\medskip

\centerline{Referee report}

\bigskip

- most appropriate session: strong converses
- reviewer familiarity with subject: high    
- recommendation: accept/tend to accept

- comments for the program committee: 

This paper definitely has merit. It addresses the
question of extending the operational interpretation of the sandwiched Renyi divergence in
the strong converse domain of quantum hypothesis testing. The extension is to the most
general infinite dimensional situation of von Neumann algebras, this result was conjectured in
the paper [Berta, Tomamichel and Scholtz, Annales Henri Poincaré 19 (6), 1843-1867
(2018)]. On the technical side, it applies the technique of Haagerup reduction and finite
spectrum reduction, which seems to be widely applicable for similar extensions of the
results of quantum information theory to the von Neumann algebra setting. So I find that
the paper is of importance and interest for the BIID audience.

On the other hand, the abstract and also the technical part is poorly written, with
exceedingly many typos and omissions, so this should be improved. The arguments of the
proofs are nevertheless sound and I am convinced that they work. Some of the Lemmas in Sec. 3 are straightforward
consequences of the results proved in  Ref. [17] or the above mentioned paper (see also
the book [F. Hiai, Quantum f-divergences in von Neumann algebras, Springer, 2021]). These
were proved without the majorization assumption Eq. (6) on the states, so it is a question
whether the results would hold under the weaker assumption that $D^*_\alpha$ is finite for
some $\alpha>1$, as in Ref. [6].

\end{document}

