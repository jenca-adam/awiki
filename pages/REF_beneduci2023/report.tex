\documentclass[12pt]{article}
\usepackage{geometry}
\usepackage{amsfonts}
\usepackage{calrsfs}
\geometry{total={210mm,290mm},
 left=23mm,right=23mm,%
 bindingoffset=0mm, top=20mm,bottom=20mm}





\begin{document}
\begin{center}
{\large   R. Beneduci, Fuzzy observables: from weak Markov kernels to Markov kernels}

\end{center}
\medskip

\centerline{Referee report}

\bigskip

It is already well understood that a fuzzy observable (which is a  commutative POVM) is
always obtained by a fuzzification of a sharp observable (a PVM) by a weak Markov kernel. Moreover, if the
observable is defined on a second countable completely metrizable space, then the weak Markov kernel can be
replaced by an equivalent  Markov kernel.  It was also proved by the present author that
under some additional conditions, we may assume that the Markov kernel is a Feller Markov
kernel, which means that it preserves continuity of functions.

As far as I understand, the present work provides a generalization of these results to arbitrary fuzzy observables
defined on a second countable metrizable space. This is achieved by studying weak Markov
kernels defined on a subset of the unit interval and the Borel sets of a  second countable
metrizable topological space. These results are quite technical and their significance is
not easily seen. In my opinion, this is largely due to  the
presentation, which should be improved before acceptance.

\medskip

\noindent

Some specific comments that might help to increase the value of the paper:
\medskip


\begin{enumerate}
\item  The Introduction could be somewhat improved, so that also a non-expert  reader
would understand the problem (e.g. the difference between Markov kernels and weak Markov
kernels)  and the obtained new results.
\item I particular, weak Markov kernels first appear on p. 3 without any explanation, it
would be better to at least add something like ...see Def. 2 below...

\item The main results are based on Sec. 2, but it is not clear what is the role of the
metrizability condition? It does not appear anywhere in the assumptions in Sec. 2, in
particular, there is no such assumption in Thm. 3. 

\item It would be good, apart from the technical proofs, to provide a paragraph in Sec. 2,
explaining the methods and perhaps comparing them to those used previously. Otherwise, the
reader is quickly lost in technical details, without seeing the whole picture.

\item In Sec. 3, the word ''smearing'' appears. It had been  used in the literature
before, but in this paper the term ''fuzzification'' is used in the previous sections, so
it would be better to stick to that.

\item On p. 9, line 7, it is written that ...avoiding to require that $F$ is regular...
But the assumption of regularity is present in Thm. 5.

\item p. 9, line 7: It is not clear what $O_2$ is.

\item  Definition 5: As far as I can see, $\mathcal E(\mathcal  H)$ was not defined
before. Also PVM appears for the first time (better use the term sharp observable).

\item Thm. 5: It is not clear what $\mathcal F(\mathcal H)$ is.





\end{enumerate}
\end{document}
