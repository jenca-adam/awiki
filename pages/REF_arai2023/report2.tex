\documentclass[12pt]{article}
\usepackage{geometry}
\usepackage{amsfonts}
\geometry{total={210mm,290mm},
 left=23mm,right=23mm,%
 bindingoffset=0mm, top=20mm,bottom=20mm}





\begin{document}
\begin{center}
{\large H. Arai, M. Hayashi:  Derivation of Standard Quantum Theory via State
Discrimination
}

\end{center}
\medskip

\centerline{Referee report to revised version}

\bigskip

The authors have corrected the mistake in Theorem 5, the statement is now correct.
However, my main concern has not been resolved, namely the novelty and significance of
their results.


The main result, Theorem 7, follows straightforwardly from the statement formulated in the present
version as Corollary 9. I never meant that this statement was trivial, but I insist that
it is an easy consequence of well known facts  about base-normed spaces, see e.g. Chap. 1
in the book E.M. Alfsen, F.W. Shultz, State Spaces of Operator Algebras: Basic Theory,
Orientations, and C*-products. It is really well known that
$D_G(\rho_0,\rho_1)=\frac12\|\rho_0-\rho_1\|_G$, where $\|\cdot\|_G$ is the base norm with
respect to the base $\mathcal S(G)$ (see e.g. Thm. 3.43 in Ref. [21]). If $\rho$ is a quantum state not belonging to
$\mathcal S(G)$, then it must be expressed in the form
\[
\rho=(1+\lambda)\rho_1-\lambda\rho_2,
\]
where $\rho_1,\rho_2\in \mathcal S(G)\subseteq \mathcal S(QT)$ are such that
$\|\rho_1-\rho_2\|_G=\|\rho_1\|_G+\|\rho_2\|_G=2$ and $\lambda\ge 0$ (see Props. 1.25 and 1.26 in the above mentioned
book). But if the condition (ii) in Corollary 9 of the present paper holds, then we also
have $\|\rho_1-\rho_2\|=2$ for the trace norm $\|\cdot\|$, which means that $\rho_1$ and
$\rho_2$ are quantum states with orthogonal supports. Hence $\lambda=0$ and $\rho=\rho_1\in \mathcal S(G)$. 
Note, by the way, that a similar argument can be used not only for QT but
for any  GPT, even for compact convex subsets in locally convex spaces, not necessarily
finite  dimensional. 

As for the other results in the paper, the results of Thms 5 and 6 can be seen as new, but
I do not think that this warrants publication. The results are obtained by straightforward
computation or an easy separation argument.  I am also not convinced about the  significance of
the results of the paper  from the point of view of physics.



\end{document}

