\documentclass[12pt]{article}
\usepackage{geometry}
\usepackage{amsfonts}
\geometry{total={210mm,290mm},
 left=23mm,right=23mm,%
 bindingoffset=0mm, top=20mm,bottom=20mm}





\begin{document}
\begin{center}
{\large H. Arai, M. Hayashi:  Derivation of Standard Quantum Theory via State Discrimination
}

\end{center}
\medskip

\centerline{Referee report}

\bigskip

This paper claims to give a characterization of quantum theory among GPTs (with certain
dimension condition) by an operational condition, namely in terms of error probabilities
in state discrimination problems. This is done in the following way: under the dimension
condition, the state space of the theory in question can be mapped into the set of quantum states by
an affine isomorphism, it is shown that the image is a strict subset (and hence the GPT is
different from the quantum theory) if and only if the error probabilities in some state
discrimination problem are smaller than the quantum ones. I have the following comments:

\begin{enumerate}
\item Inequality Eq. (7) stated in Theorem 5 is easily seen to be
wrong unless $p=1/2$, in fact, the equality denoted as (d) in the proof of this theorem
does not hold. To find a counterexample, consider the state discrimination problem for two
quantum states $\rho_0$, $\rho_1$, with $p\ne 1/2$. Let $P_\pm$ denote the projection onto
the support of $(p\rho_0-(1-p)\rho_1)_\pm$ and put $M_0=sP_++rP_-$, with $0\le r<s\le 1$.
Then $\{M_0,I-M_0\}$ is a valid quantum measurement and it is easy to compute that
\[
\mathrm{Err}(\rho_0;\rho_1;p;M)=\frac12-\frac12(s-r)\|p\rho_0-(1-p)\rho_1\|_1-\frac12(2p-1)(s+r-1).
\]
We clearly may choose $s+r>1$ if $p>1/2$ and $s+r<1$ if $p<1/2$, to violate the inequality
Eq.(7).
 Nevertheless, Theorem 7 in fact holds (note that only the case $p=1/2$ in Theorem 5
is used in the proof). 

\item The inequality Eq.(2) holds in any GPT, with the trace norm replaced by the
 base norm  corresponding to the state space. The main result of the paper just gives the fact that the  base norm increases if it is computed with respect to a smaller base, this
 is well known and easy to see.

\item An operational characterization of quantum theory would be an intrinsic property of
states and measurements of a GPT that singles out the quantum theory. This is not given in
the present paper. The main result is just an easy and well-known step away from stating
that a  state space is not quantum if and only if it is not affinely isomorphic to a set
of quantum states, which is trivial.



\item There are a lot of misspellings, incomplete or grammatically incorrect sentences,
etc. I do not point them out, since it would not improve the paper in any significant way.
\end{enumerate}





\end{document}

