\documentclass[12pt]{article}
\usepackage[utf8]{inputenc}
\usepackage[slovak]{babel}
\usepackage[T1]{fontenc}
\usepackage{enumitem}
\usepackage{geometry}
\geometry{width=150mm,
 top=25mm,bottom=25mm}
\begin{document}
\centerline{\large \textbf{Curriculum Vitae}}
\vskip 5pt
\centerline{\large \textbf{Anna Jen\v cov\'a}}
\vskip 0.5cm
\noindent
\textbf{Osobné údaje}


\begin{verbatim}http://www.mat.savba.sk/~jencova/\end{verbatim}

\begin{description}[noitemsep,leftmargin=3cm, font=\normalfont]
\item{Meno:} Anna Jen\v cov\'a
\item{Dátum a miesto narodenia:} 7. jún 1971,  Bratislava
\item{Adresa:} Matematický ústav SAV, \v Stef\'anikova 49, 814 73 Bratislava
\item{e-mail:} jenca@mat.savba.sk
\end{description}

\noindent
\textbf{Štúdium a tituly}
\begin{description}[noitemsep,leftmargin=2.6cm, font=\normalfont]
\item{1989 - 1994:} Matematicko - fyzikálna fakulta UK,  Bratislava, denné štúdium 
\item{1994:} Mgr. v odbore matematika
\item{1994 - 1998:} Matematický ústav SAV, doktorandské štúdium, školiteľ: prof.~Ľubomír Kub\'a\v cek
\item{1999:} PhD. v odbore pravdepodobnosť a matematická štatistika, dizertačná práca: Regression models with a low nonlinearity
\item{2018:} DrSc. v odbore matematika, doktorská dizertačná práca: Geometry of families of states: from classical to quantum 

\end{description}

\noindent
\textbf{Zamestnanie}
\begin{description}[noitemsep,leftmargin=3cm, font=\normalfont]
\item{1998 - 1999:} Matematický ústav SAV, odborný pracovník
\item{1999 - 2005:} Matematický ústav SAV, vedecký pracovník
\item{2005 - :} Matematický ústav SAV, samostatný vedecký pracovník
\end{description}

\noindent
\textbf{Vedecké záujmy}
\begin{description}[noitemsep,leftmargin=3cm, font=\normalfont]
\item Kvantová teória informácie, kvantová štatistika
\end{description}


\noindent
\textbf{Pedagogická prax}
\begin{description}[noitemsep,leftmargin=1.3cm, font=\normalfont]
\item{1995, 1997:} Cvičenia z matematiky, Stavebná fakulta STU, Bratislava, 2 semestre
\item{2002:} Cvičenia z lineárnej algebry, Fakulta elektortechniky a informatiky STU, 1 semester
\item{2016:} Kvantová Shannonova teória - prednáška pre doktorandov, Matematický ústav SAV
\end{description}



\noindent
\textbf{Doktorandi}
\begin{description}[noitemsep,leftmargin=1.3cm, font=\normalfont]
\item{Martin Plávala:} 2015 -  
\end{description}

\noindent
\textbf{Ocenenia}
\begin{description}[noitemsep,leftmargin=1.3cm, font=\normalfont]
\item{2003: }   2. miesto v súťaži  mladých vedeckých pracovníkov SAV o najlepšiu vedeckú prácu pri príležitosti 50. výročia SAV

\item{2014:} Birkhoff - von Neumann prize, ocenenie udelené asociáciou IQSA za vynikajúce vedecké výsledky v oblasti kvantových štruktúr
\end{description}

%\newpage
%\noindent
%\textbf{Pedagogická prax}
%\begin{description}
%\item{} cvičenia z matematiky a lineárnej algebry, Stavebná fakulta a Elektrotechnická fakulta STU
%\item{2016:} doktorandský kurz: Kvantová teória informácie
%\end{description}
\noindent
\textbf{Krátkodobé študijné pobyty}
\begin{description}[noitemsep,leftmargin=1.3cm, font=\normalfont]
\item{2004:}  Brain Science Institue, RIKEN, Tokio, 2 týždne
\item{2004:} štipendium v rámci projektu EU Research Training Network: QP Applications, Maďarsko, Budapest University of Technology and Economics, spolu 3 mesiace
\item{2005:} Tufts University, Boston, 2 týždne
\item{2007 - 2015:} Tufts University, Boston; BUTE, Rényi Institute, Budapešť; Max Planck Institute, Lipsko; Gdansk University;  Imperial Colllege, Londýn - kratšie pobyty na pozvanie
\item{2016:}  Perimeter Institute, Waterloo, 2 týždne
\end{description}




\noindent
\textbf{Pozvané prednášky}

\begin{description}[noitemsep,leftmargin=1.3cm, font=\normalfont]
\item{2018:}  \emph{Incompatible measurements in general probabilistic theories}. Quantum Information Theory and Mathematical Physics workshop, 2018, Budapest

\item{2018:} \emph{Rényi relative entropies and sufficiency of quantum channels}, Beyond I.I.D. in Information Theory, 2018, Cambridge

\item{2018:} \emph{A geometric view on quantum incompatibility}, Three Days in quantum Mechanics, 2018, Genoa


\item{2017} \emph{Rényi relative entropies and noncommutative $L_p$-spaces} Quantum Information Theory and Mathematical Physics,  Budapešť
\item{2016:} \emph{Quantum divergences and interpolation},  workshop: Quantum Information Theory and Mathematical Physics,  Budapešť  
\item{2015:} \emph{Comparison of noisy quantum channels}, Imperial College, London, prednáška na seminári
\item{2015:} \emph{Conditions for sufficiency of quantum channels}, workshop: Beyond I.I.D. in Information Theory, BIRS, Banff, Kanada
\item{2014:} \emph{Quantum versions of the randomization criterion}, workshop: New Horizons in Statistical Decision Theory,  Oberwolfach, Nemecko
%\item{2013:} \emph{Reversibility conditions for quantum operations},  Noncommutative Workshop,  Krakow, Poľsko
\item{2013:} \emph{Distinguishing quantum channels by restricted testers},  Symposium KCIK, Sopot, Poľsko
%\item{2013:} \emph{Base norms on subspaces of matrices, with applications in quantum information theory}, University of Gdansk, 
%prednáška na seminári
%\item{2012:} \emph{Reversibility conditions for quantum operations}, Max Planck Institute, Leipzig, prednáška na seminári

\item{2012:} \emph{Generalized channels and quantum networks}, workshop: Operator Structures in Quantum Information Theory,  BIRS, Banff, Kanada
\item{2012:} \emph{Basic structures of quantum information geometry} (minikurz - 3 prednášky),  Noncommutative Workshop,  Imperial College, London
%\item{2008:} \emph{Topics in quantum statistics}, R\'enyi Institute, Budapešť, prednáška na seminári

\item{2008:} \emph{On quantum information manifolds}, konferencia:  Mathematical Explorations in Contemporary Statistics,  Sestri Levante, Taliansko
\item{2007:} \emph{Quantum local asymptotic normality}, 28-th Conference on Quantum Probability and Related Topics,  Guanajuato,
Mexiko
\item{2007:} \emph{Weak convergence of quantum experiments and quantum local asymptotic
normality}, workshop: Operator Structures in Quantum Information Theory, BIRS, Banff, Kanada
\item{2006: }  \emph{Weak convergence of quantum experiments and quantum local asymptotic
normality}, Workshop on Quantum Statistics,  Erwin Schrodinger Institute, Viedeň
\item{2006: } \emph{Local asymptotic normality in quantum statistics}, konferencia: Quantum Probability,Information and Control Symposium, University of Nottingham
%\item{2004:} \emph{Divergences in von Neumann algebras},  Budapest University of Technology and Economics, prednáška na seminári

\item{2004:}  \emph{Affine connections and divergences in quantum information geometry}, 
 Workshop on Quantum Information Geometry and Quantum Computing, The Fields
Institute, Toronto
%\item{2004:} RIKEN Brain Science Institute, Tokyo, Japonsko, prednáška na seminári
%\item{2003:} \emph{On quantum information geometry}, Budapest University of Technology and Economics, prednáška na seminári

\item{2002:} \emph{Information geometry in the standard representation of matrix spaces}, konferencia: Information Geometry and its Applications, Pescara, Taliansko




 




\end{description}


\end{document}

