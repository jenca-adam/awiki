
\documentclass[A4paper]{article}
\usepackage[utf8]{inputenc}
\usepackage[slovak]{babel}
\usepackage[T1]{fontenc}

\parindent=10pt \parskip=4pt
\begin{document}
\begin{center}\textbf{CURRICULUM VITAE}
\vspace{2mm}

{Anna Jen\v cov\' a}
\end{center}
\vspace{1mm}

\newcommand{\mypoint}[1]{\vskip .75em\noindent {\bf #1}\vskip .75em}

\begin{tabular}{rl}
Naroden\'a: & 7. j\'una 1971 v Bratislave\\
1989 -- 1994:      &  \v studentka MFF UK\\
    ~           & zameranie  Pravdepodobnos\v t a matematick\'a \v statistika\\
    ~           & podzameranie: Matematick\'a anal\'yza \\
1994 -- 1998       & M\'U SAV intern\'a vedeck\'a a\v spirant\'ura\\
    ~           & \v skolite\v l \v L. Kub\'a\v cek,\\
    ~           & dizerta\v cn\'a pr\'aca: Regression models with a low
    nonlineartity\\
1998 -- 1999       & M\'U SAV, odborn\'y pracovn\'\i k\\
1999            & PhD\\
1999 -- 2005       & M\'U SAV, vedeck\'y pracovn\'\i k\\
2005 --           & MÚ SAV, samostatný vedecký pracovník
\end{tabular}

\mypoint{Vedecké záujmy:}  
klasická a kvantová informačná geometria 

kvantová štatistika a teória informácie


\mypoint{\v Clenstvo vo vedeck\'ych spolo\v cnostiach}
\v Clen Association for Quantum Probability and Infinite Dimensional Analysis
(AQPIDA), \v clen JSMF

\mypoint{Ocenenia}

2003:   2. miesto v s\'u\v ta\v zi mlad\'ych vedeck\'ych pracovn\'\i kov SAV pri pr\'\i
le\v zitosti 50. v\'yro\v cia SAV
\newpage

\mypoint{Pozvan\'e predn\'a\v sky a pobyty.}
\begin{tabular}{rl}
1.--5. 7.  2002: & predn\'a\v ska na konferencii: Information geometry and
its\\
      ~           &  applications,   Pescara, Taliansko\\
m\'aj 2003:     & {\it On quantum information geometry}, predn\'a\v ska
                           na semin\'ari\\
~              & Mathematical analysis, Budapest University of Technology\\
 ~             &  and Economics, Budape\v s\v
 t\\
11.-- 23. 1. 2004: & Brain Science Institute, RIKEN, Tokio, Japonsko,\\
~               & pracovn\'y pobyt na pozvanie, 3 predn\'a\v sky\\
 apr\'\i l 2004: &{\it Divergences in von Neumann algebras},
                predn\'a\v ska na semin\'ari\\
    ~           & Mathematical analysis, Budapest University of Technology\\
    ~           &  and Economics Budape\v s\v t\\
8.--18. 5. 2004: & Workshop on Quantum Information Geometry and Quantum\\
    ~           &  Computing, Mc Master University, Hamilton (workshop) a\\
    ~           & Fields Institute, Toronto (symposium), Kanada\\
2004:             & pobyt na BUTE, Budape\v s\v t,
 v r\'amci projektu Eur\'opskej \'unie \\
 ~      & EU Research Training Network: QP-applications (spolu 3 mes)\\
 2. -- 15. 10. 2005 &  Tufts University, Boston, pracovný pobyt na pozvanie\\
  16. -- 20. 7. 2006 & prednáška na konferencii:  Quantum Probability,
                     Information and \\
     ~           & Control Symposium, University of Nottingham\\
27. 11.  -- 1. 12. 2006  & Workshop on Quantum Statistics, Erwin Schr\"odinger
                     International\\
     ~           &Institute for Mathematical Physics, Viedeň, prednáška\\
 10. -- 17.2. 2007 & prednáška na konferencii: Operator
                    structures in quantum \\
	~	   & information theory, Banff Center, Alberta, Canada.\\
 2. -- 7.9. 2007 &  prednáška na konferencii: 28-th Conference on Quantum 
                    Probability \\
        ~         & and Related Topics, Guanajuato, Mexiko\\
 8.9. -- 16.9. 2007 &  Tufts University, Boston, pracovný pobyt na pozvanie\\
 18. -- 20.5. 2008 &  prednáška na konferencii: Mathematical Explorations in \\
      ~            &  Contemporary Statistics, Sestri Levante, Taliansko

\end{tabular}




\newpage

\begin{center} ZOZNAM P\^OVODN\'YCH VEDECK\'YCH PR\'AC \end{center}

\newcounter{foo}

\mypoint{V zahrani\v cn\'ych \v casopisoch CC:}
\begin{enumerate}
\item \label{geom} A. Jen\v cov\' a, Geometry of quantum states: dual
connections and divergence functions, Rep. Math. Phys.,47
(2001),121-138

\item \label{purif} A. Jen\v cov\'a, Quantum information geometry and standard purification,  J.Math.
Phys., 43 (2002), 2187--2201

\item \label{wyd} A. Jen\v cov\'a, Flat connections and Wigner-Yanase-Dyson metrics,
 Rep. Math.Phys. 52 (2003), 331--351, math-ph/0307057

\item \label{geodesic} A. Jen\v cov\'a, Geodesic distances on density matrices,
  J. Math. Phys. 45 (2004), 1787-1794

\item \label{contrast} A. Jen\v cov\'a,  Generalized relative entropies as contrast
functionals on density matrices, Int. J Theor. Phys. 43, (2004), 1635--1649

\item\label{lpspaces}   A. Jen\v cov\' a,  Quantum information geometry and
non-commutative Lp spaces, Infinite Dimensional Analysis, Quantum Probability
and Rel. Top. 8 (2005), 215-233

\item\label{sufficiency} A. Jen\v cov\'a, D. Petz, Sufficiency in quantum
statistical inference,  Commun. Math. Phys. 263 (2006), 259-276, math-ph/0412093


\item\label{yasp} A. Jen\v cov\'a, 
A construction of a nonparametric quantum information manifold, 
Journal of Functional Analysis 239 (2006), 1-20,  math-ph/0511065

\item\label{complete} A. Jen\v cov\'a, A relation between completely bounded norms and 
conjugate channels, Commun. Math. Phys. 266 (2006), 65-70, quant-ph/0601071

\item \label{survey} A. Jen\v cov\'a, D. Petz, Sufficiency in quantum statistical inference. A survey with examples, IDAQP 9 (2006),  331-351, quant-ph/0604091 	

\item \label{howsharp} A. Jenčová, S. Pulmannová, How sharp are PV measures?, 
Rep. Math. Phys. 59 (2007) 257-266 

\item \label{qlan} M. Guta, A. Jenčová: Local asymptotic normality in quantum statistics,  Commun. Math. Phys. 276 (2007) 341-379 


\item \label{sharpfuzzy} A. Jenčová, S. Pulmannová, E. Vinceková, Sharp and fuzzy observables on effect algebras, Int. J Theor. Phys. 47 (2008) 125-148

\item \label{af} A. Jenčová, S. Pulmannová: A note on  effect algebras and dimension theory of AF C*-algebras, Reports on Mathematical Physics 62 (2008), pp. 205-218

\item\label{characterization}  A. Jenčová, S. Pulmannová, Characterizations of commutative POV 
measures, Foundations of Physics 39 (2009), pp. 613-624 


\item \label{hypo} A. Jenčová, Quantum hypothesis testing and sufficient subalgebras, 
Lett. Math. Phys 93 (2010), 15

\item \label{ruskai} A. Jenčová, M.B. Ruskai, A unified treatment of convexity of relative
entropy and related trace functions, with conditions for equality, Reviews in Math Phys.
\setcounter{foo}{\value{enumi}}

\end{enumerate}



\mypoint{V ostatn\'ych zahrani\v cn\'ych \v casopisoch:}
\begin{enumerate}
\setcounter{enumi}{\value{foo}}
\item \label{lin_comparison}A. Jen\v cov\'a, A comparison of linearization and quadratization
 domains, Applications of Mathematics 42 (1997), no. 4, 279-291

\item \label{lin_unknown} A. Jen\v cov\'a, Linearization conditions for regression models
with unknown variance parameter, Applications of Mathematics 45
(2000), no. 2, 145-160

\item \label{weyl} A. Jenčová, D. Petz and J. Pitrik,  Markov triplets on CCR-algebras, Acta Sci. Math. (Szeged),   76(2010),  27--50
\setcounter{foo}{\value{enumi}}

\end{enumerate}
\mypoint{V zahrani\v cn\'ych zborn\'\i koch:}

\begin{enumerate}
\setcounter{enumi}{\value{foo}}

\item \label{dualistic} A. Jen\v cov\' a, Dualistic properties of the manifold of quantum
states,\\ In:{\it Disordered and complex systems},
AIP Conference Proceedings,\ Melville, New York 2001,
147--152

\item \label{yasp2} A. Jen\v cov\'a, On quantum information manifolds, In:{\it Algebraic and Geometric Methods in Statistics}, Cambridge University Press 2010
\setcounter{foo}{\value{enumi}}

\end{enumerate}

\mypoint{V dom\'acich \v casopisoch:}
\begin{enumerate}
\setcounter{enumi}{\value{foo}}

\item \label{lin_achoice} A. Jen\v cov\'a,  A Choice of Criterion Parameters in
Linearization of Regression Models, Acta Math. Univ.
Comenianae, Vol LXIV, 2(1995), 227--234

\item \label{fisher} A. Jen\v cov\'a, D. Petz, On quantum Fisher information, J.
Electrical Engineering 50, 1999, 78--81
\setcounter{foo}{\value{enumi}}

\end{enumerate}

\mypoint{Prijat\'e do tla\v ce:}





\newpage
\begin{center} CIT\'ACIE \end{center}

\newcounter{poo}
\mypoint{Cit\'acie v SCI:}

{\bf Pr\'aca \ref{geom} je citovan\'a v:}
\begin{enumerate}
\item  P. Gibilisco, T. Isola, Wigner-Yanase information on quantum state
space: The geometric approach, J Math. Phys. 44, no. 9, 2003, 3752--3762

\item  H. Hasegawa, Dual geometry of the Wigner-Yanase-Dyson information
content, IDAQP and Rel. Top., 6, no. 3, 2003, 413--430
\

\item  M.R. Grasselli, Duality, monotonicity and the Wigner-Yanase-Dyson
metrics,   IDAQP and Rel. Top. 7, no. 2, 2004, 215--232

\item  P. Gibilisco, T. Isola, On the characterization of paired monotone
metrics, Annals Inst. Stat. Math., 56, no 2., 2004, 369--381

\item  P. Gibilisco, T. Isola, On the monotonicity of scalar curvature in
classical and quantum information geometry, J. Math. Phys. 46, (2005)

\item Ma, Z. , Zhu, S., Topologies on quantum states, 	Physics Letters, Section A: General, Atomic and Solid State Physics 374 (11-12), pp. 1336-1341,  (2009)


\item Han, D., Sun, H.-F., Submanifolds of curved exponential family in quantum 
statistics, 	Beijing Ligong Daxue Xuebao/Transaction of Beijing Institute of Technology 29 (10), pp. 918-920+935 (2010)

\setcounter{poo}{\value{enumi}}

\end{enumerate}

{\bf Pr\'aca \ref{purif} je citovan\'a v:}
\begin{enumerate}
\setcounter{enumi}{\value{poo}}

\item   P. Gibilisco, T. Isola, Wigner-Yanase information on quantum state
space: The geometric approach, J Math. Phys. 44, no. 9, 2003, 3752--3762
\setcounter{poo}{\value{enumi}}

\end{enumerate}

{\bf Pr\'aca \ref{wyd} je citovan\'a v:}
\begin{enumerate}
\setcounter{enumi}{\value{poo}}

\item  S.L. Luo, Q. Zhang, On skew information, IEEE Trans. Information Theory 50, no 8., 2004, 1778--1782
\item   K. Yanagi K, S. Furuichi, K. Kuriyama, 
A generalized Skew information and uncertainty relation, 
IEEE Trans.  Information  Theory 51 (2005), 4401-4404 
\setcounter{poo}{\value{enumi}}

\end{enumerate}


{\bf Pr\'aca \ref{geodesic} je citovan\'a v:}
\begin{enumerate}
\setcounter{enumi}{\value{poo}}

\item  M.B. Ruskai, Lieb's simple proof of concavity of $(A, B) \to
{\rm Tr}\, A(p) K^\dagger B(1-p)K$ and remarks on related inequalities,
INTERNATIONAL JOURNAL OF QUANTUM INFORMATION 3 (2005): 579-590
\setcounter{poo}{\value{enumi}}

\end{enumerate}
{\bf Pr\'aca \ref{contrast} je citovan\'a v:}

\begin{enumerate}
\setcounter{enumi}{\value{poo}}

\item  P.B. Slater, 
Quantum and Fisher information from the Husimi and related distributions,
J.  Math.  Phys.  47 (2006) 
\setcounter{poo}{\value{enumi}}

\end{enumerate}

{\bf Pr\'aca \ref{sufficiency} je citovan\'a v:}

\begin{enumerate}
\setcounter{enumi}{\value{poo}}
\item  T. Ogawa, A. Sasaki, M. Iwamoto, et al., Quantum secret
sharing schemes and reversibility of quantum operations, Phys.
Rew. A, 72 (2005)

\item  M. Guta, J. Kahn, Local asymptotic normality for qubit states,
Phys. Rew  A 73 (2006)

\item  H. Moriya,
Markov property and strong additivity of von Neumann entropy for graded quantum systems,  J Math Phys 47 (2006) 

\item  J. Pitrik, Markovian quasifree states on canonical anticommutation relation algebras, J Math Phys 48 (2007) Article Number: 112110

\item  K. Lubnauer, A. Luczak,  H. Posedkowska,
Weak sufficiency of quantum statistics, Rep. Math. Phys. 60 (2007), 367-380  

\item  M. Guta, J. Kahn, Local asymptotic normality for finite dimensional
quantum systems, Communications in Mathematical Physics 289 (2009), pp. 597-652
\setcounter{poo}{\value{enumi}}

\end{enumerate}

{\bf Pr\'aca \ref{yasp} je citovan\'a v:}

\begin{enumerate}
\setcounter{enumi}{\value{poo}}
\item   Pistone, G,  kappa-exponential models from the geometrical viewpoint, 
 EUROPEAN PHYSICAL JOURNAL B,  70 (2009)  pp. 29 
\setcounter{poo}{\value{enumi}}
 
\end{enumerate}

{\bf Pr\'aca \ref{complete} je citovan\'a v:}

\begin{enumerate}
\setcounter{enumi}{\value{poo}}
\item I.  Devetak I, M. Junge, C. King, et al.
Multiplicativity of completely bounded p-norms implies a new additivity result,
Commun.  Math.  Phys.  266 (2006), 37-63

\item  Perez-Garcia D, Wolf MM, Petz D, et al.
Contractivity of positive and trace-preserving maps under L-p norms 
JOURNAL OF MATHEMATICAL PHYSICS 47 (8): Art. No. 083506 AUG 2006

\item  Johnston N, Kribs DW, Paulsen VI,  COMPUTING STABILIZED NORMS FOR QUANTUM OPERATIONS VIA THE THEORY OF COMPLETELY BOUNDED MAPS, QUANTUM INFORMATION \& COMPUTATION  9  (2009)   Pages: 16-35 
\setcounter{poo}{\value{enumi}}

\end{enumerate}


{\bf Pr\'aca \ref{survey} je citovan\'a v:}
\begin{enumerate}
\setcounter{enumi}{\value{poo}}
\item  Brody, D.C., Hook, D.W.,Information geometry in vapour-liquid
equilibrium, Journal of Physics A: Mathematical and Theoretical 42 (2009),
art. no. 023001   
\setcounter{poo}{\value{enumi}}

\end{enumerate}



{\bf Pr\'aca \ref{howsharp} je citovan\'a v:}

\begin{enumerate}
\setcounter{enumi}{\value{poo}}
\item  Beneduci, R., Unsharp number observable and Neumark theorem, Nuovo Cimento  della Societa Italiana di Fisica B 123 (2008), pp. 43-62

\item  Ali, S.T., Carmeli, C., Heinosaari, T., Toigo, A., Commutative POVMs
and fuzzy observables, Foundations of Physics 39 (2009), pp. 593-612

\item  Carmeli, C., Heinosaari, T., Pellonpää, J.-P., Toigo, A., Optimal 
covariant measurements: The case of a compact symmetry group and phase 
observables, Journal of Physics A: Mathematical and Theoretical 42 (2009), art. no. 145304 

\item R. Beneduci, Infinite sequences of linear functionals, positive operator-valued measures and Naimark extension theorem, Bull. London Math. Soc. 2010 

\item Ma, Z. , Zhu, S., Topologies on quantum effects, Reports on Mathematical Physics 64 (3), pp. 429-439, 2009
\setcounter{poo}{\value{enumi}}

\end{enumerate}

{\bf Pr\'aca \ref{qlan} je citovan\'a v:}

\begin{enumerate}
\setcounter{enumi}{\value{poo}}
\item  Hayashi, M., Matsumoto, K.,Asymptotic performance of optimal state estimation in qubit system , Journal of Mathematical Physics 49 (2008), art. no. 102101

\item  Imai, H., Hayashi, M., Fourier analytic approach to phase estimation in quantum systems, New Journal of Physics 11 (2009), art. no. 043034
\setcounter{poo}{\value{enumi}}

\end{enumerate}

{\bf Pr\'aca \ref{sharpfuzzy} je citovan\'a v:}

\begin{enumerate}
\setcounter{enumi}{\value{poo}}
\item   Beneduci, R., Unsharp number observable and Neumark theorem, Nuovo Cimento  della Societa Italiana di Fisica B 123 (2008), pp. 43-62

\item   Carmeli, C., Heinosaari, T., Pellonpää, J.-P., Toigo, A., Optimal 
covariant measurements: The case of a compact symmetry group and phase 
observables, Journal of Physics A: Mathematical and Theoretical 42 (2009), art. no. 145304 

\item Ma, Z. , Zhu, S., Topologies on quantum effects, Reports on Mathematical Physics 64 (3), pp. 429-439
\setcounter{poo}{\value{enumi}}

\end{enumerate}

{\bf Pr\'aca \ref{characterization} je citovan\'a v:}

\begin{enumerate}
\setcounter{enumi}{\value{poo}}
\item  Ali, S.T., Carmeli, C., Heinosaari, T., Toigo, A., Commutative POVMs
and fuzzy observables, Foundations of Physics 39 (2009), pp. 593-612

\setcounter{poo}{\value{enumi}}

\end{enumerate}


{\bf Pr\'aca \ref{ruskai} je citovan\'a v:}

\begin{enumerate}
\setcounter{enumi}{\value{poo}}
\item Brandao, F.G.S.L, Plenio, M.B.,  A Generalization of Quantum Stein's Lemma,
Commun. Math. Phys. 295 (2010), 791-828


\item Cai, L., Hansen, F., Metric-Adjusted Skew Information: Convexity and Restricted Forms of Superadditivity, Lett. Math. Phys. 93 (2010),  1-13 
\setcounter{poo}{\value{enumi}}

\end{enumerate}


{\bf Pr\'aca \ref{dualistic} je citovan\'a v:}

\begin{enumerate}

\setcounter{enumi}{\value{poo}}
\item   M.R. Grasselli, Duality, monotonicity and the Wigner-Yanase-Dyson
metrics,   IDAQP and Rel. Top. 7, no. 2, 2004, 215--232
\setcounter{poo}{\value{enumi}}

\end{enumerate}



\mypoint{Cit\'acie mimo SCI:}

{\bf Pr\'aca [19] je citovan\'a v:}
\begin{enumerate}

\item  A. P\'azman, Linearization of nonlinear regression models by
smoothing, Tatra Mountains Math. Publ. 22 (2001), 13--25

\item  L. Kub\'a\v cek, Linearized models with constraints of type I.,
Appl. Math. 48, 2003, 81--95

\item  L. Kub\'a\v cek, Linear versus  quadratic estimators in linearized
models, Appl. Math. 49, 2004, 81--95

\end{enumerate}

{\bf Pr\'aca [20] je citovan\'a v:}
\begin{enumerate}
\item  K Horni\v sov\'a, Approximation of intrinsic curvature in one
dimensional nonlinear regression model by moments of  prior distribution of
parameter, Measurement Science Review 6 (2006) 40 - 49
  
\end{enumerate}

{\bf Pr\'aca [18] je citovan\'a v:}
\begin{enumerate}
\item  H. Hasegawa, Quantum Fisher information and q-deformed 
relative entropies - Additivity vs nonadditivity, Progress of Theoretical Physics
Supplement No.162 (2006) pp. 183-189

\end{enumerate}

\mypoint{Cit\'acie v monografi\'ach:}

{\bf Pr\'aca \ref{contrast} je citovan\'a v:}
\begin{enumerate}
\item I. Bengtsson, K. Zyczkowski,{\it  Geometry of Quantum States: An Introduction to Quantum Entanglement}, Cambridge University Press (2006)

\item D. Petz, {\it Quantum information theory and quantum statistics}, Springer verlag 2008

\end{enumerate}
\end{document}
\mypoint{Cit\'acie v doktorandsk\'ych pr\'acach:}

{\bf Pr\'aca [1] je citovan\'a v:}
\begin{enumerate}
\item   M.R. Grasselli, Classical and quantum information geometry,
PhD. thesis, King's College, London, 2001,
\end{enumerate}

{\bf Pr\'ace  [11] a [14] s\'u citovan\'e v:}
\begin{enumerate}
\item   K. Horni\v sov\'a, Line\'arne a polynomick\'e aproxim\'acie
neline\'arnych regresn\'ych modelov vyu\v z\'\i vaj\'uce  apri\'orne rozdelenie
parametra, Dizerta\v cn\'a pr\'aca, FMFI UK Bratislava, 2004
\end{enumerate}


\end{document}
. Jenčová, Linearization conditions for regression models with unknown variance parameter, Appl. Math. 45 (2000), no. 2, 145-160
# A. Jenčová, D. Petz, On quantum Fisher information, J.Electrical Engineering 50, 1999, 78--81
# A. Jenčová, A comparison of linearization and quadratization  domains, Appl. Math. 42 (1997), no. 4, 279-291
# A. Jenčová,  A choice of criterion parameters in linearization of regression\
models, Acta Math. Univ.Comenianae, Vol LXIV, 2(1995), 227--234




\end{document}
