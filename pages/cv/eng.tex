\documentclass[12pt]{article}
\usepackage{enumitem}
\usepackage{geometry}
\geometry{width=150mm,
 top=25mm,bottom=25mm}
\begin{document}
\centerline{\large \textbf{Curriculum Vitae}}
\vskip 5pt
\centerline{\large \textbf{Anna Jen\v cov\'a}}
\vskip 0.5cm
\noindent
\textbf{Personal Information}


\begin{verbatim}http://www.mat.savba.sk/~jencova/\end{verbatim}




\begin{description}[noitemsep,leftmargin=3cm, font=\normalfont]
\item{Name:} Anna Jen\v cov\'a
\item{Date and place of birth:} 7. June 1971,  Bratislava
\item{Address:} Mathematical Institute of the Slovak Academy of Sciences, \v Stef\'anikova 49, 814 73 Bratislava
\item{e-mail:} jenca@mat.savba.sk
\vskip 5pt
\item{ORCID:}   https://orcid.org/0000-0002-4019-268X 

\end{description}

\noindent
\textbf{Education}
\begin{description}[noitemsep,leftmargin=2.6cm, font=\normalfont]
\item{1989 - 1994:} Commenius University,  Bratislava
\item{1994:} Mgr (MSc) in Mathematics
\item{1994 - 1998:} Mathematical institute, Slovak Academy of Sciences, PhD study,
supervisor: prof.~Ľubomír Kub\'a\v cek
\item{1999:} PhD in Probability and Mathematical Statistics, dissertation thesis: Regression models with a low nonlinearity

\end{description}

\noindent
\textbf{Employment}
\begin{description}[noitemsep,leftmargin=3cm, font=\normalfont]
\item{2019 - ongoing:} Mathematical Institute, Slovak Academy of Sciences, leading
researcher
\item{1999 -2019:} Mathematical Institute, Slovak Academy of Sciences, researcher

\end{description}

\noindent
\textbf{Research Interests}
\begin{description}[noitemsep,leftmargin=3cm, font=\normalfont]
\item Quantum information theory, relative entropies and quasi-entropies
\item Quantum foundations, General probabilistic theories
\item Quantum statistics, quantum information geometry

\end{description}


\noindent
\textbf{Publications}

\begin{verbatim}http://www.mat.savba.sk/~jencova/publ.html\end{verbatim}


\noindent
\textbf{PhD students}
\begin{description}[noitemsep,leftmargin=1.3cm, font=\normalfont]
\item{Martin Plávala:} 2015 - 2019, 

thesis: \emph{Non-classical effects on generalized quantum channels}
 
\end{description}

\noindent
\textbf{Projects}
\begin{description}[noitemsep,leftmargin=1.3cm, font=\normalfont]
\item{2023-2026:} Designing quantum higher order structures, APVV-22-0570, joint project
with Institute of Physics of SAS, responsible investigator for MI SAS
\item{2020-2023:} Mathematical models of non-classical events and uncertainty, VEGA
2/0142/20, principal investigator
\item{2016-2019:} Algebraic, probabilistic and categorical aspects of modelling of quantum
events and uncertainty, VEGA 2/0069/16, principal investigator
\end{description}

\noindent
\textbf{Awards}
\begin{description}[noitemsep,leftmargin=1.3cm, font=\normalfont]

\item{2014:} Birkhoff - von Neumann prize, award of the International Quantum  Structures
Association (IQSA) for excellent results in the field of quantum structures 
\item{2003: }  Best scientific paper competition of young researchers of Slovak Academy of Sciences, 2.
prize
\end{description}

%\newpage
%\noindent
%\textbf{Pedagogická prax}
%\begin{description}
%\item{} cvičenia z matematiky a lineárnej algebry, Stavebná fakulta a Elektrotechnická fakulta STU
%\item{2016:} doktorandský kurz: Kvantová teória informácie
%\end{description}
%\noindent
%\textbf{Krátkodobé študijné pobyty}
%\begin{description}[noitemsep,leftmargin=1.3cm, font=\normalfont]
%\item{2004:}  Brain Science Institue, RIKEN, Tokio, 2 týždne
%\item{2004:} štipendium v rámci projektu EU Research Training Network: QP Applications, Maďarsko, Budapest University of Technology and Economics, spolu 3 mesiace
%\item{2005:} Tufts University, Boston, 2 týždne
%\item{2007 - 2015:} Tufts University, Boston; BUTE, Rényi Institute, Budapešť; Max Planck Institute, Lipsko; Gdansk University;  Imperial Colllege, Londýn - kratšie pobyty na pozvanie
%\item{2016:}  Perimeter Institute, Waterloo, 2 týždne
%\end{description}
%


\noindent
\textbf{Invited research visits}
\begin{description}
\item{2023:} Nagoya University, Japan, 2 weeks
\item{2016:} Perimeter Institute, Waterloo, Canada, 2 weeks
\item{2005:} Tufts University, Boston, 2 weeks 
\item{2004:} Budapest, participant in EU Research Training Network QP Applications, 3 months
\item{2004:} RIKEN Brain Science Institute, Tokyo, 2 weeks

\item{Shorter (1 week) visits}

2019 Bilkent University, Turkey; 2018 University of Pavia; 2015 Imperial College, London; 2013 Gdansk University, Gdansk; 2012 and 2013 Max Planck Institute, Leipzig;
2008 BUTE Budapest; 2007 Tufts University, Boston

\end{description}

\noindent
\textbf{Recent invited talks}

\begin{description}[noitemsep,leftmargin=1.3cm, font=\normalfont]
\item{2023:} \emph{Is it possible to broadcast anything genuinely quantum?}, Quantum
Information Theory and Mathematical Physics 2023, Budapest
\item{2023:} \emph{Recoverability of quantum channels via hypothesis testing}, ILAS
Minisymposium: Linear Algebra and Quantum Information Theory,  2023, Madrid

\item{2022:} \emph{ On characterizations of quantum incompatibility and steering}, Quantum Kyoto 2022 (online)

\item{2021:} \emph{Incompatibility in GPTs, generalized spectrahedra and tensor norms},
QPL 2021, Gdansk (online)

\item{2021:} \emph{ Rényi relative entropies and noncommutative $L_p$-spaces}, Operator
Algebras and Related Topics, 2021, Istanbul (online)

\item{2019:} \emph{Randomization theorems for bipartite quantum channels}, BIRS Workshop:
Algebraic and Statistical Ways into Quantum Resource Theories, 2019, Banff 

\item{For a full list, see}
\begin{verbatim}http://www.mat.savba.sk/~jencova/talks.html\end{verbatim}

%\item{2018:}  \emph{Incompatible measurements in general probabilistic theories}, Quantum Information Theory and Mathematical Physics workshop, 2018, Budapest
%
%\item{2018:} \emph{Rényi relative entropies and sufficiency of quantum channels}, Beyond I.I.D. in Information Theory, 2018, Cambridge
%
%\item{2018:} \emph{A geometric view on quantum incompatibility}, Three Days in quantum Mechanics, 2018, Genoa
%
%
%\item{2017} \emph{Rényi relative entropies and noncommutative $L_p$-spaces} Quantum Information Theory and Mathematical Physics,  Budapešť
%\item{2016:} \emph{Quantum divergences and interpolation},  workshop: Quantum Information Theory and Mathematical Physics,  Budapešť  
%\item{2015:} \emph{Comparison of noisy quantum channels}, Imperial College, London, prednáška na seminári
%\item{2015:} \emph{Conditions for sufficiency of quantum channels}, workshop: Beyond I.I.D. in Information Theory, BIRS, Banff, Kanada
%\item{2014:} \emph{Quantum versions of the randomization criterion}, workshop: New Horizons in Statistical Decision Theory,  Oberwolfach, Nemecko
%%\item{2013:} \emph{Reversibility conditions for quantum operations},  Noncommutative Workshop,  Krakow, Poľsko
%\item{2013:} \emph{Distinguishing quantum channels by restricted testers},  Symposium KCIK, Sopot, Poľsko
%%\item{2013:} \emph{Base norms on subspaces of matrices, with applications in quantum information theory}, University of Gdansk, 
%%prednáška na seminári
%%\item{2012:} \emph{Reversibility conditions for quantum operations}, Max Planck Institute, Leipzig, prednáška na seminári
%
%\item{2012:} \emph{Generalized channels and quantum networks}, workshop: Operator Structures in Quantum Information Theory,  BIRS, Banff, Kanada
%\item{2012:} \emph{Basic structures of quantum information geometry} (minikurz - 3 prednášky),  Noncommutative Workshop,  Imperial College, London
%%\item{2008:} \emph{Topics in quantum statistics}, R\'enyi Institute, Budapešť, prednáška na seminári
%
%\item{2008:} \emph{On quantum information manifolds}, konferencia:  Mathematical Explorations in Contemporary Statistics,  Sestri Levante, Taliansko
%\item{2007:} \emph{Quantum local asymptotic normality}, 28-th Conference on Quantum Probability and Related Topics,  Guanajuato,
%Mexiko
%\item{2007:} \emph{Weak convergence of quantum experiments and quantum local asymptotic
%normality}, workshop: Operator Structures in Quantum Information Theory, BIRS, Banff, Kanada
%\item{2006: }  \emph{Weak convergence of quantum experiments and quantum local asymptotic
%normality}, Workshop on Quantum Statistics,  Erwin Schrodinger Institute, Viedeň
%\item{2006: } \emph{Local asymptotic normality in quantum statistics}, konferencia: Quantum Probability,Information and Control Symposium, University of Nottingham
%%\item{2004:} \emph{Divergences in von Neumann algebras},  Budapest University of Technology and Economics, prednáška na seminári
%
%\item{2004:}  \emph{Affine connections and divergences in quantum information geometry}, 
% Workshop on Quantum Information Geometry and Quantum Computing, The Fields
%Institute, Toronto
%%\item{2004:} RIKEN Brain Science Institute, Tokyo, Japonsko, prednáška na seminári
%%\item{2003:} \emph{On quantum information geometry}, Budapest University of Technology and Economics, prednáška na seminári
%
%\item{2002:} \emph{Information geometry in the standard representation of matrix spaces}, konferencia: Information Geometry and its Applications, Pescara, Taliansko
%
%
%

 




\end{description}

%\noindent
%\textbf{Pedagogická prax}
%\begin{description}[noitemsep,leftmargin=1.3cm, font=\normalfont]
%\item{1995, 1997:} Cvičenia z matematiky, Stavebná fakulta STU, Bratislava, 2 semestre
%\item{2002:} Cvičenia z lineárnej algebry, Fakulta elektortechniky a informatiky STU, 1 semester
%\item{2016:} Kvantová Shannonova teória - prednáška pre doktorandov, Matematický ústav SAV
%\end{description}
%
%


\end{document}

