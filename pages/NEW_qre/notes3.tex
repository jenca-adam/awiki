\documentclass[12pt]{article}

\usepackage{hyperref}
\usepackage{amsmath, amssymb, amsthm}
\usepackage[sort&compress,numbers]{natbib}
\usepackage{doi}
\usepackage[margin=0.8in]{geometry}
%\textheight23cm \topmargin-20mm  
%\textwidth175mm  
%\oddsidemargin=0mm
%\evensidemargin=0mm
%

\usepackage{amsmath, amssymb, amsthm, mathtools}

\newtheorem{lemma}{Lemma}
\newtheorem{prop}{Proposition}
\newtheorem{theorem}{Theorem}
\newtheorem{coro}{Corollary}


\theoremstyle{definition}
\newtheorem{defi}{Definition}


\theoremstyle{remark}
\newtheorem{remark}{Remark}

\def\Me{\mathcal M}
\def\Ee{\mathcal E}
\def\Ra{\mathcal R}
\def\Ne{\mathcal N}
\def \Tr{\mathrm{Tr}\,}
\def\Se {\mathcal S}
\def\supp{\mathrm{supp}}
\def\<{\langle\.}
\def\>{\.\rangle}

\title{Note on the limit $\alpha\searrow 1$}
\author{Anna Jen\v cov\'a}

\begin{document}

\maketitle


\section{Conditional expectations and $L_p$-spaces}


Let $\Ne$ be a von Neumann algebra and let $\Me\subseteq \Ne$ be a von Neumann subalgebra
such that there is a conditional expectation $\Ee$ onto $\Me$ preserving a faithful normal
state $\phi$.  Then the  modular group $\sigma^\phi$ of $\phi$
 preserves $\Me$  by
the Takesaki theorem and we have $\sigma^{\phi|_\Me}=\sigma^\phi|_\Me$.  It follows that the crossed
product $\Me\rtimes_{\sigma^{\phi|_\Me}} \mathbb R$ can be identified with a subalgebra in 
$\Ne\rtimes_{\sigma^\phi} \mathbb R$. By \cite[Thm. 4.1]{haagerup2010areduction}, the map 
\[
\hat \Ee:=(\Ee\otimes id_{B(L_2(G))})|_{\Ne\rtimes_{\sigma^\phi}\mathbb R}
\]
is a faithful normal conditional expectation of $\Ne\rtimes_{\sigma^\phi}\mathbb R$ onto
$\Me\rtimes_{\sigma^{\phi|_\Me}} \mathbb R$, moreover, we have 
\[
\hat\sigma\circ \hat\Ee=\hat\Ee\circ\hat\sigma
\]
and clearly also
\[
\hat\Ee(\pi(x)\lambda(s))=\pi(\Ee(x))\lambda(s),\qquad x\i \Me,\ s\in \mathbb R.
\]
For the dual weight $\hat\phi$, we obtain 
\[
\hat\Ee\circ\sigma^{\hat\phi}=\sigma^{\hat\phi}\circ\hat\Ee
\]
and $\hat\phi=\hat\phi\circ\hat\Ee$. Clearly, the dual weight for $\Me$ is the restriction
of $\hat\phi$. Let $\tau$ be the canonical trace and let us denote the canonical trace for
$\Me$ by $\tau_\Me$, then we have by \cite[Cor. 4.22]{takesaki2003theory2}
\[
[D\hat\phi\circ\hat\Ee:\tau_\Me\circ\hat\Ee]_t= [D\hat\phi|_{\Me\rtimes_{\sigma^{\phi|_\Me}}\mathbb
R}:D\tau_\Me]_t=\lambda(t)=[D\hat\phi\circ\hat
\Ee: D\tau]_t,
\]
it follows that $\tau=\tau_\Me\circ\hat\Ee=\tau\circ\hat\Ee$. Consequently, we
can see that the space of $\tau_\Me$-measurable elements $L_0(\Me)$ can be identified with
a *-subalgebra in $L_0(\Ne)$ and therefore also $L_p(\Me)\subseteq L_p(\Ne)$, $0<p\le \infty$. In particular, for
$p=1$ we obtain the  identification $\Me_*\subseteq \Ne_*$, given as 
\[
\omega\equiv \omega\circ \Ee, \qquad \omega\in \Me_*. 
\]
By \cite[Prop. 2.3]{junge2003noncommutative}, for $1\le p\le \infty$, $\Ee$ can be
extended to a contractive projection $\Ee_p$  of  $L_p(\Ne)$ onto $L_p(\Me)$. We have
\[
\Ee_1(h_\omega)=h_{\omega\circ\Ee},\qquad h_\omega\in L_1(\Ne)
\]
and $\Ee_q$ is the adjoint of $\Ee_p$ for $1/p+1/q=1$. The index $p$ is often dropped, so
we just write $\Ee$ instead of $\Ee_p$. For $1\le p,q,r\le \infty$ such that $1/p+1/q+1/r\le 1$, we have  
\begin{equation}\label{eq:condexp}
\Ee(hxk)=h\Ee(x)k,\qquad h\in
L_p(\Me), \ k\in L_q(\Me), \ x\in L_r(\Ne).
\end{equation}

\begin{lemma}\label{lemma:condexp} In the above situation, let $\psi,\varphi\in \Me_*^+$ and let
$\tilde\psi=\psi\circ\Ee$, $\tilde \varphi=\varphi\circ\Ee$. Then for  $1/2<\alpha/2\le z$ we
have
\[
D_{\alpha,z}(\psi\|\varphi)=D_{\alpha,z}(\tilde\psi\|\tilde\varphi).
\]

\end{lemma}


\begin{proof} Using the above identifications, we see that $h_\psi=h_{\tilde\psi}$,
$h_\varphi=h_{\tilde\varphi}$. Assume that $D_{\alpha,z}(\psi\|\varphi) <\infty$, then
there is some $y\in L_{2z}(\Me)s(\varphi)$ such that 
\begin{equation}\label{eq:fin}
h_\psi^{\frac{\alpha}{2z}}=yh_\varphi^{\frac{\alpha-1}2z}.
\end{equation}
Since $L_{2z}(\Me)\subseteq L_{2z}(\Ne)$ and $s(\varphi)=s(\tilde\varphi)$, we see that $y\in L_{2z}(\Ne)s(\tilde\varphi)$, so that
\[
Q_{\alpha,z}(\tilde\psi\|\tilde \varphi)=\|y\|_{2z}^{2z}=Q_{\alpha,z}(\psi\|\varphi).
\]
This implies that $D_{\alpha,z}(\tilde\psi\|\tilde \varphi)\le D_{\alpha,z}(\psi\|\varphi)$ in
general. Assume next that $1/2<\alpha/2\le z$ and let
$D_{\alpha,z}(\tilde\psi\|\tilde \varphi)<\infty$, so that \eqref{eq:fin} is satisfied with
some $y\in L_{2z}(\Ne)s(\tilde \varphi)$. Using the assumption on $\alpha,z$ and 
\eqref{eq:condexp}, we have
\[
h^{\frac{\alpha}{2z}}_{\psi}=\Ee(h_{\psi}^{\frac{\alpha}{2z}})=\Ee(y)h_\varphi^{\frac{\alpha-1}{2z}}.
\]
By uniqueness of $y$ and the fact
that $s(\tilde\varphi)=s(\varphi)\in \Me$, we obtain $y=\Ee(y)\in L_{2z}(\Me)s(\varphi)$.
This finishes the proof.


\end{proof}



\section{The limit $\alpha\searrow 1$}



Haagerup reduction theorem \cite[Thm. 2.1]{haagerup2010areduction} says that there is a
von Neumann algebra $\mathcal R$ with a faithful normal state $\phi$ and a sequence of von
Neumann algebras $(\Ra_n)_{n\ge 1}$ such that
\begin{enumerate}
\item[(i)] $\Me\subseteq \Ra$ and there is a conditional expectation $\Ee$ on $\Ra$ onto
$\Me$ such that $\phi\circ\Ee=\phi$,
\item[(ii)] $\Ra_n\subseteq \Ra_{n+1}$ and each $\Ra_n$ is finite,
\item[(iii)] $\bigcup_n \Ra_n$ is w*-dense in $\Ra$,
\item[(iv)] for each $n$ there is a conditional expectation on $\Ra$ onto $\Ra_n$ such
that $\phi\circ\Ee_n=\phi$.
\end{enumerate}

 

For any $\psi\in \Me_*^+$, let us denote $\hat\psi:=\psi\circ\Ee$ and $\psi_n:=\hat
\psi\circ\Ee_n$. Then $\psi_n\to \hat \psi$ in norm. 
By DPI and  martingale convergence (or DPI + LS), we have
\begin{equation}\label{eq:limit}
D_{\alpha,z}(\psi\|\varphi)=D_{\alpha,z}(\hat\psi\|\hat\varphi)=\lim_nD_{\alpha,z}(\psi_n\|\varphi_n),\qquad \text{if }
\max\{\frac{\alpha}2,\alpha-1\}\le z\le \alpha.
\end{equation}
Using Lemma \ref{lemma:condexp} and  LS, we get 
\begin{equation}\label{eq:ls}
D_{\alpha,z}(\psi\|\varphi)=D_{\alpha,z}(\hat\psi\|\hat\varphi)\le \liminf_nD_{\alpha,z}(\psi_n\|\varphi_n),\qquad \text{if }
\frac{\alpha}2\le z.
\end{equation}


\begin{prop}\label{prop:monotz}
Let $\max\{\alpha/2,\alpha-1\}\le z\le \alpha$. Then for any $z'\ge z$,
\[
D_{\alpha,z'}(\psi\|\varphi)\le D_{\alpha,z}(\psi\|\varphi).
\]
\end{prop}

\begin{proof}
By \cite[Lemma 1.3]{FHnote3}, we have $D_{\alpha,z'}(\psi_n\|\varphi_n)\le
D_{\alpha,z}(\psi_n\|\varphi_n)$ for all $n$. The statement  is proved  by using \eqref{eq:limit} for $z$ and
\eqref{eq:ls} for $z'$.

\end{proof}


\begin{coro}\label{coro:bound1}
Let $1<\alpha\le 2$. Then for any $z\ge 1$, we have
\[
 D_{\alpha,z}(\psi\|\varphi)\le D_{\alpha,1}(\psi\|\varphi).
\]


\end{coro}

\begin{proof} This follows
by putting $z=1$ and $z'=z$ is  Proposition \ref{prop:monotz}.

\end{proof}

The next statement is an extension of \cite[Lemma 2]{AJnote2}.

\begin{lemma}\label{lemma:bounds} Let $\alpha>1$ and $z\ge 1$. 
Then
\[
D_{\beta,1}(\psi\|\varphi)\le D_{\alpha,z}(\psi\|\varphi),
\]
where $\beta:=\frac{\alpha+z-1}{z}>1$.


\end{lemma}

\begin{proof} Using the scaling property of $D_{\alpha,z}$, we may assume that
$\psi(1)=1$.
We will also suppose that $D_{\alpha,z}(\psi\|\varphi)<\infty$, otherwise there is nothing to prove.  In this case, 
\[h_\psi^{\frac{\alpha}{2z}}=yh_\varphi^{\frac{\alpha-1}{2z}}
\]
for some $y\in L_{2z}(\Me)s(\varphi)$. We then get
\[
h_\psi^{\frac{\beta}2}=h_\psi^{\frac{z-1}{2z}}h_\psi^{\frac{\alpha}{2z}}=h_\psi^{\frac{z-1}{2z}}yh_\varphi^{\frac{\alpha-1}{2z}}=\eta h_{\varphi}^{\frac{\beta-1}2}
\]
with $\eta=h_\psi^{\frac{z-1}{2z}}y\in L_2(\Me)$. By \cite[Thm. 3.6]{hiai2021quantum},
it follows that
\[
Q_{\beta,1}(\psi\|\varphi)=\|\Delta_{\psi,\varphi}^{\frac{\beta}2}(h_\varphi^{1/2})\|_2^2=\|\eta\|_2^2\le
\|y\|_{2z}^2=Q_{\alpha,z}(\psi\|\varphi)^{1/z},
\]
this implies the statement.

\end{proof}



\begin{coro}\label{coro:limit}
Assume that  $D_{\alpha_0,z_0}(\psi\|\varphi)<\infty$ for some $1<\alpha_0$ and either
$z_0\ge 1$ or $\alpha_0/2\le z_0\le 1$. 
Then for any $z>1/2$ we have 
\[
\lim_{\alpha\searrow 1} D_{\alpha,z}(\psi\|\varphi)=D_1(\psi\|\varphi)<\infty.
\]

\end{coro}

\begin{proof}
We  first note  that under the above assumptions,
$D_{\beta,1}(\psi\|\varphi)<\infty$ for some $\beta>1$. Indeed, this follows from Lemma
\ref{lemma:bounds} in the first case, or by \cite[Prop. 2.3]{FHnote3} in the second case
(note that we necessarily have $\alpha_0-1\le \alpha_0/2\le z_0\le 1<\alpha_0$).

For $z\ge 1$, the statement now follows by using Lemma \ref{lemma:bounds} and Corollary
\ref{coro:bound1} for  $\alpha$ close
enough to 1. If $1/2<z\le 1$, we may use \cite[Prop. 2.3]{FHnote3} or Proposition
\ref{prop:monotz} for $\alpha$ close
enough to 1.


\end{proof}



%\bibliography{NEW_qre}
%\bibliographystyle{abbrvnat}
\begin{thebibliography}{5}
\providecommand{\natexlab}[1]{#1}
\providecommand{\url}[1]{\texttt{#1}}
\expandafter\ifx\csname urlstyle\endcsname\relax
  \providecommand{\doi}[1]{doi: #1}\else
  \providecommand{\doi}{doi: \begingroup \urlstyle{rm}\Url}\fi


\bibitem[Haagerup et~al.(2010)Haagerup, Junge, and Xu]{haagerup2010areduction}
U.~Haagerup, M.~Junge, and Q.~Xu.
\newblock {A reduction method for noncommutative $L_p$-spaces and
  applications}.
\newblock \emph{Transactions of the American Mathematical Society},
  362\penalty0 (4):\penalty0 2125--2165, 2010.
\newblock \doi{10.1090/S0002-9947-09-04935-6}.
\bibitem{hiai2021quantum}  F. Hiai, Quantum f-Divergences in von Neumann Algebras: Reversibility of Quantum Operations,  Mathematical Physics Studies, Springer Singapore, 2021

\bibitem{FHnote3} F. Hiai, Monotonicity of $z\mapsto D_{\alpha,z}(\psi\|\varphi)$, 
(12/3/2023, 12/8/2023), notes.

\bibitem{AJnote2} A. Jen\v cov\'a, Notes for $\alpha-z$-R\'enyi divergence, December 7,
2023, notes.
\bibitem[Junge and Xu(2003)]{junge2003noncommutative}
M.~Junge and Q.~Xu.
\newblock Noncommutative {B}urkholder/{R}osenthal inequalities.
\newblock \emph{The Annals of Probability}, 31\penalty0 (2):\penalty0 948--995,
  2003.

\bibitem[Takesaki(2003)]{takesaki2003theory2}
M.~Takesaki.
\newblock \emph{Theory of Operator Algebras. {II}}, volume 125 of
  \emph{Encyclopaedia of Mathematical Sciences}.
\newblock Springer-Verlag, Berlin, 2003.
\newblock ISBN 3-540-42914-X.
\newblock \doi{10.1007/978-3-662-10451-4}.


\end{thebibliography}



\end{document}

\begin{thebibliography}{5}
\providecommand{\natexlab}[1]{#1}
\providecommand{\url}[1]{\texttt{#1}}
\expandafter\ifx\csname urlstyle\endcsname\relax
  \providecommand{\doi}[1]{doi: #1}\else
  \providecommand{\doi}{doi: \begingroup \urlstyle{rm}\Url}\fi

\bibitem[Gu et~al.(2019)Gu, Yin, and Zhang]{gu2019interpolation}
J.~Gu, Z.~Yin, and H.~Zhang.
\newblock {Interpolation of quasi noncommutative $L_p$-spaces}.
\newblock \emph{arXiv:1905.08491}, 2019.

\bibitem{FHnotes} F. Hiai, Questions, note.

\bibitem{FHnote2} F. Hiai, Martingale convergence for $D_{\alpha,z}$, note.

\bibitem{AJnote} A. Jen{\v c}ov\'a, DPI for $\alpha-z$-R\'enyi divergence, note.

\bibitem[Kato(2023)]{kato2023onrenyi}
S.~Kato.
\newblock On $\alpha $-$ z $-{R}\'enyi divergence in the von
  {N}eumann algebra setting.
\newblock \emph{arXiv preprint arXiv:2311.01748}, 2023.

\bibitem{SKnote} S.~Kato, Variational expression for $\alpha>1$, note.

\bibitem[Kosaki({1984})]{kosaki1984applications}
H.~Kosaki.
\newblock {Applications of the complex interpolation method to a von Neumann
  algebra: Non-commutative $L_p$-spaces}.
\newblock \emph{{J. Funct. Anal.}}, {56}:\penalty0 {26--78}, {1984}.

\bibitem[Mosonyi(2023)]{mosonyi2023thestrong}
M.~Mosonyi.
\newblock The strong converse exponent of discriminating infinite-dimensional
  quantum states.
\newblock \emph{Communications in Mathematical Physics}, 400\penalty0
  (1):\penalty0 83--132, 2023.

\bibitem[Zhang(2020)]{zhang2020fromwyd}
H.~Zhang.
\newblock From Wigner-Yanase-Dyson conjecture to Carlen-Frank-Lieb conjecture.
\newblock \emph{Advances in Mathematics}, 365:\penalty0 107053, 2020.

\end{thebibliography}







\end{document}



