\documentclass[12pt]{article}

\usepackage{hyperref}
\usepackage{amsmath, amssymb, amsthm}
\usepackage[sort&compress,numbers]{natbib}
\usepackage{doi}
\usepackage[margin=0.8in]{geometry}
%\textheight23cm \topmargin-20mm  
%\textwidth175mm  
%\oddsidemargin=0mm
%\evensidemargin=0mm
%

\usepackage{amsmath, amssymb, amsthm, mathtools}

\newtheorem{lemma}{Lemma}
\newtheorem{prop}{Proposition}
\newtheorem{theorem}{Theorem}
\newtheorem{coro}{Corollary}


\theoremstyle{definition}
\newtheorem{defi}{Definition}


\theoremstyle{remark}
\newtheorem{remark}{Remark}

\def\Me{\mathcal M}
\def\Ee{\mathcal E}
\def\Fe{\mathcal F}
\def\Ra{\mathcal R}
\def\Ne{\mathcal N}
\def \Tr{\mathrm{Tr}\,}
\def\Se {\mathcal S}
\def\supp{\mathrm{supp}}
\def\<{\langle\.}
\def\>{\.\rangle}

\title{A note on monotonicity of  $z\mapsto D_{\alpha,z}$ and $\alpha\mapsto
D_{\alpha,z}(\psi\|\varphi)$ for $1<\alpha\le 2z$}
\author{Anna Jen\v cov\'a}

\begin{document}

\maketitle

We will assume throughout that $Q_{\alpha,z}(\psi\|\varphi)<\infty$ for some $1<\alpha\le
2z$, in which case there is some $y\in L^{2z}(\Me)$ such that
\[
h_\psi^{\frac{\alpha}{2z}}=yh_\varphi^{\frac{\alpha-1}{2z}},\qquad
Q_{\alpha,z}(\psi\|\varphi)=\|y\|_{2z}^{2z}.
\]
In particular, $e:=s(\psi)\le s(\varphi)$, so that we may assume that $\varphi$ is
faithful. Let $\sigma\in \Me_*^+$ be such that
$s(\sigma)=1-e$ and put $\psi_0:=\psi+\sigma$, so that $\psi_0$ is faithful as well. We will use the notation
$L^p_L:=L^p(\Me;\varphi)_L$, $1\le p\le \infty$.

Consider the function 
\[
f(w)=h_{\psi_0}^{\frac{\alpha}{2z}w}eh_\varphi^{1-\frac{\alpha}{2z}w},\qquad w\in S,
\]
where $S:=\{w\in \mathbb C,\ 0\le \mathrm{Re}\,w\le 1\}$. Then $f$ is a bounded continuous
function $S\to L^1(\Me)$, analytic in the interior. Further, 
\[
f(it)=h_{\psi_0}^{\frac{\alpha}{2z}it}eh_\varphi^{-\frac{\alpha}{2z}it}h_\varphi\in
L^\infty_L,\qquad t\in \mathbb R,
\]
and $\|f(it)\|_{L^\infty_L}=1$ for all $t$.
We also have 
\[
f(1+it)=h_{\psi_0}^{\frac{\alpha}{2z}it}h_\psi^{\frac{\alpha}{2z}}h_\varphi^{1-\frac{\alpha}{2z}}h_\varphi^{-\frac{\alpha}{2z}it}=
(h_{\psi_0}^{\frac{\alpha}{2z}it}yh_\varphi^{-\frac{\alpha}{2z}it})h_\varphi^{\frac{2z-1}{2z}}\in
L^{2z}_L,\qquad t\in \mathbb R.
\]
By \cite[Lemmas 10.1 and 10.2]{kosaki1984applications}, 
\[
 \|f(1+it)\|_{L^{2z}_L}=\|h_{\psi_0}^{\frac{\alpha}{2z}it}yh_\varphi^{-\frac{\alpha}{2z}it}\|_{2z}=\|y\|_{2z}
\]
and the functions $t\mapsto f(it)$ and $t\mapsto f(1+it)$ are continuous in $L^{2z}_L$. It
follows that $f\in \Fe'(L^\infty_L, L^{2z}_L)$, that is, $f$ is a function $S\to L^{2z}_L$,
bounded and  continuous on $S$ and analytic in the interior of $S$, such that the boundary
values define  bounded functions to $L^\infty_L$ resp. $L^{2z}_L$, see \cite[Definition
1.4]{kosaki1984applications}. 



\section{Monotonicity in $z$}

Let $z<z'$,  we will prove that $Q_{\alpha,z}(\psi\|\varphi)\ge Q_{\alpha,z'}(\psi\|\varphi)$.
By \cite[Remark 3.4]{kosaki1984applications}, the set of functions
$\Fe'(L^\infty_L,L^{2z}_L)$ defines the interpolation spaces
$C_\theta=C_\theta(L^\infty_L,L^{2z}_L)$, so that for any $\theta\in (0,1)$, $f(\theta)\in
C_\theta$ and 
\[
\|f(\theta)\|_{C_\theta}\le (\max_t \|f(it)\|_{L^\infty_L})^{1-\theta}(\max_t
\|f(1+it)\|_{L^{2z}_L})^\theta=\|y\|_{2z}^\theta.
\]
By the reiteration theorem, $C_\theta=L^{2z/\theta}_L$. Putting $\theta=z/z'$, we get
\[
f(z/z')=h_{\psi}^{\frac{\alpha}{2z'}}h_\varphi^{1-\frac{\alpha}{2z'}}=y'h_\varphi^{\frac{2z'-1}{2z'}}
\]
for some $y'\in L^{2z'}(\Me)$, and  $\|y'\|_{2z'}\le \|y\|_{2z}^{z/z'}$. It follows that
$h_\psi^{\frac{\alpha}{2z'}}=y'h_\varphi^{\frac{\alpha-1}{2z'}}$, so that 
\[
Q_{\alpha,z'}(\psi\|\varphi)=\|y'\|_{2z'}^{2z'}=\|f(z/z')\|_{L^{2z'}_L}^{2z'},
\]
this proves the result.


\section{Monotonicity in $\alpha$}


 The above function  allows us also to prove monotonicity in $\alpha$. Indeed,
let $1<\alpha'<\alpha$. For any $t\in \mathbb R$,
\[
f(\frac1{\alpha}+it)=h_{\psi_0}^{\frac{\alpha}{2z}it}h_\psi^{\frac{1}{2z}}h_\varphi^{-\frac{\alpha}{2z}it}h_\varphi^{\frac{2z-1}{2z}},
\]
so that $\|f(\frac1{\alpha}+it)\|_{L^{2z}_L}\le \psi(1)^{\frac1{2z}}$. Further, since
$\frac{\alpha'}{\alpha}<1$, we get $f(\frac{\alpha'}{\alpha})\in L^{2z}_L$, so that there is some $y'\in
L^{2z}(\Me)$ such that 
\[
f(\frac{\alpha'}{\alpha})=h_\psi^{\frac{\alpha'}{2z}}h_\varphi^{1-\frac{\alpha'}{2z}}=y'h_\varphi^{\frac{2z-1}{2z}}
\]
so that $h_\psi^{\frac{\alpha'}{2z}}=y'h_\varphi^{\frac{\alpha'-1}{2z}}$ and
$Q_{\alpha',z}(\psi\|\varphi)=\|y'\|^{2z}_{2z}=\|f(\frac{\alpha'}{\alpha})\|_{L^{2z}_L}^{2z}$. Now let $\lambda$ be such that
$(1-\lambda)+\lambda \alpha=\alpha'$, so that $\lambda=\frac{\alpha'-1}{\alpha-1}$, then
by the Hadamard three lines theorem, we get
\[
Q_{\alpha',z}(\psi\|\varphi)=\|f(\frac{\alpha'}{\alpha})\|^{2z}_{L^{2z}_L}\le \left(\max_t
\|f(\frac1{\alpha}+it)\|_{L^{2z}_L}^{1-\lambda}\max_t\|f(1+it)\|_{L^{2z}_L}^\lambda\right)^{2z}\le
\psi(1)^{1-\lambda}Q_{\alpha,z}(\psi\|\varphi)^\lambda,
\]
%which implies that
%\[
%\frac{Q_{\alpha',z}}{\psi(1)}\le
%\left(\frac{Q_{\alpha,z}(\psi\|\varphi)}{\psi(1)}\right)^\lambda,
%\]
this proves that $D_{\alpha',z}(\psi\|\varphi)\le D_{\alpha,z}(\psi\|\varphi)$. 

\section{The limit $\alpha\searrow 1$}

 We now try to prove the limit $\lim_{\alpha \searrow 1}
D_{\alpha,z}(\psi\|\varphi)$, using the same  ideas as in  \cite{FHnote7} but applying  analyticity of
the function f instead of the Connes cocycle.

The function $f$ is analytic in a neighborhood of
$\frac1{\alpha}$. Therefore, we have the expansion \[
f(w)=f(\frac1{\alpha})+(w-\frac1{\alpha})h+o(w-\frac1{\alpha})
\]
where $h\in L^{2z}_L$ is the derivative of $f$ at $w=\frac1{\alpha}$
and $\frac{\|o(\omega)\|_{L^{2z}_L}}{|\omega|}\to 0$ as $|\omega|\to
0$. It follows that for $1<\alpha'<\alpha$, 
\[
f(\frac{\alpha'}{\alpha})=f(\frac1{\alpha})+\frac{\alpha'-1}{\alpha}h+o(\frac{\alpha'-1}{\alpha})
\]
Using the fact that the $L^p_L$ spaces are uniformly Fr\'echet differentiable, we can
prove similarly  as in \cite{FHnote7} that
\[
\lim_{\alpha'\to 1}\frac{\|f(\frac{\alpha'}{\alpha})\|_{L^{2z}_L}-\|f(\frac1{\alpha})\|_{L^{2z}_L}
}{\frac{\alpha'-1}{\alpha}}=\<a_0, h\>
\]
where $\<\cdot,\cdot\>$ is the duality between $L^{2z}_L$ and $L^{\frac{2z}{2z-1}}_L$ and $a_0$ is the element in $L^{\frac{2z}{2z-1}}_L$ with unit norm such that $\Tr
a_0f(\frac1{\alpha})=\|f(\frac1{\alpha})\|_{L^{2z}_L}$, that is, 
\[
a_0=\left(\frac{h_\psi}{\psi(1)}\right)^{\frac{2z-1}{2z}}h_\varphi^{\frac1{2z}}.
\]
Since $f$ is uniformly differentiable and $h$ is the derivative of $f$ at
$\frac1{\alpha}$, we have
\begin{align*}
\<a_0,h\>&=\lim_{t\to 0}(it)^{-1}\<a_0,f(\frac1{\alpha}+it)-f(\frac1{\alpha})\>=\lim_{t\to
0}(it)^{-1}\<a_0,\left(h_\psi^{\frac{1}{2z}}h_{\psi_0}^{\frac{\alpha}{2z}it}h_\varphi^{-\frac{\alpha}{2z}it}-h_\psi^{\frac{1}{2z}}\right)h_\varphi^{\frac{2z-1}{2z}}\>\\
&= \psi(1)^{-\frac{2z-1}{2z}}\frac{\alpha}{2z}\lim_{t\to
0}(it)^{-1}\<h_\psi^{\frac{2z-1}{2z}}h_\varphi^{\frac1{2z}},
\left(h_\psi^{\frac{1}{2z}}h_{\psi_0}^{it}h_\varphi^{-it}-h_\psi^{\frac{1}{2z}}\right)h_\varphi^{\frac{2z-1}{2z}}\>\\
&=\psi(1)^{-\frac{2z-1}{2z}}\frac{\alpha}{2z}\lim_{t\to
0}(it)^{-1}\Tr
h_\psi\left(h_{\psi_0}^{it}h_\varphi^{-it}-1\right)=\psi(1)^{-\frac{2z-1}{2z}}\frac{\alpha}{2z}D(\psi\|\varphi),
\end{align*}
where we use \cite[Thm.5.7]{ohya1993quantum} in the last equality. We therefore have
\begin{align*}
\lim_{\alpha'\searrow 1} D_{\alpha',z}(\psi\|\varphi)=&\lim_{\alpha'\searrow 1} \frac{\log
Q_{\alpha',z}(\psi\|\varphi)-\log\psi(1)}{\alpha'-1}=\lim_{\alpha'\searrow 1}
\frac{2z\log\|f(\frac{\alpha'}{\alpha})\|_{L^{2z}_L}-2z\log\|f(\frac1\alpha)\|_{L^{2z}_L}}{\alpha'-1}\\
&= \lim_{\alpha'\to1}
\left(\frac{\log\|f(\frac{\alpha'}{\alpha})\|_{L^{2z}_L}-\log\|f(\frac1{\alpha})\|_{L^{2z}_L}}{\|f(\frac{\alpha'}{\alpha})\|_{L^{2z}_L}-
\|f(\frac1{\alpha})\|_{L^{2z}_L}}\right)\frac{2z}{\alpha} \left(\frac{\|f(\frac{\alpha'}{\alpha})\|_{L^{2z}_L}-
\|f(\frac1{\alpha}\|_{L^{2z}_L}}{\frac{\alpha'-1}{\alpha}}\right)\\
&= \psi(1)^{-1}D(\psi\|\varphi).
\end{align*}


%\bibliography{NEW_qre}
%\bibliographystyle{abbrvnat}


\begin{thebibliography}{2}
\providecommand{\natexlab}[1]{#1}
\providecommand{\url}[1]{\texttt{#1}}
\expandafter\ifx\csname urlstyle\endcsname\relax
  \providecommand{\doi}[1]{doi: #1}\else
  \providecommand{\doi}{doi: \begingroup \urlstyle{rm}\Url}\fi


\bibitem{FHnote7} F. Hiai, $\lim_{\alpha\searrow 1}$ when $1/2<z<1$, (12/27/2023) notes.

\bibitem[Kosaki({1984})]{kosaki1984applications}
H.~Kosaki.
\newblock {Applications of the complex interpolation method to a von Neumann
  algebra: Non-commutative $L_p$-spaces}.
\newblock \emph{{J. Funct. Anal.}}, {56}:\penalty0 {26--78}, {1984}.
\newblock \doi{https://doi.org/10.1016/0022-1236(84)90025-9}.

\bibitem[Ohya and Petz(1993)]{ohya1993quantum}
M.~Ohya and D.~Petz.
\newblock \emph{Quantum Entropy and Its Use}.
\newblock Lecture Notes in Computer Science. Springer-Verlag, 1993.
\newblock ISBN 9783540548812.


\end{thebibliography}

\end{document}

\begin{thebibliography}{5}
\providecommand{\natexlab}[1]{#1}
\providecommand{\url}[1]{\texttt{#1}}
\expandafter\ifx\csname urlstyle\endcsname\relax
  \providecommand{\doi}[1]{doi: #1}\else
  \providecommand{\doi}{doi: \begingroup \urlstyle{rm}\Url}\fi



\bibitem[Kosaki({1984})]{kosaki1984applications}
H.~Kosaki.
\newblock {Applications of the complex interpolation method to a von Neumann
  algebra: Non-commutative $L_p$-spaces}.
\newblock \emph{{J. Funct. Anal.}}, {56}:\penalty0 {26--78}, {1984}.
\newblock \doi{https://doi.org/10.1016/0022-1236(84)90025-9}.

\end{thebibliography}



\end{document}

