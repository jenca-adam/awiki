\documentclass[12pt]{article}

%\usepackage{hyperref}
\usepackage[dvipdfmx]{hyperref}
\usepackage{amsmath, amssymb, amsthm}
\usepackage{xcolor}
\usepackage[sort&compress,numbers]{natbib}
\usepackage{doi}
\usepackage[margin=0.8in]{geometry}
\textheight23cm \topmargin-16mm  
%\textwidth175mm  
%\oddsidemargin=0mm
%\evensidemargin=0mm
%

\usepackage{amsmath, amssymb, amsthm, mathtools}

\newtheorem{theorem}{Theorem}[section]
\newtheorem{lemma}[theorem]{Lemma}
\newtheorem{coro}[theorem]{Corollary}
\newtheorem{prop}[theorem]{Proposition}

\theoremstyle{definition}
\newtheorem{defi}[theorem]{Definition}

\theoremstyle{remark}
\newtheorem{remark}[theorem]{Remark}

\numberwithin{equation}{section}

\def\cE{\mathcal E}
\def\cM{\mathcal M}
\def\Me{\mathcal M}
\def\Ne{\mathcal N}
\def\cF{\mathcal{F}}
\def\cR{\mathcal{R}}
%\def\Tr{\mathrm{Tr}\,}%I prefer "tr" more than "Tr", though not essential.
\def\Tr{\mathrm{tr}}
\def\states {\mathfrak S}
\def\supp{\mathrm{supp}}
%\def\<{\langle\.}%This gives an error in my MAC-TeX.
%\def\>{\.\rangle}
\def\<{\langle}
\def\>{\rangle}
\def\ffi{\varphi}
\def\1{\mathbf{1}}
\def\eps{\varepsilon}
\def\bN{\mathbb{N}}
\def\bR{\mathbb{R}}
\def\bZ{\mathbb{Z}}
\def\bC{\mathbb{C}}
\def\Re{\mathrm{Re}\,}

\title{On the properties $\alpha$-$z$-R\'enyi divergences on general von Neumann algebras

{\color{red}
------------------------

$\alpha$-$z$-R\'enyi divergences in von Neumann algebras:
data-processing inequality, reversibility and monotonicity properties

-----------------------

$\alpha$-$z$-R\'enyi divergences in von Neumann algebras via non-commutative $L_p$-spaces
}}
\author{Fumio Hiai and Anna Jen\v cov\'a}


\begin{document}

\maketitle


\begin{abstract}

///////////////////////////////

//////////////////////////////

//////////////////////////////

\bigskip\noindent
{\it 2020 Mathematics Subject Classification:}

////////////////////////////////////

\medskip\noindent
{\it Keywords and phrases:}

//////////////////////////////////

\end{abstract}


\tableofcontents




\section{Preliminaries}

%\subsection{Basic definitions}

{\color{red}[Since the subsection ``Basic definitions'' is very short, it seems better to write this
without making a subsection.]}

Let $\Me$ be a von Neumann algebra and let $\Me^+$ be the cone of positive elements in $\Me$. We
denote the predual of $\Me$ by $\Me_*$, its positive part by $\Me_*^+$ and the set of normal states by
$\states_*(\Me)$. For $\psi\in \Me_*^+$, we will denote by $s(\psi)$ the support projection  of $\psi$.

For $0< p\le \infty$, let $L_p(\Me)$ be the \emph{Haagerup $L_p$-space}
\cite{haagerup1979lpspaces,terp1981lpspaces} over $\Me$ and let $L_p(\Me)^+$ its positive cone. We will
use the identifications $\Me\cong L_\infty(\Me)$ and $\Me_*\ni \psi \leftrightarrow h_\psi\in L_1(\Me)$,
so that the $\Tr$-functional on $L_1(\Me)$ defined by $\Tr\,h_\psi:=\psi(1)$ for $\psi\in \Me_*$. It this way,
$\Me_*^+$ is identified with the positive cone $L_1(\Me)^+$, and $\states_*(\Me)$ with the set of elements
$h\in L_1(\Me)^+$ with $\Tr\,h=1$. Precise definitions and further details on the spaces $L_p(\Me)$ can be
found in \cite[Chap.~9]{hiai2021lectures}, or in the notes \cite{terp1981lpspaces}. {\color{red}A short summary
on the Haagerup $L_P$-spaces and} some technical results that will be used below can be found in
Appendix \ref{app:lp}.

%\subsection{The $\alpha$-$z$-R\'enyi divergences}

In \cite{kato2023aremark, kato2023onrenyi}, the
$\alpha$-$z$-R\'enyi divergence for $\psi,\varphi\in \mathcal M_*^+$  was defined as follows:

\begin{defi}\label{defi:renyi} Let $\psi,\varphi\in \Me_*^+$, $\psi\ne 0$ and let
$\alpha,z>0$, $\alpha\ne 1$. The \emph{$\alpha$-$z$-R\'enyi divergence} is defined as 
\[
D_{\alpha,z}(\psi\|\varphi):=\frac1{\alpha-1}\log
\frac{Q_{\alpha,z}(\psi\|\varphi)}{\psi(1)},
\]
where
\[
Q_{\alpha,z}(\psi\|\varphi):=\begin{dcases} \Tr
\left(h_\varphi^{\frac{1-\alpha}{2z}}h_\psi^{\frac{\alpha}{z}}h_\varphi^{\frac{1-\alpha}{2z}}\right)^z, &
\text{if } 0<\alpha<1,\\[0.3em]
\|x\|_z^z, & \text{if } \alpha>1 \text{ and }
h_\psi^{\frac{\alpha}{z}}=h_\varphi^{\frac{\alpha-1}{2z}}xh_\varphi^{\frac{\alpha-1}{2z}}
\ \text{with}\\ & x\in s(\varphi)L_z(\Me)s(\varphi),\\[0.3em]
\infty,& \text{otherwise}.
\end{dcases}
\]
\end{defi}


In the case $\alpha>1$, the following alternative form will be useful.

\begin{lemma}[\mbox{\cite[Lemma 7]{kato2023onrenyi}}]\label{lemma:renyi_2z}
Let $\alpha>1$ and $\psi,\varphi\in \Me_*^+$. Then $Q_{\alpha,z}(\psi\|\varphi)<\infty$ if
and only if there is some $y\in L_{2z}(\Me)s(\varphi)$ such that 
\[
h_\psi^{\frac{\alpha}{2z}}=yh_\varphi^{\frac{\alpha-1}{2z}}.
\]
Moreover, in this case, such $y$ is unique and we have
$Q_{\alpha,z}(\psi\|\varphi)=\|y\|_{2z}^{2z}$. 
\end{lemma}

The \emph{standard} or \emph{Petz-type R\'enyi divergence}
\cite{petz1985quasi,hiai2018quantum,hiai2021quantum} is contained in this range as
$D_\alpha(\psi\|\varphi)=D_{\alpha,1}(\psi\|\varphi)$. Also, the \emph{sandwiched R\'enyi divergence} is
obtained as $\tilde D_\alpha(\psi\|\varphi)=D_{\alpha,\alpha}(\psi\|\varphi)$; see
\cite{berta2018renyi,hiai2021quantum,jencova2018renyi, jencova2021renyi} for some
alternative definitions and properties of $\tilde D_\alpha$. {\color{red}The definitions in
\cite{jencova2018renyi,jencova2021renyi} are based on \emph{Kosaki's interpolation $L_p$-spaces}
$L_p(\Me,\varphi)$ \cite{kosaki1984applications} with respect to $\varphi$. These spaces and the
complex interpolation method are briefly summarized in Appendix \ref{Appen-Kosaki-Lp}, and will be used
frequently in the present work.}

{\color{red}As have already been done by Kato in \cite{kato2023onrenyi}, many of the properties of
$D_{\alpha,z}(\psi\|\varphi)$ are extended from the finite-dimensional case into the general von Neumann
algebra case}. In particular, the following variational expressions will be an important tool for our work.

\begin{theorem}[Variational expressions]\label{thm:variational} Let $\psi,\varphi\in \Me_*^+$, $\psi\ne 0$. 
\begin{enumerate}
\item[(i)] Let $0<\alpha<1$, $\max\{\alpha,1-\alpha\}\le z$. Then
\begin{align}\label{F-1.1}
Q_{\alpha,z}(\psi\|\varphi)=\inf_{a\in \Me^{++}}\left\{\alpha
\Tr\left((a^{\frac12}h_\psi^{\frac{\alpha}{z}}a^{\frac12})^{\frac{z}{\alpha}}\right)+(1-\alpha)
\Tr\left((a^{-\frac12}h_\varphi^{\frac{1-\alpha}{z}}a^{-\frac12})^{\frac{z}{1-\alpha}}\right) \right\}.
\end{align}

\item[(ii)] Let $\alpha>1$, $\max\{\alpha/2,\alpha-1\}\le z$. Then
\begin{align}\label{F-1.2}
Q_{\alpha,z}(\psi\|\varphi)=\sup_{a\in \Me_+} \left\{\alpha
\Tr\left((a^{\frac12}h_\psi^{\frac{\alpha}{z}}a^{\frac12})^{\frac{z}{\alpha}}\right)-(\alpha-1)
\Tr\left((a^{\frac12}h_\varphi^{\frac{\alpha-1}{z}}a^{\frac12})^{\frac{z}{\alpha-1}}\right) \right\}.
\end{align}
{\color{red}Moreover, if $\psi\le\lambda\ffi$ for some $\lambda>0$, then \eqref{F-1.2} holds for all
$z\ge\alpha-1>0$. [Please check if this addition is OK.]}
\end{enumerate}
\end{theorem}

\begin{proof} For part (i) see \cite[Theorem 1(vi)]{kato2023onrenyi}. The inequality $\ge$ in part (ii)
holds for all $\alpha$ and $z$ and was proved in \cite[Theorem 2(vi)]{kato2023onrenyi}. We now prove
the opposite inequality. 

Assume first that $Q_{\alpha,z}(\psi\|\varphi)<\infty$, so that there is some $x\in
s(\varphi)L_z(\Me)^+s(\varphi)$ such that
$h_\psi^{\frac{\alpha}{z}}=h_\varphi^{\frac{\alpha-1}{2z}}xh_\varphi^{\frac{\alpha-1}{2z}}$. Plugging this
into the right-hand side of \eqref{F-1.2}, we obtain
\begin{align}
&\sup_{a\in \Me_+} \left\{\alpha
\Tr\left((a^{\frac12}h_\psi^{\frac{\alpha}{z}}a^{\frac12})^{\frac{z}{\alpha}}\right)-(\alpha-1)
\Tr\left((a^{\frac12}h_\varphi^{\frac{\alpha-1}{2z}}a^{\frac12})^{\frac{z}{\alpha-1}}\right) \right\}
\nonumber\\
&\quad=\sup_{a\in \Me_+} \left\{\alpha
\Tr\left((a^{\frac12}h_\varphi^{\frac{\alpha-1}{2z}}xh_\varphi^{\frac{\alpha-1}{2z}}
a^{\frac12})^{\frac{z}{\alpha}}\right)-(\alpha-1)
\Tr\left((a^{\frac12}h_\varphi^{\frac{\alpha-1}{2z}}a^{\frac12})^{\frac{z}{\alpha-1}}\right) \right\}
\nonumber\\
&\quad=\sup_{a\in \Me_+} \left\{\alpha
\Tr\left((x^{\frac12}h_\varphi^{\frac{\alpha-1}{2z}}ah_\varphi^{\frac{\alpha-1}{2z}}
x^{\frac12})^{\frac{z}{\alpha}}\right)-(\alpha-1)
\Tr\left((h_\varphi^{\frac{\alpha-1}{2z}}a h_\varphi^{\frac{\alpha-1}{2z}}
)^{\frac{z}{\alpha-1}}\right)\right\}\nonumber\\
&\quad=\sup_{w\in L_{\frac{z}{\alpha-1}}(\Me)^+} \left\{\alpha
\Tr\left((x^{\frac12}wx^{\frac12})^{\frac{z}{\alpha}}\right)-(\alpha-1)
\Tr\left(w^{\frac{z}{\alpha-1}}\right)
\right\}, \label{F-1.3}
\end{align}
where we have used the fact that $\Tr\left((h^*h)^p\right)=\Tr\left((hh^*)^p\right)$ for
$p>0$, $h\in L_{\frac{p}{2}}(\Me)$, and Lemma \ref{lemma:cone}.
 Putting $w=x^{\alpha-1}$ we
get
\[
\sup_{w\in L_{\frac{z}{\alpha-1}}(\Me)^+} \left\{\alpha
\Tr\left((x^{\frac12}wx^{\frac12})^{\frac{z}{\alpha}}\right)-(\alpha-1)
\Tr\left(w^{\frac{z}{\alpha-1}}\right)
\right\}\ge \Tr(x^z)=\|x\|_z^z= Q_{\alpha,z}(\psi\|\varphi).
\]
This finishes the proof of \eqref{F-1.2} in the case that $Q_{\alpha,z}(\psi\|\varphi)<\infty$. Note that
this holds if $\psi\le \lambda\varphi$ for some $\lambda>0$. {\color{red}Indeed, since
$\frac{\alpha-1}{z}\in (0,1]$ by the assumption, we then have 
$h_\psi^{\frac{\alpha-1}{z}}\le \lambda^{\frac{\alpha-1}{z}}h_\varphi^{\frac{\alpha-1}{z}}$.
Hence by \cite[Lemma A.58]{hiai2021quantum} there is some $b\in \Me$ such that 
\[
h_\psi^{\frac{\alpha-1}{2z}}=bh_\varphi^{\frac{\alpha-1}{2z}},
\]
so that $h_\psi^{\alpha\over2z}=yh_\ffi^{\alpha-1\over2z}$, where
$y:=h_\varphi^{\frac{1}{2z}}b\in L_{2z}(\Me)$. By Lemma \ref{lemma:renyi_2z},
$Q_{\alpha,z}(\psi\|\varphi)=\|y\|_{2z}^{2z}<\infty$. This shows the latter assertion too.}
{\color{red}[My previous proof here is bad. Is this OK?]}

In the general case, the variational expression holds for
$Q_{\alpha,z}(\psi\|\varphi+\varepsilon\psi)$  for all $\varepsilon>0$, so that we have
\begin{align*}
Q_{\alpha,z}(\psi\|\varphi+\varepsilon\psi)&=\sup_{a\in \Me_+} \left\{\alpha
\Tr\left((a^{\frac12}h_\psi^{\frac{\alpha}{z}}a^{\frac12})^{\frac{z}{\alpha}}\right)
-(\alpha-1)\Tr\left((a^{\frac12}h_{\varphi+\varepsilon \psi}^{\frac{\alpha-1}{z}}
a^{\frac12})^{\frac{z}{\alpha-1}}\right)\right\}\\
&\le\sup_{a\in \Me_+} \left\{\alpha
\Tr\left((a^{\frac12}h_\psi^{\frac{\alpha}{z}}a^{\frac12})^{\frac{z}{\alpha}}\right)-(\alpha-1)
\Tr\left((a^{\frac12}h_\varphi^{\frac{\alpha-1}{z}}a^{\frac12})^{\frac{z}{\alpha-1}}\right)
\right\},
\end{align*}
where the inequality above follows by Lemma \ref{lemma:order}. Therefore,
{\color{red}since $z\ge\alpha/2$, from lower semicontinuity \cite[Theorem 2(iv)]{kato2023onrenyi}
we have}
\[
Q_{\alpha,z}(\psi\|\varphi)\le \liminf_{\varepsilon\searrow 0}
Q_{\alpha,z}(\psi\|\varphi+\varepsilon \psi),
\]
so that the desired inequality is obtained.
\end{proof}

We finish this section by investigation of the properties of the variational expression for
$0<\alpha<1$, in the case when $\lambda^{-1}\ffi\le \psi\le \lambda \ffi$ for some
$\lambda>0$. This case will be denoted as $\psi\sim \ffi$. 

\begin{lemma}\label{lemma:variational_majorized}
Assume that $\psi\sim\ffi$. Then the infimum in \eqref{F-1.1} of Theorem \ref{thm:variational}(i) is
attained at a unique element $\bar a\in \Me^{++}$. This element satisfies
\begin{align}
h_\psi^{\alpha\over2z}\bar ah_\psi^{\alpha\over2z}
&=\Bigl(h_\psi^{\alpha\over2z}h_\ffi^{1-\alpha\over z}h_\psi^{\alpha\over2z}\Bigr)^\alpha
\label{eq:minimizer1}\\
h_\ffi^{1-\alpha\over2z}\bar a^{-1}h_\ffi^{1-\alpha\over2z}
&=\Bigl(h_\ffi^{1-\alpha\over2z}h_\psi^{\alpha\over
z}h_\ffi^{1-\alpha\over2z}\Bigr)^{1-\alpha}.
\label{eq:minimizer2}
\end{align}
\end{lemma}

\begin{proof} We may assume that $\varphi$ and hence also $\psi$ {\color{red}are(?)} faithful. Following
 the proof of \cite[Theorem 1(vi)]{kato2023onrenyi},  we may use the
assumptions and \cite[Lemma A.58]{hiai2021quantum} to show  that there are $b,c\in\cM$  such that
\begin{align}\label{eq:bc}
h_\ffi^{1-\alpha\over2z}
=b\Bigl(h_\ffi^{1-\alpha\over2z}h_\psi^{\alpha\over z}h_\ffi^{1-\alpha\over2z}\Bigr)^{1-\alpha\over2},\qquad
\Bigl(h_\ffi^{1-\alpha\over2z}h_\psi^{\alpha\over z}h_\ffi^{1-\alpha\over2z}\Bigr)^{1-\alpha\over2}
=ch_\ffi^{1-\alpha\over2z}.
\end{align}
With $\bar a:=bb^*\in\cM^{++}$ we have  $\bar a^{-1}=c^*c$ and $\bar a$ is indeed a minimizer of
\eqref{F-1.2}, equivalently,
\begin{align}\label{eq:infimum}
Q_{\alpha,z}(\psi\|\ffi)
=\inf_{a\in\cM^{++}}\biggl\{\alpha\Big\|h_\psi^{\alpha\over2z}ah_\psi^{\alpha\over2z}\Big\|_{z\over\alpha}^{z\over\alpha}
+(1-\alpha)\Big\|h_\ffi^{1-\alpha\over2z}a^{-1}h_\ffi^{1-\alpha\over2z}\Big\|_{z\over
1-\alpha}^{z\over 1-\alpha}\biggr\}.
\end{align}
We next observe that such a minimizer is unique. Indeed, suppose that the infimum is
attained  at some $a_1,a_2\in \Me^{++}$. Let $a_0:=(a_1+a_2)/2$. Since the map 
$L_{p}(\cM)\ni k\mapsto\|k\|_{p}^{p}$ is convex for any $p\ge 1$ and
$a_0^{-1}\le(a_1^{-1}+a_2^{-1})/2$, we have 
\[
\Big\|h_\psi^{\alpha\over2z}a_0h_\psi^{\alpha\over2z}\Big\|_{z\over\alpha}^{z\over\alpha}
\le{1\over2}\biggl\{\Big\|h_\psi^{\alpha\over2z}a_1
h_\psi^{\alpha\over2z}\Big\|_{z\over\alpha}^{z\over\alpha}
+\Big\|h_\psi^{\alpha\over2z}a_2h_\psi^{\alpha\over2z}\Big\|_{z\over\alpha}^{z\over\alpha}\biggr\}.
\]
Moreover, using Lemma \ref{lemma:order1} we have
\begin{align*}
\Big\|h_\ffi^{1-\alpha\over2z}a_0^{-1}h_\ffi^{1-\alpha\over2z}\Big\|_{z\over1-\alpha}^{z\over1-\alpha}
&\le\Big\|h_\ffi^{1-\alpha\over2z}\biggl({a_1^{-1}+a_2^{-1}\over2}\biggr)
h_\ffi^{1-\alpha\over2z}\Big\|_{z\over1-\alpha}^{z\over1-\alpha} \\
&\le{1\over2}\biggl\{\Big\|h_\ffi^{1-\alpha\over2z}a_1^{-1}
h_\ffi^{1-\alpha\over2z}\Big\|_{z\over1-\alpha}^{z\over1-\alpha}
+\Big\|h_\ffi^{1-\alpha\over2z}a_2^{-1}
h_\ffi^{1-\alpha\over2z}\Big\|_{z\over1-\alpha}^{z\over1-\alpha}\biggr\}.
\end{align*}
Hence the assumption of $a_1,a_2$ being a minimizer gives
\[
\Big\|h_\ffi^{1-\alpha\over2z}a_0^{-1}h_\ffi^{1-\alpha\over2z}\Big\|_{z\over1-\alpha}
=\Big\|h_\ffi^{1-\alpha\over2z}\biggl({a_1^{-1}+a_2^{-1}\over2}\biggr)
h_\ffi^{1-\alpha\over2z}\Big\|_{z\over1-\alpha},
\]
which implies that $a_0^{-1}={a_1^{-1}+a_2^{-1}\over2}$, as easily verified by Lemma \ref{lemma:order1}
again. From this we easily have $a_1=a_2$.

The equality  \eqref{eq:minimizer2} is obvious from the second equality in \eqref{eq:bc} and
$\bar a^{-1}=c^*c$. Since $Q_{\alpha,z}(\psi\|\ffi)=Q_{1-\alpha,z}(\ffi\|\psi)$, we see by uniqueness that
the minimizer of the infimum expression for $Q_{1-\alpha,z}(\ffi\|\psi)$ (instead of \eqref{eq:infimum}) is
$\bar a^{-1}$ (instead of $\bar a$). This says that \eqref{eq:minimizer1} is the equality corresponding to
\eqref{eq:minimizer2} when $\psi,\ffi,\alpha$ are replaced with $\ffi,\psi,1-\alpha$, respectively. 
\end{proof}

To make the next lemma more readable, we will use the following notations:
\[
p:={z\over\alpha},\qquad r:={z\over 1-\alpha},\qquad
\xi_p(a):=h_\psi^{1\over2p}ah_\psi^{1\over 2p},\qquad
\eta_r(a)=h_\ffi^{1\over2r}a^{-1}h_\ffi^{1\over 2r}.
\]
We will also denote the function under the infimum in the variational expression in
Theorem \ref{thm:variational}(i) by $f$, that is,
\begin{equation}\label{func-variational}
f(a)=\alpha\|\xi_p(a)\|_p^p
+(1-\alpha)\|\eta_r(a)\|_r^r,\qquad a\in \Me^{++}.
\end{equation}
{\color{red}When $p\in(1,\infty)$, recall that $L_p(\Me)$ is uniformly convex (see
\cite{haagerup1979lpspaces}, \cite[Theorem 4.2]{kosaki1984applications}), so that the norm
$\|\cdot\|_p$ is uniformly Fr\'echet differentiable (see, e.g.,
\cite[Part 3, Chap.~II]{beauzamy1982introduction}). Hence $a\mapsto\|\xi_p(a)\|_p^p$
and $a\mapsto\|\eta_r(a)\|_r^r$ are Fr\'echet differentiable on $\Me^{++}$. Since differentiability of
these functions is obvious when $p=1$ and $r=1$, we see that the function $f$ is Fr\'echet differentiable
on $\Me^{++}$ for any $p,r\ge1$, whose Fr\'echet derivative at $a$ will be denoted by $\nabla f(a)$.}

\begin{lemma}\label{lemma:variational_majorized2}
Assume that $\psi\sim\ffi$  and let $0<\alpha<1$, $\max\{\alpha,1-\alpha\}\le z$. {\color{red}Let
$\bar a\in\Me^{++}$ be as given in Lemma \ref{lemma:variational_majorized}.} If $p>1$, then for
every $C\ge Q_{\alpha,z}(\psi\|\ffi)$ and $\eps>0$ there is some $\delta>0$ such that whenever
$\|\xi_p(b)\|^p_p\le C$ and $\|\xi_p(b)-\xi_p(\bar a)\|_p\ge \eps$, we have
\[
f(b)-Q_{\alpha,z}(\psi\|\ffi)\ge \delta.
\]
A similar statement holds if $r>1$.
\end{lemma}

\begin{proof} By assumptions, $p,r\ge 1$.  For
$a,b\in \Me^{++}$ and $s\in (0,1/2)$, we have
\begin{align*}
\|\xi_p(sb+(1-s)a)\|_p^p&=\|s\xi_p(b)+(1-s)\xi_p(a)\|_p^p\\
&=\Big\|(1-2s)\xi_p(a)+2s\frac12(\xi_p(a)+\xi_p(b))\Big\|_p^p\\
&\le (1-2s)\|\xi_p(a)\|_p^p+2s\Big\|\frac12(\xi_p(a)+\xi_p(b))\Big\|_p^p.
\end{align*}
Similarly,
\[
\|\eta_r(sb+(1-s)a)\|_r^r\le
(1-2s)\|\eta_r(a)\|_r^r+2s\Big\|\frac12(\eta_r(a)+\eta_r(b))\Big\|_r^r,
\]
where we have also used the fact that $(t a+(1-t)b)^{-1}\le t
a^{-1}+(1-t)b^{-1}$ for $t\in (0,1)$ and Lemma \ref{lemma:order1}. It follows that
\begin{align*}
&\<\nabla f(a),b-a\>\\
&\quad=\lim_{s\to 0^+} {1\over s}[ f(sb+(1-s)a)-f(a)]\\
&\quad\le 2\alpha\biggl[\Big\|\frac12(\xi_p(a)+\xi_p(b))\Big\|_p^p-\|\xi_p(a)\|_p^p\biggr]
+2(1-\alpha)\biggl[\Big\|\frac12(\eta_r(a)+\eta_r(b))\Big\|_r^r-\|\eta_r(a)\|_r^r\biggr]\\
&\quad=f(b)-f(a)-2\Biggl\{\alpha\biggl[\frac12 \|\xi_p(a)\|_p^p
+\frac12 \|\xi_p(b)\|_p^p-\Big\|\frac12(\xi_p(a)+\xi_p(b))\Big\|_p^p\biggr]\\
&\qquad\qquad\qquad+(1-\alpha)\biggl[\frac12 \|\eta_r(a)\|_r^r+\frac12 \|\eta_r(b)\|_r^r
-\Big\|\frac12(\eta_r(a)+\eta_r(b))\Big\|_r^r\biggr]\Biggr\}.
%&\quad=\begin{multlined}[t]f(b)-f(a)-2\Biggl\{\alpha\biggl[\frac12 \|\xi_p(a)\|_p^p
%+\frac12 \|\xi_p(b)\|_p^p-\Big\|\frac12(\xi_p(a)+\xi_p(b))\Big\|_p^p\biggr]\\
%+(1-\alpha)\biggl[\frac12 \|\eta_r(a)\|_r^r+\frac12 \|\eta_r(b)\|_r^r
%-\Big\|\frac12(\eta_r(a)+\eta_r(b))\Big\|_r^r\biggr]\Biggr\}.\end{multlined}
\end{align*}
Since $p,r\ge 1$, both terms in brackets
{\color{red}[I have changed parentheses $(,)$ into bracket $[,]$ in the above expression.]}
in the last expression above are nonnegative. Assume now that $p>1$. 
Let $\bar a\in \Me^{++}$ be the minimizer as in Lemma \ref{lemma:variational_majorized},
then $f(\bar a)=Q_{\alpha,z}(\psi\|\varphi)$ and $\nabla f(\bar a)=0$, so that we get
\begin{align*}
f(b)-Q_{\alpha,z}(\psi\|\varphi)\ge 2\alpha\biggl[\frac12 \|\xi_p(a)\|_p^p+\frac12 \|\xi_p(b)\|_p^p
-\Big\|\frac12(\xi_p(a)+\xi_p(b))\Big\|_p^p\biggr].
\end{align*}
Since $L_p(\Me)$ is uniformly convex, note (see, e.g., \cite[Thmeorem 3.7.7]{zalinescu2002convex})
that the function $h\mapsto \|h\|_p^p$ is uniformly convex on each bounded subset of $L_p(\Me)$.
Hence for each $C>0$ and $\varepsilon>0$ there is some $\delta>0$ such that for every $h,k$
with $\|h\|_p^p,\|k\|_p^p\le C$ and $\|h-k\|_p\ge \varepsilon$, we have
\[
\frac12\|h\|_p^p+\frac12\|k\|_p^p-\Big\|\frac12(h+k)\Big\|_p^p\ge \delta
\]
(see \cite[p.~288, Exercise 3.3]{zalinescu2002convex}).  The proof in the case $r>1$ is similar. 
\end{proof}


\section{Data processing inequality and reversibility of channels}

Let  $\gamma: \Ne\to \Me$ be a normal positive unital map. Then the  predual of $\gamma$  defines a 
positive linear map $\gamma_*: L_1(\Me)\to L_1(\Ne)$ that preserves the {\color{red}$\Tr$-functional,}
acting as
\[
L_1(\Me)\ni h_\rho\mapsto h_{\rho\circ\gamma} \in L_1(\Ne).
\]
The support
of $\gamma$ will be denoted by $s(\gamma)$, recall that this is defined as the smallest projection
$e\in \Ne$ such that $\gamma(e)=1$ and in this case, $\gamma(a)=\gamma(eae)$ for any $a\in
\Ne$. For any $\rho\in \Me_*^+$ we clearly have
$s(\rho\circ\gamma)\le s(\gamma)$, with equality if $\rho$ is faithful. 
It follows that $\gamma_*$ maps $L_1(\Me)$ to $s(\gamma)L_1(\Ne)s(\gamma)\equiv
L_1(s(\gamma)\Ne s(\gamma))$.  For any $\rho\in \Me_+^*$, $\rho\ne 0$, the map
\[
\gamma_0: {\color{red}s(\rho\circ\gamma)\Ne s(\rho\circ\gamma)}\to s(\rho)\Me s(\rho),
\qquad a\mapsto s(\rho) \gamma(a)s(\rho)
\]
is a faithful normal positive unital {\color{red}(i.e., $\gamma_0(\rho\circ\gamma)=s(\rho)$)} map;
{\color{red}see \cite[Remark 6.7]{hiai2021lectures}.} Moreover, for any $\ffi\in \Me_*^+$ such that
$s(\ffi)\le s(\rho)$, we have for any $a\in \Ne$,
\[
\ffi(\gamma_0(s(\gamma)as(\gamma)))=\ffi(s(\rho)\gamma(a)s(\rho))={\color{red}\ffi(\gamma(a)).}
\]
Replacing $\gamma$ by $\gamma_0$ and $\rho$ by the restriction
{\color{red}$\rho|_{s(\rho\circ\gamma)\Me s(\rho\circ\gamma)}$,} we may  assume that both $\rho$
and $\rho\circ\gamma$ are faithful, {\color{red}as far as we are concerned with $\ffi\in\Me_*^+$ and
$\ffi\circ\gamma\in\Ne_*^+$ with $s(\ffi)\le s(\rho)$.}

The \emph{Petz dual} of $\gamma$ with respect to $\rho\in \Me_*^+$ is a map
$\gamma_\rho^*:\Me\to \Ne$, introduced in \cite{petz1988sufficiency} {\color{red}when $\rho$ and
$\rho\circ\gamma$ are faithful (hence so is $\gamma$ as well).} It was proved that
$\gamma_\rho^*$ is again normal, positive and unital, and in addition, it is $n$-positive whenever
$\gamma$ is. More generally, {\color{red}even though none of $\rho$, $\rho\circ\gamma$ and
$\gamma$ is faithful, letting $e:=s(\rho)$ and $e_0:=s(\rho\circ\gamma)$,} we may use the restriction
$\gamma_0$ as mentioned above to define the Petz dual $\gamma^*_\rho: e\Me e\to e_0\Ne e_0$. 
As explained in \cite{jencova2018renyi}, in this general setting, $\gamma^*_\rho$ is determined by
the equality
{\color{red}
\[
h_{\rho\circ\gamma}^{1/2}\gamma_\rho^*(a)h_{\rho\circ\gamma}^{1/2}
=\gamma_*\bigl(h_\rho^{1/2}ah_\rho^{1/2}\bigr),
\qquad a\in\Me,
\]
equivalently,
\begin{equation}\label{eq:petzdual}
(\gamma^*_\rho)_*\bigl(h_{\rho\circ\gamma}^{1/2}bh_{\rho\circ\gamma}^{1/2}\bigr)
=h_\rho^{1/2}\gamma(b)h_\rho^{1/2},\qquad b\in\Ne^+,
\end{equation}
where $\gamma_*$ and $(\gamma^*_\rho)_*$ are the predual maps of $\gamma$ and $\gamma^*_\rho$,
respectively. [This seems better (?)]}
We also have
\begin{equation}\label{eq:petzdual2}
\rho\circ\gamma\circ\gamma^*_\rho=\rho,\qquad (\gamma_\rho^*)_{\rho\circ\gamma}^*=\gamma.
\end{equation}
In the special case where $\gamma$ is the
inclusion map $\gamma: \Ne\hookrightarrow \Me$ for a subalgebra $\Ne\subseteq \Me$, the Petz dual
is the \emph{generalized conditional expectation} $\cE_{\Ne,\rho}:\Me\to \Ne$, as introduced in
\cite{accardi1982conditional}; see, e.g., \cite[Proposition 6.5]{hiai2021quantum}. Hence 
$\cE_{\Ne,\rho}$ is a normal completely positive unital with range in $\Ne$ and such that 
\[
\rho\circ \cE_{\Ne,\rho}=\rho.
\]


\subsection{Data processing inequality}

In this {\color{red}subsection} we prove the \emph{data processing inequality} (\emph{DPI}) for
$D_{\alpha,z}$ with respect to normal positive unital maps. For standard R\'enyi divergence, that is,
for $z=1$, {\color{red}the} DPI is known to hold for $\alpha\in (0,1)\cup (1,2]$ under stronger positivity
assumptions \cite{hiai2018quantum}. In the case of the sandwiched divergences $\tilde D_\alpha$ with
{\color{red}$\alpha\in[1/2,1)\cup(1,\infty)$,} DPI was proved in \cite{jencova2018renyi,jencova2021renyi};
see also \cite{berta2018renyi} for an alternative proof in the case when the maps are assumed
completely positive. In the finite-dimensional case, {\color{red}the DPI for $D_{\alpha,z}$ under completely
positive maps was proved in \cite{zhang2020fromwyd}, for $\alpha,z$ in the range specified as
in Theorem \ref{thm:dpi} below.}

{\color{red}The first part of the next lemma was essentially shown in
\cite[Proposition 3.12]{jencova2018renyi}, while we give the proof for the convenience of the reader.}

\begin{lemma}\label{lemma:pcontraction} Let $\gamma:\Ne\to \Me$ be a normal positive unital map.
Let $\rho\in \Me_*^+$, {\color{red}$\rho\ne0$,} $e:=s(\rho)$ and $e_0:=s(\rho\circ\gamma)$. For any
$p\ge 1$, the map $\gamma^*_{\rho,p}:L_p(e_0 \Ne e_0)\to L_p(e\Me e)$, determined by
\begin{align}\label{gamma-rho-p1}
\gamma^*_{\rho,p}\Bigl(h_{\rho\circ\gamma}^{1\over 2p}bh^{1\over2p}_{\rho\circ\gamma}\Bigr)
=h_\rho^{1\over 2p}\gamma(b) h_\rho^{1\over 2p},\qquad b\in \Ne,
\end{align}
is a contraction such that 
\begin{align}\label{gamma-rho-p2}
(\gamma^*_\rho)_*\Bigl(h_{\rho\circ\gamma}^{p-1\over 2p}k
h_{\rho\circ\gamma}^{p-1\over2p}\Bigr)
=h_\rho^{p-1\over 2p}\gamma^*_{\rho,p}(k)h_\rho^{p-1\over 2p},\qquad
k\in L_p(e_0 \Ne e_0).
\end{align}
Moreover, if $\rho_n\in \Me_*^+$ are such that $s(\rho)\le s(\rho_n)$ and
$\|\rho_n-\rho\|_1\to 0$, then for any $k\in L_p(e_0 \Ne e_0)$ we have
$\gamma^*_{\rho_n,p}(k)\to \gamma^*_{\rho,p}(k)$ in $L_p(\Me)$.

\end{lemma}

\begin{proof}
%For $b\in \Ne$, let $\sigma\in e_0 (\Ne_*)e_0$ be such that
%$h_\sigma=h_{\rho\circ\gamma}^{\frac12}b h_{\rho\circ\gamma}^{\frac12}$. Then
%\[
%(\gamma^*_\rho)_*(h_{\rho\circ\gamma}^{\frac12}b
%h_{\rho\circ\gamma}^{\frac12})=h_\rho^{\frac12}\gamma(b)h_\rho^{\frac12}=h_\rho^{p-1\over
%2p}\gamma^*_{\rho,p}(h_{\rho\circ\gamma}^{1\over 2p}bh_{\rho\circ\gamma}^{1\over
%2p})h_\rho^{p-1\over 2p}.
%\]
%Since $\gamma^*_\rho$ is a normal positive unital map, its predual $(\gamma^*_\rho)_*$
%defines a contraction {\color{red}between Kosaki's symmetric} $L_p$-spaces
%$L_p(\Ne,\rho\circ\gamma)\to L_p(\Me,\rho)$ (see \eqref{F-C.4} in Appendix \ref{Appen-Kosaki-Lp}),
{\color{red}
We use Kosaki's symmetric $L_p$-spaces $L_p(e_0\Ne e_0,\rho\circ\gamma)$ and
$L_p(e\Me e,\rho)$ (see \eqref{F-C.4} in Appendix \ref{Appen-Kosaki-Lp}). The map
$(\gamma_\rho^*)_*:L_1(e_0\Ne e_0)\to L_1(e\Me e)$ is contractive
with respect to $\|\cdot\|_1$. Its restriction to $h_{\rho\circ\gamma}^{1/2}\Ne h_{\rho\circ\gamma}^{1/2}$
($\subseteq L_1(e_0\Ne e_0)$) is given by \eqref{eq:petzdual}, which is also contractive with respect
to $\|\cdot\|_{\infty,\rho\circ\gamma}$ and $\|\cdot\|_{\infty,\rho}$. Hence it follows from the Riesz--Thorin
theorem that $(\gamma_\rho^*)_*$ is a contraction from $L_p(e_0\Ne e_0,\rho\circ\gamma)$ to
$L_p(e\Me e,\rho)$ for any $p\in(1,\infty)$. By \eqref{F-C.4} note that we have isometric isomorphisms
\begin{align*}
k\in L_p(e_0\Ne e_0)&\mapsto h_{\rho\circ\gamma}^{p-1\over2p}kh_{\rho\circ\gamma}^{p-1\over2p}
\in L_p(e_0\Ne e_0,\rho\circ\gamma), \\
h\in L_p(e\Me e)&\mapsto h_\rho^{p-1\over2p}hh_\rho^{p-1\over2p}\in L_p(e\Me e,\rho).
\end{align*}
Hence we can define a contraction $\gamma_{\rho,p}^*:L_p(e_0\Ne e_0)\to L_p(e\Me e)$ by
\eqref{gamma-rho-p2}. Then, for $k=h_{\rho\circ\gamma}^{1\over2p}bh_{\rho\circ\gamma}^{1\over2p}$
with $b\in\Ne$ we have
\[
h_\rho^{p-1\over2p}\gamma_{\rho,p}^*\Bigl(h_{\rho\circ\gamma}^{1\over2p}b
h_{\rho\circ\gamma}^{1\over2p}\Bigr)h_\rho^{p-1\over2p}
=(\gamma_\rho^*)_*\Bigl(h_{\rho\circ\gamma}^{1\over2}bh_{\rho\circ\gamma}^{1\over2}\Bigr)
=h_\rho^{1\over2}\gamma(b)h_\rho^{1\over2},
\]
so that \eqref{gamma-rho-p1} holds.
%\[
%\Big\|h_\rho^{1\over 2p}\gamma(b) h_\rho^{1\over 2p}\Big\|_p
%=\Big\|(\gamma^*_\rho)_*\Bigl(h_{\rho\circ\gamma}^{\frac12}b
%h_{\rho\circ\gamma}^{\frac12}\Bigr)\Big\|_{p,\rho}\le \Big\|h_{\rho\circ\gamma}^{\frac12}b
%h_{\rho\circ\gamma}^{\frac12}\Big\|_{p,\rho\circ\gamma}
%=\Big\|h_{\rho\circ\gamma}^{1\over 2p}bh_{\rho\circ\gamma}^{1\over 2p}\Big\|_p.
%\]
Since $h_{\rho\circ\gamma}^{1\over 2p}\Ne h^{1\over
2p}_{\rho\circ\gamma}$ is dense in $L_p(e_0 \Ne e_0)$, this proves the first part of the statement.
[I have written here slightly in more detail.]}

Let $\rho_n$ be a sequence as required and let $k\in L_p(e_0 \Ne e_0)$. By the
assumptions on the supports, $\gamma^*_{\rho_n,p}$ is well defined on $k$ for all $n$.
Further, we may assume that $k=h_{\rho\circ\gamma}^{1\over 2p}b
h_{\rho\circ\gamma}^{1\over 2p}$ for some $b\in \Ne$, since the set of such elements is
dense in $L_p(e_0 \Ne e_0)$ and all the maps are contractions. 
Put $k_n:=h_{\rho_n\circ\gamma}^{1\over 2p}b
h_{\rho_n\circ\gamma}^{1\over 2p}$, then we have
\[
\gamma^*_{\rho,p}(k)={\color{red}h_\rho^{1\over2p}\gamma(b)h_\rho^{1\over2p},}\qquad
\gamma^*_{\rho_n,p}(k_n)= {\color{red}h_{\rho_n}^{1\over2p}\gamma(b)h_{\rho_n}^{1\over2p},}
\]
and we have $k_n\to k$ in $L_p(\Ne)$ and  $\gamma^*_{\rho_n,p}(k_n)\to
\gamma^*_{\rho,p}(k)$ in $L_p(\Me)$. Indeed, this follows by the H\"older inequality and continuity
of the map $L_1(\Me)^+\ni h\mapsto {\color{red}h^{1\over2p}\in L_{2p}(\Me)^+}$; see
\cite[Lemma 3.4]{kosaki1984applicationsuc}. Therefore
\[
\|\gamma^*_{\rho_n,p}(k)-\gamma^*_{\rho,p}(k)\|_p\le
\|\gamma^*_{\rho_n,p}(k-k_n)\|_p+\|\gamma^*_{\rho_n,p}(k_n)-\gamma^*_{\rho,p}(k)\|_p\to 0,
\]
showing the latter statement.
\end{proof}


\begin{lemma}\label{lemma:dpi} Let $\gamma:\Ne\to \Me$ be a normal positive unital map, and
let $\rho\in \Me_*^+$, {\color{red}$\rho\ne0$, and} $b\in \Ne^+$. 
\begin{enumerate}
\item[(i)]  If $p\in [1/2,1)$, then 
\[
\Big\|h_{\rho\circ\gamma}^{\frac{1}{2p}}bh_{\rho\circ\gamma}^{\frac{1}{2p}}\Big\|_p\le
\Big\|h_{\rho}^{\frac{1}{2p}}\gamma(b)h_{\rho}^{\frac{1}{2p}}\Big\|_p.
\]

\item[(ii)]  If $p\in [1,\infty]$, the inequality reverses.
\end{enumerate}
\end{lemma}

\begin{proof} Let us denote $\beta:=\gamma_\rho^*$ and let $\omega\in {\color{red}\Ne_*^+}$ be such
that 
$h_\omega:=h_{\rho\circ\gamma}^{\frac12}bh_{\rho\circ\gamma}^{\frac12}\in L_1(\Ne)^+$. Then
$\beta$ is a normal positive unital map, and {\color{red}by \eqref{eq:petzdual}} we have
\[
\beta_*(h_\omega)=h_\rho^{\frac12}\gamma(b)h_\rho^{\frac12},\qquad
\beta_*(h_{\rho\circ\gamma})=h_\rho.
\]
Let $p\in [1/2,1)$, then  
\begin{align*}
\Big\|h_{\rho}^{\frac{1}{2p}}\gamma(b)h_{\rho}^{\frac{1}{2p}}\Big\|^p_p
&=\Big\|h_\rho^{\frac{1-p}{2p}}\beta_*(h_\omega)h_\rho^{\frac{1-p}{2p}}\Big\|_p^p
=Q_{p,p}(\beta_*(h_\omega)\|h_\rho)=Q_{p,p}(\beta_*(h_\omega)\|\beta_*(h_{\rho\circ\gamma}))\\
&\ge  Q_{p,p}(h_\omega\|h_{\rho\circ\gamma})=\Big\|h_{\rho\circ\gamma}^{\frac{1-p}{2p}}h_\omega
h_{\rho\circ\gamma}^{\frac{1-p}{2p}}\Big\|_p^p
=\Big\|h_{\rho\circ\gamma}^{\frac{1}{2p}}bh_{\rho\circ\gamma}^{\frac{1}{2p}}\Big\|^p_p.
\end{align*}
Here we have used the DPI for the sandwiched R\'enyi  divergence $D_{\alpha,\alpha}$ for
$\alpha\in [1/2,1)$; see \cite[Theorem 4.1]{jencova2021renyi}. This proves (i).
The case (ii) is immediate from Lemma \ref{lemma:pcontraction}. This was proved also 
in \cite{kato2023onrenyi} (see Eq.~(22) therein), by using the same argument.
\end{proof}

\begin{theorem}[DPI] \label{thm:dpi}
Let $\psi,\varphi\in \Me_*^+$, $\psi\ne 0$ and let $\gamma:\Ne\to \Me$ be a normal positive
unital map. Assume either of the following conditions:
\begin{enumerate}
\item[(i)] $0<\alpha<1$, $\max\{\alpha,1-\alpha\}\le z$,
\item[(ii)] $\alpha>1$, $\max\{\alpha/2,\alpha-1\}\le z\le \alpha$.
\end{enumerate}
Then we have
\[
D_{\alpha,z}(\psi\circ\gamma\|\varphi\circ\gamma)\le D_{\alpha,z}(\psi\|\varphi).
\]
\end{theorem}

\begin{proof}
Under the conditions (i), the DPI was proved in \cite[Theorem 1(viii)]{kato2023onrenyi}.
Since parts of this proof will be used below, we repeat it here.

Assume the conditions in (i) and put $p:=\frac{z}{\alpha}$, $r:=\frac{z}{1-\alpha}$, so that $p,r\ge 1$. 
For any $b\in \Ne^{++}$, we have by  the Choi inequality \cite{choi1974aschwarz} 
that  $\gamma(b)^{-1}\le \gamma(b^{-1})$, so that  by Lemmas \ref{lemma:order1} and
\ref{lemma:dpi}(ii), we have 
\begin{equation}\label{eq:ineq}
\Big\|h_\varphi^{\frac{1}{2r}}\gamma(b)^{-1}\varphi^{\frac{1}{2r}}\Big\|_r\le
\Big\|h_\varphi^{\frac{1}{2r}}\gamma(b^{-1})\varphi^{\frac{1}{2r}}\Big\|_r\le
\Big\|h_{\varphi\circ\gamma}^{\frac{1}{2r}}b^{-1}h_{\varphi\circ\gamma}^{\frac{1}{2r}}\Big\|_r^r .
\end{equation}
Using the variational expression in Theorem \ref{thm:variational}(i), we have
\begin{align*}
Q_{\alpha,z}(\psi\|\varphi)&\le \alpha\|h_\psi^{\frac{1}{2p}}\gamma(b)h_\psi^{\frac{1}{2p}}\|_p^p+
(1-\alpha)\|h_\varphi^{\frac{1}{2r}}\gamma(b)^{-1}h_\varphi^{\frac{1}{2r}}\|_r^r\\
&\le  \alpha\|h_{\psi\circ\gamma}^{\frac{1}{2p}}bh_{\psi\circ\gamma}^{\frac{1}{2p}}\|_p^p+
(1-\alpha)\|h_{\varphi\circ\gamma}^{\frac{1}{2r}}b^{-1}h_{\varphi\circ\gamma}^{\frac{1}{2r}}\|_r^r.
\end{align*}
 Since this holds for all
$b\in \Ne^{++}$, it follows that $Q_{\alpha,z}(\psi\|\varphi)\le
Q_{\alpha}(\psi\circ\gamma\|\varphi\circ\gamma)$, which proves the DPI in this case.

Assume next the condition (ii), and put $p:=\frac{z}{\alpha}$, $q:=\frac{z}{\alpha-1}$, so
that $p\in [1/2,1)$ and $q\ge 1$. Using Theorem
\ref{thm:variational}(ii), we get for any $b\in \Ne^+$,
\begin{align*}
Q_{\alpha,z}(\psi\|\varphi)&\ge
\alpha\|h_\psi^{\frac{1}{2p}}\gamma(b)h_\psi^{\frac{1}{2p}}\|_p^p-
(\alpha-1)\|h_\varphi^{\frac{1}{2q}}\gamma(b)h_\varphi^{\frac{1}{2q}}\|_q^q\\
&\ge \alpha\|h_{\psi\circ\gamma}^{\frac{1}{2p}}bh_{\psi\circ\gamma}^{\frac{1}{2p}}\|_p^p-
(\alpha-1)\|h_{\varphi\circ\gamma}^{\frac{1}{2r}}bh_{\varphi\circ\gamma}^{\frac{1}{2q}}\|_q^q,
\end{align*}
here we used both (i) and (ii) in Lemma \ref{lemma:dpi}. Again, since this holds for all
$b\in \Ne^+$, we get the desired inequality.
\end{proof}


\subsection{Martingale convergence}

An important consequence of DPI is the martingale convergence property that will be
proved in this {\color{red}subsection. Here assume that $\cM$ is a $\sigma$-finite von Neumann algebra.}

\begin{theorem}\label{thm:martingale}
Let $\psi,\varphi\in \Me_*^+$, $\psi\ne 0$, and let $\{\cM_i\}$ be an increasing net of von Neumann
subalgebras of $\cM$ containing the unit of $\cM$ such that $\cM=\bigl(\bigcup_i\cM_i\bigr)''$.
Assume that $\alpha$ and $z$ satisfy the DPI bounds (that is, conditions (i) or (ii)
{\color{red}["condition" is correct grammatically? If then, "bounds" $\to$ "bound"?]}
in Theorem \ref{thm:dpi}). Then we have
\begin{align}\label{eq:martingale}
D_{\alpha,z}(\psi\|\ffi)=\lim_iD_{\alpha,z}(\psi|_{\cM_i}\|\ffi|_{\cM_i})
\quad\mbox{increasingly}.
\end{align}
\end{theorem}

\begin{proof}
Let $\ffi_i:=\ffi|_{\cM_i}$ and $\psi_i:=\psi|_{\cM_i}$. From Theorem
\ref{thm:dpi}, it
follows that $D_{\alpha,z}(\psi\|\ffi)\ge D_{\alpha,z}(\psi_i\|\ffi_i)$ for all $i$ and
$i\mapsto D_{\alpha,z}(\psi_i\|\ffi_i)$ is increasing. Hence, to show \eqref{eq:martingale}, it suffices to
prove that
\begin{align}\label{eq:martingale1}
D_{\alpha,z}(\psi\|\ffi)\le\sup_iD_{\alpha,z}(\psi_i\|\ffi_i).
\end{align}
To do this, we may assume that $\ffi$ is faithful. Indeed, assume that
\eqref{eq:martingale1} has been shown when $\ffi$ is
faithful. For general $\ffi\in\cM_*^+$, from the assumption of $\cM$ being $\sigma$-finite, there exists
a $\ffi_0\in\cM_*^+$ with $s(\ffi_0)=\1-s(\ffi)$. Let $\ffi^{(n)}:=\ffi+n^{-1}\ffi_0$ and
$\ffi_i^{(n)}:=\ffi^{(n)}|_{\cM_i}$ for each $n\in\bN$. Thanks to the lower semi-continuity
\cite[Theorems 1(iv) and 2(iv)]{kato2023onrenyi} and the order relation
\cite[Theorems 1(iii) and 2(iii)]{kato2023onrenyi} we have
\begin{align*}
D_{\alpha,z}(\psi\|\ffi)&\le\liminf_{n\to\infty}D_{\alpha,z}(\psi\|\ffi^{(n)}) \\
&\le\liminf_{n\to\infty}\sup_iD_{\alpha,z}(\psi_i\|\ffi_i^{(n)}) \\
&\le\sup_iD_{\alpha,z}(\psi_i\|\ffi_i),
\end{align*}
proving \eqref{eq:martingale1} for general $\ffi$. Below we assume the faithfulness of $\ffi$ and write
$\cE_{\cM_i,\ffi}$ for the generalized conditional expectation from $\cM$ to $\cM_i$ with respect to $\ffi$. 
Then we note that we have by \cite[Theorem 3]{hiai1984strong},  
\begin{align}\label{eq:martingaleHT}
\psi_i\circ\cE_{\cM_i,\ffi}=\psi\circ\cE_{\cM_i,\ffi}\to\psi\quad\mbox{in the norm},
\end{align}
as well as
\begin{align}\label{eq:condexp}
\ffi_i\circ\cE_{\cM_i,\ffi}=\ffi\circ\cE_{\cM_i,\ffi}=\ffi.
\end{align}
Using lower semicontinuity and DPI, we obtain
\[
D_{\alpha,z}(\psi\|\ffi)\le \liminf_{i}
D_{\alpha,z}(\psi_i\circ\cE_{\cM_i,\varphi}\|\ffi)\le \liminf_i
D_{\alpha,z}(\psi_i\|\varphi)\le \sup_i D_{\alpha,z}(\psi_i\|\ffi).
\]
\end{proof}

{\color{red}[I like to add:]
The following proposition is another martingale type convergence, which is not included in
Theorem \ref{thm:martingale} since $e_i\Me e_i$'s do not contain the unit of $\Me$.

\begin{prop}
Let $\psi,\varphi\in \Me_*^+$, $\psi\ne 0$, and let $\{e_i\}$ be an increasing net of projections in $\Me$
such that $e_i\nearrow\1$. If $\alpha$ and $z$ satisfy the DPI bounds [bound?], then we have
\[
D_{\alpha,z}(\psi\|\ffi)=\lim_iD_{\alpha,z}(e_i\psi e_i\|e_i\ffi e_i),
\]
where $e_i\psi e_i$, $e_i\ffi e_i$ are the restrictions of $\psi$, $\ffi$ to the reduced von Neumann
subalgebra $e_i\Me e_i$.
\end{prop}

\begin{proof}
It suffices to show that
\[
Q_{\alpha,z}(\psi\|\ffi)=\lim_iQ_{\alpha,z}(e_i\psi e_i\|e_i\ffi e_i).
\]
Let $\Me_i:=e_i\Me e_i+\bC(\1-e_i)$; then $\{\Me_i\}$ is an increasing net of von Neumann subalgebras
of $\Me$ containing the unit of $\Me$ with $\Me=\bigl(\bigcup_i\Me_i\bigr)''$. By
Theorem \ref{thm:martingale} and \cite[Theorems 1(ii) and 2(ii)]{kato2023onrenyi} we have
\[
Q_{\alpha,z}(\psi\|\ffi)
=\lim_i\bigl[Q_{\alpha,z}(e_i\psi e_i\|e_i\ffi e_i)+\psi(\1-e_i)^\alpha\ffi(\1-e_i)^{1-\alpha}\bigr].
\]
Here, $\psi(\1-e_i)^\alpha\ffi(\1-e_i)^{1-\alpha}$ is defined with the conventions
\[
0^{1-\alpha}:=\begin{cases}0 & (0<\alpha<1), \\ \infty & (\alpha>1),\end{cases}\qquad
\lambda\cdot\infty:=\begin{cases}0 & (\lambda=0), \\ \infty & (\lambda>0).\end{cases}
\]
Then the statement holds if we show the following:
\begin{itemize}
\item[(1)] If $Q_{\alpha,z}(\psi\|\ffi)=\infty$, then $\lim_iQ_{\alpha,z}(e_i\psi e_i\|e_i\ffi e_i)=\infty$.
\item[(2)] If $Q_{\alpha,z}(\psi\|\ffi)<\infty$, then $\lim_i\psi(\1-e_i)^\alpha\ffi(\1-e_i)^{1-\alpha}=0$.
\end{itemize}
These two facts can be shown in the same way as in the proof of \cite[Theorem 4.5]{hiai2018quantum},
whose details are omitted here.
\end{proof}}


\section{Equality in DPI and reversibility of channels}

In what follows, a channel is a normal 2-positive unital map $\gamma: \Ne\to \Me$.

\begin{defi} Let $\gamma:\Ne\to \Me$ be a channel and let $\mathcal S \subset
\Me_*^+$. We say that $\gamma$ is reversible (or sufficient) with respect to $\mathcal S$
if there exists a channel $\beta:\Me\to \Ne$ such that
\[
\rho\circ\gamma\circ\beta=\rho\quad\mbox{for all}\ \rho\in \mathcal S.
\]
\end{defi}

The notion of sufficient channels was introduced by Petz
\cite{petz1986sufficient,petz1988sufficiency}, who also obtained a number of conditions
characterizing this situation. It particular, it was proved in \cite{petz1988sufficiency}
that sufficient channels can be characterized by equality in {\color{red}the} DPI for the relative entropy
$D(\psi\|\varphi)$: if $D(\psi\|\varphi)<\infty$, then a channel $\gamma$ is sufficient
with respect to $\{\psi,\varphi\}$ if and only if 
\[
D(\psi\circ\gamma\|\varphi\circ\gamma)=D(\psi\|\varphi). 
\]
This characterization has been proved for a number of other divergence measures, including the
standard R\'enyi divergences $D_{\alpha,1}$ with $0<\alpha<2$  and the sandwiched
R\'enyi divergences $D_{\alpha,\alpha}$ for $\alpha>1/2$
(\cite{hiai2021quantum,jencova2018renyi,jencova2021renyi}).
Another important result of \cite{petz1988sufficiency} shows that the Petz dual $\gamma_\varphi^*$
is a universal recovery map, in the sense given in the proposition below. 

\begin{prop}\label{prop:universal}
Let $\gamma:\Ne\to \Me$ be a channel and let $\varphi\in \Me_*^+$ be such that both $\ffi$ and
$\ffi\circ\gamma$ are faithful. Then the following hold:
\begin{itemize}
\item[(i)] For any $\psi\in \Me_*^+$, $\gamma$ is sufficient with respect to $\{\psi,\varphi\}$ if and only
if $\psi\circ\gamma\circ\gamma_\varphi^*=\psi$.

\item[(ii)]
There is a faithful normal conditional expectation $\mathcal E$ from $\Me$ onto a von Neumann
subalgebra of $\Me$ such that $\varphi\circ \mathcal E=\varphi$, and $\gamma$ is sufficient with
respect to $\{\psi,\varphi\}$ if and only if also $\psi\circ\mathcal E=\psi$.
\end{itemize}
\end{prop}

Note {\color{red}(see \cite[Theorem 2]{petz1988sufficiency} and the proof of
\cite[Theorem 3]{petz1988sufficiency})}
that the range of the conditional expectation $\mathcal E$ in statement (ii) above is the set of
fixed points of the channel $\gamma\circ\gamma_\varphi^*$.
{\color{red}[I write references in detail, since statement (ii) is implicit in \cite{petz1988sufficiency}.]}


Our aim in this section is to prove that equality in {\color{red}the} DPI for $D_{\alpha,z}$ with 
values of the parameters (strictly) contained in the DPI bounds of Theorem \ref{thm:dpi}
characterizes sufficiency of channels. Throughout this section, we use the
notations $\psi_0:=\psi\circ\gamma$ and $\ffi_0:=\ffi\circ\gamma$. We also denote
\[
p:=\frac{z}{\alpha},\qquad r:=\frac{z}{1-\alpha},\qquad q:=-r=\frac{z}{\alpha-1}. 
\]


\subsection{The case $\alpha\in (0,1)$}

Here we study equality in {\color{red}the} DPI for $D_{\alpha,z}$ with $\alpha\in (0,1)$,  for a pair of positive
normal functionals $\psi,\ffi\in \Me_*^+$ and a normal positive unital map $\gamma:\Ne\to \Me$. We first
prove some equality conditions  in the case $\psi\sim \ffi$. 

\begin{prop}\label{prop:DPI_equality}
Let $0<\alpha<1$, $\max\{\alpha,1-\alpha\}\le z$.  Assume that $\psi\sim \ffi$ and  let
$\gamma:\Ne\to \Me$ be a normal positive unital map. Let 
$\bar a\in \Me^{++}$  
be the unique minimizer as in Lemma \ref{lemma:variational_majorized} for
$Q_{\alpha,z}(\psi\|\ffi)$ and let $\bar a_0\in \Ne^{++}$ be the minimizer for
$Q_{\alpha,z}(\psi_0\|\ffi_0)$. The following conditions are equivalent:
\begin{itemize}
\item[(i)] $D_{\alpha,z}(\psi_0\|\ffi_0)=D_{\alpha,z}(\psi\|\ffi)$, i.e.,
$Q_{\alpha,z}(\psi_0\|\ffi_0)=Q_{\alpha,z}(\psi\|\ffi)$.
\item[(ii)] $\gamma(\bar a_0)=\bar a$ and
$\Big\|h_\psi^{1\over2p}\gamma(\bar a_0)h_\psi^{1\over2p}\Big\|_{p}
=\Big\|h_{\psi_0}^{1\over2p}\bar a_0h_{\psi_0}^{1\over2p}\Big\|_{p}$.
\item[(iii)] $\Big\|h_\psi^{1\over2p}\bar ah_\psi^{1\over2p}\Big\|_{p}
=\Big\|h_{\psi_0}^{1\over2p}\bar a_0h_{\psi_0}^{1\over2p}\Big\|_{p}$.
\item[(iv)] $\gamma(\bar a_0^{-1})=\bar a^{-1}$ and
$\Big\|h_\ffi^{1\over2r}\gamma(\bar a_0^{-1})h_\ffi^{1\over2r}\Big\|_{r}
=\Big\|h_{\ffi_0}^{1\over2r}\bar a_0^{-1}h_{\ffi_0}^{1\over2r}\Big\|_{r}$.
\item[(v)] $\Big\|h_\ffi^{1\over2r}\bar a^{-1}h_\ffi^{1\over2r}\Big\|_{r}
=\Big\|h_{\ffi_0}^{1\over2r}\bar a_0^{-1}h_{\ffi_0}^{1\over2r}\Big\|_{r}$.
\end{itemize}
\end{prop}


\begin{proof}
{\color{red}Since $\psi\sim\ffi$ by assumption and hence $\psi_0\sim \ffi_0$, we have
$s(\psi)=s(\ffi)$ and $s(\psi_0)=s(\ffi_0)$.} Using restrictions {\color{red}explained in the beginning
of Sec.~2}, we may assume that all $\psi,\ffi,\psi_0,\ffi_0$ are faithful.

(i)$\implies$(ii) \& (iv).\enspace
By Lemma \ref{lemma:dpi}(ii)
\begin{align}
\Big\|h_\psi^{1\over2p}\gamma(\bar a_0)h_\psi^{1\over2p}\Big\|_{p}
\le\Big\|h_{\psi_0}^{1\over2p}\bar a_0h_{\psi_0}^{1\over2p}\Big\|_{p},
\label{eq:ineq2}
\end{align}
and by \eqref{eq:ineq} we have
\begin{align}
\Big\|h_\ffi^{1\over2r}\gamma(\bar a_0)^{-1}h_\ffi^{1\over2r}\Big\|_{r}
\le\Big\|h_\ffi^{1\over2r}\gamma(\bar a_0^{-1})h_\ffi^{1\over2r}\Big\|_{r}
\le\Big\|h_{\ffi_0}^{1\over2r}\bar
a_0^{-1}h_{\ffi_0}^{1\over2r}\Big\|_{r}. \label{eq:ineq3}
\end{align}
From \eqref{eq:ineq2} and \eqref{eq:ineq3} it follows that
\begin{align*}
Q_{\alpha,z}(\psi\|\ffi)&\le \alpha\Big\|h_\psi^{1\over2p}\gamma(\bar
a_0)h_\psi^{1\over2p}\Big\|^p_{p}
+(1-\alpha)\Big\|h_\ffi^{1\over2r}\gamma(\bar a_0)^{-1}h_\ffi^{1\over2r}\Big\|^r_{r} \\
&\le\alpha\Big\|h_{\psi_0}^{1\over2p}\bar a_0h_{\psi_0}^{1\over2p}\Big\|^p_{p}
+(1-\alpha)\Big\|h_{\ffi_0}^{1\over2r}\bar a_0^{-1}h_{\ffi_0}^{1\over2r}\Big\|^r_{r}
=Q_{\alpha,z}(\psi_0\|\ffi_0)=Q_{\alpha,z}(\psi\|\ffi).
\end{align*}
By uniqueness in Lemma \ref{lemma:variational_majorized} we find that $\gamma(\bar
a_0)=\bar a$ and all the inequalities in \eqref{eq:ineq2} and \eqref{eq:ineq3} must
become equalities. Since $\gamma(\bar a_0^{-1})\ge\gamma(\bar a_0)^{-1}$, we see by
Lemma \ref{lemma:order1} that the equality in
\eqref{eq:ineq3} yields $\gamma(\bar a_0^{-1})=\gamma(\bar a_0)^{-1}=\bar a^{-1}$. Therefore,
(ii) and (iv) hold.

The implications (ii)$\implies$(iii) and (iv)$\implies$(v) are obvious.

(iii)$\implies$(i).\enspace
By (iii) with the equality \eqref{eq:minimizer1} in Lemma \ref{lemma:variational_majorized} we have
\begin{align*}
Q_{\alpha,z}(\psi\|\ffi)
&=\Tr\Bigl(h_\psi^{1\over2p}h_\ffi^{1\over r}h_\psi^{1\over2p}\Bigr)^z
=\Tr\Bigl(h_\psi^{1\over2p}\bar ah_\psi^{1\over2p}\Bigr)^{p} \\
&=\Tr\Bigl(h_{\psi_0}^{1\over2p}\bar a_0h_{\psi_0}^{1\over2p}\Bigr)^{p}
=\Tr\Bigl(h_{\psi_0}^{\1\over2p}h_{\ffi_0}^{1\over r}h_{\psi_0}^{1\over2p}\Bigr)^z
=Q_{\alpha,z}(\psi_0\|\ffi_0).
\end{align*}

(v)$\implies$(i).\enspace
Similarly, by (v) with the equality \eqref{eq:minimizer2} in Lemma
\ref{lemma:variational_majorized} we have
\begin{align*}
Q_{\alpha,z}(\psi\|\ffi)
&=\Tr\Bigl(h_\ffi^{1\over2r}h_\psi^{1\over p}h_\ffi^{1\over2r}\Bigr)^z
=\Tr\Bigl(h_\ffi^{1\over2r}\bar a^{-1}h_\ffi^{1\over2r}\Bigr)^{r} \\
&=\Tr\Bigl(h_{\ffi_0}^{1\over2r}\bar a_0^{-1}h_{\ffi_0}^{1\over2r}\Bigr)^{r}
=\Tr\Bigl(h_{\ffi_0}^{1\over2r}h_{\psi_0}^{1\over p}h_{\ffi_0}^{1\over2r}\Bigr)^z
=Q_{\alpha,z}(\psi_0\|\ffi_0).
\end{align*}
\end{proof}


\begin{remark}\label{rem:conditions} Note that the above conditions extend the results
obtained in \cite{leditzky2017data} and \cite{zhang2020equality} in the finite-dimensional case.
Indeed, the {\color{red}first} condition in (ii) with $\alpha=z$ is equivalent to the condition in
\cite[Theorem 1]{leditzky2017data}, {\color{red}as seen from \eqref{eq:minimizer2}.}
{\color{red}[Better to mention that \cite[Theorem 1]{leditzky2017data} was shown when
$s(\psi)\le s(\ffi)$, while $\psi\sim\ffi$ is assumed in the above proposition.]}
Note here that in this case the second condition in (ii) is automatic. {\color{red}[Is this directly
checked? Or, do you mean this is a consequence of the above proposition?]}
Moreover, (ii) extends the necessary condition in \cite[Theorem 1.2(2)]{zhang2020equality}
to a necessary and sufficient one. While in both {\color{red}of} these works $\gamma$ was required
to be completely positive, only positivity is enough for our result. 
\end{remark}


\begin{theorem}\label{thm:suffle1} Let $0<\alpha<1$ and $\max\{\alpha,1-\alpha\}\le
z$. Let $\psi,\varphi\in \Me_*^+$, $\psi\ne0$, and assume either that $\alpha<z$ and $s(\ffi)\le
s(\psi)$, or that $1-\alpha<z$ and $s(\psi)\le s(\ffi)$. 
Then a channel $\gamma:\Ne \to \Me$ is reversible with respect to
$\{\psi,\varphi\}$ if and only if
\[
D_{\alpha,z}(\psi\|\varphi)=D_{\alpha,z}(\psi\circ\gamma\|\varphi\circ\gamma).
\]

\end{theorem}

\begin{proof} This proof is a modification of the proof of \cite[Theorem
5.1]{jencova2021renyi}.  
We will assume that $s(\ffi)\le s(\psi)$ and $\alpha<z$, that is, $p>1$. In the other case we may
exchange the roles of $p$, $r$ and of $\psi$, $\ffi$ by the equality
$Q_{\alpha,z}(\psi\|\varphi)=Q_{1-\alpha,z}(\varphi\|\psi)$. As before, we may assume that
both $\psi$ and $\psi_0$ are faithful.

The strategy of the proof is to use known results in \cite{jencova2018renyi} for the sandwiched
R\'enyi divergence $D_{p,p}$ with $p>1$. For this, {\color{red}let $\mu,\omega\in\Me_*^+$ be such that
\[
h_{\mu}=\Big|h_{\ffi}^{1\over 2r}h_{\psi}^{1\over 2p}\Big|^{2z},\qquad
h_\omega=h_\psi^{p-1\over 2p}h_\mu^{\frac1p}h_\psi^{p-1\over 2p},
\]}
and notice that
\[
Q_{z,\alpha}(\psi\|\ffi)=Q_{p,p}(\omega\|\psi).
\]
Let $\mu_0,\omega_0\in \Ne_*^+$ be similar functionals obtained from $\psi_0,\ffi_0$. Then
we have the equality
\begin{equation}\label{eq:sandwiched}
Q_{p,p}(\omega_0\|\psi_0)=Q_{\alpha,z}(\psi_0\|\ffi_0)=Q_{\alpha,z}(\psi\|\ffi)=Q_{p,p}(\omega\|\psi).
\end{equation}
Our first goal is to show that $\omega_0=\omega\circ\gamma$, which implies by
\cite[Theorem 4.6]{jencova2018renyi} that $\gamma$ is sufficient with respect to $\{\omega,\psi\}$. 
We then apply Proposition \ref{prop:universal} and the properties of the extensions of the
conditional expectation $\cE$ to the Haagerup $L_p$-spaces proved in
\cite{junge2003noncommutative}. 

Let us remark here that if $\psi\sim \ffi$, {\color{red}it follows from \eqref{eq:minimizer1} that}
$h_\omega=h_\psi^{\frac12}\bar ah_\psi^{\frac12}$ and
$h_{\omega_0}=h_{\psi_0}^{\frac12}\bar a_0h_{\psi_0}^{\frac12}$. Hence from {\color{red}\eqref{eq:petzdual}
and condition (ii)} in Proposition \ref{prop:DPI_equality}, we immediately have
\[
(\gamma^*_{\psi})_*(h_{\omega_0})=h_\psi^{\frac12}\gamma(\bar
a_0)h_\psi^{\frac12}=h_\omega,\quad
{\color{red}\mbox{i.e.},\quad\omega_0\circ\gamma_\psi^*=\omega,}
\]
{\color{red}as well as $\psi_0\circ\gamma_\psi^*=\psi$ by \eqref{eq:petzdual2}.}
These and \eqref{eq:sandwiched} {\color{red}show} that $\gamma^*_\psi$ is sufficient with respect to
$\{\omega_0,\psi_0\}$. By Proposition \ref{prop:universal} and the fact that the
Petz dual  $(\gamma_\psi^*)_{\psi_0}^*$ is $\gamma$ itself, this implies the desired
equality
\[
\omega\circ\gamma= \omega_0\circ \gamma_\psi^*\circ\gamma=\omega_0.
\]

In the case $\psi\not\sim \ffi$ some more work is required. Let $\psi_n:=\psi+\frac1n \ffi$ and
$\ffi_n:=\ffi+\frac1n \psi$. Then all $\psi_n$, $\ffi_n$ are faithful,  $\psi_n\to \psi$, $\ffi_n\to \ffi$ in
$\Me_*^+$, {\color{red}and} moreover,  $\psi_n\sim \ffi_n$ for all $n$. Then
$\psi_n\circ\gamma\sim \ffi_n\circ\gamma$, $\psi_n\circ\gamma\to \psi_0$,
$\ffi_n \circ \gamma\to \ffi_0$ and by joint continuity of $Q_{\alpha,z}$ {\color{red}in the norm}
(\cite[Theorem 1(iv)]{kato2023onrenyi}), we have
\[
\lim_n
Q_{\alpha,z}(\psi_n\circ\gamma\|\ffi_n\circ\gamma)=Q_{\alpha,z}(\psi_0\|\ffi_0)
=Q_{\alpha,z}(\psi\|\ffi)=\lim_nQ_{\alpha,z}(\psi_n\|\ffi_n).
\]
Let $\bar b_{n}\in \Ne^{++}$ be the minimizer for the variational expression for
$Q_{\alpha,z}(\psi_n\circ\gamma\|\ffi_n\circ\gamma)$ {\color{red}given in \eqref{F-1.1}.}
Let also $\bar a_n$ be the minimizer for $Q_{\alpha,z}(\psi_n\|\ffi_n)$, and let
$f_n:\Me^{++}\to \mathbb R^+$ be the function minimized in the expression for
$Q_{\alpha,z}(\psi_n\|\ffi_n)$ {\color{red}(see \eqref{func-variational}).} We then have 
\begin{align*}
Q_{\alpha,z}(\psi_n\circ\gamma\|\ffi_n\circ\gamma)-f_n(\gamma(\bar
b_{n}))&=\alpha\left(\Big\|h_{\psi_n\circ\gamma}^{\frac1{2p}}\bar b_{n}
h_{\psi_n\circ\gamma}^{\frac1{2p}}\Big\|_p^p-\Big\|h_{\psi_n}^{1\over 2p}\gamma(\bar
b_n)h_{\psi_n}^{1\over 2p}\Big\|_p^p\right)\\
&\quad + (1-\alpha)\left(\Big\|h_{\ffi_n\circ\gamma}^{\frac1{2r}}\bar b_{n}^{-1}
h_{\ffi_n\circ\gamma}^{\frac1{2r}}\Big\|_r^r-\Big\|h_{\ffi_n}^{1\over 2r}\gamma(\bar
b_n)^{-1}h_{\ffi_n}^{1\over 2r}\Big\|_r^r\right)\ge 0,
\end{align*}
where the inequality follows from Lemma \ref{lemma:dpi}(ii) and \eqref{eq:ineq}. We
obtain
\begin{equation}\label{eq:qfn}
Q_{\alpha,z}(\psi_n\circ\gamma\|\ffi_n\circ\gamma)-Q_{\alpha,z}(\psi_n\|\ffi_n)\ge f_n(\gamma(\bar
b_{n}))-Q_{\alpha,z}(\psi_n\|\ffi_n)\ge 0.
\end{equation}

Now let $\mu_{n,0}\in \Ne_*^+$ and $\mu_n\in \Me_*^+$ be such that (using  \eqref{eq:minimizer1}
in Lemma \ref{lemma:variational_majorized})
\begin{align*}
h_{\mu_{n,0}}^{\frac 1p}=
\Big|h_{\ffi_n\circ\gamma}^{1\over 2r}h_{\psi_n\circ\gamma}^{1\over 2p}\Big|^{2\alpha}=
h_{\psi_n\circ\gamma}^{\frac 1{2p}}\bar b_{n}
h_{\psi_n\circ\gamma}^{\frac1{2p}},\qquad
h_{\mu_{n}}^{\frac1p}=\Big|h_{\ffi_n}^{1\over 2r}h_{\psi_n}^{1\over 2p}\Big|^{2\alpha}=
h_{\psi_n}^{\frac 1{2p}}\bar a_{n}h_{\psi_n}^{\frac1{2p}}. 
\end{align*}
Then $h_{\mu_{n,0}}^{\frac1p}\to h_{\mu_0}^{\frac1p}$ in $L_p(\Ne)$, this follows from the H\"older
inequality and the fact \cite{kosaki1984applicationsuc} that the map
$L_{2z}(\Ne)\ni h\mapsto |h|^{2\alpha}\in L_p(\Ne)$ is continuous in the norm. Similarly,
$h_{\mu_n}^{\frac1p}\to h_\mu^{\frac1p}$ in $L_p(\Me)$. 
Next, note that since
$Q_{\alpha,z}(\psi_n\circ\gamma\|\ffi_n\circ\gamma)$ and $Q_{\alpha,z}(\psi_n\|\ffi_n)$
have the same limit, we see from \eqref{eq:qfn} that
$f_n(\gamma(\bar b_{n}))-Q_{\alpha,z}(\psi_n\|\ffi_n)\to0$.
{\color{red}Moreover, by Lemma \ref{lemma:dpi}(ii) note that
\[
\sup_n\Big\|h_{\psi_n}^{\frac 1{2p}}\gamma(\bar b_{n})h_{\psi_n}^{1\over 2p}\Big\|_p^p
\le\sup_n\Big\|h_{\psi_n\circ\gamma}^{\frac 1{2p}}\bar b_{n}
h_{\psi_n\circ\gamma}^{\frac 1{2p}}\Big\|_p^p
\le{1\over\alpha}\sup_nD_{\alpha,z}(\psi_n\circ\gamma\|\ffi_n\circ\gamma)<\infty.
\]}
Therefore, {\color{red}since
$\big\|h_{\psi_n}^{\frac 1{2p}}\gamma(\bar b_{n})h_{\psi_n}^{1\over 2p}-h_{\mu_n}^{\frac1p}\big\|_p$
means $\|\xi_p(\gamma(\bar b_n))-\xi_p(\bar a_n)\|_p$ defined for $\psi_n$ (in place of $\psi$),
it follows} from Lemma \ref{lemma:variational_majorized2} that 
$h_{\psi_n}^{\frac 1{2p}}\gamma(\bar b_{n})h_{\psi_n}^{1\over 2p}-h_{\mu_n}^{\frac1p}\to
0$ in $L_p(\Me)$.  On the other hand, let $\gamma^*_{\psi_n,p}, \gamma^*_{\psi,p}$ be the
contractions defined in Lemma \ref{lemma:pcontraction}. We
then have 
\[
h_{\psi_n}^{\frac 1{2p}}\gamma(\bar b_{n})h_{\psi_n}^{1\over
2p}=\gamma_{\psi_{n},p}^*\bigl(h_{\mu_{n,0}}^{\frac1p}\bigr)
\]
and since $\gamma^*_{\psi_{n},p}(k)\to \gamma^*_{\psi,p}(k)$ in $L_p(\Me)$ for any
$k\in L_p(s(\psi\circ\gamma)\Ne s(\psi\circ\gamma))$ by Lemma \ref{lemma:pcontraction}, we have  
\[
\Big\|\gamma^*_{\psi,p}\bigl(h_{\mu_0}^{\frac1p}\bigr)-
\gamma_{\psi_{n},p}^*\bigl(h_{\mu_{n,0}}^{\frac1p}\bigr)\Big\|_p\le
\Big\|(\gamma^*_{\psi,p}-\gamma^*_{\psi_{n},p})\bigl(h_{\mu_0}^{\frac1p}\bigr)\Big\|_p+
\Big\|\gamma^*_{\psi_{n},p}\bigl(h_{\mu_0}^{\frac1p}-h_{\mu_{n,0}}^{\frac1p}\bigr)\Big\|_p\to 0.
\]
Putting all together, we obtain that 
\[
h_\mu^{\frac1p}=\lim_n h_{\mu_n}^{\frac1p}=\lim_n
\gamma^*_{\psi_{n},p}(h_{\mu_{n,0}}^{\frac1p})=\gamma^*_{\psi,p}(h_{\mu_0}^{\frac1p}).
\]
It follows from \eqref{gamma-rho-p2} that 
\[
(\gamma^*_{\psi})_*(h_{\omega_0})=
h_{\psi}^{\frac{p-1}{2p}}\gamma^*_{\psi,p}(h_{\mu_0}^{\frac1p})h_{\psi}^{\frac{p-1}{2p}}=
h_{\psi}^{\frac{p-1}{2p}}h_\mu^{\frac1p}h_{\psi}^{\frac{p-1}{2p}}=h_\omega.
\]
As we have seen {\color{red}in the case $\psi\sim \ffi$ above,} this and \eqref{eq:sandwiched} imply that
\[
\omega\circ\gamma= \omega_0\circ \gamma_\psi^*\circ\gamma=\omega_0.
\]
{\color{red}Therefore, we have shown that $\gamma$ is sufficient with respect to $\{\omega,\psi\}$.}

Next, let $\mathcal E$ be the faithful  normal conditional expectation onto the
set of fixed points of $\gamma\circ\gamma^*_\psi$ (see a note after Proposition \ref{prop:universal}). Then
by Proposition \ref{prop:universal}(ii), $\mathcal E$ preserves both $\psi$ and $\omega$, which by
\cite{junge2003noncommutative} implies that 
\[
h_\psi^{p-1\over 2p}h_\mu^{\frac1p}h_\psi^{p-1\over 2p}=h_\omega=\mathcal
E_*(h_\omega)=h_\psi^{p-1\over 2p}\cE_p\bigl(h_\mu^{\frac1p}\bigr)h_\psi^{p-1\over 2p},
\]
so that $\big|h_\ffi^{1\over 2r}h_\psi^{1\over 2p}\big|^{2\alpha}= h_\mu^{\frac1p}\in
L_p(\cE(\Me))$ and consequently $\big|h_\varphi^{\frac1{2r}}h_\psi^{\frac1{2p}}\big|=h_\mu^{1\over
2z}\in L_{2z}(\mathcal E(\Me))^+$. By the assumption $2z>1$ note that we may use the
{\color{red}bimodule property} of the extension of $\mathcal E$; {\color{red}see
\cite[Proposition 2.3(ii)]{junge2003noncommutative}.}
Let  $h_\varphi^{\frac1{2r}}h_\psi^{\frac1{2p}}=u\big|h_\varphi^{\frac1{2r}}h_\psi^{\frac1{2p}}\big|$ be the
polar decomposition in $L_{2z}(\Me)$; then we have 
\[
u^*h_\varphi^{\frac1{2r}}h_\psi^{\frac1{2p}}=\mathcal
E_{2z}\bigl(u^*h_\varphi^{\frac1{2r}}h_\psi^{\frac1{2p}}\bigr)=\mathcal
E_{2r}\bigl(u^*h_\varphi^{\frac1{2r}}\bigr)h_\psi^{\frac1{2p}},
\]
which implies that $u^*h_\varphi^{\frac1{2r}}\in L_{2r}(\cE(\Me))$. Since $\psi$ is
faithful, we have
{\color{red}
\[
\ker\bigl(h_\ffi^{1\over2r}h_\psi^{1\over2p}\bigr)=\ker\bigl(h_\psi^{1\over2p}h_\ffi^{1\over2r}\bigr)
=\ker h_\ffi^{1\over2r}=\ker h_\ffi,
\]
which implies that $uu^*=s(\ffi)$. [This seems more transparent.]}
Hence by uniqueness of the polar decomposition in $L_{2r}(\Me)$ and $L_{2r}(\cE(\Me))$,
we obtain that $h_\ffi^{1\over 2r}\in L_{2r}(\cE(\Me))^+$ and $u\in \cE(\Me)$. 
Therefore, we must have $h_\ffi\in L_1(\cE(\Me))$, so that $\ffi\circ\mathcal E=\ffi$ and $\gamma$ is
sufficient with respect to $\{\psi,\varphi\}$ by Proposition \ref{prop:universal}(ii) again.
\end{proof}


\subsection{The case $\alpha>1$}

We now turn to the case $\alpha>1$. %We will put $p:=\frac z\alpha$ and $q:=\frac z{\alpha-1}$,
Then within the DPI bounds{\color{red}[bound (?)]}, we have $p:=\frac z\alpha\in [1/2,1]$ and
$q:=\frac z{\alpha-1}\ge 1$, and we note that we always have $p<q$. Here we need to assume that
$D_{\alpha,z}(\psi\|\ffi)<\infty$, so that by Lemma \ref{lemma:renyi_2z} there is some
(unique) $y\in L_{2z}(\Me)$ such that
\[
h_{\psi}^{1\over 2p}=y h_\ffi^{1\over 2q}.
\]
By the proof of Theorem \ref{thm:variational}, we have the following variational expression
\begin{align}\label{eq:variationalq}
Q_{\alpha,z}(\psi\|\varphi) =\sup_{w\in
L_q(\Me)^+}\bigl\{\alpha\Tr\bigl((ywy^*)^p\bigr)-(\alpha-1)\Tr\bigl(w^q\bigr)\bigr\}.
\end{align}
{\color{red}Indeed, we note that $x$ in the proof of Theorem \ref{thm:variational} is $y^*y$ and
$\Tr\bigl((x^{1\over2}wx^{1\over2})^p\bigr)$ in expression \eqref{F-1.3} is rewritten as
$\Tr\bigl((|y|w|y|)^p\bigr)=\Tr\bigl((ywy^*)^p\bigr)$.}
The supremum is attained at a unique point $\bar
w=(y^*y)^{\alpha-1}\in L_q(\Me)^+$, uniqueness follows from strict concavity of the
function $w\mapsto \alpha\Tr\bigl((ywy^*)^p\bigr)-(\alpha-1)\Tr\bigl( w^q\bigr)$.


By DPI, we have $D_{\alpha,z}(\psi_0\|\varphi_0)\le D_{\alpha,z}(\psi\|\varphi)<\infty$,
so that there is some (unique) $y_0\in L_{2z}(\Ne)$ such that 
\[
h_{\psi_0}^{\frac1{2p}}=y_0h_{\varphi_0}^{\frac1{2q}}.
\]
Since $D_{\alpha,z}(\psi\|\ffi)<\infty$ implies that $s(\psi)\le s(\ffi)$, we may assume that both $\ffi$ and
$\ffi_0$ are faithful.


\begin{lemma}\label{lemma:le} Let $\gamma:\Ne\to\Me$ be a normal positive unital map. 
Let $\gamma^*_{\ffi,q}:L_q(\Ne)\to L_q(\Me)$ be the contraction as in lemma \ref{lemma:pcontraction}. 
Let $\bar w:=(y^*y)^{\alpha-1}\in L_q(\Me)$ and $\bar w_0:=(y_0^*y_0)^{\alpha-1}\in
L_q(\Ne)$. Then equality in {\color{red}the DPI for $D_{\alpha,z}(\psi\|\ffi)$} holds if and only if
\begin{equation}\label{eq:dpiw}
\bar w=\gamma^*_{\ffi,q}(\bar w_0)\quad \text{and}\quad  
\Tr\bigl(\bar w_0^q\bigr)=\Tr\bigl(\gamma^*_{\varphi,q}(\bar w_0)^q\bigr).
\end{equation}
\end{lemma}

\begin{proof} We first show that  for any $w_0\in L_q(\Ne)^+$,
\begin{equation}\label{eq:lemma}
\Tr\bigl((y\gamma^*_{\ffi,q}(w_0)y^*)^p\bigr)\ge
\Tr\bigl((y_0w_0y_0^*)^p\bigr).
\end{equation}
Let us first assume that
$w_0=h_{\varphi_0}^{\frac1{2q}}bh_{\varphi_0}^{\frac1{2q}}$ for some $b\in \Ne_+$. Then 
$\gamma^*_{\varphi,q}(w_0)=h_{\varphi}^{\frac1{2q}}\gamma(b)h_{\varphi}^{\frac1{2q}}$.
Therefore
\begin{align*}
\Tr\bigl((y\gamma^*_{\varphi,q}(w_0)y^*)^p\bigr)
&=\Tr\Bigl(\bigl(yh_{\varphi}^{\frac1{2q}}\gamma(b)h_{\varphi}^{\frac1{2q}}y^*\bigr)^p\Bigr)=
\Tr\Bigl(\bigl(h_\psi^{\frac1{2p}}\gamma(b)h_\psi^{\frac1{2p}}\bigr)^p\Bigr)\ge
\Tr\Bigl(\bigl(h_{\psi_0}^{\frac1{2p}}bh_{\psi_0}^{\frac1{2p}}\bigr)^p\Bigr)\\
&=\Tr\Bigl(\bigl(y_0h_{\varphi_0}^{\frac1{2q}}bh_{\varphi_0}^{\frac1{2q}}y_0^*\bigr)^p\Bigr)
=\Tr\bigl((y_0w_0y_0^*)^p\bigr),
\end{align*}
where the inequality is from Lemma \ref{lemma:dpi}(i). The proof of inequality
\eqref{eq:lemma} is  finished  by Lemma \ref{lemma:cone}.
By using this and the fact that $\gamma^*_{\ffi,q}$ is a contraction, if follows from the variational
expression in \eqref{eq:variationalq} that
\begin{align*}
Q_{\alpha,z}(\psi\|\varphi)
&\ge \alpha\Tr\bigl((y\gamma^*_{\varphi,q}(\bar w_0)y^*)^p\bigr)-
(\alpha-1)\Tr\bigl(\gamma^*_{\varphi,q}(\bar w_0)^q\bigr)\\
&\ge \alpha\Tr\bigr((y_0\bar w_0 y_0^*)^p\bigr)-(\alpha-1)\Tr\bigl(\bar
w_0^q\bigr)=Q_{\alpha,z}(\psi_0\|\varphi_0).
\end{align*}
Supose that $Q_{\alpha,z}(\psi_0\|\varphi_0)=Q_{\alpha,z}(\psi\|\varphi)$, then both the
inequalities must be equalities. Since $\bar
w\in L_q(\Me)^+$ and $\bar w_0\in L_q(\Ne)^+$ are the unique elements such that the suprema
in the {\color{red}respective} variational expressions in \eqref{eq:variationalq} for
$Q_{\alpha,z}(\psi\|\varphi)$ and $Q_{\alpha,z}(\psi_0\|\varphi_0)$ are attained, this proves
\eqref{eq:dpiw}. Conversely, if the equalities in \eqref{eq:dpiw} hold, then
\[
Q_{\alpha,z}(\psi_0\|\ffi_0)=\Tr\bigl((y_0^*y_0)^z\bigr)=\Tr\bigl(\bar w_0^q\bigr)
=\Tr\bigl(\bar w^q\bigr)=\Tr\bigl((y^*y)^z\bigr)=Q_{\alpha,z}(\psi\|\ffi).
\]
\end{proof}


\begin{theorem}\label{thm:suffge1}
Let $\gamma:\Ne\to \Me$ be a channel and let $\psi,\varphi\in \Me_*^+$ be such that 
$s(\psi)\le s(\ffi)$ and $D_{\alpha,z}(\psi\|\varphi)<\infty$. Then
$D_{\alpha,z}(\psi\circ\gamma\|\ffi\circ\gamma)=D_{\alpha,z}(\psi\|\varphi)$ if and only if
$\gamma$ is sufficient with respect to $\{\psi,\ffi\}$.

\end{theorem}

\begin{proof} Let $\bar w$ and $\bar w_0$ be as in Lemma \ref{lemma:le}. Let $\omega\in
\Me_*^+$ and $\omega_0\in \Ne_*^+$ be such that
{\color{red}\[
h_\omega=h_\ffi^{\frac{q-1}{2q}}\bar wh_\ffi^{\frac{q-1}{2q}},\qquad
h_{\omega_0}=h_{\ffi_0}^{\frac{q-1}{2q}}\bar w_0h_{\ffi_0}^{\frac{q-1}{2q}}.
\]}
 Assume that the equality in DPI holds; then by Lemma
\ref{lemma:le} we have
\[
Q_{\alpha,z}(\omega_0\|\varphi_0)=\Tr\bigl(\bar w_0^q\bigr)=\Tr\bigl(\bar
w^q\bigr)=Q_{\alpha,z}(\omega\|\varphi).
\]
and using also Lemma \ref{lemma:pcontraction}, we have
\[
(\gamma^*_\ffi)_*(h_{\omega_0})=h_\ffi^{\frac1{2\alpha}}\gamma^*_{\ffi,q}(\bar
w_0)h_\ffi^{\frac1{2\alpha}}=h_\omega.
\]
Similarly as in the proof of {\color{red}Theorem \ref{thm:suffle1},} this shows that $\gamma$ is sufficient
with respect to $\{\omega,\ffi\}$. Hence $\omega\circ \cE=\omega$, where $\cE$ is the
conditional expectation onto the fixed points of $\gamma\circ\gamma^*_\ffi$.  Using the
extensions of $\cE$ and their properties {\color{red}in \cite{junge2003noncommutative}, we have}
\[
h_\varphi^{q-1\over2q}\bar
wh_\varphi^{q-1\over 2q}=h_\omega=\cE(h_\omega)=h_\varphi^{q-1\over 2q}\cE(\bar
w)h_\varphi^{q-1\over 2q},
\]
which implies that $\bar w=\cE(\bar w)\in L_q(\cE(\Me))^+$. But then we also have
\[
|y|=\bar w^{\frac1{2(\alpha-1)}}=\bar w^{\frac{q}{2z}}\in L_{2z}(\cE(\Me))^+
\]
Let $y=u|y|$ be the polar decomposition of $y$; then we obtain from the definition of $y$ that
$uu^*=s(yy^*)=s(\psi)$. Furthermore, since
\[
u^*h_\psi^{\frac1{2p}}=|y|h_\varphi^{\frac1{2q}}\in L_{2p}(\cE(\Me)),
\]
by uniqueness of the polar decomposition in $L_{2p}(\Me)$ and $L_{2p}(\cE(\Me))$, we
obtain that $h_{\psi}^{\frac1{2p}}\in L_{2p}(\cE(\Me))^+$ and $u\in \cE(\Me)$. Hence we must
have $h_\psi\in L_1(\cE(\Me))$ so that $\psi\circ\cE=\psi$ and $\gamma$ is sufficient with
respect to $\{\psi,\ffi\}$ by Proposition \ref{prop:universal}(ii). The converse is clear from DPI.
\end{proof}


\section{Monotonicity in the parameter $z$}

It is well known \cite{berta2018renyi,hiai2018quantum,jencova2018renyi} that the standard R\'enyi divergence
$D_{\alpha,1}(\psi\|\ffi)$ is monotone increasing in $\alpha\in(0,1)\cup(1,\infty)$ and the sandwiched R\'enyi
divergence $D_{\alpha,\alpha}(\psi\|\ffi)$ is monotone increasing in $\alpha\in[1/2,1)\cup(1,\infty)$. It is also
known \cite{berta2018renyi,hiai2018quantum,jencova2018renyi} that
\begin{align}\label{F-4.1}
\lim_{\alpha\nearrow1}D_{\alpha,1}(\psi\|\ffi)=\lim_{\alpha\nearrow1}D_{\alpha,\alpha}(\psi\|\ffi)
=D_1(\psi\|\ffi),
\end{align}
and if $D_{\alpha,1}(\psi\|\ffi)<\infty$ (resp., $D_{\alpha,\alpha}(\psi\|\ffi)<\infty$) for some $\alpha>1$, then
\begin{align}\label{F-4.2}
\lim_{\alpha\searrow1}D_{\alpha,1}(\psi\|\ffi)=D_1(\psi\|\ffi)\quad
\Bigl(\mbox{resp.,}\ \lim_{\alpha\searrow1}D_{\alpha,\alpha}(\psi\|\ffi)=D_1(\psi\|\ffi)\Bigr),
\end{align}
{\color{red}where $D_1(\psi\|\ffi):=D(\psi\|\ffi)/\psi(1)$, the normalized relative entropy.}
In the rest of the paper we will discuss similar monotonicity properties and limits for $D_{\alpha,z}(\psi\|\ffi)$.
We consider monotonicity in the parameter $z$ in Sec.~4 and monotonicity in the parameter $\alpha$ in Sec.~5.

\subsection{The finite von Neumann algebra case}

In this subsection we show monotonicity of $D_{\alpha,z}$ in the parameter $z$ in the finite von Neumann
algebra setting. Recall that if $(\Me,\tau)$ is a semi-finite von Neumann algebra $\Me$ with a faithful normal
semi-finite trace $\tau$, then the Haagerup $L_p$-space $L_p(\Me)$ is identified with the $L_p$-space
$L_p(\Me,\tau)$ with respect to $\tau$ \cite[Example 9.11]{hiai2021lectures}. Hence one can define
$Q_{\alpha,z}(\psi\|\ffi)$ for $\psi,\ffi\in\Me_*^+$ by replacing, in Definition \ref{defi:renyi}, $L_p(\Me)$ with
$L_p(\Me,\tau)$ and $h_\psi\in L_1(\Me)_+$ with the Radon--Nikodym derivative $d\psi/d\tau\in L_1(\Me,\tau)^+$.
Below we use the symbol $h_\psi$ to denote $d\psi/d\tau$ as well. Note that $\tau$ on $\Me^+$ is naturally
extended to the positive part $\widetilde\Me^+$ of the space $\widetilde\Me$ of $\tau$-measurable operators.
We then have \cite[Proposition 2.7]{fack1986generalized} (also \cite[Proposition 4.20]{hiai2021lectures})
\begin{align}\label{F-4.3}
\tau(a)=\int_0^\infty\mu_s(a)\,ds,\qquad a\in\widetilde\Me^+,
\end{align}
where $\mu_s(a)$ is the generalized $s$-number of $a$ \cite{fack1986generalized}.

Now we assume that $(\Me,\tau)$ is a finite von Neumann algebra with a faithful normal finite trace $\tau$. In
this setting note that $\widetilde\Me^+$ consists of all positive self-adjoint operators affiliated with $\Me$. Our
discussions below are essentially along the same lines as in the finite-dimensional case
\cite{lin2015investigating,mosonyi2023somecontinuity}, where the integral expression in \eqref{F-4.3} is useful.


\begin{lemma}\label{L-4.1}
For every $\psi,\ffi\in\Me_*^+$ with $\psi\ne0$ and for any $\alpha,z>0$ with $\alpha\ne1$,
\begin{align}\label{F-4.4}
D_{\alpha,z}(\psi\|\ffi)&=\lim_{\eps\searrow0}D_{\alpha,z}(\psi\|\ffi+\eps\tau)\quad\mbox{increasingly},
\end{align}
and hence $D_{\alpha,z}(\psi\|\ffi)=\sup_{\eps>0}D_{\alpha,z}(\psi\|\ffi+\eps\tau)$.
\end{lemma}

\begin{proof}
{\it Case $0<\alpha<1$}.\enspace
We need to prove that
\begin{align}\label{F-4.5}
Q_{\alpha,z}(\psi\|\ffi)&=\lim_{\eps\searrow0}Q_{\alpha,z}(\psi\|\ffi+\eps\tau)\quad\mbox{decreasingly}.
\end{align}
In the present setting we have by \eqref{F-4.3}
\begin{align}\label{F-4.6}
Q_{\alpha,z}(\psi\|\ffi)
=\tau\Bigl(\bigl(h_\psi^{\alpha/2z}h_\ffi^{1-\alpha\over z}h_\psi^{\alpha/2z}\bigr)^z\Bigr)
=\int_0^\infty\mu_s\Bigl(h_\psi^{\alpha/2z}h_\ffi^{1-\alpha\over z}h_\psi^{\alpha/2z}\Bigr)^z\,ds,
\end{align}
and similarly
\[
Q_{\alpha,z}(\psi\|\ffi+\eps\tau)
=\int_0^\infty\mu_s\Bigl(h_\psi^{\alpha/2z}h_{\ffi+\eps\tau}^{1-\alpha\over z}h_\psi^{\alpha/2z}\Bigr)^z\,ds.
\]
Since $h_{\ffi+\eps\tau}^{1-\alpha\over z}=(h_\ffi+\eps\1)^{1-\alpha\over z}$ decreases to
$h_\ffi^{1-\alpha\over z}$ in the measure topology as $\eps\searrow0$, it follows that
$h_\psi^{\alpha/2z}h_{\ffi+\eps\tau}^{1-\alpha\over z}h_\psi^{\alpha/2z}$ decreases to
$h_\psi^{\alpha/2z}h_\ffi^{1-\alpha\over z}h_\psi^{\alpha/2z}$ in the measure topology. Hence by
\cite[Lemma 3.4]{fack1986generalized} we have
$\mu_s\bigl(h_\psi^{\alpha/2z}h_{\ffi+\eps\tau}^{1-\alpha\over z}h_\psi^{\alpha/2z}\bigr)
\searrow\mu_s\bigl(h_\psi^{\alpha/2z}h_\ffi^{1-\alpha\over z}h_\psi^{\alpha/2z}\bigr)$
as $\eps\searrow0$ for almost every $s>0$. Since
$s\mapsto\mu_s\bigl(h_\psi^{\alpha/2z}h_{\ffi+\tau}^{1-\alpha\over z}h_\psi^{\alpha/2z}\bigr)$ is
integrable on $(0,\infty)$, the Lebesgue convergence theorem gives \eqref{F-4.5}.

{\it Case $\alpha>1$}.\enspace
We need to prove that
\begin{align}\label{F-4.7}
Q_{\alpha,z}(\psi\|\ffi)&=\lim_{\eps\searrow0}Q_{\alpha,z}(\psi\|\ffi+\eps\tau)\quad\mbox{increasingly}.
\end{align}
For any $\eps>0$, since $h_{\ffi+\eps\tau}=h_\ffi+\eps\1$ has the bounded inverse
$h_{\ffi+\eps\tau}^{-1}=(h_\ffi+\eps\1)^{-1}\in\Me^+$, one can define
$x_\eps:=(h_\ffi+\eps\1)^{-{\alpha-1\over2z}}h_\psi^{\alpha/z}(h_\ffi+\eps\1)^{-{\alpha-1\over2z}}
\in\widetilde\Me^+$ so that
\[
h_\psi^{\alpha/z}=(h_\ffi+\eps\1)^{\alpha-1\over2z}x_\eps(h_\ffi+\eps\1)^{\alpha-1\over2z}.
\]
In the present setting one can write by \eqref{F-4.3}
\begin{align}\label{F-4.8}
Q_{\alpha,z}(\psi\|\ffi+\eps\tau)=\tau(x_\eps^z)=\int_0^\infty\mu_s(x_\eps)^z\,ds\ (\in[0,\infty]).
\end{align}
Let $0<\eps\le\eps'$. Since $(h_\ffi+\eps\1)^{-{\alpha-1\over z}}\ge(h_\ffi+\eps'\1)^{-{\alpha-1\over z}}$,
one has $\mu_s(x_\eps)\ge\mu_s(x_{\eps'})$ for all $s>0$, so that
\[
Q_{\alpha,z}(\psi\|\ffi+\eps\tau)\ge Q_{\alpha,z}(\psi\|\ffi+\eps'\tau).
\]
Hence $\eps>0\mapsto D_{\alpha,z}(\psi\|\ffi+\eps\tau)$ is decreasing.

First, assume that $s(\psi)\not\le s(\ffi)$. Then
$\mu_{s_0}\Bigl(h_\psi^{\alpha/2z}s(\ffi)^\perp h_\psi^{\alpha/2z}\Bigr)>0$ for some $s_0>0$; indeed,
otherwise, $h_\psi^{\alpha/2z}s(\ffi)^\perp h_\psi^{\alpha/2z}=0$ so that $s(\psi)\le s(\ffi)$. Hence we have
\[
\mu_s(x_\eps)=\mu_s\Bigl(h_\psi^{\alpha/2z}(h_\ffi+\eps\1)^{-{\alpha-1\over z}}h_\psi^{\alpha/2z}\Bigr)
\ge\eps^{-{\alpha-1\over z}}\mu_s\Bigl(h_\psi^{\alpha/2z}s(\ffi)^\perp h_\psi^{\alpha/2z}\Bigr)
\nearrow\infty\quad\mbox{as $\eps\searrow0$}
\]
for all $s\in(0,s_0]$. Therefore, it follows from \eqref{F-4.8} that
$Q_{\alpha,z}(\psi\|\ffi+\eps\tau)\nearrow\infty=Q_{\alpha,z}(\psi\|\ffi)$.

Next, assume that $s(\psi)\le s(\ffi)$. Take the spectral decomposition $h_\ffi=\int_0^\infty t\,de_t$ and
define $y,x\in\widetilde\Me_+$ by
\[
y:=h_\ffi^{-{\alpha-1\over z}}s(\ffi)=\int_{(0,\infty)}t^{-{\alpha-1\over z}}\,de_t,
\qquad x:=y^{1/2}h_\psi^{\alpha/z}y^{1/2}.
\]
Since
\[
h_\psi^{\alpha/z}=s(\ffi)h_\psi^{\alpha/z}s(\ffi)
=h_\ffi^{\alpha-1\over2z}y^{1/2}h_\psi^{\alpha/z}y^{1/2}h_\ffi^{\alpha-1\over2z}
=h_\ffi^{\alpha-1\over2z}xh_\ffi^{\alpha-1\over2z},
\]
one has, similarly to \ref{F-4.8},
\begin{align}\label{F-4.9}
Q_{\alpha,z}(\psi\|\ffi)=\tau(x^z)=\int_0^\infty\mu_s(x)^z\,ds.
\end{align}
We write $(h_\ffi+\eps\1)^{-{\alpha-1\over z}}s(\ffi)=\int_{(0,\infty)}(t+\eps)^{-{\alpha-1\over z}}\,de_t$,
and for any $\delta>0$ choose a $t_0>0$ such that $\tau(e_{(0,t_0)})<\delta$. Then, since
$\int_{[t_0,\infty)}(t+\eps)^{-{\alpha-1\over z}}\,de_t\to\int_{[t_0,\infty)}t^{-{\alpha-1\over z}}\,de_t$
in the operator norm as $\eps\searrow0$, we obtain $(h_\ffi+\eps\1)^{-{\alpha-1\over z}}s(\ffi)\nearrow y$
in the measure topology (see \cite[1.5]{fack1986generalized}), so that
$h_\psi^{\alpha/2z}(h_\ffi+\eps\1)^{-{\alpha-1\over z}}h_\psi^{\alpha/2z}
\nearrow h_\psi^{\alpha/2z}yh_\psi^{\alpha/2z}$ in the measure topology as $\eps\searrow0$. Hence
we have by \cite[Lemma 3.4]{fack1986generalized}
\begin{align}\label{F-4.10}
\mu_s(x_\eps)=\mu_s\Bigl(h_\psi^{\alpha/2z}(h_\ffi+\eps\1)^{-{\alpha-1\over z}}h_\psi^{\alpha/2z}\Bigr)
\nearrow\mu_s\Bigl(h_\psi^{\alpha/2z}yh_\psi^{\alpha/2z}\Bigr)=\mu_s(x)
\end{align}
for all $s>0$. Therefore, by \eqref{F-4.8} and \eqref{F-4.9} the monotone convergence theorem gives
\eqref{F-4.7}.
\end{proof}

\begin{lemma}\label{L-4.2}
Let $(\Me,\tau)$ and $\psi,\ffi$ be as above, and let $0<z\le z'$. Then
\[
\begin{cases}
D_{\alpha,z}(\psi\|\ffi)\le D_{\alpha,z'}(\psi\|\ffi), & \text{$0<\alpha<1$},\\
D_{\alpha,z}(\psi\|\ffi)\ge D_{\alpha,z'}(\psi\|\ffi), & \text{$\alpha>1$}.
\end{cases}
\]
\end{lemma}

\begin{proof}
The case $0<\alpha<1$ was shown in \cite[Theorem 1(x)]{kato2023onrenyi} for general von Neumann algebras.
For the case $\alpha>1$, by Lemma \ref{L-4.1} it suffices to show that, for every $\eps>0$,
\[
\tau\Bigl(\Bigl(y_\eps^{\alpha-1\over2z}h_\psi^{\alpha/z}y_\eps^{\alpha-1\over2z}\Bigr)^z\Bigr)
\ge\tau\Bigl(\Bigl(y_\eps^{\alpha-1\over2z'}h_\psi^{\alpha/z'}y_\eps^{\alpha-1\over2z'}\Bigr)^{z'}\Bigr),
\]
where $y_\eps:=(h_\ffi+\eps\1)^{-1}\in\Me_+$. The above is equivalently written as
\[
\tau\Bigl(\big|(h_\psi^{\alpha/2z'})^r(y_\eps^{(\alpha-1)/2z'})^r\big|^{2z}\Bigr)
\ge\tau\Bigl(\big|h_\psi^{\alpha/2z'} y_\eps^{(\alpha-1)/2z'}\big|^{2zr}\Bigr),
\]
where $r:=z'/z\ge1$. Hence the desired inequality follows from Kosaki's ALT inequality
\cite[Corollary 3]{kosaki1992aninequality}.
\end{proof}

When $(\Me,\tau)$ and $\psi,\ffi$ are as in Lemma \ref{L-4.1}, one can define, thanks to Lemma \ref{L-4.2},
for any $\alpha\in(0,\infty)\setminus\{1\}$,
\begin{align}
Q_{\alpha,\infty}(\psi\|\ffi)&:=\lim_{z\to\infty}Q_{\alpha,z}(\psi\|\ffi)
=\inf_{z>0}Q_{\alpha,z}(\psi\|\ffi), \nonumber\\
D_{\alpha,\infty}(\psi\|\ffi)&:={1\over\alpha-1}\log{Q_{\alpha,\infty}(\psi\|\ffi)\over\psi(\1)} \nonumber\\
&\ =\lim_{z\to\infty}D_{\alpha,z}(\psi\|\ffi)
=\begin{cases}\sup_{z>0}D_{\alpha,z}(\psi\|\ffi), & \text{$0<\alpha<1$},\\
\inf_{z>0}D_{\alpha,z}(\psi\|\ffi), & \text{$\alpha>1$}.\end{cases} \label{F-4.11}
\end{align}
If $h_\psi,h_\ffi\in\Me^{++}$ (i.e., $\delta\tau\le\psi,\ffi\le\delta^{-1}\tau$ for some $\delta\in(0,1)$), then
the Lie--Trotter formula gives
\begin{align}\label{F-4.12}
Q_{\alpha,\infty}(\psi\|\ffi)=\tau\bigl(\exp(\alpha\log h_\psi+(1-\alpha)\log h_\ffi)\bigr).
\end{align}

\begin{lemma}\label{L-4.3}
Let $(\Me,\tau)$ and $\psi,\ffi$ be as above. Then for any $z>0$,
\[
\begin{cases}
D_{\alpha,z}(\psi\|\ffi)\le D_1(\psi\|\ffi), & \text{$0<\alpha<1$},\\
D_{\alpha,z}(\psi\|\ffi)\ge D_1(\psi\|\ffi), & \text{$\alpha>1$}.
\end{cases}
\]
\end{lemma}

\begin{proof}
First, assume that $h_\psi,h_\ffi\in\Me^{++}$. Set self-adjoint $H:=\log h_\psi$ and $K:=\log h_\ffi$ in $\Me$
and define $F(\alpha):=\log\tau\bigl(e^{\alpha H+(1-\alpha)K}\bigr)$ for $\alpha>0$.
Then by \eqref{F-4.12}, $F(\alpha)=\log Q_{\alpha,\infty}(\psi\|\ffi)$ for all $\alpha\in(0,\infty)\setminus\{1\}$,
and we compute
\begin{align*}
F'(\alpha)&={\tau\bigl(e^{\alpha H+(1-\alpha)K}(H-K)\bigr)\over\tau\bigl(e^{\alpha H+(1-\alpha)K}\bigr)}, \\
F''(\alpha)&={\tau\bigl(e^{\alpha H+(1-\alpha)K}(H-K)^2\bigr)\tau\bigl(e^{\alpha
H+(1-\alpha)K}\bigr)-\bigl\{\tau\bigl(e^{\alpha H+(1-\alpha)K}(H-K)\bigr)\bigr\}^2
\over\bigl\{\tau\bigl(e^{\alpha H+(1-\alpha)K}\bigr)\bigr\}^2}.
\end{align*}
Since $F''(\alpha)\ge0$ on $(0,\infty)$ thanks to the Schwarz inequality, we see that $F(\alpha)$ is
convex on $(0,\infty)$ and hence
\[
D_{\alpha,\infty}(\psi\|\ffi)={F(\alpha)-F(1)\over\alpha-1}
\]
is increasing in $\alpha\in(0,\infty)$, where for $\alpha=1$ the above RHS is understood as
\[
F'(1)={\tau(e^H(H-K))\over\tau(e^H)}={\tau\bigl(h_\psi(\log h_\psi-\log h_\ffi)\bigr)\over\tau(h_\psi)}
=D_1(\psi\|\ffi).
\]
Hence by \eqref{F-4.11} the assertion holds when $h_\psi,h_\ffi\in\Me^{++}$. Below we extend it to general
$\psi,\ffi\in\Me_*^+$.

{\it Case $0<\alpha<1$}.\enspace
Let $\psi,\ffi\in\Me_*^+$ and $z>0$. From \cite[Theorem 1(iv)]{kato2023onrenyi} and
\cite[Corollary 2.8(3)]{hiai2021quantum} we have
\begin{align*}
D_{\alpha,z}(\psi\|\ffi)&=\lim_{\eps\searrow0}D_{\alpha,z}(\psi+\eps\tau\|\ffi+\eps\tau), \\
D_1(\psi\|\ffi)&=\lim_{\eps\searrow0}D_1(\psi+\eps\tau\|\ffi+\eps\tau),
\end{align*}
so that we may assume that $\psi,\ffi\ge\eps\tau$ for some $\eps>0$. Take the spectral decompositions
$h_\psi=\int_0^\infty t\,de_t^\psi$ and $h_\ffi=\int_0^\infty t\,de_t^\ffi$, and define
$e_n:=e_n^\psi\wedge e_n^\ffi$ for each $n\in\bN$. Then
$\tau(e_n^\perp)\le\tau((e_n^\psi)^\perp)+\tau((e_n^\ffi)^\perp)\to0\quad\mbox{as $n\to\infty$}$,  so that
$e_n\nearrow\1$. We set $\psi_n:=\psi(e_n\cdot e_n)$ and $\ffi_n:=\ffi(e_n\cdot e_n)$; then
$h_{\psi_n}=e_nh_\psi e_n$ and $h_{\ffi_n}=e_nh_\ffi e_n$ are in $(e_n\Me e_n)^{++}$. Note that
\begin{align*}
\|h_\psi-e_nh_\psi e_n\|_1&\le\|(\1-e_n)h_\psi\|_1+\|e_nh_\psi(\1-e_n)\|_1 \\
&\le\|(\1-e_n)h_\psi^{1/2}\|_2\|h_\psi^{1/2}\|_2+\|e_nh_\psi^{1/2}\|_2\|h_\psi^{1/2}(\1-e_n)\|_2 \\
&=\psi(\1-e_n)^{1/2}\psi(\1)^{1/2}+\psi(e_n)^{1/2}\psi(\1-e_n)^{1/2}\to0\quad\mbox{as $n\to\infty$},
\end{align*}
and similarly $\|h_\ffi-e_nh_\ffi e_n\|_1\to0$. Hence by \cite[Theorem 1(iv)]{kato2023onrenyi} one has
$D_{\alpha,z}(e_n\psi e_n\|e_n\ffi e_n)\to D_{\alpha,z}(\psi\|\ffi)$. On the other hand, one has
$D_1(e_n\psi e_n\|e_n\ffi e_n)\to D_1(\psi\|\ffi)$ by \cite[Proposition 2.10]{hiai2021quantum}. Since
$D_{\alpha,z}(e_n\psi e_n\|e_n\ffi e_n)\le D_1(e_n\psi e_n\|e_n\ffi e_n)$ holds by regarding
$e_n\psi e_n,e_n\ffi e_n$ as functionals on the reduced von Neumann algebra $e_n\Me e_n$, we obtain
the desired inequality for general $\psi,\ffi\in\Me_*^+$.

{\it Case $\alpha>1$}.\enspace
We show the extension to general $\psi,\ffi\in\Me_*^+$ by dividing four steps as follows, where
$h_\psi=\int_0^\infty t\,e_t^\psi$ and $h_\ffi=\int_0^\infty t\,de_t^\ffi$ are the spectral decompositions.

(1)\enspace
Assume that $h_\psi\in\Me^+$ and $h_\ffi\in\Me^{++}$. Set $\psi_n\in\Me_*^+$ by
$h_{\psi_n}=(1/n)e_{[0,1/n]}^\psi+\int_{(1/n,\infty)}t\,de_t^\psi$ ($\in\Me^{++}$). Since
$h_{\psi_n}^{\alpha/z}\searrow h_\psi^{\alpha/z}$ in the operator norm, we have by \eqref{F-4.6} and
\cite[Lemma 3.4]{fack1986generalized}
\begin{equation}\label{F-4.13}
\begin{aligned}
Q_{\alpha,z}(\psi\|\ffi)&=\int_0^\infty\mu_s\Bigl((h_\ffi^{-1})^{\alpha-1\over2z}h_\psi^{\alpha/z}
(h_\ffi^{-1})^{\alpha-1\over2z}\Bigr)^z\,ds \\
&=\lim_{n\to\infty}\int_0^\infty\mu_s\Bigl((h_\ffi^{-1})^{\alpha-1\over2z}h_{\psi_n}^{\alpha/z}
(h_\ffi^{-1})^{\alpha-1\over2z}\Bigr)^z\,ds
=\lim_{n\to\infty}Q_{\alpha,z}(\psi_n\|\ffi).
\end{aligned}
\end{equation}
From this and the lower semicontinuity of $D_1$ the extension holds in this case.

(2)\enspace
Assume that $h_\psi\in\Me^+$ and $h_\ffi\ge\delta\1$ for some $\delta>0$. Set $\ffi_n\in\Me_*^+$
by $h_{\ffi_n}=\int_{[\delta,n]}t\,de_t^\ffi+ne_{(n,\infty)}^\ffi$ ($\in\Me^{++}$). Since
$h_{\ffi_n}^{-{\alpha-1\over z}}\searrow h_\ffi^{-{\alpha-1\over z}}$ in the operator norm, we have by
\eqref{F-4.6} and \cite[Lemma 3.4]{fack1986generalized} again
\begin{align*}
Q_{\alpha,z}(\psi\|\ffi)&=\int_0^\infty\mu_s\Bigl(h_\psi^{\alpha/2z}h_\ffi^{-{\alpha-1\over z}}
h_\psi^{\alpha/2z}\Bigr)^z\,ds \\
&=\lim_{n\to\infty}\int_0^\infty\mu_s\Bigl(h_\psi^{\alpha/2z}h_{\ffi_n}^{-{\alpha-1\over z}}
h_\psi^{\alpha/2z}\Bigr)^z\,ds
=\lim_{n\to\infty}Q_{\alpha,z}(\psi\|\ffi_n).
\end{align*}
From this and (1) above the extension holds in this case too.

(3)\enspace
Assume that $\psi$ is general and $\ffi\ge\delta\tau$ for some $\delta>0$. Set $\psi_n\in\Me_*^+$
by $h_{\psi_n}=\int_{[0,n]}t\,de_t^\psi+ne_{(n,\infty)}^\ffi$ ($\in\Me_+$). Since
$h_{\psi_n}^{\alpha/z}\nearrow h_\psi^{\alpha/z}$ in the measure topology, one can argue as in \eqref{F-4.13}
with use of the monotone convergence theorem to see from (2) that the extension holds in this case too.

(4)\enspace
Finally, from (3) with Lemma \ref{L-4.1} and \cite[Corollary 2.8(3)]{hiai2021quantum} it follows that
the desired extension hods for general $\psi,\ffi\in\Me_*^+$.
\end{proof}

In the next proposition we summarize inequalities for $D_{\alpha,z}$ obtained so far in Lemmas \ref{L-4.2}
and \ref{L-4.3}.

\begin{prop}\label{P-4.4}
Assume that $\Me$ is a finite von Neumann algebra with a faithful normal finite trace $\tau$. Let
$\psi,\ffi\in\Me_*^+$, $\psi\ne0$. If $0<\alpha<1<\alpha'$ and $0<z\le z'\le\infty$, then
\[
D_{\alpha,z}(\psi\|\ffi)\le D_{\alpha,z'}(\psi\|\ffi)\le D_1(\psi\|\ffi)
\le D_{\alpha',z'}(\psi\|\ffi)\le D_{\alpha',z}(\psi\|\ffi).
\]
\end{prop}

In view of \eqref{F-4.1} and \eqref{F-4.2}, the above proposition gives the next limits in the restrictive
situation of this subsection, while in Sec.~5.3 we will obtain similar results in general von Neumann algebras
by using the complex interpolation method.

\begin{coro}\label{C-4.5}
Let $(\Me,\tau)$ and $\psi,\ffi$ be as in Proposition \ref{P-4.4}. Then for any $z\in[1,\infty]$,
\begin{align}\label{F-4.14}
\lim_{\alpha\nearrow1}D_{\alpha,z}(\psi\|\ffi)=D_1(\psi\|\ffi).
\end{align}
Moreover, if $D_{\alpha,\alpha}(\psi\|\ffi)<\infty$ for some $\alpha>1$ then for any $z\in(1,\infty]$,
\begin{align}\label{F-4.15}
\lim_{\alpha\searrow1}D_{\alpha,z}(\psi\|\ffi)=D_1(\psi\|\ffi).
\end{align}
\end{coro}

\begin{proof}
For every $z\in[1,\infty]$ and $\alpha\in(0,1)$, Proposition \ref{P-4.4} gives
\[
D_{\alpha,1}(\psi\|\ffi)\le D_{\alpha,z}(\psi\|\ffi)\le D_1(\psi\|\ffi).
\]
Hence \eqref{F-4.14} follows since it holds for $D_{\alpha,1}$ {\color{red}as stated in \eqref{F-4.1};
see \cite[Proposition 5.3(3)]{hiai2018quantum}.}

Next, assume that $D_{\alpha,\alpha}(\psi\|\ffi)<\infty$ for some $\alpha>1$. For every $z\in(1,\infty]$ and
$\alpha\in(1,z]$, Proposition \ref{P-4.4} gives
\[
D_1(\psi\|\ffi)\le D_{\alpha,z}(\psi\|\ffi)\le D_{\alpha,\alpha}(\psi\|\ffi).
\]
Hence \eqref{F-4.15} follows since it holds for $D_{\alpha,\alpha}$ {\color{red}as stated in \eqref{F-4.2};
see \cite[Proposition 3.8(ii)]{jencova2018renyi}.}
\end{proof}

\medskip
In this subsection, in the specialized setting of finite von Neumann algebras, we have given monotonicity
of $D_{\alpha,z}$ in the parameter $z$ in an essentially similar way to the finite-dimensional case
\cite{mosonyi2023somecontinuity}. In the next subsection we will extend it to general von Neumann algebras
under certain restrictions of $\alpha,z$.


\subsection{The general von Neumann algebra case}

In this subsection we show monotonicity of $D_{\alpha,z}$ in the parameter $z$ in general von Neumann
algebras under certain restrictions of $\alpha,z$. From now on let $\Me$ be a general $\sigma$-finite von
Neumann algebra. 

First, we extend Proposition \ref{P-4.4} to general von Neumann algebras, based on Haagerup's reduction
theorem \cite{haagerup2010areduction} that is briefly explained in Appendix \ref{Appen-reduction} for the
convenience of the reader. Let $\omega$ be a faithful normal state of $\Me$, and let
 \[
 \hat\Me,\qquad\hat\omega,\qquad E_\Me:\hat\Me\to\Me,\qquad\Me_n,\qquad E_{\Me_n}:\hat\Me\to\Me_n
 \qquad(n\ge1)
 \]
 be given as in Theorem \ref{T-B.1}. We then give the next lemma.

\begin{lemma}\label{L-4.6}
In the above situation, for any $\psi,\ffi\in\Me_*^+$ with $\psi\ne0$ let $\hat\psi:=\psi\circ E_\Me$ and
$\hat\ffi:=\ffi\circ E_\Me$. If $\alpha,z>0$ with $\alpha\ne1$ satisfy either
\[
0<\alpha<1,\qquad \max\{\alpha,1-\alpha\}\le z,
\]
or
\[
\alpha>1,\qquad\max\{\alpha/2,\alpha-1\}\le z\le\alpha,
\]
then we have
\begin{align}
D_{\alpha,z}(\psi\|\ffi)&=D_{\alpha,z}(\hat\psi\|\hat\ffi)
=\lim_{n\to\infty}D_{\alpha,z}(\hat\psi|_{\Me_n}\|\hat\ffi|_{\Me_n})\quad\mbox{increasingly}, \label{F-4.16}\\
D_1(\psi\|\ffi)&=D_1(\hat\psi\|\hat\ffi)
=\lim_{n\to\infty}D_1(\hat\psi|_{\Me_n}\|\hat\ffi|_{\Me_n})\quad\mbox{increasingly}. \label{F-4.17}
\end{align}
\end{lemma}

\begin{proof}
Apply the DPI for $D_{\alpha,z}$ proved in Theorem \ref{thm:dpi} to the injection $\Me\hookrightarrow\hat\Me$
and to the conditional expectation $E_\Me:\hat\Me\to\Me$. We then have the first equality in \eqref{F-4.16}.
By Theorem \ref{T-B.1} we can apply the martingale convergence in Theorem \ref{eq:martingale} to obtain
the latter equality in \eqref{F-4.16} with increasing convergence. The assertion of $D_1$ in \eqref{F-4.17} is
included in \cite[Proposition 2.2]{fawzi2023asymptotic}, while this is the well-known martingale convergence
of the relative entropy \cite{kosaki1986relative}.
\end{proof}

\begin{theorem}\label{T-4.7}
For every $\psi,\ffi\in\Me_*^+$, $\psi\ne0$, we have:
\begin{itemize}
\item[(1)] If $0<\alpha<1$ and {\color{red}$0\le z\le z'$,} then
\[
D_{\alpha,z}(\psi\|\ffi)\le D_{\alpha,z'}(\psi\|\ffi)\le D_1(\psi\|\ffi).
\]
\item[(2)] If $\alpha>1$ and $\max\{\alpha/2,\alpha-1\}\le z\le z'\le\alpha$, then
\[
D_1(\psi\|\ffi)\le D_{\alpha,z'}(\psi\|\ffi)\le D_{\alpha,z}(\psi\|\ffi).
\]
\end{itemize}
\end{theorem}

{\color{red}
\begin{proof}
The first inequality in (1) was shown in \cite[Theorem 1(x)]{kato2023onrenyi}. The other inequalities in (1)
and (2) immediately follow from Proposition \ref{P-4.4} and Lemma \ref{L-4.6}.
\end{proof}
}

For the proof of Theorem \ref{T-4.7} using Theorem \ref{T-B.1} it is inevitable to restrict the parameter $z$
to the DPI range for each $\alpha$. But the next theorem strengthens Theorem \ref{T-4.7}(2) into a wider
range of $z$, whose proof is based on Kosaki's interpolation $L_p$-spaces \cite{kosaki1984applications}.
(A brief explanation on Kosaki's interpolation $L_p$-spaces is given in Appendix \ref{Appen-Kosaki-Lp} for
the convenience of the reader.)

\begin{theorem}\label{T-4.8}
For every $\psi,\ffi\in\Me_*^+$, $\psi\ne0$, and $\alpha>1$, the function $z\mapsto D_{\alpha,z}(\psi\|\varphi)$
is monotone decreasing on $[\alpha/2,\infty)$.
\end{theorem}

\begin{proof}
Let $\alpha>1$ and $z,z'\in[\alpha/2,\infty)$ be such that $z<z'$. We need to prove that
$Q_{\alpha,z}(\psi\|\ffi)\ge Q_{\alpha,z'}(\psi\|\ffi)$. To do this, we may assume that
$Q_{\alpha,z}(\psi\|\ffi)<\infty$. Hence by Lemma \ref{lemma:renyi_2z}, there is some $y\in L_{2z}(\Me)$
such that
\[
h_\psi^{\alpha\over2z}=yh_\ffi^{\alpha-1\over2z},\qquad
Q_{\alpha,z}(\psi\|\ffi)=\|y\|_{2z}^{2z}.
\]
In particular, $e:=s(\psi)\le s(\ffi)$, so that we may assume that $\ffi$ is faithful. Let $\sigma\in\Me_*^+$
be such that $s(\sigma)=\1-e$, and set $\psi_0:=\psi+\sigma$, so that $\psi_0$ is faithful too. Let us use
for simplicity the notation $L_{p,L}$ for Kosaki's left $L_p$-space $L_p(\Me,\ffi)_L$ for $1\le p\le\infty$;
see \eqref{F-C.2} in Appendix \ref{Appen-Kosaki-Lp}.

Consider the function
\begin{align}\label{F-4.18}
f(w):=h_{\psi_0}^{{\alpha\over2z}w}eh_\ffi^{1-{\alpha\over2z}w},\qquad w\in S,
\end{align}
where $S:=\{w\in\bC:0\le\Re w\le1\}$. Then, for $w=s+it$ with $0\le s\le1$ and $t\in\bR$, since
\[
f(s+it)=h_{\psi_0}^{{\alpha\over2z}it}h_{\psi_0}^{{\alpha\over2z}s}e
h_\ffi^{1-{\alpha\over2z}s}h_\ffi^{-{\alpha\over2z}it},
\]
it is easy to see that $f$ is a bounded continuous function on $S$ into $L_1(\Me)$ and it is analytic in the
interior (see, e.g., \cite[Lemma 9.19, Theorem 9.18(c)]{hiai2021lectures}). Furthermore, we have
\begin{align*}
&f(it)=h_{\psi_0}^{{\alpha\over2z}it}eh_\ffi^{-{\alpha\over2z}it}h_\ffi\in L_{\infty,L}, \\
&\|f(it)\|_{L_{\infty,L}}=\Big\|h_{\psi_0}^{{\alpha\over2z}it}eh_\ffi^{-{\alpha\over2z}it}\Big\|=1,
\qquad t\in\bR,
\end{align*}
and
\begin{align*}
&f(1+it)=h_{\psi_0}^{{\alpha\over2z}it}h_\psi^{\alpha\over2z}
h_\ffi^{1-{\alpha\over2z}}h_\ffi^{-{\alpha\over2z}it}
=\Bigl(h_{\psi_0}^{{\alpha\over2z}it}yh_\ffi^{-{\alpha\over2z}it}\Bigr)h_\ffi^{2z-1\over2z}\in L_{2z,L}, \\
&\|f(1+it)\|_{L_{2z,L}}=\Big\|h_{\psi_0}^{{\alpha\over2z}it}yh_\ffi^{-{\alpha\over2z}it}\Big\|_{2z}=\|y\|_{2z},
\qquad t\in\bR.
\end{align*}
Therefore, $f$ belongs to the set $\cF'(L_{\infty,L},L_{2z,L})$ of $L_1(\Me)$-valued functions given in
\cite[Definition 1.4]{kosaki1984applications}. Since $L_{2z,L}$ is reflexive thanks to $1<\alpha\le2z<\infty$,
it follows from \cite[Theorems 1.5, Remark 3.4]{kosaki1984applications} that the set
$\cF'(L_{\infty,L},L_{2z,L})$ defines the interpolation space $C_\theta=C_\theta(L_{\infty,L},L_{2z,L})$ in
\cite[Definition 1.1]{kosaki1984applications}. Hence for any $\theta\in(0,1)$, we have $f(\theta)\in C_\theta$
and
\[
\|f(\theta)\|_{C_\theta}\le\biggl(\sup_{t\in\bR}\|f(it)\|_{L_{\infty,L}}\biggr)^{1-\theta}
\biggl(\sup_{t\in\bR}\|f(1+it)\|_{L_{2z,L}}\biggr)^\theta=\|y\|_{2z}^\theta
\]
{\color{red}(see \cite[Lemma 4.3.2(ii)]{bergh1976interpolation} for the last inequality).} By
\cite[Theorem 1.9]{kosaki1984applications} and the reiteration theorem {\color{red}(see
\cite{cwikel1978complex}),} $C_\theta=L_{2z/\theta,L}$ with {\color{red}equal norms}, so that putting
$\theta=z/z'$ we have
\[
f(z/z')=h_\psi^{\alpha\over2z'}{\color{red}h_\ffi^{1-{\alpha\over2z'}}}=y'h_\ffi^{2z'-1\over2z'}
\]
for some $y'\in L^{2z'}(\Me)$, and $\|y'\|_{2z'}\le\|y\|_{2z}^{z/z'}$. This implies that
$h_\psi^{\alpha\over2z'}=y'h_\ffi^{\alpha-1\over2z'}$ so that
$Q_{\alpha,z'}(\psi\|\ffi)=\|y'\|_{2z'}^{2z'}\le\|y\|_{2z}^{2z}$, and the assertion follows.
\end{proof}


\section{Monotonicity in the parameter $\alpha$}

In this section we show monotonicity of $D_{\alpha,z}$ in the parameter $\alpha$ as well as limits of
$D_{\alpha,z}$ as $\alpha\nearrow1$ and $\alpha\searrow1$.

\subsection{The case $\alpha<1$ and all $z>0$}

The aim of this subsection is to prove the next theorem.

\begin{theorem}\label{T-5.1}
Let $\psi,\ffi\in\Me_*^+$ and $z>0$. Then we have
\begin{itemize}
\item[(1)] $\alpha\mapsto\log Q_{\alpha,z}(\psi\|\ffi)$ is convex on $(0,1)$,
\item[(2)] $\alpha\mapsto D_{\alpha,z}(\psi\|\ffi)$ is monotone increasing on $(0,1)$.
\end{itemize}
\end{theorem}

Below we will supply two different proofs of the theorem. The first proof for all $z>0$ is given in the
real analysis method, and the second one when $z>1/2$ is in the complex analysis method.

\subsubsection{The first proof}

To give the first proof, we obtain a certain ``log-majorization'' result for positive $\tau$-measurable
operators, which might be meaningful in its own. Assume that $(\Ne,\tau)$ be a semi-finite von Neumann
algebra $\Ne$ with a faithful normal semi-finite trace $\tau$. Let $\widetilde\Ne$ denote the space of
$\tau$-measurable operators affiliated with $\Ne$. For each $a\in\widetilde\Ne$ we write $\mu_t(a)$, $t>0$,
for the ($t$th) \emph{generalized $s$-number} of $a$ \cite{fack1986generalized}. We consider operators
$a\in\widetilde\Ne$ satisfying
\begin{align}\label{F-5.1}
\mbox{$a\in\Ne$\ \ or\ \ $\mu_t(a)\le Ct^{-\gamma}$ ($t>0$) for some $C,\gamma>0$.}
\end{align}
For each $a\in\widetilde\Ne$ satisfying \eqref{F-5.1} we define \cite{fack1986generalized}
\[
\Lambda_t(a):=\exp\int_0^t\log\mu_s(a)\,ds,\qquad t>0.
\]
Note \cite[]{fack1986generalized} that $\Lambda_t(a)\in[0,\infty)$, $t>0$, are well defined whenever $a$
satisfies \eqref{F-5.1}. Also, note that if $a,b\in\widetilde\Ne$ satisfy \eqref{F-5.1}, then $|a|^p$ $(p>0$)
and $ab$ satisfy \eqref{F-5.1} too, as it is clear since $\mu_t(ab)\le\|a\|\mu_t(b)$ for $a\in\Ne$,
$\mu_t(|a|^p)=\mu_t(a)^p$, and $\mu_t(ab)\le\mu_{t/2}(a)\mu_{t/2}(b)$ (see
\cite[Lemma 2.5]{fack1986generalized}).

\begin{prop}\label{P-5.2}
Let $a_j,b_j\in\widetilde\Ne_+$, $j=1,2$, satisfying \eqref{F-5.1} and assume that $a_1a_2=a_2a_1$
and $b_1b_2=b_2b_1$. Then for every $\theta\in(0,1)$ and any $t>0$,
\begin{align}\label{F-5.2}
\Lambda_t\Bigl((a_1^\theta a_2^{1-\theta})^{1/2}(b_1^\theta b_2^{1-\theta})
(a_1^\theta a_2^{1-\theta})^{1/2}\Bigr)
&\le\Lambda_t\bigl(a_1^\theta b_1^\theta\bigr)\Lambda_t\bigl(a_2^{1-\theta}b_2^{1-\theta}\bigr).
\end{align}
In particular,
\begin{align}\label{F-5.3}
\Lambda_t\Bigl((a_1^{1/2}a_2^{1/2})^{1/2}(b_1^{1/2}b_2^{1/2})(a_1^{1/2}a_2^{1/2})^{1/2}\Bigr)
\le\Lambda_t\bigl(a_1^{1/2}b_1a_1^{1/2}\bigr)^{1/2}\Lambda_t\bigl(a_2^{1/2}b_2a_2^{1/2}\bigr)^{1/2}.
\end{align}
\end{prop}

\begin{proof}
Let $\theta\in(0,1)$ and $t>0$ be arbitrary. For any $k\in\bN$ we note that
\begin{align*}
&\bigl((a_1^\theta a_2^{1-\theta})^{1/2}(b_1^\theta b_2^{1-\theta})
(a_1^\theta a_2^{1-\theta})^{1/2}\bigr)^k \\
&\quad=(a_1^\theta a_2^{1-\theta})^{1/2}(b_1^\theta b_2^{1-\theta})(a_1^\theta a_2^{1-\theta})
(b_1^\theta b_2^{1-\theta})\cdots(a_1^\theta a_2^{1-\theta})(b_1^\theta b_2^{1-\theta})
(a_1^\theta a_2^{1-\theta})^{1/2} \\
&\quad=(a_1^\theta a_2^{1-\theta})^{1/2}b_2^{1-\theta}(b_1^\theta a_1^\theta)
(a_2^{1-\theta}b_2^{1-\theta})\cdots(b_1^\theta a_1^\theta)(a_2^{1-\theta}b_2^{1-\theta})
b_1^\theta(a_1^\theta a_2^{1-\theta})^{1/2}.
\end{align*}
Since $(a_1^\theta a_2^{1-\theta})^{1/2}(b_1^\theta b_2^{1-\theta})$, $(a_1^\theta a_2^{1-\theta})^{1/2}$,
etc.\ are $\tau$-measurable operators satisfying \eqref{F-5.1}, we have
by \cite[Theorem 4.2(ii)]{fack1986generalized}
\begin{align*}
&\Lambda_t\bigl((a_1^\theta a_2^{1-\theta})^{1/2}(b_1^\theta b_2^{1-\theta})
(a_1^\theta a_2^{1-\theta})^{1/2}\bigr)^k \\
&\quad\le\Lambda_t\bigl((a_1^\theta a_2^{1-\theta})^{1/2}b_2^{1-\theta}\bigr)
\Lambda_t(b_1^\theta a_1^\theta)^{k-1}\Lambda_t\bigl(a_2^{1-\theta}b_2^{1-\theta}\bigr)^{k-1}
\Lambda_t\bigl(b_1^\theta(a_1^\theta a_2^{1-\theta})^{1/2}\bigr),
\end{align*}
so that
\begin{align*}
&\Lambda_t\bigl((a_1^\theta a_2^{1-\theta})^{1/2}(b_1^\theta b_2^{1-\theta})
(a_1^\theta a_2^{1-\theta})^{1/2}\bigr) \\
&\quad\le\Lambda_t\bigl((a_1^\theta a_2^{1-\theta})^{1/2}b_2^{1-\theta}\bigr)^{1/k}
\Lambda_t(b_1^\theta a_1^\theta)^{1-{1\over k}}\Lambda_t\bigl(a_2^{1-\theta}b_2^{1-\theta}\bigr)^{1-{1\over k}}
\Lambda_t\bigl(b_1^\theta(a_1^\theta a_2^{1-\theta})^{1/2}\bigr)^{1/k}.
\end{align*}
Letting $k\to\infty$ gives \eqref{F-5.2}. When $\theta=1/2$, \eqref{F-4.16} is rewritten as \eqref{F-5.3}.
\end{proof}



\begin{remark}\label{R-5.3}
Since $\Lambda_t(a_j^rb_j^r)\le\Lambda_t(a_jb_j)^r$ for any $r\in(0,1)$ by \cite{kosaki1992aninequality},
inequality \eqref{F-5.2} implies that
\[
\Lambda_t\bigl((a_1^\theta a_2^{1-\theta})^{1/2}(b_1^\theta b_2^{1-\theta})
(a_1^\theta a_2^{1-\theta})^{1/2}\bigr)
\le\Lambda_t(a_1b_1)^\theta\Lambda_t(a_2b_2)^{1-\theta}
=\Lambda_t\bigl(a_1b_1^2a_1\bigr)^{\theta\over2}\Lambda_t\bigl(a_2b_2^2a_2)^{1-\theta\over2}.
\]
We are indeed interested in whether a stronger inequality
\[
\Lambda_t\bigl((a_1^\theta a_2^{1-\theta})^{1/2}(b_1^\theta b_2^{1-\theta})
(a_1^\theta a_2^{1-\theta})^{1/2}\bigr)
\le\Lambda_t\bigl(a_1^{1/2}b_1a_1^{1/2}\bigr)^\theta
\Lambda_t\bigl(a_2^{1/2}b_2a_2^{1/2}\bigr)^{1-\theta}
\]
hold or not in the situation of Proposition \ref{P-5.2}. The last inequality is known to hold in the
finite-dimensional setting \cite[Theorem 2.1]{hiai2024log-majorization}.
\end{remark}

\begin{lemma}\label{L-5.4}
Let $\psi,\ffi\in\Me_*^+$ with $\psi\ne0$, and assume that $s(\psi)\not\perp s(\ffi)$. Then for every $z>0$,
$Q_{\alpha,z}(\psi\|\ffi)>0$ for all $\alpha\in(0,1)$, and $\alpha\mapsto Q_{\alpha,z}(\psi\|\ffi)$ is continuous
on $(0,1)$.
\end{lemma}

\begin{proof}
Assume that $Q_{\alpha,z}(\psi\|\ffi)=0$ for some $z>0$ and $\alpha\in(0,1)$. Then
$h_\psi^{\alpha/2z}h_\ffi^{(1-\alpha)/2z}=0$ as a $\tau$-measurable operator affiliated with
$\Ne:=\Me\rtimes_\sigma\bR$ that is a semi-finite von Neumann algebra with the canonical semi-finite
trace $\tau$, where $\sigma_t$ is the modular automorphism group associated with some faithful normal
state (or weight). Since $s(\psi)=s(h_\psi^{\alpha/2z})$ and $s(\ffi)=s(h_\ffi^{(1-\alpha)/2z})$, it is easy to
see that $s(\psi)\perp s(\ffi)$. Hence the first assertion follows.

Next, since $p>0\mapsto a^p\in\widetilde\Ne$ is differentiable in the measure topology for any
$a\in\widetilde\Ne_+$ (see, e.g., \cite[Lemma 9.19]{hiai2021lectures}), we see that
$\alpha\mapsto h_\psi^{\alpha/2z}h_\ffi^{(1-\alpha)/z}h_\psi^{\alpha/2z}$ is differentiable (hence continuous)
on $(0,1)$ in the measure topology. Hence by \cite[Lemma 9.14]{hiai2021lectures}, the function
$\alpha\mapsto Q_{\alpha,z}(\psi\|\ffi)=\|h_\psi^{\alpha/2z}h_\ffi^{(1-\alpha)/z}h_\psi^{\alpha/2z}\|_z^z$ is
continuous. Here, when $z<1$, note \cite[Theorem 4.9(iii)]{fack1986generalized} that
$|\,\|a\|_z^z-\|b\|_z^z|\le\|a-b\|_z^z$ for $a,b\in L_z(\Me)$.
\end{proof}

\noindent
{\bf The first proof of Theorem \ref{T-5.1}.}\enspace
We may assume that $s(\psi)\not\perp s(\ffi)$; otherwise, $D_{\alpha,z}(\psi\|\ffi)=0$ for all $\alpha\in(0,1)$.
Then by Lemma \ref{L-5.4}, $Q_{\alpha,z}(\psi\|\ffi)\in(0,\infty)$ for all $\alpha\in(0,1)$, and
$\alpha\mapsto Q_{\alpha,z}(\psi\|\ffi)$ is continuous on $(0,1)$. Hence $\alpha\mapsto D_{\alpha,z}(\psi\|\ffi)$
is continuous on $(0,1)$ too.

Let $\alpha_1,\alpha_2\in(0,1)$ and $z>0$. Let $(\Ne,\tau)$ be as in the proof of Lemma \ref{L-5.4}. Consider
$a_j:=h_\psi^{\alpha_j/z}$ and $b_j:=h_\ffi^{(1-\alpha_j)/z}$ in $\widetilde\Ne_+$. Since
$a_j\in L^{z/\alpha_j}(\Me)$, we note by \cite[Lemma 4.8]{fack1986generalized} that
$\mu_t(a_j)=t^{-\alpha_j/z}\|a_j\|_{z/\alpha_j}$ $t>0$, and hence $a_j$ satisfies \eqref{F-5.1}. Similarly, $b_j$
does so. Therefore, we can apply \eqref{F-5.3} to $a_j,b_j$ with $t=1$ to obtain
\begin{align}
&\int_0^1\log\mu_s\Bigl(h_\psi^{\alpha_1+\alpha_2\over4z}
h_\ffi^{2-\alpha_1-\alpha_2\over2z}h_\psi^{\alpha_1+\alpha_2\over4z}\Bigr)\,ds \nonumber\\
&\quad\le{1\over2}\biggl[\int_0^1\log\mu_s\Bigl(h_\psi^{\alpha_1\over2z}
h_\ffi^{1-\alpha_1\over z}h_\psi^{\alpha_1\over2z}\Bigr)\,ds
+\int_0^1\log\mu_s\Bigl(h_\psi^{\alpha_2\over2z}
h_\ffi^{1-\alpha_2\over z}h_\psi^{\alpha_2\over2z}\Bigr)\,ds\biggr]. \label{F-5.4}
\end{align}
Since $h_\psi^{\alpha_1+\alpha_2\over4z}h_\ffi^{2-\alpha_1-\alpha_2\over2z}h_\psi^{\alpha_1+\alpha_2\over4z}$
is in $L^z(\Me)$, we have by \cite[Lemma 4.8]{fack1986generalized} again
\[
\mu_s\Bigl(h_\psi^{\alpha_1+\alpha_2\over4z}
h_\ffi^{2-\alpha_1-\alpha_2\over2z}h_\psi^{\alpha_1+\alpha_2\over4z}\Bigr)
=s^{-1/z}\Big\|h_\psi^{\alpha_1+\alpha_2\over4z}
h_\ffi^{2-\alpha_1-\alpha_2\over2z}h_\psi^{\alpha_1+\alpha_2\over4z}\Big\|_z
\]
so that
\begin{align}\label{F-5.5}
\log\mu_s\Bigl(h_\psi^{\alpha_1+\alpha_2\over4z}
h_\ffi^{2-\alpha_1-\alpha_2\over2z}h_\psi^{\alpha_1+\alpha_2\over4z}\Bigr)^z
=-\log s+\log Q_{{\alpha_1+\alpha_2\over2},z}(\psi\|\ffi).
\end{align}
Similarly,
\begin{align}\label{F-5.6}
\log\mu_s\Bigl(h_\psi^{\alpha_j\over2z}h_\ffi^{1-\alpha_j\over z}h_\psi^{\alpha_j\over2z}\Bigr)^z
=-\log s+\log Q_{\alpha_j,z}(\psi\|\ffi),\qquad j=1,2.
\end{align}
Multiply $z$ to both sides of \eqref{F-5.4} and insert \eqref{F-5.5} and \eqref{F-5.6} into it. Since
$\int_0^1(-\log s)\,ds=1$, we then arrive at
\[
1+\log Q_{{\alpha_1+\alpha_2\over2},z}(\psi\|\ffi)
\le{1\over2}\bigl[2+\log Q_{\alpha_1,z}(\psi\|\ffi)+\log Q_{\alpha_2,z}(\psi\|\ffi)\bigr],
\]
which implies that $\alpha\mapsto\log Q_{\alpha,z}(\psi\|\ffi)$ is midpoint convex on $(0,1)$.
Since midpoint convexity implies convexity for continuous functions, (1) holds. Moreover, by
\cite[Theorem 1(vii)]{kato2023onrenyi} we find that
$\lim_{\alpha\nearrow1}Q_{\alpha,z}(\psi\|\ffi)\le\psi(1)$. Therefore, (1) implies (2) from the defining
formula of $D_{\alpha,z}$ in Definition \ref{defi:renyi}.\qed


\subsubsection{The second proof when $z>1/2$}\label{Sec-5.1.2}

Assume that $z>1/2$ and let $p:=2z$ and $q:={2z\over2z-1}$ so that $1/p+1/q=1$. Let
$\rho,\sigma\in\Me_*^+$ be such that $s(\rho)=\1-s(\psi)$ and $s(\sigma)=\1-s(\ffi)$. Put
$\psi_0:=\psi+\rho$ and $\ffi_0:=\ffi+\sigma$, which are faithful elements in $\Me_*^+$. Below, for each
$p\in(1,\infty)$ and $\eta\in(0,1)$, we use for simplicity the notations
\[
L_{p,L}:=L_p(\Me,\ffi_0)_L,\qquad L_{p,R}:=L_p(\Me,\psi_0)_R,\qquad
L_{p,\eta}:=L_p(\Me,\psi_0,\ffi_0)_\eta,
\]
(see \eqref{F-C.1}--\eqref{F-C.3} in Appendix \ref{Appen-Kosaki-Lp}). Then, by
\cite[Theorem 11.1]{kosaki1984applications} (also see \eqref{F-C.6}) note that
\begin{align}\label{F-5.7}
L_{p,\eta}=C_\eta(L_{p,L},L_{p,R})
\end{align}
with equal norms. We divide the second proof into the cases $z\ge1$ and $1/2<z<1$.

\bigskip\noindent
{\bf The second proof of Theorem \ref{T-5.1} when $z\ge1$.}\enspace
Assume that $z\ge1$, and put $h_0:=h_\psi^{1/2}h_\ffi^{1/2}\in L_1(\Me)$. For each $\alpha\in(0,1)$ put
$\eta:={z-\alpha\over2z-1}$; then we have $0<\eta<1$ as
\[
0\le1-{q\over2}={z-1\over2z-1}<\eta<{z\over2z-1}={q\over2}\le1.
\]
Since
\[
h_0=h_\psi^{\eta\over q}\Bigl(h_\psi^{\alpha\over2z}h_\ffi^{1-\alpha\over2z}\Bigr)h_\ffi^{1-\eta\over q}
=h_{\psi_0}^{\eta\over q}\Bigl(h_\psi^{\alpha\over2z}h_\ffi^{1-\alpha\over2z}\Bigr)h_{\ffi_0}^{1-\eta\over q},
\]
we have $h_0\in L_{p,\eta}$ and
\begin{align}\label{F-5.8}
\|h_0\|_{p,\psi_0,\ffi_0,\eta}^p=\Big\|h_\psi^{\alpha\over2z}h_\ffi^{1-\alpha\over2z}\Big\|_p^p
=\Big\|h_\psi^{\alpha\over2z}h_\ffi^{1-\alpha\over2z}\Big\|_{2z}^{2z}=Q_{\alpha,z}(\psi\|\ffi).
\end{align}
Now let $\alpha_1,\alpha_2\in(0,1)$ and for each $\theta\in(0,1)$ let
$\alpha:=(1-\theta)\alpha_1+\theta\alpha_2$. Put $\eta_j:={z-\alpha_j\over2z-1}$, $j=1,2$, so that
$\eta:={z-\alpha\over2z-1}=(1-\theta)\eta_1+\theta\eta_2$. To show (1), it suffices to prove that
\begin{align}\label{F-5.9}
Q_{\alpha,z}(\psi\|\ffi)\le Q_{\alpha_1,z}(\psi\|\ffi)^{1-\theta}Q_{\alpha_2,z}(\psi\|\ffi)^\theta.
\end{align}
From the complex interpolation space in \eqref{F-5.7} and the reiteration theorem \cite{cwikel1978complex},
we have
\begin{align}\label{F-5.10}
L_{p,\eta}=C_\theta(L_{p,\eta_1},L_{p,\eta_2}).
\end{align}
Since $h_0\in L_{p,\eta_1}\cap L_{p,\eta_2}$ as shown above, it follows that
\begin{align}\label{F-5.11}
\|h_0\|_{p,\psi_0,\ffi_0,\eta}\le\|h_0\|_{p,\psi_0,\ffi_0,\eta_1}^{1-\theta}\|h_0\|_{p,\psi_0,\ffi_0,\eta_2}^\theta.
\end{align}
(Indeed, this is a special case of the Riesz--Thorin theorem applied to the map $T(z):=zh_0$, $z\in\bC$.)
From \eqref{F-5.11} and \eqref{F-5.8} (for $\eta,\eta_1,\eta_2$) we have \eqref{F-5.9}, so that (1) has been
shown. Moreover, (1) immediately implies (2) as in the last part of the first proof of Theorem \ref{T-5.1}.
{\color{red}[I don't see how to modify Remark 2 in your notes (Jan.\ 10, 2024) in the case $\alpha<1$. On
the other hand, (2) is immediate from (1) in this case, as in the last part of the first proof of Th.~\ref{T-5.1}.]}
\qed

\bigskip
Next we turn to the case $1/2<z<1$. Here we will need a bit more of the complex interpolation method.
Let us denote $\Sigma=\Sigma(L_{p,L},L_{p,R}):=L_{p,L}+L_{p,R}$, and let $\cF'(L_{p,L},L_{p,R})$ be
the set of functions $f:S:=\{w\in\bC:0\le\Re w\le1\}\to\Sigma$ satisfying
\begin{itemize}
\item[(i)] $f$ is bounded, continuous on $S$ and analytic in the interior of $S$ (with respect to the norm
in $\Sigma$),
\item[(ii)] $f(it)\in L_{p,L}$ and $f(1+it)\in L_{p,R}$ for all $t\in\bR$,
\item[(iii)] the maps $t\in\bR\mapsto f(it)\in L_{p,L}$ and $t\in\bR\mapsto f(1+it)\in L_{p,R}$ are continuous
and
\[
\max\biggl\{\sup_{t\in\bR}\|f(it)\|_{L_{p,L}},\sup_{t\in\bR}\|f(1+it)\|_{L_{p,R}}\biggr\}<\infty.
\]
\end{itemize}
(See \cite[Definition 1.4]{kosaki1984applications}.)

Consider the function $f:S\to L_1(\Me)$ defined by
\begin{align}\label{F-5.12}
f(w):=h_\psi^{{w\over q}+{1-w\over p}}h_\ffi^{{1-w\over q}+{w\over p}},\qquad w\in S.
\end{align}
The next lemma shows that $f$ has values in $\Sigma$.

\begin{lemma}\label{L-5.5}
We have $f\in\cF'(L_{p,L},L_{p,R})$. Moreover, for each $\eta\in(0,1)$ and $t\in\bR$,
$f(\eta+it)\in L_{p,\eta}$ and
\[
\|f(\eta+it)\|_{p,\ffi_0,\psi_0,\eta}^p=Q_{1-\eta,z}(\psi\|\ffi).
\]
\end{lemma}

\begin{proof}
For any $\eta\in[0,1]$ we have
\[
f(\eta+it)=h_\psi^{\eta\over q}h_\psi^{i({1\over q}-{1\over p})t}h_\psi^{1-\eta\over p}
h_\ffi^{\eta\over p}h_\ffi^{i({1\over p}-{1\over q})t}h_\ffi^{1-\eta\over q}
=h_{\psi_0}^{\eta\over q}\Bigl(h_{\psi_0}^{i({1\over q}-{1\over p})t}h_\psi^{1-\eta\over p}
h_\ffi^{\eta\over p}h_{\ffi_0}^{i({1\over p}-{1\over q})t}\Bigr)h_{\ffi_0}^{1-\eta\over q}.
\]
Recall \cite[Lemmas 10.1 and 10.2]{kosaki1984applications} that $h_{\psi_0}^{it}\cdot h_{\ffi_0}^{-it}$
defines a strongly continuous one-parameter group of isometries on $L_p(\Me)$ for every $p\in[1,\infty)$.
This implies properties (ii) and (iii) in the definition of $\cF'(L_{p,L},L_{p,R})$. Furthermore, for $\eta\in(0,1)$
we see that $f(\eta+it)\in L_{p,\eta}$ and
\[
\|f(\eta+it)\|_{p,\ffi_0,\psi_0,\eta}^p
=\Big\|h_{\psi_0}^{i({1\over q}-{1\over p})t}h_\psi^{1-\eta\over p}
h_\ffi^{\eta\over p}h_{\ffi_0}^{i({1\over p}-{1\over q})t}\Big\|_p^p
=\Big\|h_\psi^{1-\eta\over p}h_\ffi^{\eta\over p}\Big\|_p^p=Q_{1-\eta,z}(\psi\|\ffi).
\]
Since $L_{p,\eta}$ for each $\eta\in(0,1)$ is continuouly embedded in $\Sigma$, this implies that $f$ is
$\Sigma$-valued. Since the H\"older inequality gives
$\Big\|h_\psi^{1-\eta\over p}h_\ffi^{\eta\over p}\Big\|_p\le\psi(1)^{1-\eta\over p}\ffi(1)^{\eta\over p}$ for all
$\eta\in(0,1)$, $f$ is also bounded on $S$. Note that as a function with values in $L_1(\Me)$, $f$ is bounded,
continuous on $S$ and analytic in the interior. To confirm (i), we now prove (similarly to
\cite[Secs.~9.1 and 29.1]{calderon1964intermediate} that the continuity and analyticity properties also hold
in $\Sigma$. Let $\mu_0(w,t)$ and $\mu_1(w,t)$ be the Poisson kernels associated with $S$. We then have
\[
f(w)=\int_\bR f(it)\mu_0(w,t)\,dt+\int_\bR f(1+it)\mu_1(w,t)\,dt.
\]
The integrals are in $L_1(\Me)$, but since $t\mapsto f(it)\in L_{p,L}$ and $t\mapsto f(1+it)\in\L_{p,R}$ are
continuous and bounded in the respective norms, we see that the integrals also exist in $\Sigma$. Hence
the above equality holds in $\Sigma$ since $\Sigma$ is continuously embedded in $L_1(\Me)$. This shows
that $f:S\to\Sigma$ is continuous. Therefore, for any $w$, $0<\Re w<1$, the expression
\[
{1\over2\pi i}\int_\Gamma{f(\xi)\over\xi-w}\,d\xi
\]
for a suitable circle $\Gamma$ around $w$ in the interior of $S$ is defined in $\Sigma$. Since $f$ is
analytic in $L_1(\Me)$, the expression is equal to $f(w)$, which shows that $f$ is analytic in $\Sigma$ in
the interior of $S$.
\end{proof}

\noindent
{\bf The second proof of Theorem \ref{T-5.1} when $1/2<z<1$.}\enspace
Let $\alpha_1,\alpha_2\in(0,1)$ and for each $\theta\in(0,1)$ let $\alpha:=(1-\theta)\alpha_1+\theta\alpha_2$.
Put $\eta_j:=1-\alpha_j$, $j=1,2$, so that $\eta:=1-\alpha=(1-\theta)\eta_1+\theta\eta_2$. With the function
$f$ given in \eqref{F-5.12} define $f_1(w):=f((1-w)\eta_1+w\eta_2)$, which belongs to the set
$\cF(L_{p,\eta_1},L_{p,\eta_2})$ (see \cite[Definition 1.1]{kosaki1984applications}) by Lemma \ref{L-5.5}
and \eqref{F-C.6}. Since $L_{p,\eta}=C_\theta(L_{p,\eta_1},L_{p,\eta_2})$ by the iteration theorem, we have
by usual arguments
\[
\|f(\eta)\|_{p,\ffi_0,\psi_0,\eta}=\|f_1(\theta)\|_{C_\theta(L_{p,\eta_1},L_{p,\eta_2})}
\le\biggl(\sup_{t\bR}\|f_1(it)\|_{L_{p,\eta_1}}\biggr)^{1-\theta}
\biggl(\sup_{t\in\bR}\|f_1(1+it)\|_{L_{p,\eta_2}}\biggr)^\theta.
\]
Since $f_1(it)=f(\eta_1+i(\eta_2-\eta_1)t)$ and $f_1(1+it)=f(\eta_2+i(\eta_2-\eta_1)t)$, it follows from
Lemma \ref{L-5.5} that
\[
Q_{1-\eta,z}(\psi\|\ffi)\le Q_{1-\eta_1,z}(\psi\|\ffi)^{1-\theta}Q_{1-\eta_2,z}(\psi\|\ffi)^\theta.
\]
This shows (1), which implies (2) as in the previous proof.
%Furthermore, since $f(it)\in L_{p,L}$ and $\|f(it)\|_{L_{p,L}}^p=\psi(1)$, (2) is seen as in the above case.
\qed


\subsection{The case $1<\alpha\le2z$}

In this subsection let us show monotonicity of $D_{\alpha,z}$ in the parameter $\alpha\in(1,2z]$ when $z>1/2$,
based on the complex interpolation as in Sec.~\ref{Sec-5.1.2}.

\begin{theorem}
Let $\psi,\ffi\in\Me_*^+$ and $z>1/2$. Then we have
\begin{itemize}
\item[(1)] $\alpha\mapsto\log Q_{\alpha,z}(\psi\|\ffi)$ is convex on $(1,2z]$,
\item[(2)] $\alpha\mapsto D_{\alpha,z}(\psi\|\ffi)$ is monotone increasing on $(1,2z]$.
\end{itemize}
\end{theorem}

\begin{proof}
Assume that $z>1/2$ and let $p$, $q$, $\psi_0$ and $\ffi_0$ be defined in the same way as in the beginning
of Sec.~\ref{Sec-5.1.2}. For each $\alpha\in(1,2z]$ put $\eta:={2z-\alpha\over2z-1}\in[0,1)$. Assume that
$Q_{\alpha,z}(\psi\|\ffi)<\infty$ (hence $s(\psi)\le s(\ffi)$), so that there exists a unique
$y\in s(\psi)L^p(\Me)s(\ffi)$ such that $h_\psi^{\alpha\over2z}=yh_\ffi^{\alpha-1\over2z}$. Since
\[
h_\psi=h_\psi^{2z-\alpha\over2z}yh_\ffi^{\alpha-1\over2z}
=h_{\psi_0}^{\eta\over q}yh_{\ffi_0}^{1-\eta\over q},
\]
we have $h_\psi\in L_{p,\eta}$ and
\[
\|h_\psi\|_{p,\psi_0,\ffi_0,\eta}^p=\|y\|_p^p=Q_{\alpha,z}(\psi\|\ffi),
\]
where for $\eta-0$ ($\alpha=2z$) the left-hand side is $\|h_\psi\|_{L_{p,L}}^p$. Now let
$\alpha_1,\alpha_2\in(1,2z]$ and $\alpha=(1-\theta)\alpha_1+\theta\alpha_2$ for any $\theta\in(0,1)$.
Put $\eta_j:={2z-\alpha_j\over2z-1}$, $j=1,2$, and $\eta:={2z-\alpha\over2z-1}=(1-\theta)\eta+\theta\eta_1$.
To show (1), it suffices to prove that \eqref{F-5.9} holds in the present situation. For this, we may assume
that $Q_{\alpha_j,z}(\psi\|\ffi)<\infty$, $j=1,2$. Then we can use \eqref{F-5.10} similarly to the proof in
Sec.~\ref{Sec-5.1.2} with $h_\psi$ instead of $h_0$. Hence we have \eqref{F-5.9}, and (1) holds.

As for (2), note that $h_\psi=h_{\psi_0}^{1/q}h_\psi^{1/p}\in K_{p,R}^p$ (see \eqref{F-C.3}) and
$\|h_\psi\|_{L_{p,R}}^p=\|h_\psi^{1/p}\|_p^p=\psi(1)$. Assume that $1<\alpha<\alpha_1\le2z$ and
$Q_{\alpha_1,z}(\psi\|\ffi)<\infty$, so that $\alpha=(1-\theta)\alpha_1+\theta$ for some $\theta\in(0,1)$.
Let $\eta:={2z-\alpha\over2z-1}$ and $\eta_1:={2z-\alpha_1\over2z-1}$. Since
\[
L_{p,\eta}=C_\theta(L_{p,\eta_1},L_{p,R})
\]
by the reiteration theorem, it follows that
\[
Q_{\alpha,z}(\psi\|\ffi)\le Q_{\alpha_1,z}(\psi\|\ffi)^{1-\theta}\psi(1)^\theta.
\]
Taking the logarithm and noting $\theta={\alpha_1-\alpha\over\alpha_1-1}$, we obtain
$D_{\alpha,z}(\psi\|\ffi)\le D_{\alpha_1,z}(\psi\|\ffi)$, proving (2).
\end{proof}


\subsection{Limits as $\alpha\nearrow1$ and $\alpha\searrow1$}

The aim of this last subsection is to show the limits of $D_{\alpha,z}$ as $\alpha\nearrow1$ and
$\alpha\searrow1$, extending the limits in \eqref{F-4.1} and \eqref{F-4.2}.

\begin{theorem}
Let $\psi,\ffi\in\Me_*^+$, $\psi\ne0$. For every $z>0$ we have
\[
\lim_{\alpha\nearrow1}D_{\alpha,z}(\psi\|\ffi)=D_1(\psi\|\ffi).
\]
\end{theorem}

\begin{proof}
Assume first that $z\in(0,1]$ and $0\le1-z<\alpha<1$. Let $\beta:={\alpha-1+z\over z}$; then
$0<\beta<1$ and $\beta\nearrow1$ as $\alpha\nearrow1$. Hence the result follows from Lemma \ref{L-5.8}
below and \eqref{F-4.14} for $D_{\alpha,1}$. On the other hand, for the case $z\in[1,\infty)$ the result
follows from Theorem \ref{T-4.7} as \eqref{F-4.14} does from Proposition \ref{P-4.4}.
\end{proof}

\begin{lemma}\label{L-5.8}
Assume that $z\in(0,1]$ and $0\le1-z<\alpha<1$. Let $\beta:={\alpha-1+z\over z}$. Then for any
$\psi,\ffi\in\Me_*^+$, $\psi\ne0$,
\[
D_{\beta,1}(\psi\|\ffi)\le D_{\alpha,z}(\psi\|\ffi)\le D_{\alpha,1}(\psi\|\ffi).
\]
\end{lemma}

\begin{proof}
Since the statement is trivial for $z=1$, we may assume that $z\in(0,1)$. The second inequality follows
from Theorem \ref{T-4.7}(1). For the first inequality, noting that $\beta\in(0,1)$ by assumption and usnig
by the H\"older inequality with ${1\over2z}={1-z\over2z}+{1\over2}$, we have
\begin{align*}
Q_{\alpha,z}(\psi\|\ffi)&=\Big\|h_\psi^{\alpha\over2z}h_\ffi^{1-\alpha\over2z}\Big\|_{2z}^{2z}
=\Big\|h_\psi^{1-z\over2z}h_\psi^{\beta\over2}h_\ffi^{1-\beta\over2}\Big\|_{2z}^{2z} \\
&\le\Big\|h_\psi^{1-z\over2z}\Big\|_{2z\over1-z}^{2z}
\Big\|h_\psi^{\beta\over2}h_\ffi^{1-\beta\over2}\Big\|_2^{2z}
=\psi(1)^{1-z}Q_{\beta,1}(\psi\|\ffi)^z,
\end{align*}
which proves the second inequality since $\alpha-1=z(\beta-1)$.
\end{proof}

\begin{theorem}\label{T-5.9}
Let $\psi,\ffi\in\Me_*^+$, $\psi\ne0$, and $z>1/2$. Assume that $D_{\alpha,z}(\psi\|\ffi)<\infty$ for some
$\alpha\in(1,2z]$. Then we have
\[
\lim_{\alpha\searrow1}D_{\alpha,z}(\psi\|\ffi)=D_1(\psi\|\ffi).
\]
\end{theorem}

\begin{proof}
Assume that $z>1/2$ and $D_{\alpha,z}(\psi\|\ffi)<\infty$ for some $\alpha\in(1,2z]$. We may assume that
$\ffi$ is faithful. We utilize the function $f$ on $S$ given in \eqref{F-4.18}, whose values are in $L_{2z,L}$
as seen from the proof of Theorem \ref{T-4.8}. Since $f$ is analytic in a neighborhood of $1/\alpha$, we have
the expansion
\[
f(w)=f\biggl({1\over\alpha}\biggr)+\biggl(w-{1\over\alpha}\biggr)h+o\biggl(w-{1\over\alpha}\biggr),
\]
where $h\in L_{2z,L}$ is the derivative of $f$ at $w=1/\alpha$ and $\|o(\zeta)\|_{L_{2z,L}}/|\zeta|\to0$ as
$|\zeta|\to0$. For each $\alpha'\in(1,\alpha)$ it follows that
\[
f\biggl({\alpha'\over\alpha}\biggr)=f\biggl({1\over\alpha}\biggr)+{\alpha'-1\over\alpha}\,h
+o\biggl({\alpha'-1\over\alpha}\biggr)\quad\mbox{as $\alpha'\searrow1$}.
\]
Furthermore, as in the proof of Theorem \ref{T-4.8}, we have
$f(\alpha'/\alpha)=h_\psi^{\alpha'\over2z}h_\ffi^{1-{\alpha'\over2z}}=y'h_\ffi^{2z-1\over2z}$ for some
$y'\in L_{2z}(\Me)$, so that $Q_{\alpha'z}(\psi\|\ffi)=\|y\|_{2z}^{2z}=\|f(\alpha'/\alpha)\|_{L_{2z,L}}$.

Now let us recall that $L_{2z,L}$ is uniformly convex thanks to $2z>1$ (see \cite{haagerup1979lpspaces},
\cite[Theorem 4.2]{kosaki1984applications}), so that the norm $\|\cdot\|_{2z,L}$ is uniformly Fr\'echet
differentiable (see, e.g., \cite[Part 3, Chap.~II]{beauzamy1982introduction}). {\color{red}We set
$a_0\in L_{{2z\over2z-1},L}$ with the unit norm by
\[
a_0:=\biggl({h_\psi\over\psi(1)}\biggr)^{2z-1\over2z}h_\ffi^{1\over2z},
\]
so that $\<a_0,f(1/\alpha)\>=\|f(1/\alpha)\|_{L_{2z,L}}$, where the dual pairing of $L_{{2z\over2z-1},L}$
and $L_{2z,L}$ is given in \eqref{F-C.5} in Appendix \ref{Appen-Kosaki-Lp} with $p={2z\over2z-1}$.} Then
the uniform Fr\'echet differentiability of $\|\cdot\|_{2z,L}$ at $1/\alpha$ implies that
\begin{align}\label{F-5.13}
\<a_0,f(1/\alpha)\>
=\lim_{\alpha'\searrow\1}{\|f(\alpha'/\alpha)\|_{L_{2z,L}}-\|f(1/\alpha)\|_{L_{2z,L}}
\over{\alpha'-1\over\alpha}}
\end{align}
and also
\begin{align}
\<a_0,f(1/\alpha)\>
&=\lim_{t\to0}{1\over it}\<a_0,f((1/\alpha)+it)-f(1/\alpha)\> \nonumber\\
&=\psi(\1)^{-{2z-1\over2z}}\lim_{t\to0}{1\over it}\Bigl\<h_\psi^{2z-1\over2z}h_\ffi^{1\over2z},
h_\psi^{1\over2z}\Bigl(h_{\psi_0}^{{\alpha\over2z}it}h_\ffi^{-{\alpha\over2z}it}-\1\Bigr)
h_\ffi^{2z-1\over2z}\Bigr\> \nonumber\\
&=\psi(\1)^{-{2z-1\over2z}}\lim_{t\to0}{1\over it}\Tr\Bigl[h_\psi^{2z-1\over2z}h_\psi^{1\over2z}
\bigl(h_{\psi_0}^{{\alpha\over2z}it}h_\ffi^{-{\alpha\over2z}it}-1\bigr)\Bigr] \nonumber\\
&=\psi(\1)^{-{2z-1\over2z}}{\alpha\over2z}
\lim_{t\to0}\Tr\bigl[h_\psi\bigl(h_{\psi_0}^{it}h_\ffi^{-it}-\1\bigr)\bigr] \nonumber\\
&=\psi(\1)^{-{2z-1\over2z}}{\alpha\over2z}\,D(\psi\|\ffi), \label{F-5.14}
\end{align}
where we have used {\color{red}\eqref{F-C.5} for the third equality and} \cite[Theorem 5.7]{ohya1993quantum}
for the last equality. Since $\psi(\1)=\|f(1/\alpha)\|_{L_{2z,L}}^{2z}$, it follows from \eqref{F-5.13} and
\eqref{F-5.14} that
\begin{align*}
D_{\alpha',z}(\psi\|\ffi)
&={\log Q_{\alpha',z}(\psi\|\ffi)-\log\psi(\1)\over\alpha'-1}
={2z\log\|f(\alpha'/\alpha)\|_{L_{2z,L}}-2z\log\|f(1/\alpha)\|_{L_{2z,L}}\over\alpha'-1} \\
&=\biggl({\log\|f(\alpha'/\alpha)\|_{L_{2z,L}}-\log\|f(1/\alpha)\|_{L_{2z,L}}\over
\|f(\alpha'/\alpha)\|_{L_{2z,L}}-\|f(1/\alpha)\|_{L_{2z,L}}}\biggr)
{2z\over\alpha}\biggl({\|f(\alpha'/\alpha)\|_{L_{2z,L}}-\|f(1/\alpha)\|_{L_{2z,L}}\over
{\alpha'-1\over\alpha}}\biggr) \\
&\to{1\over\psi(\1)^{1\over2z}}\,{2z\over\alpha}\,\psi(\1)^{-{2z-1\over2z}}{\alpha\over2z}\,D(\psi\|\ffi)
={D(\psi\|\ffi)\over\psi(\1)}=D_1(\psi\|\ffi)
\end{align*}
as $\alpha\searrow1$, as desired.
\end{proof}

\section{Concluding remarks}

////////////////////////


\subsection*{Acknowledgments}

////////////////////////


\appendix

\section{Haagerup  $L_p$-spaces}\label{app:lp}

{\color{red}
Let $\cR:=\Me\rtimes_{\sigma^\omega}\bR$ be the crossed product of $\Me$ by the
\emph{modular automorphism group} $\sigma_t^\omega$, $t\in\bR$, for a faithful normal semi-finite weight
$\omega$ on $\Me$. Then $\cR$ is a semi-finite von Neumann algebra with the canonical trace $\tau$.
Let $\theta_s$, $s\in\bR$, be the \emph{dual action} on $\cR$ having the $\tau$-scaling property
$\tau\circ\theta_s=e^{-s}\tau$, $s\in\bR$; see \cite[Chap.~X]{takesaki2003theoryof} (also
\cite[Chap.~8]{hiai2021lectures}). Let $\widetilde\cR$ denote the space of $\tau$-measurable operators
affiliated with $\cR$; see \cite{fack1986generalized} (also \cite[Chap.~4]{hiai2021lectures}). For $0<p\le\infty$,
the \emph{Haagerup $L_p$-space} $L_p(\Me)$ \cite{haagerup1979lpspaces,terp1981lpspaces} (also
\cite[Chap.~9]{hiai2021lectures}) is defined by
\[
L_p(\Me):=\{a\in\widetilde\cR:\theta_s(a)=e^{-s/p}a,\ s\in\bR\}.
\]
In particular, $\Me=L_\infty(\Me)$ and we have an order isomorphism $\Me_*\cong L_1(\Me)$ given as
$\psi\in\Me_*\leftrightarrow h_\psi\in L_1(\Me)$, so that $\Tr\,h_\psi=\psi(1)$, $\psi\in L_1(\Me)$, defines a
positive linear functional $\Tr$ on $L_1(\Me)$. For $0<p<\infty$ the $L_p$-norm (quasi-norm for $0<p<1$)
of $a\in L_p(\Me)$ is defined by $\|a\|_p:=(\Tr\,|a|^p)^{1/p}$, and the $L_\infty$-norm $\|\cdot\|_\infty$ is the
operator norm $\|\cdot\|$ on $\Me$. For $1\le p<\infty$, $L_p(\Me)$ is a Banach space whose dual Banach
space is $L^q(\Me)$, where $1/p+1/q=1$, by the duality pairing
\begin{align}\label{F-A.1}
(a,b)\in L_p(\Me)\times L_q(\Me)\mapsto\Tr(ab)=\Tr(ba).
\end{align}
}

The following lemmas are well known, proofs are given for completeness.

\begin{lemma}\label{lemma:cone}
For any $0<p<\infty$ and $\varphi\in \Me_*^+$, 
$h_\varphi^{\frac1{2p}}\Me^+h_\varphi^{\frac1{2p}}$ is dense in $L_p(\Me)^+$ with respect
to the (quasi)-norm $\|\cdot\|_p$.
\end{lemma}

\begin{proof} We may assume that $\varphi$ is faithful. By \cite[Lemma 1.1]{junge2003noncommutative}, $\Me
h_\varphi^{\frac1{2p}}$ is dense in $L_{2p}(\Me)$ for any $0<p<\infty$. Let $y\in L_p(\Me)^+$, then
$y^{\frac12}\in L_{2p}(\Me)$, hence there is a sequence $a_n\in \Me$ such that
$\|a_nh^{\frac1{2p}}_\varphi-y^{\frac12}\|_{2p}\to 0$. Then also 
\[
\Big\|h^{\frac1{2p}}_\varphi a_n^*-y^{\frac12}\Big\|_p
=\Big\|(a_nh^{\frac1{2p}}_\varphi-y^{\frac12})^*\Big\|_p
=\Big\|a_nh^{\frac1{2p}}_\varphi-y^{\frac12}\Big\|_p\to 0
\]
and 
\[
\Big\|h^{\frac1{2p}}_\varphi a_n^*a_nh^{\frac1{2p}}_\varphi-y\Big\|_p
=\Big\|(h^{\frac1{2p}}_\varphi a_n^*-y^{\frac12})a_nh^{\frac1{2p}}_\varphi
+y^{\frac12}(a_nh^{\frac1{2p}}_\varphi-y^{\frac12})\Big\|_p.
\]
Since $\|\cdot\|_p$ is a (quasi)-norm, the above expression goes to 0 by the H\"older
inequality.
\end{proof}


\begin{lemma}\label{lemma:order1}
Let $0<p\le \infty$ and let $h,k\in L_p(\Me)^+$ be such
that $h\le k$. Then 
$\|h\|_p\le \|k\|_p$. Moreover, if $1\le p<\infty$, then 
\[
\|k-h\|_p^p\le \|k\|_p^p-\|h\|_p^p.
\]
\end{lemma}

\begin{proof} The first statement follows from \cite[Lemmas 2.5(iii) and 4.8]{fack1986generalized}.
The second statement is from \cite[Lemma 5.1]{fack1986generalized}.
\end{proof}

\begin{lemma}\label{lemma:order}
Let $\psi,\varphi\in \Me_*^+$ with $\psi\le \varphi$.
Then for any $a\in \Me$ and $p\in [1,\infty)$,
\[
\Tr\Bigl(\bigl(a^*h_\psi^{1/p}a\bigr)^p\Bigr)\le\Tr\Bigl(\bigl(a^*h_\varphi^{1/p}a\bigr)^p\Bigr).
\]
\end{lemma}

\begin{proof} Since $1/p\in ({\color{red}0},1]$, it follows (see \cite[Lemma B.7]{hiai2021quantum} and
\cite[Lemma 3.2]{hiai2021connections}) that $h_\psi^{1/p}\le h_\varphi^{1/p}$.  Hence
$a^*h_\psi^{1/p}a\le a^*h_\varphi^{1/p}a$. Therefore, by Lemma \ref{lemma:order1}, 
we have the statement.
\end{proof}


\section{Haagerup's reduction theorem}\label{Appen-reduction}

In this appendix let us recall Haagerup's reduction theorem, which was presented in
\cite[Sec.~2]{haagerup2010areduction} (a compact survey is also found in
\cite[Sec.~2.5]{fawzi2023asymptotic}). Let $\Me$ be a general $\sigma$-finite von Neumann algebra.
Let $\omega$ be a faithful normal state of $\Me$ and $\sigma_t^\omega$ ($t\in\bR$) be the associated
modular automorphism group. Consider the discrete additive group $G:=\bigcup_{n\in\bN}2^{-n}\bZ$ and
define $\hat\Me:=\Me\rtimes_{\sigma^\omega}G$, the crossed product of $\Me$ by the action
$\sigma^\omega|_G$. Then the dual weight $\hat\omega$ is a faithful normal state of $\hat\Me$, and
we have $\hat\omega=\omega\circ E_\Me$, where $E_\Me:\hat\Me\to\Me$ is the canonical conditional
expectation (see, e.g., \cite[Sec.~8.1]{hiai2021lectures}, also \cite[Sec.~2.5]{fawzi2023asymptotic}).

Haagerup's reduction theorem is summarized as follows:

\begin{theorem}[\cite{haagerup2010areduction}]\label{T-B.1}
In the above setting, there exists an increasing sequence $\{\Me_n\}_{n\ge1}$ of von Neumann
subalgebras of $\hat\Me$, containing the unit of $\hat\Me$, such that the following hold:
\begin{itemize}
\item[(i)] Each $\Me_n$ is finite with a faithful normal tracial state $\tau_n$.
\item[(ii)] $\left(\bigcup_{n\ge1}\Me_n\right)''=\hat\Me$.
\item[(iii)] For every $n$ there exist a (unique) faithful normal conditional expectation
$E_{\Me_n}:\hat\Me\to\Me_n$ satisfying
\[
\hat\omega\circ E_{\Me_n}=\hat\omega,\qquad
\sigma_t^{\hat\omega}\circ E_{\Me_n}=E_{\Me_n}\circ\sigma_t^{\hat\omega},\quad t\in\bR.
\]
Moreover, for any $x\in\hat\Me$, $E_{\Me_n}(x)\to x$ in the $\sigma$-strong topology.
\end{itemize}
\end{theorem}

Furthermore, for any $\psi\in\Me_*^+$, if we define $\hat\psi:=\psi\circ E_\Me$ then
$\hat\psi\circ E_{\Me_n}\to\hat\psi$ in the norm, as seen from \cite[Theorem 4]{hiai1984strong}.
{\color{red}[It is written in \cite[Sec.~2.5]{fawzi2023asymptotic} that this was proved in
\cite[Theorem 3.1]{haagerup2010areduction}. However it is not clear to me.]}

\section{Kosaki's interpolation $L_p$-spaces}\label{Appen-Kosaki-Lp}

Assume that $\Me$ is a $\sigma$-finite von Neumann algebra and let faithful
$\psi_0,\ffi_0\in\Me_*^+$ be given. For each $\eta\in[0,1]$ consider an embedding
$\Me\hookrightarrow L^1(\Me)$ by $x\mapsto h_{\psi_0}^\eta xh_{\ffi_0}^{1-\eta}$. Defining
$\|h_{\psi_0}^\eta xh_{\ffi_0}^{1-\eta}\|_\infty:=\|x\|$ (the operator norm of $x$) on
$h_{\psi_0}^\eta\Me h_{\ffi_0}^{1-\eta}$ we have a pair
$\bigl(h_{\psi_0}^\eta\Me h_{\ffi_0}^{1-\eta},L^1(\Me)\bigr)$ of compatible Banach spaces (see, e.g.,
\cite{bergh1976interpolation}). For $1<p<\infty$, \emph{Kosaki's interpolation $L_p$-space}
with respect to $\psi_0,\ffi_0$ and $\eta$ \cite{kosaki1984applications} (also see
\cite[Sec.~9.3]{hiai2021lectures} for a compact survey) is defined as the complex interpolation
Banach space:
\begin{align}\label{F-C.1}
L_p(\Me,\psi_0,\ffi_0)_\eta:=C_{1/p}\bigl(h_{\psi_0}^\eta\Me h_{\ffi_0}^{1-\eta},L^1(\Me)\bigr)
\end{align}
equipped with the interpolation norm $\|\cdot\|_{p,\psi_0,\ffi_0,\eta}:=\|\cdot\|_{C_{1/p}}$. Then,
 Kosaki's theorem \cite[Theorem 9.1]{kosaki1984applications} says that for every $\eta\in[0,1]$ and
 $p\in(1,\infty)$ with $1/p+1/q=1$,
\begin{align*}
&L_p(\Me,\psi_0,\ffi_0)_\eta=h_{\psi_0}^{\eta/q}L_p(\Me)h_{\ffi_0}^{(1-\eta)/q}\ (\subset L_1(\Me)), \\
&\|h_{\psi_0}^{\eta/q}ah_{\ffi_0}^{(1-\eta)/q}\|_{p,\psi_0,\ffi_0,\eta}=\|a\|_p,\qquad a\in L^p(\Me),
\end{align*}
that is, $L_p(\Me)\cong L_p(\Me,\psi_0,\ffi_0)_\eta$ by the isometry
$a\mapsto h_{\psi_0}^{\eta/q}ah_{\ffi_0}^{(1-\eta)/q}$. {\color{red}In the main body of this paper we use the
special cases where $\eta=0,1$, that is,}
\begin{align}
L_p(\Me,\ffi_0)_L&:=C_{1/p}\bigl(\Me h_{\ffi_0},L_1(\Me)\bigr)=L_p(\Me)h_{\ffi_0}^{1/q},
\label{F-C.2}\\
L_p(\Me,\psi_0)_R&:=C_{1/p}\bigl(h_{\psi_0}\Me,L_1(\Me)\bigr)=h_{\psi_0}^{1/q}L_p(\Me),
\label{F-C.3}
\end{align}
which are called Kosaki's \emph{left and right $L_p$-spaces}, respectively. {\color{red}Another special case
we use is the \emph{symmetric $L_p$-space} $L_p(\Me,\ffi_0)$ where $\eta=1/2$ and $\psi_0=\ffi_0$, i.e.,
\begin{align}\label{F-C.4}
L_p(\Me,\ffi_0)=C_{1/p}\bigl(h_{\ffi_0}^{1/2}\Me h_{\ffi_0}^{1/2},L_1(\Me)\bigr)
=h_{\ffi_0}^{1/2q}L_p(\Me)h_{\ffi_0}^{1/2q},
\end{align}
whose interpolation norm is denoted by $\|\cdot\|_{p,\ffi_0}$. The $L_p$-$L_q$ duality of Kosaki's
$L_p$-spaces can be given by transforming the duality paring in \eqref{F-A.1}; in particular,
the duality pairing between $L_p(\Me,\ffi_0)_L$ and $L_q(\Me,\ffi_0)_L$ for $1\le p<\infty$ and
$1/p+1/q=1$ is written as
\begin{align}\label{F-C.5}
\<ah_{\ffi_0}^{1,q},bh_{\ffi_0}^{1/p}\>=\Tr(ab),\qquad
a\in L_p(\Me),\ b\in L_q(\Me).
\end{align}}
Kosaki's non-commutative Stein--Weiss interpolation theorem \cite[Theorem 11.1]{kosaki1984applications}
says that for each $\eta\in(0,1)$ and $p\in(1,\infty)$, Kosaki's $L_p$-space $L_p(\Me,\psi_0,\ffi_0)_\eta$
given in \eqref{F-C.1} is the complex interpolation space of the left and right $L_p$-spaces in \eqref{F-C.2}
and \eqref{F-C.3} with equal norms, that is,
\begin{align}\label{F-C.6}
L_p(\Me,\psi_0,\ffi_0)_\eta
=C_{1/p}\bigl(h_{\psi_0}^\eta\Me h_{\ffi_0}^{1-\eta},L^1(\Me)\bigr)
=C_\eta(L_p(\Me,\ffi_0)_L,L_p(\Me,\psi_0)_R).
\end{align}

\addcontentsline{toc}{section}{References}

\begin{thebibliography}{27}
%\providecommand{\natexlab}[1]{#1}
\providecommand{\url}[1]{\texttt{#1}}
\expandafter\ifx\csname urlstyle\endcsname\relax
  \providecommand{\doi}[1]{doi: #1}\else
  \providecommand{\doi}{doi: \begingroup \urlstyle{rm}\Url}\fi

\bibitem[Accardi and Cecchini(1982)]{accardi1982conditional}
L.~Accardi and C.~Cecchini.
\newblock Conditional expectations in von {Neumann} algebras and a theorem of
  {T}akesaki.
\newblock \emph{Journal of Functional Analysis}, 45:\penalty0 245--273, 1982.
\newblock \doi{10.1016/0022-1236(82)90022-2}.

\bibitem[Beauzamy(1982)]{beauzamy1982introduction}
B.~Beauzamy.
\newblock \emph{Introduction to Banach Spaces and their Geometry}.
\newblock Mathematics Studies, 68, North-Holland, Amsterdam, 1982.

\bibitem[Bergh and L\"ofstr\"om(1976)]{bergh1976interpolation}
J.~Bergh and J.~L\"ofstr\"om.
\newblock \emph{Interpolation Spaces: An Introduction}.
\newblock Springer, Berlin-Heidelberg-New York, 1976.

\bibitem[Berta et~al.(2018)Berta, Scholz, and Tomamichel]{berta2018renyi}
M.~Berta, V.~B. Scholz, and M.~Tomamichel.
\newblock {R{\'e}nyi divergences as weighted non-commutative vector valued
  $L_p$-spaces}.
\newblock \emph{Annales Henri Poincar{\'e}}, 19:\penalty0 1843--1867, 2018.
\newblock \doi{https://doi.org/10.48550/arXiv.1608.05317}.

\bibitem[Calder\'on(1964)]{calderon1964intermediate}
A.~Calder\'on.
\newblock Intermediate spaces and interpolation, the complex method.
\newblock \emph{Studia Mathematics}, 24\penalty0 (2):\penalty0 113--190, 1964.

\bibitem[Choi(1974)]{choi1974aschwarz}
M.-D.~Choi.
\newblock {A Schwarz inequality for positive linear maps on $C$*-algebras}.
\newblock \emph{Illinois Journal of Mathematics}, 18\penalty0 (4):\penalty0
  565--574, 1974.
\newblock \doi{10.1215/ijm/1256051007}.

\bibitem[Cwikel(1978)]{cwikel1978complex}
M.~Cwikel.
\newblock Complex interpolation spaces, a discrete definition and reiteration.
\newblock \emph{Indiana University Mathematics Journal}, 27\penalty0 (6):\penalty 0
  1005--1009, 1978.

\bibitem[Fack and Kosaki(1986)]{fack1986generalized}
T.~Fack and H.~Kosaki.
\newblock {Generalized $s$-numbers of $\tau$-measurable operators.}
\newblock \emph{Pacific Journal of Mathematics}, 123\penalty0 (2):\penalty0 269
  -- 300, 1986.
  
\bibitem[Fawzi et~al.(2023)Fawzi, Gao, and Rahaman]{fawzi2023asymptotic}
O.~Fawzi, L.~Gao, and M.~Rahaman.
\newblock Asymptotic equipartition theorems in von Neumann algebras.
\newblock \emph{arXiv preprint arXiv:2212.14700v2 [quant-ph]}, 2023.

\bibitem[Haagerup(1979)]{haagerup1979lpspaces}
U.~Haagerup.
\newblock {$L_p$-spaces associated with an arbitrary von Neumann algebra}.
\newblock In \emph{Algebres d'op{\'e}rateurs et leurs applications en
  physique math{\'e}matique (Proc. Colloq., Marseille, 1977)}, volume 274,
  pages 175--184, 1979.
  
\bibitem[Haagerup et~al.(2010)]{haagerup2010areduction}
U.~Haagerup, M.~Junge, and Q.~Xu.
\newblock A reduction method for noncommutative $L_p$-spaces and applications.
\newblock \emph{Transactions of the American Mathematical Society}, 362\penalty0
  (4):\penalty0 2125--2165, 2010.

\bibitem[Hiai(2018)]{hiai2018quantum}
F.~Hiai.
\newblock {Quantum $f$-divergences in von Neumann algebras. I. Standard
  $f$-divergences}.
\newblock \emph{Journal of Mathematical Physics}, 59\penalty0 (10):\penalty0
  102202, 2018.

\bibitem[Hiai(2021{\natexlab{a}})]{hiai2021lectures}
F.~Hiai.
\newblock \emph{Lectures on Selected Topics in von Neumann Algebras}.
\newblock EMS Press, Berlin, 2021{\natexlab{a}}.
\newblock \doi{10.4171/ELM/32}.

\bibitem[Hiai(2021{\natexlab{b}})]{hiai2021quantum}
F.~Hiai.
\newblock \emph{Quantum $f$-Divergences in von Neumann Algebras: Reversibility
  of Quantum Operations}.
\newblock Mathematical Physics Studies. Springer, Singapore,
  2021{\natexlab{b}}.
\newblock ISBN 9789813341999.
\newblock \doi{10.1007/978-981-33-4199-9}.

\bibitem[Hiai(2024)]{hiai2024log-majorization}
F.~Hiai.
\newblock Log-majorization and matrix norm inequalities with application to quantum
information.
\newblock \emph{arXiv preprint arXiv:2402.16067}, 2024.

\bibitem[Hiai and Kosaki(2021)]{hiai2021connections}
F.~Hiai and H.~Kosaki.
\newblock Connections of unbounded operators and some related topics: von
  {N}eumann algebra case.
\newblock \emph{International Journal of Mathematics}, 32\penalty0
  (05):\penalty0 2150024, 2021.
\newblock \doi{10.1142/S0129167X21500245}.

\bibitem[Hiai and Tsukada(1984)]{hiai1984strong}
F.~Hiai and M.~Tsukada.
\newblock Strong martingale convergence of generalized conditional expectations
  on von {N}eumann algebras.
\newblock \emph{Transactions of the American Mathematical Society},
  282\penalty0 (2):\penalty0 791--798, 1984.
\newblock \doi{10.1090/S0002-9947-1984-0732120-1}.

\bibitem[Jen{\v c}ov{\'a}(2018)]{jencova2018renyi}
A.~Jen{\v c}ov{\'a}.
\newblock {R{\'e}nyi relative entropies and noncommutative $L_p$-spaces}.
\newblock \emph{Annales Henri Poincar{\'e}}, 19:\penalty0 2513--2542, 2018.
\newblock \doi{10.1007/s00023-018-0683-5}.

\bibitem[Jen{\v{c}}ov{\'a}(2021)]{jencova2021renyi}
A.~Jen{\v{c}}ov{\'a}.
\newblock {R{\'e}nyi relative entropies and noncommutative $L_p$-spaces II}.
\newblock \emph{Annales Henri Poincar{\'e}}, 22:\penalty0 3235–3254, 2021.
\newblock \doi{10.1007/s00023-021-01074-9}.

\bibitem[Junge and Xu(2003)]{junge2003noncommutative}
M.~Junge and Q.~Xu.
\newblock Noncommutative {B}urkholder/{R}osenthal inequalities.
\newblock \emph{The Annals of Probability}, 31\penalty0 (2):\penalty0 948--995,
  2003.

\bibitem[Kato(2023)]{kato2023onrenyi}
S.~Kato.
\newblock On $\alpha $-$ z $-{R}\'enyi divergence in the von {N}eumann algebra
  setting.
\newblock \emph{arXiv preprint arXiv:2311.01748}, 2023.

\bibitem[Kato and Ueda(2023)]{kato2023aremark}
S.~Kato and Y.~Ueda.
\newblock A remark on non-commutative {$L^p$}-spaces.
{\color{red}\newblock \emph{Studia Math.}, to appear.}
\newblock \emph{arXiv preprint arXiv:2307.01790}, 2023.

\bibitem[Kosaki(1984)]{kosaki1984applications}
H.~Kosaki.
\newblock {Applications of the complex interpolation method to a von Neumann
  algebra: Non-commutative $L_p$-spaces}.
\newblock \emph{J. Funct. Anal.}, {56}:\penalty0 26--78,
  {1984}{\natexlab{a}}.
\newblock \doi{https://doi.org/10.1016/0022-1236(84)90025-9}.

\bibitem[Kosaki(1984)]{kosaki1984applicationsuc}
H.~Kosaki.
\newblock {Applications of uniform convexity of noncommutative $L\sp{p}$-spaces}.
\newblock \emph{Trans. Amer. Math. Soc.}, {283}:\penalty0 265--282,
  {1984}{\natexlab{b}}.

\bibitem[Kosaki(1986)]{kosaki1986relative}
H.~Kosaki.
\newblock Relative entropy of states: A variational expression.
\newblock \emph{J. Operator Theory}, 16:\penalty335--348, 1986.

\bibitem[Kosaki(1992)]{kosaki1992aninequality}
H.~Kosaki.
\newblock {An inequality of Araki-Lieb-Thirring (von Neumann algebra case)}.
\newblock \emph{Proceedings of the American Mathematical Society}, 114\penalty0
  (2):\penalty0 477--481, 1992.
\newblock \doi{10.1090/S0002-9939-1992-1065951-1}.

\bibitem[Leditzky et~al.(2017)Leditzky, Rouz{\'e}, and Datta]{leditzky2017data}
F.~Leditzky, C.~Rouz{\'e}, and N.~Datta.
\newblock Data processing for the sandwiched {R}{\'e}nyi divergence: a
  condition for equality.
\newblock \emph{Letters in Mathematical Physics}, 107\penalty0 (1):\penalty0
  61--80, 2017.
\newblock \doi{10.1007/s11005-016-0896-9}.

\bibitem[Lin and Tomamichel(2023)]{lin2015investigating}
M.~S.~Lin and M.~Tomamichel.
\newblock Investigating properties of a family of quantum Renyi divergences.
\newblock \emph{Quantum Information Processing}, 14\penalty0
 (4):\penalty0 1501--1512, 2015.

\bibitem[Mosonyi and Hiai(2023)]{mosonyi2023somecontinuity}
M.~Mosonyi and F.~Hiai.
\newblock Some continuity properties of quantum {R}ényi divergences.
\newblock \emph{IEEE Transactions on Information Theory}, 2023.
\newblock \doi{10.1109/TIT.2023.3324758}.

\bibitem[Ohya and Petz(1993)]{ohya1993quantum}
M.~Ohya and D.~Petz.
\newblock \emph{Quantum Entropy and Its Use}.
\newblock Texts and Monographs in Physics, 2nd ed., Springer, Berlin, 2004.

\bibitem[Petz(1985)]{petz1985quasi}
D.~Petz.
\newblock Quasi-entropies for states of a von {N}eumann algebra.
\newblock \emph{Publications of the Research Institute for Mathematical
  Sciences}, 21\penalty0 (4):\penalty0 787--800, 1985.
\newblock \doi{10.2977/prims/1195178929}.

\bibitem[Petz(1986)]{petz1986sufficient}
D.~Petz.
\newblock Sufficient subalgebras and the relative entropy of states of a von
  {Neumann} algebra.
\newblock \emph{Communications in Mathematical Physics}, 105\penalty0
  (1):\penalty0 123--131, 1986.
\newblock \doi{10.1007/BF01212345}.

\bibitem[Petz(1988)]{petz1988sufficiency}
D.~Petz.
\newblock Sufficiency of channels over von {Neumann} algebras.
\newblock \emph{The Quarterly Journal of Mathematics}, 39\penalty0
  (1):\penalty0 97--108, 1988.
\newblock \doi{10.1093/qmath/39.1.97}.

\bibitem[Takesaki(2003)]{takesaki2003theoryof}
M.~Takesaki.
\newblock \emph{Theory of Operator Algebras II}.
\newblock Encyclopaedia of Mathematical Sciences, vol. 125, Springer, Berlin, 2003.

\bibitem[Terp({1981})]{terp1981lpspaces}
M.~Terp.
\newblock {$L_p$-spaces associated with von Neumann algebras}.
\newblock {Notes, Copenhagen University}, {1981}.

\bibitem[Zalinescu(2002)]{zalinescu2002convex}
C.~Zalinescu.
\newblock \emph{Convex Analysis in General Vector Spaces}.
\newblock World scientific, Singapore, 2002.

\bibitem[Zhang(2020{\natexlab{b}})]{zhang2020fromwyd}
H.~Zhang.
\newblock {From Wigner-Yanase-Dyson conjecture to Carlen-Frank-Lieb
  conjecture}.
\newblock \emph{Advances in Mathematics}, 365:\penalty0 107053,
  2020{\natexlab{b}}.
\newblock \doi{10.1016/j.aim.2020.107053}.

\bibitem[Zhang(2020{\natexlab{a}})]{zhang2020equality}
H.~Zhang.
\newblock Equality conditions of data processing inequality for $\alpha$-z
  {R}{\'e}nyi relative entropies.
\newblock \emph{Journal of Mathematical Physics}, 61\penalty0 (10),
  2020{\natexlab{a}}.
\newblock \doi{10.1063/5.0022787}.

\end{thebibliography}

%\bibliography{alphaz}
%\bibliographystyle{abbrvnat}

\end{document}

\end{document}
