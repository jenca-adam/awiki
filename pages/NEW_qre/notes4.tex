\documentclass[12pt]{article}

\usepackage{hyperref}
\usepackage{amsmath, amssymb, amsthm}
\usepackage[sort&compress,numbers]{natbib}
\usepackage{doi}
\usepackage[margin=0.8in]{geometry}
%\textheight23cm \topmargin-20mm  
%\textwidth175mm  
%\oddsidemargin=0mm
%\evensidemargin=0mm
%

\usepackage{amsmath, amssymb, amsthm, mathtools}

\newtheorem{lemma}{Lemma}
\newtheorem{prop}{Proposition}
\newtheorem{theorem}{Theorem}
\newtheorem{coro}{Corollary}


\theoremstyle{definition}
\newtheorem{defi}{Definition}


\theoremstyle{remark}
\newtheorem{remark}{Remark}

\def\Me{\mathcal M}
\def\Ee{\mathcal E}
\def\Ra{\mathcal R}
\def\Ne{\mathcal N}
\def \Tr{\mathrm{Tr}\,}
\def\Se {\mathcal S}
\def\supp{\mathrm{supp}}
\def\<{\langle\.}
\def\>{\.\rangle}

\title{DPI and sufficiency, $\alpha>1$}
\author{Anna Jen\v cov\'a}

\begin{document}

\maketitle

Throughout these notes, we will assume that $\alpha>1$ and $\max\{\alpha/2,\alpha-1\}\le
z\le \alpha$. Let $\psi,\varphi\in \Me_*^+$ be such that
$D_{\alpha,z}(\psi\|\varphi)<\infty$, we will also assume that $\varphi$ is faithul. 

We put $p:=\frac z{\alpha}$ and $q:=\frac z{\alpha-1}$, so that $1/2\le p\le 1\le q$. By the
assumptions, there is some unique $y\in L_{2z}(\Me)$ such that 
\begin{equation}\label{eq:y}
h_\psi^{\frac1{2p}}=yh_\varphi^{\frac1{2q}}.
\end{equation}
By \cite{kato2023onrenyi,SKnote}, we have the variational formula
\begin{align*}
Q_{\alpha,z}(\psi\|\varphi)&=\sup_{a\in \Me_+} \alpha
\Tr(h_\psi^{\frac1{2p}}ah_\psi^{\frac1{2p}})^p-(\alpha-1)\Tr(h_\varphi^{\frac1{2q}}ah_\varphi^{\frac1{2q}})^q\\
&=\sup_{a\in \Me_+} \alpha
\Tr(yh_\varphi^{\frac1{2q}}ah_\varphi^{\frac1{2q}}y^*)^p-(\alpha-1)\Tr(h_\varphi^{\frac1{2q}}ah_\varphi^{\frac1{2q}})^q\\
&=\sup_{w\in L_q(\Me)^+}\alpha\Tr(ywy^*)^p-(\alpha-1)\Tr w^q,
\end{align*}
this follows from the fact that $h_\varphi^{\frac1{2q}}\Me_+h_\varphi^{\frac1{2q}}$ is
dense in $L_q(\Me)^+$. The supremum is attained at a unique point $\bar
w=(y^*y)^{\alpha-1}\in L_1(\Me)^+$, uniqueness follows from strict concavity of the
function $w\mapsto \alpha\Tr(ywy^*)^p-(\alpha-1)\Tr w^q$. 

Let $\Phi:\Me_*\to \Ne_*$ be a 2-positive trace preserving map and let 
$\varphi_0:=\Phi(\varphi)$, $\psi_0:=\Phi(\psi)$. Assume that also $\varphi_0$ is faithful. Let $\Phi_\varphi:
\Ne_*\to\Me_*$ be the Petz dual of $\Phi$ with respect to $\varphi$, then we have
\[
\Phi(h_\varphi^{1/2}ah_\varphi^{1/2})=h_{\varphi_0}^{1/2}\Phi_\varphi^*(a)h_{\varphi_0}^{1/2},\qquad 
\Phi_\varphi(h_{\varphi_0}^{1/2}bh_{\varphi_0}^{1/2})=h_{\varphi}^{1/2}\Phi^*(b)h_{\varphi}^{1/2},
\qquad a\in \Me,\ b\in \Ne,
\]
here $\Phi^*:\Ne\to\Me$ and $\Phi^*_\varphi:\Me\to \Ne$ are the 2-positive unital normal
maps that are adjoints of $\Phi$ resp. $\Phi_\varphi$. More generally, since for any $r\ge
1$, $\Phi$ is a contraction $L_r(\Me,\varphi)$ to $L_r(\Ne,\varphi_0)$, and similarly for
$\Phi_\varphi$, there are positive contractions $\Phi_{r,\varphi}:L_r(\Me)\to L_r(\Ne)$
and $\Phi_{r,\varphi_0}:L_r(\Ne)\to L_r(\Me)$ such that 
\[
\Phi(h_\varphi^{\frac1{2r'}}ah_\varphi^{\frac1{2r'}})=h_{\varphi_0}^{\frac1{2r'}}\Phi_{r,\varphi}(a)h_{\varphi_0}^{\frac1{2r'}},\qquad 
\Phi_\varphi(h_{\varphi_0}^{\frac1{2r'}}bh_{\varphi_0}^{\frac1{2r'}})=h_{\varphi}^{\frac1{2r'}}\Phi_{r,\varphi_0}(b)h_{\varphi}^{\frac1{2r'}},
\qquad a\in L_r(\Me),\ b\in L_r(\Ne)
\]
here $r'$ is such that $\frac1r+\frac1{r'}=1$.


By DPI, we have $D_{\alpha,z}(\psi_0\|\varphi_0)\le D_{\alpha,z}(\psi\|\varphi)<\infty$,
so that there is some unique $y_0\in L_{2z}(\Ne)$ such that 
\[
h_{\psi_0}^{\frac1{2p}}=y_0h_{\varphi_0}^{\frac1{2q}}.
\]

\begin{lemma}\label{lemma:le} Keeping the above assumptions and notations, we have for any $w_0\in
L_q(\Ne)^+$
\[
\Tr\Phi_{q,\varphi_0}(w_0)^q\le \Tr w_0^q,\qquad \Tr(y\Phi_{q,\varphi_0}(w_0)y^*)^p\ge
\Tr (y_0w_0y_0^*)^p.
\]

\end{lemma}

\begin{proof} The first inequality is immediate from the fact that $\Phi_{q,\varphi_0}$ is
a contraction. For the second inequality, let us first assume that
$w_0=h_{\varphi_0}^{\frac1{2q}}bh_{\varphi_0}^{\frac1{2q}}$ for some $b\in \Ne_+$. Then 
\[
h_{\varphi}^{\frac1{2q'}}\Phi_{q,\varphi_0}(w_0)h_{\varphi}^{\frac1{2q'}}=
\Phi_\varphi(h_{\varphi_0}^{\frac1{2q'}}w_0h_{\varphi_0}^{\frac1{2q'}})=\Phi_\varphi(h_{\varphi_0}^{\frac1{2}}bh_{\varphi_0}^{\frac1{2}})=h_{\varphi}^{\frac1{2q'}}h_{\varphi}^{\frac1{2q}}\Phi^*(b)h_{\varphi}^{\frac1{2q}}h_{\varphi}^{\frac1{2q'}},
\]
so that
$\Phi_{q,\varphi_0}(w_0)=h_{\varphi}^{\frac1{2q}}\Phi^*(b)h_{\varphi}^{\frac1{2q}}$.
Therefore
\begin{align*}
\Tr(y\Phi_{q,\varphi_0}(w_0)y^*)^p&=\Tr(yh_{\varphi}^{\frac1{2q}}\Phi^*(b)h_{\varphi}^{\frac1{2q}}y^*)^p=
\Tr(h_\psi^{\frac1{2p}}\Phi^*(b)h_\psi^{\frac1{2p}})^p\ge
\Tr(h_{\psi_0}^{\frac1{2p}}bh_{\psi_0}^{\frac1{2p}})^p\\
&=
\Tr(y_0h_{\varphi_0}^{\frac1{2q}}bh_{\varphi_0}^{\frac1{2q}}y_0^*)^p=\Tr(y_0w_0y_0^*)^p,
\end{align*}
here the inequality was proved in \cite{AJnotes}. Since
$h_{\varphi_0}^{\frac1{2q}}\Ne_+h_{\varphi_0}^{\frac1{2q}}$ is dense in $L_q(\Ne)^+$, the
statement follows.
\end{proof}

\begin{theorem} Let $\Phi:\Me_*\to \Ne_*$ be a 2-positive trace preserving map and let $\psi,\varphi\in \Me_*^+$ be such that
$D_{\alpha,z}(\psi\|\varphi)<\infty$. Then
$D_{\alpha,z}(\Phi(\psi)\|\Phi(\varphi))=D_{\alpha,z}(\psi\|\varphi)$ if and only if
$\Phi_\varphi\circ\Phi(\psi)=\psi$.

\end{theorem}

\begin{proof} By usual arguments, we may assume that both $\varphi$ and $\varphi_0$ are
faithful. Then there is a  conditional expectation $\Ee$ onto the set of fixed points of
$\Phi^*\circ\Phi_\varphi^*$ such that $\varphi\circ \Ee=\varphi$ and  
$\Phi_\varphi\circ\Phi(\psi)=\psi$ if and only if also $\psi\circ \Ee$. This is what we
are going to prove, using the extensions of conditional expectations to the Haagerup
$L_p$-spaces in \cite{junge2003noncommutative}, see also \cite[Sec. 1]{AJnote3}. 

So assume that $D_{\alpha,z}(\psi_0\|\varphi_0)=D_{\alpha,z}(\psi\|\varphi)$. Let $\bar
w\in L_q(\Me)^+$ and $\bar w_0\in L_q(\Ne)^+$ be the unique elements such that the suprema
in the variational formulas for $D_{\alpha,z}(\psi\|\varphi)$ resp.
$D_{\alpha,z}(\psi_0\|\varphi_0)$ are attained. We have by Lemma \ref{lemma:le}
\begin{align*}
D_{\alpha,z}(\psi\|\varphi)&\ge \alpha\Tr(y\Phi_{q,\varphi_0}(\bar w_0)y^*)^p-(\alpha-1)\Tr
\Phi_{q,\varphi_0}(\bar w_0)^q\\
&\ge \alpha\Tr(y_0\bar w_0 y_0^*)^p-(\alpha-1)\Tr \bar
w_0^q=D_{\alpha,z}(\psi_0\|\varphi_0)=D_{\alpha,z}(\psi\|\varphi),
\end{align*}
so that both inequalities must be equalities. This implies that in particular
\[
\Tr \bar w_0^q=\Tr \Phi_{q,\varphi_0}(\bar w_0)^q.
\]
By uniqueness, we must also have  $\bar w=\Phi_{q,\varphi_0}(\bar w_0)$. Let now $\omega\in \Me_*^+$ be given by
$h_{\omega}=h_\varphi^{\frac1{2q'}}\bar wh_\varphi^{\frac1{2q'}}$ and similarly
$h_{\omega_0}=h_{\varphi_0}^{\frac1{2q'}}\bar w_0h_{\varphi_0}^{\frac1{2q'}}$, then 
we get  $\Phi_\varphi(\omega_0)=\omega$ and also by
definition of the sandwiched R\'enyi divergence,
\[
\tilde D_\alpha(\omega_0\|\varphi_0)=\Tr \bar w_0^q=\Tr \Phi_{q,\varphi_0}(\bar
w_0)^q=\tilde D_\alpha(\Phi_\varphi(\omega_0)\|\Phi_\varphi(\varphi_0)).
\]
By \cite{jencova2018renyi}, this implies that $\Phi_\varphi$ is sufficient with respect to
$\{\omega_0,\varphi_0\}$ and hence $\Phi\circ\Phi_\varphi(\omega_0)=\omega_0$. It follows
that
\[
\Phi_\varphi\circ\Phi(\omega)=\Phi_\varphi\circ \Phi\circ
\Phi_\varphi(\omega_0)=\Phi_\varphi(\omega_0)=\omega,
\]
which implies that $\omega\circ \Ee=\omega$. Using the extensions of $\Ee$ and their
properties, we get
\[
h_\varphi^{\frac1{2q'}}\bar
wh_\varphi^{\frac1{2q'}}=h_\omega=\Ee(h_\omega)=h_\varphi^{\frac1{2q'}}\Ee(\bar
w)h_\varphi^{\frac1{2q'}},
\]
which implies that $\bar w=\Ee(\bar w)\in L_q(\Ee(\Me))^+$. But then also 
\[
|y|=\bar w^{\frac1{2(\alpha-1)}}=\bar w^{\frac{q}{2z}}\in L_{2z}(\Ee(\Me))^+
\]
Let $y=u|y|$ be the polar decomposition of $y$, then we obtain from \eqref{eq:y} that
$uu^*=s(\psi)$. Further,
\[
u^*h_\psi^{\frac1{2p}}=|y|h_\varphi^{\frac1{2q}}\in L_{2p}(\Ee(\Me))
\]
and by uniqueness of the polar decomposition in $L_{2p}(\Me)$ and $L_{2p}(\Ee(\Me))$, we
obtain that $h_{\psi}^{\frac1{2p}}\in L_{2p}(\Ee(\Me))^+$, $u\in \Ee(\Me)$. Hence we must
have $h_\psi\in L_1(\Ee(\Me))$ so that $\psi\circ\Ee=\psi$.


\end{proof}




\begin{thebibliography}{5}
\providecommand{\natexlab}[1]{#1}
\providecommand{\url}[1]{\texttt{#1}}
\expandafter\ifx\csname urlstyle\endcsname\relax
  \providecommand{\doi}[1]{doi: #1}\else
  \providecommand{\doi}{doi: \begingroup \urlstyle{rm}\Url}\fi

\bibitem{jencova2018renyi} A. Jen\v cov\'a, Rényi relative entropies and noncommutative
$L_p$-spaces, Ann. Henri Poincaré 19, 2513-2542, (2018)

\bibitem{AJnotes} A. Jen\v cov\'a, DPI for $\alpha-z$- R\'enyi divergence, Nov. 23, notes.
\bibitem{AJnote3} A. Jen\v cov\'a, Note on the limit $\alpha\searrow 1$, December 19,
2023, notes.
\bibitem[Junge and Xu(2003)]{junge2003noncommutative}
M.~Junge and Q.~Xu.
\newblock Noncommutative {B}urkholder/{R}osenthal inequalities.
\newblock \emph{The Annals of Probability}, 31\penalty0 (2):\penalty0 948--995,
  2003.
\bibitem[Kato(2023)]{kato2023onrenyi}
S.~Kato.
\newblock On $\alpha $-$ z $-{R}\'enyi divergence in the von
  {N}eumann algebra setting.
\newblock \emph{arXiv preprint arXiv:2311.01748}, 2023.

\bibitem{SKnote} S.~Kato, Variational expression for $\alpha>1$, note.


\end{thebibliography}










\end{document}


\section{Conditional expectations and $L_p$-spaces}


Let $\Ne$ be a von Neumann algebra and let $\Me\subseteq \Ne$ be a von Neumann subalgebra
such that there is a conditional expectation $\Ee$ onto $\Me$ preserving a faithful normal
state $\phi$.  Then the  modular group $\sigma^\phi$ of $\phi$
 preserves $\Me$  by
the Takesaki theorem and we have $\sigma^{\phi|_\Me}=\sigma^\phi|_\Me$.  It follows that the crossed
product $\Me\rtimes_{\sigma^{\phi|_\Me}} \mathbb R$ can be identified with a subalgebra in 
$\Ne\rtimes_{\sigma^\phi} \mathbb R$. By \cite[Thm. 4.1]{haagerup2010areduction}, the map 
\[
\hat \Ee:=(\Ee\otimes id_{B(L_2(G))})|_{\Ne\rtimes_{\sigma^\phi}\mathbb R}
\]
is a faithful normal conditional expectation of $\Ne\rtimes_{\sigma^\phi}\mathbb R$ onto
$\Me\rtimes_{\sigma^{\phi|_\Me}} \mathbb R$, moreover, we have 
\[
\hat\sigma\circ \hat\Ee=\hat\Ee\circ\hat\sigma
\]
and clearly also
\[
\hat\Ee(\pi(x)\lambda(s))=\pi(\Ee(x))\lambda(s),\qquad x\i \Me,\ s\in \mathbb R.
\]
For the dual weight $\hat\phi$, we obtain 
\[
\hat\Ee\circ\sigma^{\hat\phi}=\sigma^{\hat\phi}\circ\hat\Ee
\]
and $\hat\phi=\hat\phi\circ\hat\Ee$. Clearly, the dual weight for $\Me$ is the restriction
of $\hat\phi$. Let $\tau$ be the canonical trace and let us denote the canonical trace for
$\Me$ by $\tau_\Me$, then we have by \cite[Cor. 4.22]{takesaki2003theory2}
\[
[D\hat\phi\circ\hat\Ee:\tau_\Me\circ\hat\Ee]_t= [D\hat\phi|_{\Me\rtimes_{\sigma^{\phi|_\Me}}\mathbb
R}:D\tau_\Me]_t=\lambda(t)=[D\hat\phi\circ\hat
\Ee: D\tau]_t,
\]
it follows that $\tau=\tau_\Me\circ\hat\Ee=\tau\circ\hat\Ee$. Consequently, we
can see that the space of $\tau_\Me$-measurable elements $L_0(\Me)$ can be identified with
a *-subalgebra in $L_0(\Ne)$ and therefore also $L_p(\Me)\subseteq L_p(\Ne)$, $0<p\le \infty$. In particular, for
$p=1$ we obtain the  identification $\Me_*\subseteq \Ne_*$, given as 
\[
\omega\equiv \omega\circ \Ee, \qquad \omega\in \Me_*. 
\]
By \cite[Prop. 2.3]{junge2003noncommutative}, for $1\le p\le \infty$, $\Ee$ can be
extended to a contractive projection $\Ee_p$  of  $L_p(\Ne)$ onto $L_p(\Me)$. We have
\[
\Ee_1(h_\omega)=h_{\omega\circ\Ee},\qquad h_\omega\in L_1(\Ne)
\]
and $\Ee_q$ is the adjoint of $\Ee_p$ for $1/p+1/q=1$. The index $p$ is often dropped, so
we just write $\Ee$ instead of $\Ee_p$. For $1\le p,q,r\le \infty$ such that $1/p+1/q+1/r\le 1$, we have  
\begin{equation}\label{eq:condexp}
\Ee(hxk)=h\Ee(x)k,\qquad h\in
L_p(\Me), \ k\in L_q(\Me), \ x\in L_r(\Ne).
\end{equation}

\begin{lemma}\label{lemma:condexp} In the above situation, let $\psi,\varphi\in \Me_*^+$ and let
$\tilde\psi=\psi\circ\Ee$, $\tilde \varphi=\varphi\circ\Ee$. Then for  $1/2<\alpha/2\le z$ we
have
\[
D_{\alpha,z}(\psi\|\varphi)=D_{\alpha,z}(\tilde\psi\|\tilde\varphi).
\]

\end{lemma}


\begin{proof} Using the above identifications, we see that $h_\psi=h_{\tilde\psi}$,
$h_\varphi=h_{\tilde\varphi}$. Assume that $D_{\alpha,z}(\psi\|\varphi) <\infty$, then
there is some $y\in L_{2z}(\Me)s(\varphi)$ such that 
\begin{equation}\label{eq:fin}
h_\psi^{\frac{\alpha}{2z}}=yh_\varphi^{\frac{\alpha-1}2z}.
\end{equation}
Since $L_{2z}(\Me)\subseteq L_{2z}(\Ne)$ and $s(\varphi)=s(\tilde\varphi)$, we see that $y\in L_{2z}(\Ne)s(\tilde\varphi)$, so that
\[
Q_{\alpha,z}(\tilde\psi\|\tilde \varphi)=\|y\|_{2z}^{2z}=Q_{\alpha,z}(\psi\|\varphi).
\]
This implies that $D_{\alpha,z}(\tilde\psi\|\tilde \varphi)\le D_{\alpha,z}(\psi\|\varphi)$ in
general. Assume next that $1/2<\alpha/2\le z$ and let
$D_{\alpha,z}(\tilde\psi\|\tilde \varphi)<\infty$, so that \eqref{eq:fin} is satisfied with
some $y\in L_{2z}(\Ne)s(\tilde \varphi)$. Using the assumption on $\alpha,z$ and 
\eqref{eq:condexp}, we have
\[
h^{\frac{\alpha}{2z}}_{\psi}=\Ee(h_{\psi}^{\frac{\alpha}{2z}})=\Ee(y)h_\varphi^{\frac{\alpha-1}{2z}}.
\]
By uniqueness of $y$ and the fact
that $s(\tilde\varphi)=s(\varphi)\in \Me$, we obtain $y=\Ee(y)\in L_{2z}(\Me)s(\varphi)$.
This finishes the proof.


\end{proof}



\section{The limit $\alpha\searrow 1$}



Haagerup reduction theorem \cite[Thm. 2.1]{haagerup2010areduction} says that there is a
von Neumann algebra $\mathcal R$ with a faithful normal state $\phi$ and a sequence of von
Neumann algebras $(\Ra_n)_{n\ge 1}$ such that
\begin{enumerate}
\item[(i)] $\Me\subseteq \Ra$ and there is a conditional expectation $\Ee$ on $\Ra$ onto
$\Me$ such that $\phi\circ\Ee=\phi$,
\item[(ii)] $\Ra_n\subseteq \Ra_{n+1}$ and each $\Ra_n$ is finite,
\item[(iii)] $\bigcup_n \Ra_n$ is w*-dense in $\Ra$,
\item[(iv)] for each $n$ there is a conditional expectation on $\Ra$ onto $\Ra_n$ such
that $\phi\circ\Ee_n=\phi$.
\end{enumerate}

 

For any $\psi\in \Me_*^+$, let us denote $\hat\psi:=\psi\circ\Ee$ and $\psi_n:=\hat
\psi\circ\Ee_n$. Then $\psi_n\to \hat \psi$ in norm. 
By DPI and  martingale convergence (or DPI + LS), we have
\begin{equation}\label{eq:limit}
D_{\alpha,z}(\psi\|\varphi)=D_{\alpha,z}(\hat\psi\|\hat\varphi)=\lim_nD_{\alpha,z}(\psi_n\|\varphi_n),\qquad \text{if }
\max\{\frac{\alpha}2,\alpha-1\}\le z\le \alpha.
\end{equation}
Using Lemma \ref{lemma:condexp} and  LS, we get 
\begin{equation}\label{eq:ls}
D_{\alpha,z}(\psi\|\varphi)=D_{\alpha,z}(\hat\psi\|\hat\varphi)\le \liminf_nD_{\alpha,z}(\psi_n\|\varphi_n),\qquad \text{if }
\frac{\alpha}2\le z.
\end{equation}


\begin{prop}\label{prop:monotz}
Let $\max\{\alpha/2,\alpha-1\}\le z\le \alpha$. Then for any $z'\ge z$,
\[
D_{\alpha,z'}(\psi\|\varphi)\le D_{\alpha,z}(\psi\|\varphi).
\]
\end{prop}

\begin{proof}
By \cite[Lemma 1.3]{FHnote3}, we have $D_{\alpha,z'}(\psi_n\|\varphi_n)\le
D_{\alpha,z}(\psi_n\|\varphi_n)$ for all $n$. The statement  is proved  by using \eqref{eq:limit} for $z$ and
\eqref{eq:ls} for $z'$.

\end{proof}


\begin{coro}\label{coro:bound1}
Let $1<\alpha\le 2$. Then for any $z\ge 1$, we have
\[
 D_{\alpha,z}(\psi\|\varphi)\le D_{\alpha,1}(\psi\|\varphi).
\]


\end{coro}

\begin{proof} This follows
by putting $z=1$ and $z'=z$ is  Proposition \ref{prop:monotz}.

\end{proof}

The next statement is an extension of \cite[Lemma 2]{AJnote2}.

\begin{lemma}\label{lemma:bounds} Let $\alpha>1$ and $z\ge 1$. 
Then
\[
D_{\beta,1}(\psi\|\varphi)\le D_{\alpha,z}(\psi\|\varphi),
\]
where $\beta:=\frac{\alpha+z-1}{z}>1$.


\end{lemma}

\begin{proof} Using the scaling property of $D_{\alpha,z}$, we may assume that
$\psi(1)=1$.
We will also suppose that $D_{\alpha,z}(\psi\|\varphi)<\infty$, otherwise there is nothing to prove.  In this case, 
\[h_\psi^{\frac{\alpha}{2z}}=yh_\varphi^{\frac{\alpha-1}{2z}}
\]
for some $y\in L_{2z}(\Me)s(\varphi)$. We then get
\[
h_\psi^{\frac{\beta}2}=h_\psi^{\frac{z-1}{2z}}h_\psi^{\frac{\alpha}{2z}}=h_\psi^{\frac{z-1}{2z}}yh_\varphi^{\frac{\alpha-1}{2z}}=\eta h_{\varphi}^{\frac{\beta-1}2}
\]
with $\eta=h_\psi^{\frac{z-1}{2z}}y\in L_2(\Me)$. By \cite[Thm. 3.6]{hiai2021quantum},
it follows that
\[
Q_{\beta,1}(\psi\|\varphi)=\|\Delta_{\psi,\varphi}^{\frac{\beta}2}(h_\varphi^{1/2})\|_2^2=\|\eta\|_2^2\le
\|y\|_{2z}^2=Q_{\alpha,z}(\psi\|\varphi)^{1/z},
\]
this implies the statement.

\end{proof}



\begin{coro}\label{coro:limit}
Assume that  $D_{\alpha_0,z_0}(\psi\|\varphi)<\infty$ for some $1<\alpha_0$ and either
$z_0\ge 1$ or $\alpha_0/2\le z_0\le 1$. 
Then for any $z>1/2$ we have 
\[
\lim_{\alpha\searrow 1} D_{\alpha,z}(\psi\|\varphi)=D_1(\psi\|\varphi)<\infty.
\]

\end{coro}

\begin{proof}
We  first note  that under the above assumptions,
$D_{\beta,1}(\psi\|\varphi)<\infty$ for some $\beta>1$. Indeed, this follows from Lemma
\ref{lemma:bounds} in the first case, or by \cite[Prop. 2.3]{FHnote3} in the second case
(note that we necessarily have $\alpha_0-1\le \alpha_0/2\le z_0\le 1<\alpha_0$).

For $z\ge 1$, the statement now follows by using Lemma \ref{lemma:bounds} and Corollary
\ref{coro:bound1} for  $\alpha$ close
enough to 1. If $1/2<z\le 1$, we may use \cite[Prop. 2.3]{FHnote3} or Proposition
\ref{prop:monotz} for $\alpha$ close
enough to 1.


\end{proof}



%\bibliography{NEW_qre}
%\bibliographystyle{abbrvnat}
\begin{thebibliography}{5}
\providecommand{\natexlab}[1]{#1}
\providecommand{\url}[1]{\texttt{#1}}
\expandafter\ifx\csname urlstyle\endcsname\relax
  \providecommand{\doi}[1]{doi: #1}\else
  \providecommand{\doi}{doi: \begingroup \urlstyle{rm}\Url}\fi


\bibitem[Haagerup et~al.(2010)Haagerup, Junge, and Xu]{haagerup2010areduction}
U.~Haagerup, M.~Junge, and Q.~Xu.
\newblock {A reduction method for noncommutative $L_p$-spaces and
  applications}.
\newblock \emph{Transactions of the American Mathematical Society},
  362\penalty0 (4):\penalty0 2125--2165, 2010.
\newblock \doi{10.1090/S0002-9947-09-04935-6}.
\bibitem{hiai2021quantum}  F. Hiai, Quantum f-Divergences in von Neumann Algebras: Reversibility of Quantum Operations,  Mathematical Physics Studies, Springer Singapore, 2021

\bibitem{FHnote3} F. Hiai, Monotonicity of $z\mapsto D_{\alpha,z}(\psi\|\varphi)$, 
(12/3/2023, 12/8/2023), notes.

\bibitem{AJnote2} A. Jen\v cov\'a, Notes for $\alpha-z$-R\'enyi divergence, December 7,
2023, notes.
\bibitem[Junge and Xu(2003)]{junge2003noncommutative}
M.~Junge and Q.~Xu.
\newblock Noncommutative {B}urkholder/{R}osenthal inequalities.
\newblock \emph{The Annals of Probability}, 31\penalty0 (2):\penalty0 948--995,
  2003.

\bibitem[Takesaki(2003)]{takesaki2003theory2}
M.~Takesaki.
\newblock \emph{Theory of Operator Algebras. {II}}, volume 125 of
  \emph{Encyclopaedia of Mathematical Sciences}.
\newblock Springer-Verlag, Berlin, 2003.
\newblock ISBN 3-540-42914-X.
\newblock \doi{10.1007/978-3-662-10451-4}.


\end{thebibliography}



\end{document}

\begin{thebibliography}{5}
\providecommand{\natexlab}[1]{#1}
\providecommand{\url}[1]{\texttt{#1}}
\expandafter\ifx\csname urlstyle\endcsname\relax
  \providecommand{\doi}[1]{doi: #1}\else
  \providecommand{\doi}{doi: \begingroup \urlstyle{rm}\Url}\fi

\bibitem[Gu et~al.(2019)Gu, Yin, and Zhang]{gu2019interpolation}
J.~Gu, Z.~Yin, and H.~Zhang.
\newblock {Interpolation of quasi noncommutative $L_p$-spaces}.
\newblock \emph{arXiv:1905.08491}, 2019.

\bibitem{FHnotes} F. Hiai, Questions, note.

\bibitem{FHnote2} F. Hiai, Martingale convergence for $D_{\alpha,z}$, note.

\bibitem{AJnote} A. Jen{\v c}ov\'a, DPI for $\alpha-z$-R\'enyi divergence, note.

\bibitem[Kato(2023)]{kato2023onrenyi}
S.~Kato.
\newblock On $\alpha $-$ z $-{R}\'enyi divergence in the von
  {N}eumann algebra setting.
\newblock \emph{arXiv preprint arXiv:2311.01748}, 2023.

\bibitem{SKnote} S.~Kato, Variational expression for $\alpha>1$, note.

\bibitem[Kosaki({1984})]{kosaki1984applications}
H.~Kosaki.
\newblock {Applications of the complex interpolation method to a von Neumann
  algebra: Non-commutative $L_p$-spaces}.
\newblock \emph{{J. Funct. Anal.}}, {56}:\penalty0 {26--78}, {1984}.

\bibitem[Mosonyi(2023)]{mosonyi2023thestrong}
M.~Mosonyi.
\newblock The strong converse exponent of discriminating infinite-dimensional
  quantum states.
\newblock \emph{Communications in Mathematical Physics}, 400\penalty0
  (1):\penalty0 83--132, 2023.

\bibitem[Zhang(2020)]{zhang2020fromwyd}
H.~Zhang.
\newblock From Wigner-Yanase-Dyson conjecture to Carlen-Frank-Lieb conjecture.
\newblock \emph{Advances in Mathematics}, 365:\penalty0 107053, 2020.

\end{thebibliography}







\end{document}



