\documentclass[12pt]{article}
%DIF LATEXDIFF DIFFERENCE FILE
%DIF DEL alphaz-1.tex   Wed Feb 28 13:46:44 2024
%DIF ADD alphaz.tex     Mon Feb 26 15:30:00 2024

\usepackage{hyperref}
%DIF 4d4
%DIF < %\usepackage[dvipdfmx]{hyperref}
%DIF -------
\usepackage{amsmath, amssymb, amsthm}
\usepackage{xcolor}
\usepackage[sort&compress,numbers]{natbib}
\usepackage{doi}
\usepackage[margin=0.8in]{geometry}
%\textheight23cm \topmargin-20mm  
%\textwidth175mm  
%\oddsidemargin=0mm
%\evensidemargin=0mm
%

\usepackage{amsmath, amssymb, amsthm, mathtools}

\newtheorem{theorem}{Theorem}[section]
%DIF 19a18-19
\newtheorem{coro}{Corollary} %DIF > 
 %DIF > 
%DIF -------
\newtheorem{lemma}[theorem]{Lemma}
%DIF 20-21c21
%DIF < \newtheorem{coro}[theorem]{Corollary}
%DIF < \newtheorem{prop}[theorem]{Proposition}
%DIF -------
\newtheorem{prop}{Proposition} %DIF > 
%DIF -------

\theoremstyle{definition}
\newtheorem{defi}[theorem]{Definition}

%DIF 26a26
 %DIF > 
%DIF -------
\theoremstyle{remark}
%DIF 27c28
%DIF < \newtheorem{remark}[theorem]{Remark}
%DIF -------
\newtheorem{remark}{Remark} %DIF > 
%DIF -------

%DIF 29-30c30
%DIF < \numberwithin{equation}{section}
%DIF < 
%DIF -------
\def\cM{\mathcal{M}} %DIF > 
%DIF -------
\def\Me{\mathcal M}
\def\Ne{\mathcal N}
\def \Tr{\mathrm{Tr}\,}
\def\states {\mathfrak S}
\def\supp{\mathrm{supp}}
\def\<{\langle\.}
\def\>{\.\rangle}
\def\ffi{\varphi}
%DIF 39-41d39
%DIF < \def\1{\mathbf{1}}
%DIF < \def\eps{\varepsilon}
%DIF < \def\bN{\mathbb{N}}
%DIF -------

\title{On the properties $\alpha-z$ R\'enyi divergences on general von Neumann algebras}
\author{Fumio Hiai and Anna Jen\v cov\'a}
%DIF PREAMBLE EXTENSION ADDED BY LATEXDIFF
%DIF UNDERLINE PREAMBLE %DIF PREAMBLE
\RequirePackage[normalem]{ulem} %DIF PREAMBLE
\RequirePackage{color}\definecolor{RED}{rgb}{1,0,0}\definecolor{BLUE}{rgb}{0,0,1} %DIF PREAMBLE
\providecommand{\DIFaddtex}[1]{{\protect\color{blue}\uwave{#1}}} %DIF PREAMBLE
\providecommand{\DIFdeltex}[1]{{\protect\color{red}\sout{#1}}}                      %DIF PREAMBLE
%DIF SAFE PREAMBLE %DIF PREAMBLE
\providecommand{\DIFaddbegin}{} %DIF PREAMBLE
\providecommand{\DIFaddend}{} %DIF PREAMBLE
\providecommand{\DIFdelbegin}{} %DIF PREAMBLE
\providecommand{\DIFdelend}{} %DIF PREAMBLE
\providecommand{\DIFmodbegin}{} %DIF PREAMBLE
\providecommand{\DIFmodend}{} %DIF PREAMBLE
%DIF FLOATSAFE PREAMBLE %DIF PREAMBLE
\providecommand{\DIFaddFL}[1]{\DIFadd{#1}} %DIF PREAMBLE
\providecommand{\DIFdelFL}[1]{\DIFdel{#1}} %DIF PREAMBLE
\providecommand{\DIFaddbeginFL}{} %DIF PREAMBLE
\providecommand{\DIFaddendFL}{} %DIF PREAMBLE
\providecommand{\DIFdelbeginFL}{} %DIF PREAMBLE
\providecommand{\DIFdelendFL}{} %DIF PREAMBLE
%DIF HYPERREF PREAMBLE %DIF PREAMBLE
\providecommand{\DIFadd}[1]{\texorpdfstring{\DIFaddtex{#1}}{#1}} %DIF PREAMBLE
\providecommand{\DIFdel}[1]{\texorpdfstring{\DIFdeltex{#1}}{}} %DIF PREAMBLE
%DIF COLORLISTINGS PREAMBLE %DIF PREAMBLE
\RequirePackage{listings} %DIF PREAMBLE
\RequirePackage{color} %DIF PREAMBLE
\lstdefinelanguage{DIFcode}{ %DIF PREAMBLE
%DIF DIFCODE_UNDERLINE %DIF PREAMBLE
  moredelim=[il][\color{red}\sout]{\%DIF\ <\ }, %DIF PREAMBLE
  moredelim=[il][\color{blue}\uwave]{\%DIF\ >\ } %DIF PREAMBLE
} %DIF PREAMBLE
\lstdefinestyle{DIFverbatimstyle}{ %DIF PREAMBLE
	language=DIFcode, %DIF PREAMBLE
	basicstyle=\ttfamily, %DIF PREAMBLE
	columns=fullflexible, %DIF PREAMBLE
	keepspaces=true %DIF PREAMBLE
} %DIF PREAMBLE
\lstnewenvironment{DIFverbatim}{\lstset{style=DIFverbatimstyle}}{} %DIF PREAMBLE
\lstnewenvironment{DIFverbatim*}{\lstset{style=DIFverbatimstyle,showspaces=true}}{} %DIF PREAMBLE
%DIF END PREAMBLE EXTENSION ADDED BY LATEXDIFF

\begin{document}

\maketitle


\section{Introduction}

\section{Preliminaries}

\subsection{Basic definitions}

Let $\Me$ be a von Neumann algebra  and let $\Me^+$ be the cone of positive elements in $\Me$. We denote the predual by $\Me_*$, its positive part by $\Me_*^+$ and the set of normal states by $\states_*(\Me)$. For $\psi\in \Me_*^+$, we will denote by $s(\psi)$ the support projection  of $\psi$.

For $0< p\le \infty$, let $L_p(\Me)$ be the Haagerup $L_p$-space over $\Me$ and let
$L_p(\Me)$ its positive cone, 
\cite{haagerup1979lpspaces}. We will use the identifications $\Me\simeq L_\infty(\Me)$, $\Me_*\ni \psi \leftrightarrow h_\psi\in L_1(\Me)$ and the notation $\Tr h_\psi=\psi(1)$ for the trace in $L_1(\Me)$. It this way,
 $\Me_*^+$  is identified with the positive cone $L_1(\Me)^+$ and $\states_*(\Me)$ with subset of elements in $L_1(\Me)^+$ 
 with unit trace. 
Precise definitions and further details on the spaces $L_p(\Me)$ can be found in the notes \cite{terp1981lpspaces}.



%Kosaki, complex interpolation. Generalized s-numbers. Haagerup reduction. Martingale
%convergence


\subsection{The $\alpha-z$-R\'enyi divergences}

In \cite{kato2023aremark, kato2023onrenyi}, the
$\alpha-z$-R\'enyi divergence for $\psi,\varphi\in \mathcal M_*^+$  was defined as
follows: 
\begin{defi}\label{defi:renyi} Let $\psi,\varphi\in \Me_*^+$, $\psi\ne 0$ and let
$\alpha,z>0$, $\alpha\ne 1$. The $\alpha-z$-R\'enyi divergence is defined as 
\[
D_{\alpha,z}(\psi\|\varphi):=\frac{1}{\alpha-1}\log
\frac{Q_{\alpha,z}(\psi\|\varphi)}{\psi(1)},
\]
where 
\[
Q_{\alpha,z}=\begin{dcases} \Tr
\left(h_\varphi\DIFdelbegin \DIFdel{^{(1-\alpha)/2z}}\DIFdelend \DIFaddbegin \DIFadd{^{\frac{1-\alpha}{2z}}}\DIFaddend h_\psi\DIFdelbegin \DIFdel{^{\alpha/z}}\DIFdelend \DIFaddbegin \DIFadd{^{\frac{\alpha}{z}}}\DIFaddend h_\varphi\DIFdelbegin \DIFdel{^{(1-\alpha)/2z}}\DIFdelend \DIFaddbegin \DIFadd{^{\frac{1-\alpha}{2z}}}\DIFaddend \right)^z, &
\text{if } 0<\alpha<1\\[0.3em]
\|x\|_z^z, & \text{if } \alpha>1 \text{ and }\\
\ &
h_\psi\DIFdelbegin \DIFdel{^{\alpha/z}}\DIFdelend \DIFaddbegin \DIFadd{^{\frac{\alpha}{z}}}\DIFaddend =h_\varphi\DIFdelbegin \DIFdel{^{(\alpha-1)/2z}}\DIFdelend \DIFaddbegin \DIFadd{^{\frac{\alpha-1}{2z}}}\DIFaddend xh_\varphi\DIFdelbegin \DIFdel{^{(\alpha-1)/2z}}\DIFdelend \DIFaddbegin \DIFadd{^{\frac{\alpha-1}{2z}}}\DIFaddend ,\text{ with }
x\in s(\varphi)L_z(\Me)s(\varphi)\\[0.3em]
\infty& \text{otherwise}.
\end{dcases}
\]
\end{defi}

%It is easily checked that this definition coincides with \eqref{eq:fdrenyi} in the finitedimensional case.

In the case $\alpha>1$, the following alternative form will be useful.

\begin{lemma}\DIFdelbegin \DIFdel{\mbox{%DIFAUXCMD
\cite{kato2023onrenyi} }\hskip0pt%DIFAUXCMD
}\DIFdelend \DIFaddbegin \DIFadd{\mbox{%DIFAUXCMD
\cite[Lemma 7]{kato2023onrenyi} }\hskip0pt%DIFAUXCMD
}\DIFaddend \label{lemma:renyi_2z}
Let $\alpha>1$ and $\psi,\varphi\in \Me_*^+$. Then $Q_{\alpha,z}(\psi\|\varphi)<\infty$ if
and only if there is some $y\in L_{2z}(\Me)s(\varphi)$ such that 
\[
h_\psi\DIFdelbegin \DIFdel{^{\alpha/2z}}\DIFdelend \DIFaddbegin \DIFadd{^{\frac{\alpha}{2z}}}\DIFaddend =yh_\varphi\DIFdelbegin \DIFdel{^{(\alpha-1)/2z}}\DIFdelend \DIFaddbegin \DIFadd{^{\frac{\alpha-1}{2z}}}\DIFaddend .
\]
Moreover, in this case, such $y$ is unique and we have
$Q_{\alpha,z}(\psi\|\varphi)=\|y\|_{2z}^{2z}$. 
\end{lemma}

The standard R\'enyi divergence \cite{petz1985quasi, hiai2018quantum, hiai2021quantum} is
contained in this range as $D_\alpha(\psi\|\varphi)=D_{\alpha,1}(\psi\|\varphi)$. The
sandwiched R\'enyi divergence  is obtained as $\tilde
D_\alpha(\psi\|\varphi)=D_{\alpha,\alpha}(\psi\|\varphi)$, see
\cite{berta2018renyi,hiai2021quantum,jencova2018renyi, jencova2021renyi} for some
alternative definitions and properties of $\tilde D_\alpha$. The definition in
\cite{jencova2018renyi} and \cite{jencova2021renyi} is based on the Kosaki interpolation
spaces  $L_p(\Me,\varphi)$ with respect to a state \cite{kosaki1984applications}. These spaces and
complex interpolation method will be used frequently also in the present work. 


Many of the  properties of $D_{\alpha,z}(\psi\|\varphi)$ 
were extended from the finite dimensional case in \cite{kato2023onrenyi}. In particular,
\DIFdelbegin \DIFdel{the following variational expressions will be an important tool for our work. }\DIFdelend \DIFaddbegin \DIFadd{a variational expression for $Q_{\alpha,z}$ in the case $0<\alpha<1$ was proved there,
see part (i) in the theorem below. We will prove a similar variational expression also in
the case when $\alpha>1$. 
}\DIFaddend 

\begin{theorem}[Variational expressions]\label{thm:variational} Let $\psi,\varphi\in \Me_*^+$, $\psi\ne 0$. 
\begin{enumerate}
\item[(i)] Let $0<\alpha<1$ and $\max\{\alpha,1-\alpha\}\le z$. Then
\[
Q_{\alpha,z}(\psi\|\varphi)=\inf_{a\in \Me^{++}}\left\{\alpha
\Tr\left((a^{\frac{1}{2}}h_\psi^{\frac{\alpha}{z}}a^{\frac{1}{2}})^{\frac{z}{\alpha}}\right)+(1-\alpha)
\Tr\left((a^{\frac{1}{2}}h_\varphi^{\frac{1-\alpha}{z}}a^{\frac{1}{2}})^{\frac{z}{1-\alpha}}\right) \right\}.
\]
\DIFdelbegin \DIFdel{Moreover, if $\lambda^{-1}\varphi\le \psi\le \lambda \varphi$ for some $\lambda>0$, then
the infimum is attained at a unique element  $\bar a\in \Me^{++}$. This element  satisfies
}\[
\DIFdel{h_\psi^{\frac{\alpha}{2z}}\bar ah_\psi^{\frac{\alpha}{2z}}=\left(h_\psi^{\frac{\alpha}{2z}}h_\varphi^{\frac{1-\alpha}{2z}}h_\psi^{\frac{\alpha}{2z}}\right)^\alpha
}\]%DIFAUXCMD
\DIFdel{and
}\[
\DIFdel{h_\varphi^{\frac{1-\alpha}{2z}}\bar a^{-1}h_\varphi^{\frac{1-\alpha}{2z}}=\left(h_\varphi^{\frac{1-\alpha}{2z}}h_\psi^{\frac{\alpha}{z}}h_\varphi^{\frac{1-\alpha}{2z}}\right)^{1-\alpha}.
}\]%DIFAUXCMD
\DIFdelend 

\item[(ii)] Let \DIFdelbegin \DIFdel{$1<\alpha\le 2z$, then
}\DIFdelend \DIFaddbegin \DIFadd{$1<\alpha$, $\max\{\frac{\alpha}{2},\alpha-1\}\le z$. Then
}\DIFaddend \[
Q_{\alpha,z}(\psi\|\varphi)=\sup_{a\in \Me_+} \left\{\alpha
\Tr\left((a^{\frac{1}{2}}h_\psi^{\frac{\alpha}{z}}a^{\frac{1}{2}})^{\frac{z}{\alpha}}\right)-(\alpha-1)
\Tr\left((a^{\frac{1}{2}}h_\varphi^{\frac{\alpha-1}{2z}}a^{\frac{1}{2}})^{\frac{z}{\alpha-1}}\right) \right\}.
\]

\end{enumerate}


\end{theorem}


\begin{proof} For part (i) see \cite[Theorem 1 (vi)]{kato2023onrenyi}\DIFdelbegin \DIFdel{and its proof}\DIFdelend . The
inequality $\ge$ in part (ii) holds for all $\alpha$ and $z$ and was proved in
\cite[Theorem 2 (vi)]{kato2023onrenyi}. We now prove the opposite inequality. 

Assume first that $Q_{\alpha,z}(\psi\|\varphi)<\infty$, so that there is some $x\in
s(\varphi)L_z(\Me)^+s(\varphi)$ such that
$h_\psi^{\frac{\alpha}{z}}=h_\varphi^{\frac{\alpha-1}{2z}}xh_\varphi^{\frac{\alpha-1}{2z}}$. Plugging this
into the right hand side, we obtain
\begin{align*}
&\sup_{a\in \Me_+} \left\{\alpha
\Tr\left((a^{\frac{1}{2}}h_\psi^{\frac{\alpha}{z}}a^{\frac{1}{2}})^{\frac{z}{\alpha}}\right)-(\alpha-1)
\Tr\left((a^{\frac{1}{2}}h_\varphi^{\frac{\alpha-1}{2z}}a^{\frac{1}{2}})^{\frac{z}{\alpha-1}}\right) \right\}\\
&=\sup_{a\in \Me_+} \left\{\alpha
\Tr\left((a^{\frac{1}{2}}h_\varphi^{\frac{\alpha-1}{2z}}xh_\varphi^{\frac{\alpha-1}{2z}}
a^{\frac{1}{2}})^{\frac{z}{\alpha}}\right)-(\alpha-1)
\Tr\left((a^{\frac{1}{2}}h_\varphi^{\frac{\alpha-1}{2z}}a^{\frac{1}{2}})^{\frac{z}{\alpha-1}}\right) \right\}\\
&=\sup_{a\in \Me_+} \left\{\alpha
\Tr\left((x^{\frac{1}{2}}h_\varphi^{\frac{\alpha-1}{2z}}ah_\varphi^{\frac{\alpha-1}{2z}}
x^{\frac{1}{2}})^{\frac{z}{\alpha}}\right)-(\alpha-1)
\Tr\left((h_\varphi^{\frac{\alpha-1}{2z}}a h_\varphi^{\frac{\alpha-1}{2z}}
)^{\frac{z}{\alpha-1}}\right)
\right\}\\
&=\sup_{w\in L_{\frac{z}{\alpha-1}}(\Me)^+} \left\{\alpha
\Tr\left((x^{\frac{1}{2}}wx^{\frac{1}{2}})^{\frac{z}{\alpha}}\right)-(\alpha-1)
\Tr\left(w^{\frac{z}{\alpha-1}}\right)
\right\},
\end{align*}
where we used the fact that \DIFdelbegin \DIFdel{$\Tr \left((a^*a)^p\right)=\Tr \left((aa^*)^p\right)$ }\DIFdelend \DIFaddbegin \DIFadd{$\Tr \left((h^*h)^p\right)=\Tr \left((hh^*)^p\right)$ }\DIFaddend for
$p>0$ and \DIFdelbegin \DIFdel{$a\in L_{\frac{p}{2}}(\Me)$ and the fact that the set of
elements of the form $h_\varphi^{\frac{\alpha-1}{2z}}a h_\varphi^{\frac{\alpha-1}{2z}}$ with $a \in
\Me^+$ is dense in the positive cone $L_{\frac{z}{\alpha-1}}(\Me)^+$}\DIFdelend \DIFaddbegin \DIFadd{$h\in L_{\frac{p}{2}}(\Me)$, and Lemma \ref{lemma:cone}}\DIFaddend .  Putting $w=x^{\alpha-1}$ we
get
\[
\sup_{w\in L_{\frac{z}{\alpha-1}}(\Me)^+} \left\{\alpha
\Tr\left((x^{\frac{1}{2}}wx^{\frac{1}{2}})^{\frac{z}{\alpha}}\right)-(\alpha-1)
\Tr\left(w^{\frac{z}{\alpha-1}}\right)
\right\}\ge \Tr(x^z)=\|x\|_z^z= Q_{\alpha,z}(\psi\|\varphi).
\]
This finishes the proof in the case that $Q_{\alpha,z}(\psi\|\varphi)<\infty$.  Note that
this holds if $\psi\le \lambda\varphi$ for some $\lambda>0$. Indeed, since
$\frac{\alpha}{2z}\in (0,1]$ by the assumption, we then have 
\[
h_\psi^{\frac{\alpha}{2z}}\le \lambda^{\frac{\alpha}{2z}}h_\varphi^{\frac{\alpha}{2z}},
\]
hence by \cite[Lemma A.58]{hiai2021quantum} there is some $b\in \Me$ such that 
\[
h_\psi^{\frac{\alpha}{2z}}=bh_\varphi^{\frac{\alpha}{2z}}=yh_\varphi^{\frac{\alpha-1}{2z}},
\]
where $y=bh_\varphi^{\frac{1}{2z}}\in L_{2z}(\Me)$. By Lemma \ref{lemma:renyi_2z} we get 
$Q_{\alpha,z}(\psi\|\varphi)=\|y\|_{2z}^{2z}<\infty$. 

In the general case, \DIFdelbegin \DIFdel{note that lower semicontinuity \mbox{%DIFAUXCMD
\cite[]{kato2023onrenyi}}\hskip0pt%DIFAUXCMD
, we have
}\[
\DIFdel{Q_{\alpha,z}(\psi\|\varphi)\le \liminf_{\epsilon\searrow 0}
Q_{\alpha,z}(\psi\|\varphi+\epsilon \psi) 
}\]%DIFAUXCMD
\DIFdel{and }\DIFdelend the variational expression holds for
$Q_{\alpha,z}(\psi\|\varphi+\epsilon\psi)$  for all $\epsilon>0$\DIFdelbegin \DIFdel{. The proof is finished 
by using norm continuity of the map $L_1(\Me)^+\ni h\mapsto h^{\frac{1}p}\in L_p(\Me)^+$ for $p>1$. }\DIFdelend \DIFaddbegin \DIFadd{, so that we have
}\begin{align*}
\DIFadd{Q_{\alpha,z}(\psi\|\varphi+\epsilon\psi)}&\DIFadd{=\sup_{a\in \Me_+} \left\{\alpha
\Tr\left((a^{\frac{1}{2}}h_\psi^{\frac{\alpha}{z}}a^{\frac{1}{2}})^{\frac{z}{\alpha}}\right)-(\alpha-1)
\Tr\left((a^{\frac{1}{2}}h_{\varphi+\epsilon \psi}^{\frac{\alpha-1}{2z}}a^{\frac{1}{2}})^{\frac{z}{\alpha-1}}\right) \right\}}\\
&\DIFadd{\le\sup_{a\in \Me_+} \left\{\alpha
\Tr\left((a^{\frac{1}{2}}h_\psi^{\frac{\alpha}{z}}a^{\frac{1}{2}})^{\frac{z}{\alpha}}\right)-(\alpha-1)
\Tr\left((a^{\frac{1}{2}}h_\varphi^{\frac{\alpha-1}{2z}}a^{\frac{1}{2}})^{\frac{z}{\alpha-1}}\right)
\right\},
}\end{align*}
\DIFadd{where the inequality above follows by Lemma \ref{lemma:order}. Therefore, since lower
semicontinuity  \mbox{%DIFAUXCMD
\cite[Theorem 2 (iv)]{kato2023onrenyi} }\hskip0pt%DIFAUXCMD
gives
}\[
\DIFadd{Q_{\alpha,z}(\psi\|\varphi)\le \liminf_{\epsilon\searrow 0}
Q_{\alpha,z}(\psi\|\varphi+\epsilon \psi) 
}\]
\DIFadd{the desired inequality follows.
}\DIFaddend 


\DIFdelbegin %DIFDELCMD < {\color{red} %%%
\DIFdel{I am not entirely sure about this proof , since the
convergence of 
the
 expressions 
}\[
 \DIFdel{\Tr\left((a^{\frac{1}{2}}(h_\varphi+\epsilon h_\psi)^{\frac{\alpha-1}{2z}}a^{\frac{1}{2}})^{\frac{z}{\alpha-1}}\right)\to
 \Tr\left((a^{\frac{1}{2}}h_\varphi^{\frac{\alpha-1}{2z}}a^{\frac{1}{2}})^{\frac{z}{\alpha-1}}\right)}\]%DIFAUXCMD
\DIFdelend \DIFaddbegin \end{proof}

\begin{lemma}\label{lemma:variational_majorized}  \DIFadd{Assume that $\lambda^{-1}\varphi\le
\psi\le \lambda \varphi$ for some $\lambda>0$. Then
the infimum in the variational expression in Theorem \ref{thm:variational}(i) is attained at a unique element  $\bar a\in \Me^{++}$. This element  satisfies
}\begin{align}
\DIFadd{h_\psi^{\alpha\over2z}ah_\psi^{\alpha\over2z}
}&\DIFadd{=\bigl(h_\psi^{\alpha\over2z}h_\ffi^{1-\alpha\over z}h_\psi^{\alpha\over2z}\bigr)^\alpha
\label{eq:minimizer1}}\\
\DIFadd{h_\ffi^{1-\alpha\over2z}a^{-1}h_\ffi^{1-\alpha\over2z}
}&\DIFadd{=\bigl(h_\ffi^{1-\alpha\over2z}h_\psi^{\alpha\over
z}h_\ffi^{1-\alpha\over2z}\bigr)^{1-\alpha}.
\label{eq:minimizer2}
}\end{align}\DIFaddend 
\DIFdelbegin \DIFdel{also depends
 on $\|a\|$, which is not bounded. But probably I misunderstand something, or I do not see something trivial. }%DIFDELCMD < }
%DIFDELCMD <  %%%
\DIFdelend \DIFaddbegin \end{lemma}
\DIFaddend 


\DIFaddbegin \begin{proof} \DIFadd{We may assume that $\varphi$ and hence also $\psi$ is faithful. Following
 the proof of \mbox{%DIFAUXCMD
\cite[Theorem 1 (vi)]{kato2023onrenyi}}\hskip0pt%DIFAUXCMD
,  we may use the
assumptions and \mbox{%DIFAUXCMD
\cite[Lemma A.58]{hiai2021quantum} }\hskip0pt%DIFAUXCMD
to show  that there are $b,c\in\cM$  such that
}\begin{align}\DIFadd{\label{eq:bc}
h_\ffi^{1-\alpha\over2z}
=b\bigl(h_\ffi^{1-\alpha\over2z}h_\psi^{\alpha\over z}h_\ffi^{1-\alpha\over2z}\bigr)^{1-\alpha\over2},\qquad
\bigl(h_\ffi^{1-\alpha\over2z}h_\psi^{\alpha\over z}h_\ffi^{1-\alpha\over2z}\bigr)^{1-\alpha\over2}
=ch_\ffi^{1-\alpha\over2z}.
}\end{align}
\DIFadd{Put $\bar a:=bb^*\in\cM^{++}$, then we have  $\bar a^{-1}=c^*c$ and $\bar a$ is indeed a
minimizer of 
}\begin{align}\DIFadd{\label{eq:infimum}
Q_{\alpha,z}(\psi\|\ffi)
=\inf_{a\in\cM^{++}}\biggl\{\alpha\Big\|h_\psi^{\alpha\over2z}ah_\psi^{\alpha\over2z}\Big\|_{z\over\alpha}^{z\over\alpha}
+(1-\alpha)\Big\|h_\ffi^{1-\alpha\over2z}a^{-1}h_\ffi^{1-\alpha\over2z}\Big\|_{z\over
1-\alpha}^{z\over 1-\alpha}\biggr\}.
}\end{align}
\DIFadd{We next observe that such  minimizer is unique. Indeed, suppose that the infimum is
attained  at some $a_1,a_2\in \Me^{++}$. Let $a_0:=(a_1+a_2)/2$. Since the map 
$L^{p}(\cM)\ni k\mapsto\|k\|_{p}^{p}$ is convex for any $p\ge 1$ and
$a_0^{-1}\le(a_1^{-1}+a_2^{-1})/2$, we have
}\begin{align*}
\DIFadd{\Big\|h_\psi^{\alpha\over2z}a_0h_\psi^{\alpha\over2z}\Big\|_{z\over\alpha}^{z\over\alpha}
}&\DIFadd{\le{1\over2}\biggl\{\Big\|h_\psi^{\alpha\over2z}a_1h_\psi^{\alpha\over2z}\Big\|_{z\over\alpha}^{z\over\alpha}
+\Big\|h_\psi^{\alpha\over2z}a_2h_\psi^{\alpha\over2z}\Big\|_{z\over\alpha}^{z\over\alpha}\biggr\}, }\\
\DIFadd{\Big\|h_\ffi^{1-\alpha\over2z}a_0^{-1}h_\ffi^{1-\alpha\over2z}\Big\|_{z\over1-\alpha}^{z\over1-\alpha}
}&\DIFadd{\le\Big\|h_\ffi^{1-\alpha\over2z}\biggl(}{\DIFadd{a_1^{-1}+a_2^{-1}\over2}}\DIFadd{\biggr)
h_\ffi^{1-\alpha\over2z}\Big\|_{z\over1-\alpha}^{z\over1-\alpha} }\\
&\DIFadd{\le{1\over2}\biggl\{\Big\|h_\ffi^{1-\alpha\over2z}a_1^{-1}h_\ffi^{1-\alpha\over2z}\Big\|_{z\over1-\alpha}^{z\over1-\alpha}
+\Big\|h_\ffi^{1-\alpha\over2z}a_2^{-1}h_\ffi^{1-\alpha\over2z}\Big\|_{z\over1-\alpha}^{z\over1-\alpha}\biggr\}.
}\end{align*}
\DIFadd{Hence we have
}\[
\DIFadd{\Big\|h_\ffi^{1-\alpha\over2z}a_0^{-1}h_\ffi^{1-\alpha\over2z}\Big\|_{z\over1-\alpha}
=\Big\|h_\ffi^{1-\alpha\over2z}\biggl(}{\DIFadd{a_1^{-1}+a_2^{-1}\over2}}\DIFadd{\biggr)
h_\ffi^{1-\alpha\over2z}\Big\|_{z\over1-\alpha},
}\]
\DIFadd{which implies that $a_0^{-1}={a_1^{-1}+a_2^{-1}\over2}$, as easily verified. From this we easily have $a_1=a_2$.
}

\DIFadd{The equality  \eqref{eq:minimizer2} is obvious from the second equality in \eqref{eq:bc}
and $\bar a^{-1}=c^*c$. Since
$Q_{\alpha,z}(\psi\|\ffi)=Q_{1-\alpha,z}(\ffi\|\psi)$, we see by uniqueness that the minimizer of the
infimum expression for $Q_{1-\alpha,z}(\ffi\|\psi)$ (instead of \eqref{eq:infimum}) is
$\bar a^{-1}$ (instead of $\bar a$). This
says that \eqref{eq:minimizer1} is the equality corresponding to \eqref{eq:minimizer2} when $\psi,\ffi,\alpha$ are replaced with
$\ffi,\psi,1-\alpha$, respectively. 
}



\DIFaddend \end{proof}








%DIF > h_\psi^{\frac{\alpha}{2z}}\bar
%DIF > ah_\psi^{\frac{\alpha}{2z}}=\left(h_\psi^{\frac{\alpha}{2z}}h_\varphi^{\frac{1-\alpha}{2z}}h_\psi^{\frac{\alpha}{2z}}\right)^\alpha,\qquad
%DIF > h_\varphi^{\frac{1-\alpha}{2z}}\bar a^{-1}h_\varphi^{\frac{1-\alpha}{2z}}=\left(h_\varphi^{\frac{1-\alpha}{2z}}h_\psi^{\frac{\alpha}{z}}h_\varphi^{\frac{1-\alpha}{2z}}\right)^{1-\alpha}.
%DIF > \]
%DIF > 
%DIF > 
\section{Data processing inequality and reversibility of channels}

Let  $\gamma: \Ne\to \Me$ be a normal positive unital map. Then the  predual of $\gamma$  defines a 
positive linear map $\gamma_*: L_1(\Me)\to L_1(\Ne)$ that preserves the trace, acting as
\[
L_1(\Me)\ni h_\rho\mapsto h_{\rho\circ\gamma} \in L_1(\Ne).
\]
The support
of $\gamma$ will be denoted by $s(\gamma)$, recall that this is defined as the largest projection
$p\in \Ne$ such that $\gamma(p)=1$. For any $\rho\in \Me_*^+$ we clearly have
$s(\rho\circ\gamma)\le s(\gamma)$, with equality if $\rho$ is faithful. 
It follows that $\gamma_*$ maps $L_1(\Me)$ to $s(\gamma)L_1(\Ne)s(\gamma)\equiv
L_1(s(\gamma)\Ne s(\gamma))$.  For any $\rho\in \Me_+^*$, $\rho\ne 0$, the map
\[
s(\gamma)\Ne s(\gamma)\to s(\rho)\Me s(\rho),\qquad a\mapsto s(\rho) \gamma(a)s(\rho)
\]
is a faithful normal positive unital map, so using such  restrictions we may always assume that both $\rho$ and $\rho\circ
\gamma$ are faithful.

The Petz dual  of $\gamma$ with respect to a faithful  $\rho\in \Me_*^+$
is a map $\gamma_\rho^*:\Me\to \Ne$,
introduced in \cite{petz1988sufficiency}. It was proved that it is again
normal, positive and unital, in addition, it is $n$-positive whenever $\gamma$ is. 
As explained in \cite{jencova2018renyi} $\gamma^*_\rho$ is determined by the equality
\begin{equation}\label{eq:petzdual}
(\gamma^*_\rho)_*(h_{\rho\circ\gamma}^{\frac{1}{2}}bh_{\rho\circ\gamma}^{\frac{1}{2}})=h_\rho^{\frac{1}{2}}\gamma(b)h_\rho^{\frac{1}{2}},
\end{equation}
for all $b\in \Ne^+$, here $(\gamma^*_\rho)_*$ is the predual map of $\gamma^*_\rho$. We
also have
\[
(\gamma^*_\rho)_*(h_{\rho\circ\gamma})=(\gamma^*_\rho)_*\circ \gamma_*(h_\rho)=h_\rho
\]
and $(\gamma_\rho^*)_{\rho\circ\gamma}^*=\gamma$.

\subsection{Data processing inequality}


In this paragraph we prove the data processing inequality (DPI) for $D_{\alpha,z}$ with respect to normal
positive unital maps. In the case of the sandwiched divergences $\tilde D_\alpha$ with
$1/2\le \alpha \ne 1$, DPI was proved in \cite{jencova2018renyi,
jencova2021renyi}, see also \cite{berta2018renyi} for an alternative proof in the case
when the maps are  also completely positive.

\begin{lemma}\label{lemma:dpi} Let $\gamma:\Ne\to \Me$ be a normal positive unital map and
let $\rho\in \Me_*^+$, $b\in \Ne^+$. 
\begin{enumerate}
\item[(i)]  If $p\in [1/2,1)$, then 
\[
\|h_{\rho\circ\gamma}^{\frac{1}{2p}}bh_{\rho\circ\gamma}^{\frac{1}{2p}}\|_p\le
\|h_{\rho}^{\frac{1}{2p}}\gamma(b)h_{\rho}^{\frac{1}{2p}}\|_p.
\]

\item[(ii)]  If $p\in [1,\infty]$, the inequality reverses.

\end{enumerate}


\end{lemma}

\begin{proof} Let us denote $\beta:=\gamma_\rho^*$ and let $\omega\in \Me_*^+$ be such
that 
$h_\omega:=h_{\rho\circ\gamma}^{\frac{1}{2}}bh_{\rho\circ\gamma}^{\frac{1}{2}}\in L_1(\Ne)^+$. Then
$\beta$ is a normal positive unital map and  we have 
\[
\beta_*(h_\omega)=h_\rho^{\frac{1}{2}}\gamma(b)h_\rho^{\frac{1}{2}},\qquad
\beta_*(h_{\rho\circ\gamma})=h_\rho.
\]
Let $p\in [1/2,1)$, then  
\begin{align*}
\|h_{\rho}^{\frac{1}{2p}}\gamma(b)h_{\rho}^{\frac{1}{2p}}\|^p_p&=
\|h_\rho^{\frac{1-p}{2p}}\beta_*(h_\omega)h_\rho^{\frac{1-p}{2p}}\|_p^p=
Q_{p,p}(\beta_*(h_\omega)\|h_\rho)=Q_{p,p}(\beta_*(h_\omega)\|\beta_*(h_{\rho\circ\gamma}))\\
&\ge  Q_{p,p}(h_\omega\|h_{\rho\circ\gamma})=\|h_{\rho\circ\gamma}^{\frac{1-p}{2p}}h_\omega
h_{\rho\circ\gamma}^{\frac{1-p}{2p}}\|_p^p=\|h_{\rho\circ\gamma}^{\frac{1}{2p}}bh_{\rho\circ\gamma}^{\frac{1}{2p}}\|^p_p.
\end{align*}
Here we have used the DPI for the sandwiched R\'enyi  divergence $D_{\alpha,\alpha}$ for
$\alpha\in [1/2,1)$, \cite[Theorem 4.1]{jencova2021renyi}.  This proves (i). 
The case (ii) was proved in \cite{kato2023onrenyi} (see Eq. (22) therein), using the
relation of the sandwiched R\'enyi divergence to the Kosaki $L_p$ norms. In our setting,
the proof can be written as 
\begin{align*}
\|h_\rho^{\frac{1}{2p}}\gamma(b)h_\rho^{\frac{1}{2p}}\|_p^p&=Q_{p,p}(h_\rho^{\frac{1}{2}}\gamma(b)h_\rho^{\frac{1}{2}}\|h_\rho)=Q_{p,p}(\beta_*(h_\omega)\|\beta_*(h_{\rho\circ\gamma}))\\
&\le
Q_{p,p}(h_\omega\|h_{\rho\circ\gamma})=\|h_{\rho\circ\gamma}^{\frac{1}{2p}}bh_{\rho\circ\gamma}^{\frac{1}{2p}}\|^p_p,
\end{align*}
here the inequality follows from the DPI for sandwiched R\'enyi divergence
$D_{\alpha,\alpha}$ with
$\alpha>1$, \cite[]{jencova2018renyi}.


\end{proof}




\begin{theorem}[DPI] \label{thm:dpi} Let $\psi,\varphi\in \Me_*^+$, $\psi\ne 0$ and let $\gamma:
\Ne\to \Me$ be a normal positive unital map. Assume either of the following conditions:
\begin{enumerate}
\item[(i)] $0<\alpha<1$, $\max\{\alpha,1-\alpha\}\le z$
\item[(ii)] $\alpha>1$, $\max\{\alpha/2,\alpha-1\}\le z\le \alpha$.
\end{enumerate}
Then we have
\[
D_{\alpha,z}(\psi\circ\gamma\|\varphi\circ\gamma)\le D_{\alpha,z}(\psi\|\varphi).
\]


\end{theorem}


\begin{proof} Under the conditions (i), the DPI was proved in \cite[Theorem 1
(viii)]{kato2023onrenyi}.
%There an additional assumption was used, namely  that $\Ne$ is
%$\sigma$-finite, to construct a faithful map out of $\gamma$. This can be done using the
%restriction described above, so the additional condition is not needed. 
Since parts of this proof will be used below, we repeat it here.

 Assume the conditions in (i) and put $p:=\frac{z}{\alpha}$, $r:=\frac{z}{1-\alpha}$, so
 that  $p,r\ge 1$. 
For any   $b\in \Ne^{++}$, we have by  the Choi inequality \cite{choi1974aschwarz} 
that  $\gamma(b)^{-1}\le \gamma(b^{-1})$, so that  
\[
\|h_\varphi^{\frac{1}{2r}}\gamma(b)^{-1}\varphi^{\frac{1}{2r}}\|_r\le
\|h_\varphi^{\frac{1}{2r}}\gamma(b^{-1})\varphi^{\frac{1}{2r}}\|_r.
\]
Using the variational expression in Theorem \ref{thm:variational} (i), we have
\begin{align}\label{eq:dpif}
Q_{\alpha,z}(\psi\|\varphi)&\le \alpha\|h_\psi^{\frac{1}{2p}}\gamma(b)h_\psi^{\frac{1}{2p}}\|_p^p+
(1-\alpha)\|h_\varphi^{\frac{1}{2r}}\gamma(b)^{-1}h_\varphi^{\frac{1}{2r}}\|_r^r\\
&\le  \alpha\|h_\psi^{\frac{1}{2p}}\gamma(b)h_\psi^{\frac{1}{2p}}\|_p^p+
(1-\alpha)\|h_\varphi^{\frac{1}{2r}}\gamma(b^{-1})h_\varphi^{\frac{1}{2r}}\|_r^r\\
&  \alpha\|h_{\psi\circ\gamma}^{\frac{1}{2p}}bh_{\psi\circ\gamma}^{\frac{1}{2p}}\|_p^p+
(1-\alpha)\|h_{\varphi\circ\gamma}^{\frac{1}{2r}}b^{-1}h_{\varphi\circ\gamma}^{\frac{1}{2r}}\|_r^r,\label{eq:dpil}
\end{align}
here we used Lemma \ref{lemma:dpi} (ii) for the last inequality. Since this holds for all
$b\in \Ne^++$, it follows that $Q_{\alpha,z}(\psi\|\varphi)\le
Q_{\alpha}(\psi\circ\gamma\|\varphi\circ\gamma)$, which proves the DPI in this case.


Assume next the condition (ii), and put $p:=\frac{z}{\alpha}$, $q:=\frac{z}{\alpha-1}$, so
that $p\in [1/2,1)$ and $q\ge 1$. Using Theorem
\ref{thm:variational} (ii), we get for any $b\in \Ne^+$,
\begin{align*}
Q_{\alpha,z}(\psi\|\varphi)&\ge
\alpha\|h_\psi^{\frac{1}{2p}}\gamma(b)h_\psi^{\frac{1}{2p}}\|_p^p-
(\alpha-1)\|h_\varphi^{\frac{1}{2q}}\gamma(b)h_\varphi^{\frac{1}{2q}}\|_q^q\\
&\ge \alpha\|h_{\psi\circ\gamma}^{\frac{1}{2p}}bh_{\psi\circ\gamma}^{\frac{1}{2p}}\|_p^p-
(\alpha-1)\|h_{\varphi\circ\gamma}^{\frac{1}{2r}}bh_{\varphi\circ\gamma}^{\frac{1}{2q}}\|_q^q,
\end{align*}
here we used both (i) and (ii) in Lemma \ref{lemma:dpi}. Again, since this holds for all
$b\in \Ne^+$, we get the desired inequality.



\end{proof}



\subsection{Martingale convergence}




\subsection{Equality in DPI and reversibility of channels}

In what follows, a channel is a normal 2-positive unital map $\gamma: \Ne\to \Me$.

\begin{defi} Let $\gamma:\Ne\to \Me$ be a channel and let $\mathcal S \subset
\Me_*^+$. We say that $\gamma$ is reversible (or sufficient) with respect to $\mathcal S$
if there exists a channel $\beta:\Me\to \Ne$ such that
\[
\rho\circ\gamma\circ\beta=\rho,\qquad \forall \rho\in \mathcal S.
\]

\end{defi}

The notion of sufficient channels was introduced by Petz
\cite{petz1986sufficient,petz1988sufficiency}, who also obtained a number of conditions
characterizing this situation. It particular, it was proved in \cite{petz1988sufficiency}
that sufficient channels can be characterized by equality in DPI for the relative entropy
$D(\psi\|\varphi)$: if $D(\psi\|\varphi)<\infty$, then a channel $\gamma$ is sufficient
with respect to $\{\psi,\varphi\}$ if and only if 
\[
D(\psi\circ\gamma\|\varphi\circ\gamma)=D(\psi\|\varphi). 
\]
This result has been proved for a number of other divergence measures, including the
standard R\'enyi divergences $D_{\alpha,1}$ with $0<\alpha<2$ (\cite{}) and the sandwiched
R\'enyi divergences $D_{\alpha,\alpha}$ for $\alpha>1/2$
(\cite{jencova2018renyi,jencova2021renyi}).
Our aim in this section is to prove that a similar statement holds for $D_{\alpha,z}$ for
values of the parameters strictly contained in the DPI bounds of Theorem \ref{thm:dpi}. 

Throughout this section, we will assume that  $\psi,\varphi\in\Me_*^+$ are such that
$s(\psi)\le s(\varphi)$. As noted above, we may replace the channel $\gamma$ by its
restriction so that we may assume that both $\varphi$ and $\varphi_0:=\varphi\circ\gamma$
are faithful. 

Another important result of \cite{petz1988sufficiency} shows that the Petz dual $\gamma_\varphi^*$ is a universal
recovery map, in the sense given in the proposition below. 

\begin{prop}\label{prop:universal} Let $\varphi\in \Me_*^+$ be faithful and let 
$\gamma:\Ne\to \Me$ be a faithful channel. Then for any $\psi\in \Me_*^+$, $\gamma$ is
reversible with respect to $\{\psi,\varphi\}$ if and only if $\psi\circ\gamma\circ
\gamma_\varphi^*=\psi$.

Consequently, there is a faithful normal conditional expectation $\mathcal E$ on $\Me$
such that $\varphi\circ \mathcal E=\varphi$ and $\gamma$ is sufficient with respect to
$\{\psi,\varphi\}$ if and only if also $\psi\circ\mathcal E=\psi$.

\end{prop}

Note that the range of the conditional expectation $\mathcal E$ in the above proposition
is the set  of fixed points of the channel $\gamma\circ\gamma_\varphi^*$. 

\subsubsection{The case $\alpha\in (0,1)$}

\begin{theorem}\label{thm:suff<1} Let $0<\alpha<1$ and $\alpha,1-\alpha\le
z$ where at least one of the inequalities is strict. Let $\psi,\varphi\in \Me_*^+$ be
such that $s(\psi)\le s(\varphi)$.
Then $\gamma$ is reversible with respect to
$\{\psi,\varphi\}$ if and only if
\[
D_{\alpha,z}(\psi\|\varphi)=D_{\alpha,z}(\psi\circ\gamma\|\varphi\circ\gamma).
\]

\end{theorem}

\begin{proof} Let us denote $\psi_0:=\psi\circ\gamma$, $\varphi_0:=\varphi\circ\gamma$.
Using restrictions as before, we may assume that both $\varphi$ and $\varphi_0$ are
faithful.

We first treat the case when  $\lambda^{-1}\varphi\le \psi\le \lambda\varphi$ for
some $\lambda>0$, then
$\psi_0$ and $\varphi_0$ also satisfy this condition an all the states
$\psi,\varphi,\psi_0,\varphi_0$ are faithful. By Theorem \ref{thm:variational}
(i), there are  some $\bar a\in \Me^{++}$ and  $\bar a_0\in \Ne^{++}$ such that the
infimum in the variational formula for $D_{\alpha,z}(\psi\|\varphi)$ resp.
$D_{\alpha,z}(\psi_0\|\varphi_0)$ is attained. Using the inequalities in \eqref{eq:dpif} -
\eqref{eq:dpil}, we obtain
\begin{align*}
Q_{\alpha,z}(\psi\|\varphi)&\le \alpha\|h_\psi^{\frac{1}{2p}}\gamma(\bar a_0)h_\psi^{\frac{1}{2p}}\|_p^p+
(1-\alpha)\|h_\varphi^{\frac{1}{2r}}\gamma(\bar a_0)^{-1}h_\varphi^{\frac{1}{2r}}\|_r^r\\
&\le \alpha\|h_{\psi_0}^{\frac{1}{2p}}\bar a_0h_{\psi_0}^{\frac{1}{2p}}\|_p^p+
(1-\alpha)\|h_{\varphi_0}^{\frac{1}{2r}}\bar
a_0^{-1}h_{\varphi_0}^{\frac{1}{2r}}\|_r^r\\
&= Q_{\alpha,z}(\psi_0\|\varphi_0),
\end{align*}
where we again put $p=\frac{z}{\alpha}$, $r=\frac{z}{1-\alpha}$. Assume
$D_{\alpha,z}(\psi\|\varphi)=D_{\alpha,z}(\psi_0\|\varphi_0)$, then all the above
inequalities must be equalities. 

This has several consequences. First, by uniqueness of $\bar a$ in Theorem
\ref{thm:variational} (i), we have $\gamma(\bar a_0)=\bar a$. Furthermore, by Lemma
\ref{lemma:dpi} (ii), we obtain that 
\[
\|h_\psi^{\frac{1}{2p}}\gamma(\bar
a_0)h_\psi^{\frac{1}{2p}}\|_p^p=\|h_{\psi_0}^{\frac{1}{2p}}\bar
a_0h_{\psi_0}^{\frac{1}{2p}}\|_p^p,\qquad \|h_\varphi^{\frac{1}{2r}}\gamma(\bar
a_0)^{-1}h_\varphi^{\frac{1}{2r}}\|_r^r=\|h_{\varphi_0}^{\frac{1}{2r}}\bar
a_0^{-1}h_{\varphi_0}^{\frac{1}{2r}}\|_r^r.
\]
By the assumptions, at least one of $p$ and $r$ must be strictly larger than 1. Assume
that $r>1$ (the case $p>1$ is similar, even slightly easier). 
Since $h_\varphi^{\frac{1}{2r}}\gamma(\bar a_0)^{-1}h_\varphi^{\frac{1}{2r}} \le
h_\varphi^{\frac{1}{2r}}\gamma(\bar a_0^{-1})h_\varphi^{\frac{1}{2r}}$, Lemma
\ref{lemma:dpi} and the equality above imply that
\begin{equation}\label{eq:norms}
\|h_\varphi^{\frac{1}{2r}}\gamma(\bar a_0)^{-1}h_\varphi^{\frac{1}{2r}}\|_r^r =
\|h_\varphi^{\frac{1}{2r}}\gamma(\bar a_0^{-1})h_\varphi^{\frac{1}{2r}}\|_r^r=\|h_{\varphi_0}^{\frac{1}{2r}}\bar
a_0^{-1}h_{\varphi_0}^{\frac{1}{2r}}\|_r^r.
\end{equation}

Using \cite[Lemma 5.1]{fack1986generalized}, this shows that we must have 
\[
h_\varphi^{\frac{1}{2r}}\bar a^{-1}h_\varphi^{\frac{1}{2r}}=
h_\varphi^{\frac{1}{2r}}\gamma(\bar a_0)^{-1}h_\varphi^{\frac{1}{2r}}=
h_\varphi^{\frac{1}{2r}}\gamma(\bar a_0^{-1})h_\varphi^{\frac{1}{2r}}.
\]
Put $h_\omega:= h_\varphi^{\frac{1}{2}} \bar a^{-1} h_\varphi^{\frac{1}{2}}$,
$h_{\omega_0}:=h_{\varphi_0}^{\frac{1}{2}} \bar a_0^{-1} h_{\varphi_0}^{\frac{1}{2}}$.
Then we have
\begin{equation}\label{eq:omg}
(\gamma_\varphi^*)_* (h_{\omega_0})=h_\varphi^{\frac{1}{2}}\gamma(\bar
a_0^{-1})h_\varphi^{\frac{1}{2}}=h_\varphi^{\frac{1}{2}}\gamma(\bar
a_0)^{-1}h_\varphi^{\frac{1}{2}}=h_\varphi^{\frac{1}{2}}\bar a^{-1}h_\varphi^{\frac{1}{2}}=h_\omega.
\end{equation}

Using \eqref{eq:norms}, we obtain
\[
Q_{r,r}((\gamma_\varphi^*)_*(h_{\omega_0})\|(\gamma_\varphi^*)_*(h_{\varphi_0}))=\|h_\varphi^{\frac{1}{2r}}\gamma(\bar a_0^{-1})h_\varphi^{\frac{1}{2r}}\|_r^r=\|h_{\varphi_0}^{\frac{1}{2r}}\bar
a_0^{-1}h_{\varphi_0}^{\frac{1}{2r}}\|_r^r=Q_{r,r}(h_{\omega_0}\|h_{\varphi_0}),
\]
which by the properties of the sandwiched R\'enyi divergence \cite[Thm.
]{jencova2018renyi} implies that $\gamma_\varphi^*$ is sufficient with respect to
$\{\omega_0,\varphi_0\}$. By Proposition \ref{prop:universal} and the fact that the
Petz dual  $(\gamma_\varphi^*)_{\varphi_0}^*$ is $\gamma$ itself, this  is equivalent to 
\[
\gamma_*\circ (\gamma_\varphi^*)_*(h_{\omega_0})=h_{\omega_0},
\]
so that  by \eqref{eq:omg},
\[
(\gamma_\varphi^*)_*\circ \gamma_*(h_\omega)=(\gamma_\varphi^*)_*\circ
\gamma_*\circ
(\gamma_\varphi^*)_*(h_{\omega_0})=(\gamma_\varphi^*)_*(h_{\omega_0})=h_\omega.
\]
Hence $\gamma$ is sufficient with respect to $\{\omega,\varphi\}$. Let $\mathcal E$ be the
faithful normal conditional expectation as in Proposition \ref{prop:universal}. Then
$\mathcal E$ preserves both $h_\omega$ and $h_\varphi$, which by
\cite{junge2003noncommutative} implies that 
\[
h_\omega=\mathcal E_*(h_\omega)=h_\varphi^{\frac{1}{2}}\mathcal E(\bar a^{-1}) h_\varphi^{\frac{1}{2}},
\]
so that $\mathcal E(\bar a^{-1})=\bar a^{-1}$. It follows that 
\[
\left(h_\varphi^{\frac{1}{2r}}h_\psi^{\frac{1}{p}}h_\varphi^{\frac{1}{2r}}\right)^{1-\alpha}=
h_\varphi^{\frac{1}{2r}}\bar a^{-1}h_\varphi^{\frac{1}{2r}}\in L_r(\mathcal E(\Me))
\]
and consequently $|h_\psi^{\frac{1}{2p}}h_\varphi^{\frac{1}{2r}}|\in L_{2z}(\mathcal E(\Me))$.
Note that by the assumptions $2z>1$, so that we may use the multiplicativity properties
of the extension of $\mathcal E$ \cite{junge2003noncommutative}. Let 
\[
h_\psi^{\frac{1}{2p}}h_\varphi^{\frac{1}{2r}}=u|h_\psi^{\frac{1}{2p}}h_\varphi^{\frac{1}{2r}}|
\]
be the polar decomposition in $L_{2z}(\Me)$, then we have 
\[
u^*h_\psi^{\frac{1}{2p}}h_\varphi^{\frac{1}{2r}}=\mathcal
E_{2z}(u^*h_\psi^{\frac{1}{2p}}h_\varphi^{\frac{1}{2r}})=\mathcal
E_{2p}(u^*h_\psi^{\frac{1}{2p}})h_\varphi^{\frac{1}{2r}},
\]
which implies that 
\[
\mathcal E_p(h_\psi^{\frac{1}p})=\mathcal
E_p(h_\psi^{\frac{1}{2p}}uu^*h_\psi^{\frac{1}{2p}})=h_\psi^{\frac{1}{2p}}uu^*h_\psi^{\frac{1}{2p}}=h_\psi^{\frac{1}{p}}
\]
Consequently, $\psi\circ\mathcal E=\psi$ and $\gamma$ is sufficient with respect to
$\{\psi,\varphi\}$. 


\end{proof}

\subsubsection{The case $\alpha>1$}


\DIFdelbegin \DIFdel{////////////////////////
}\DIFdelend \DIFaddbegin \appendix
\DIFaddend 

\section{\DIFdelbegin \DIFdel{Monotonicity in the parameter $z$}\DIFdelend \DIFaddbegin \DIFadd{Haagerup  $L_p$-spaces}\DIFaddend }


\DIFdelbegin \DIFdel{It is well known\mbox{%DIFAUXCMD
\cite{berta2018renyi,hiai2018quantum,jencova2018renyi} }\hskip0pt%DIFAUXCMD
that the standard R\'enyi divergence
$D_{\alpha,1}(\psi\|\ffi)$ is monotone increasing in $\alpha\in(0,1)\cup(1,\infty)$ and the sandwiched R\'enyi
divergence $D_{\alpha,\alpha}(\psi\|\ffi)$ is monotone increasing in $\alpha\in[1/2,1)\cup(1,\infty)$. It is also
known \mbox{%DIFAUXCMD
\cite{berta2018renyi,hiai2018quantum,jencova2018renyi} }\hskip0pt%DIFAUXCMD
that
}\[
\DIFdel{\lim_{\alpha\nearrow1}D_{\alpha,1}(\psi\|\ffi)=\lim_{\alpha\nearrow1}D_{\alpha,\alpha}(\psi\|\ffi)
=D_1(\psi\|\ffi),
}\]%DIFAUXCMD
\DIFdel{and if $D_{\alpha,1}(\psi\|\ffi)<\infty$ (resp., $D_{\alpha,\alpha}(\psi\|\ffi)<\infty$) for some $\alpha>1$, then
}\[
\DIFdel{\lim_{\alpha\searrow1}D_{\alpha,1}(\psi\|\ffi)=D_1(\psi\|\ffi)\quad
\Bigl(\mbox{resp.,}\ \lim_{\alpha\searrow1}D_{\alpha,\alpha}(\psi\|\ffi)=D_1(\psi\|\ffi)\Bigr).
}\]%DIFAUXCMD
\DIFdel{In the rest of the paper we will discuss similar monotonicity properties and limits for $D_{\alpha,z}(\psi\|\ffi)$.
We consider monotonicity in the parameter $z$ in Sec.~4 and monotonicity in the parameter $\alpha$ in Sec.~5.
}\DIFdelend \DIFaddbegin \DIFadd{The following lemmas are well known, proofs are given for completeness.
}\DIFaddend 

\DIFdelbegin \subsection{\DIFdel{The finite von Neumann algebra case}}
%DIFAUXCMD
\addtocounter{subsection}{-1}%DIFAUXCMD
%DIFDELCMD < 

%DIFDELCMD < %%%
\DIFdel{Assume that $(\Me,\tau)$ is a semi-finite von Neumann algebra $\Me$ with a faithful normal semi-finite trace
$\tau$. Then the Haagerup $L_p$-space $L_p(\Me)$ is identified with the $L_p$-space $L_p(\Me,\tau)$ }\DIFdelend \DIFaddbegin \begin{lemma}\label{lemma:cone} \DIFadd{For any $0<p<\infty$ and $\varphi\in \Me_*^+$, 
$h_\varphi^{\frac{1}{2p}}\Me^+h_\varphi^{\frac{1}{2p}}$ is dense in $L_p(\Me)^+$ }\DIFaddend with respect
to \DIFdelbegin \DIFdel{$\tau$ \mbox{%DIFAUXCMD
\cite[Example 9.11]{hiai2021lectures}}\hskip0pt%DIFAUXCMD
. Hence one can define $Q_{\alpha,z}(\psi\|\ffi)$ for
$\psi,\ffi\in\Me_*^+$ by replacing, in Definition \ref{defi:renyi}, $L_p(\Me)$ with $L_p(\Me,\tau)$ and
$h_\psi\in L_1(\Me)_+$ with the Radon--Nikodym derivative $d\psi/d\tau\in L_1(\Me,\tau)^+$. Below we use the
symbol $h_\psi$ to denote $d\psi/d\tau$ as well. Note that $\tau$ on $\Me_+$ is naturally extended to the positive
part $\widetilde\Me^+$ of the space $\widetilde\Me$ of $\tau$-measurable operators. We then have 
\mbox{%DIFAUXCMD
\cite[Proposition 4.20]{hiai2021lectures}
}\hskip0pt%DIFAUXCMD
}\begin{align*}\DIFdel{%DIFDELCMD < \label{F-4.1}%%%
\tau(a)=\int_0^\infty\mu_s(a)\,ds,\qquad a\in\widetilde\Me^+,
}\end{align*}%DIFAUXCMD
\DIFdel{where $\mu_s(a)$ is the generalized $s$-number of $a$ \mbox{%DIFAUXCMD
\cite{fack1986generalized}}\hskip0pt%DIFAUXCMD
.
}\DIFdelend \DIFaddbegin \DIFadd{the (quasi)-norm $\|\cdot\|_p$.
}\DIFaddend 

\DIFdelbegin \DIFdel{Throughout this subsection we assume that $\Me$ is a finite von Neumann algebra with a faithful normal finite
trace $\tau$; then $\widetilde\Me^+$ consists of all positive self-adjoint operators affiliated with $\Me$.
}%DIFDELCMD < 

%DIFDELCMD < \begin{lemma}\label{L-4.1}
%DIFDELCMD < %%%
\DIFdel{For every $\psi,\ffi\in\Me_*^+$ with $\psi\ne0$ and for any $\alpha,z>0$ with $\alpha\ne1$,
}\begin{align*}\DIFdel{%DIFDELCMD < \label{F-4.2}%%%
D_{\alpha,z}(\psi\|\ffi)}&\DIFdel{=\lim_{\eps\searrow0}D_{\alpha,z}(\psi\|\ffi+\eps\tau)\quad\mbox{increasingly},
}\end{align*}%DIFAUXCMD
\DIFdel{and hence $D_{\alpha,z}(\psi\|\ffi)=\sup_{\eps>0}D_{\alpha,z}(\psi\|\ffi+\eps\tau)$.
}\DIFdelend \end{lemma}

\begin{proof} \DIFdelbegin %DIFDELCMD < {\it %%%
\DIFdel{Case $0<\alpha<1$}%DIFDELCMD < }%%%
\DIFdel{.}%DIFDELCMD < \enspace
%DIFDELCMD < %%%
\DIFdel{We need to prove that
}\begin{align*}\DIFdel{%DIFDELCMD < \label{F-4.3}%%%
Q_{\alpha,z}(\psi\|\ffi)}&\DIFdel{=\lim_{\eps\searrow0}Q_{\alpha,z}(\psi\|\ffi+\eps\tau)\quad\mbox{decreasingly}.
}\end{align*}%DIFAUXCMD
\DIFdel{In the present setting we have by \eqref{F-4.1}
}\begin{align*}\DIFdel{%DIFDELCMD < \label{F-4.4}%%%
Q_{\alpha,z}(\psi\|\ffi)
=\tau\Bigl(\bigl(h_\psi^{\alpha/2z}h_\ffi^{1-\alpha\over z}h_\psi^{\alpha/2z}\bigr)^z\Bigr)
=\int_0^\infty\mu_s\bigl(h_\psi^{\alpha/2z}h_\ffi^{1-\alpha\over z}h_\psi^{\alpha/2z}\bigr)^z\,ds,
}\end{align*}%DIFAUXCMD
\DIFdel{and similarly
}\[
\DIFdel{Q_{\alpha,z}(\psi\|\ffi+\eps\tau)
=\int_0^\infty\mu_s\bigl(h_\psi^{\alpha/2z}h_{\ffi+\eps\tau}^{1-\alpha\over z}h_\psi^{\alpha/2z}\bigr)^z\,ds.
}\]%DIFAUXCMD
\DIFdel{Since $h_{\ffi+\eps\tau}^{1-\alpha\over z}=(h_\ffi+\eps\1)^{1-\alpha\over z}$ decreases to
$h_\ffi^{1-\alpha\over z}$ in the measure topology as $\eps\searrow0$, 
it follows that
$h_\psi^{\alpha/2z}h_{\ffi+\eps\tau}^{1-\alpha\over z}h_\psi^{\alpha/2z}$ decreases to
$h_\psi^{\alpha/2z}h_\ffi^{1-\alpha\over z}h_\psi^{\alpha/2z}$ in the measure topology. Hence by
\mbox{%DIFAUXCMD
\cite[Lemma 3.4]{fack1986generalized} }\hskip0pt%DIFAUXCMD
we have
$\mu_s\bigl(h_\psi^{\alpha/2z}h_{\ffi+\eps\tau}^{1-\alpha\over z}h_\psi^{\alpha/2z}\bigr)
\searrow\mu_s\bigl(h_\psi^{\alpha/2z}h_\ffi^{1-\alpha\over z}h_\psi^{\alpha/2z}\bigr)$
as $\eps\searrow0$ for almost every $s>0$. Since
$s\mapsto\mu_s\bigl(h_\psi^{\alpha/2z}h_{\ffi+\tau}^{1-\alpha\over z}h_\psi^{\alpha/2z}\bigr)$ is integrable on $(0,\infty)$, the Lebesgue convergence theorem gives \eqref{F-4.3}.
}%DIFDELCMD < 

%DIFDELCMD < {\it %%%
\DIFdel{Case $\alpha>1$}%DIFDELCMD < }%%%
\DIFdel{.}%DIFDELCMD < \enspace
%DIFDELCMD < %%%
\DIFdel{We need to prove that
}\begin{align*}\DIFdel{%DIFDELCMD < \label{F-4.5}%%%
Q_{\alpha,z}(\psi\|\ffi)}&\DIFdel{=\lim_{\eps\searrow0}Q_{\alpha,z}(\psi\|\ffi+\eps\tau)\quad\mbox{increasingly}.
}\end{align*}%DIFAUXCMD
\DIFdel{For any $\eps>0$, since $h_{\ffi+\eps\tau}=h_\psi+\eps\1$ has the bounded inverse
$h_{\ffi+\eps\tau}^{-1}=(h_\ffi+\eps\1)^{-1}\in\Me^+$, one can define
$x_\eps:=(h_\ffi+\eps\1)^{-{\alpha-1\over2z}}h_\psi^{\alpha/z}(h_\ffi+\eps\1)^{-{\alpha-1\over2z}}
\in\widetilde\Me^+$ so that
}\[
\DIFdel{h_\psi^{\alpha/z}=(h_\ffi+\eps\1)^{\alpha-1\over2z}x_\eps(h_\ffi+\eps\1)^{\alpha-1\over2z}.
}\]%DIFAUXCMD
\DIFdel{In the present setting one can write by \eqref{F-4.1}
}\begin{align*}\DIFdel{%DIFDELCMD < \label{F-4.6}%%%
Q_{\alpha,z}(\psi\|\ffi+\eps\tau)=\tau(x_\eps^z)=\int_0^\infty\mu_s(x_\eps)^z\,ds\ (\in[0,\infty]).
}\end{align*}%DIFAUXCMD
\DIFdelend \DIFaddbegin \DIFadd{We may assume that $\varphi$ is faithful. By \mbox{%DIFAUXCMD
\cite[Lemma 1.1]{junge2003noncommutative}}\hskip0pt%DIFAUXCMD
, $\Me
h_\varphi^{\frac{1}{2p}}$ is dense in $L_{2p}(\Me)$ for any $0<p<\infty$. }\DIFaddend Let \DIFdelbegin \DIFdel{$0<\eps\le\eps'$. Since $(h_\ffi+\eps\1)^{-{\alpha-1\over z}}\ge(h_\ffi+\eps'\1)^{-{\alpha-1\over z}}$,
one has $\mu_s(x_\eps)\ge\mu_s(x_{\eps'})$ for all $s>0$, so that
}\[
\DIFdel{Q_{\alpha,z}(\psi\|\ffi+\eps\tau)\ge Q_{\alpha,z}(\psi\|\ffi+\eps'\tau).
}\]%DIFAUXCMD
\DIFdel{Hence $\eps>0\mapsto D_{\alpha,z}(\psi\|\ffi+\eps\tau)$ is decreasing.
}%DIFDELCMD < 

%DIFDELCMD < %%%
\DIFdel{First, assume that $s(\psi)\not\le s(\ffi)$. Then
$\mu_{s_0}(h_\psi^{\alpha/2z}s(\ffi)^\perp h_\psi^{\alpha/2z})>0$ for some $s_0>0$; indeed, otherwise,
$h_\psi^{\alpha/2z}s(\ffi)^\perp h_\psi^{\alpha/2z}=0$ so that $s(\psi)\le s(\ffi)$. Hence we have
}\[
\DIFdel{\mu_s(x_\eps)=\mu_s\bigl(h_\psi^{\alpha/2z}(h_\ffi+\eps\1)^{-{\alpha-1\over z}}h_\psi^{\alpha/2z}\bigr)
\ge\eps^{-{\alpha-1\over z}}\mu_s(h_\psi^{\alpha/2z}s(\ffi)^\perp h_\psi^{\alpha/2z})
\nearrow\infty\quad\mbox{as $\eps\searrow0$}
}\]%DIFAUXCMD
\DIFdel{for all $s\in(0,s_0]$. Therefore, it follows from \eqref{F-4.6} that
$Q_{\alpha,z}(\psi\|\ffi+\eps\tau)\nearrow\infty=Q_{\alpha,z}(\psi\|\ffi)$.
}%DIFDELCMD < 

%DIFDELCMD < %%%
\DIFdel{Next, assume that $s(\psi)\le s(\ffi)$. Take the spectral decomposition $h_\ffi=\int_0^\infty t\,de_t$ and
define $y,x\in\widetilde\Me_+$ by
}\[
\DIFdel{y:=h_\ffi^{-{\alpha-1\over z}}s(\ffi)=\int_{(0,\infty)}t^{-{\alpha-1\over z}}\,de_t,
\qquad x:=y^{1/2}h_\psi^{\alpha/z}y^{1/2}.
}\]%DIFAUXCMD
\DIFdel{Since
}\[
\DIFdel{h_\psi^{\alpha/z}=s(\ffi)h_\psi^{\alpha/z}s(\ffi)
=h_\ffi^{\alpha-1\over2z}y^{1/2}h_\psi^{\alpha/z}y^{1/2}h_\ffi^{\alpha-1\over2z}
=h_\ffi^{\alpha-1\over2z}xh_\ffi^{\alpha-1\over2z},
}\]%DIFAUXCMD
\DIFdel{one has, similarly to \ref{F-4.6},
}\begin{align*}\DIFdel{%DIFDELCMD < \label{F-4.7}%%%
Q_{\alpha,z}(\psi\|\ffi)=\tau(x^z)=\int_0^\infty\mu_s(x)^z\,ds.
}\end{align*}%DIFAUXCMD
\DIFdel{We write $(h_\ffi+\eps\1)^{-{\alpha-1\over z}}s(\ffi)=\int_{(0,\infty)}(t+\eps)^{-{\alpha-1\over z}}\,de_t$,
and for any $\delta>0$ choose a $t_0>0$ }\DIFdelend \DIFaddbegin \DIFadd{$y\in L_p(\Me)^+$, then
$y^{\frac{1}{2}}\in L_{2p}(\Me)$, hence there is a sequence $a_n\in \Me$ }\DIFaddend such that
\DIFdelbegin \DIFdel{$\tau(e_{(0,t_0)})<\delta$. Then, since
$\int_{[t_0,\infty)}(t+\eps)^{-{\alpha-1\over z}}\,de_t\to\int_{[t_0,\infty)}t^{-{\alpha-1\over z}}\,de_t$
in the operator norm as $\eps\searrow0$, we obtain $(h_\ffi+\eps\1)^{-{\alpha-1\over z}}s(\ffi)\nearrow y$
in the measure topology (see \mbox{%DIFAUXCMD
\cite[1.5]{fack1986generalized}}\hskip0pt%DIFAUXCMD
), so that
$h_\psi^{\alpha/2z}(h_\ffi+\eps\1)^{-{\alpha-1\over z}}h_\psi^{\alpha/2z}
\nearrow h_\psi^{\alpha/2z}yh_\psi^{\alpha/2z}$ in the measure topology as $\eps\searrow0$.
Hence
we have by \mbox{%DIFAUXCMD
\cite[Lemma 3.4]{fack1986generalized}
}\hskip0pt%DIFAUXCMD
}\begin{align*}\DIFdel{%DIFDELCMD < \label{F-4.8}%%%
\mu_s(x_\eps)=\mu_s\bigl(h_\psi^{\alpha/2z}(h_\ffi+\eps\1)^{-{\alpha-1\over z}}h_\psi^{\alpha/2z}\bigr)
\nearrow\mu_s(h_\psi^{\alpha/2z}yh_\psi^{\alpha/2z})=\mu_s(x)
}\end{align*}%DIFAUXCMD
\DIFdelend \DIFaddbegin \DIFadd{$\|a_nh^{\frac{1}{2p}}_\varphi-y^{\frac{1}{2}}\|_{2p}\to 0$. Then also 
}\[
\DIFadd{\|h^{\frac{1}{2p}}_\varphi
a_n^*-y^{\frac{1}{2}}\|_p=\|(a_nh^{\frac{1}{2p}}_\varphi-y^{\frac{1}{2}})^*\|_p=\|a_nh^{\frac{1}{2p}}_\varphi-y^{\frac{1}{2}}\|_p\to 0
}\]\DIFaddend 
\DIFdelbegin \DIFdel{for all $s>0$. Therefore, by \eqref{F-4.6} and \eqref{F-4.7} the monotone convergence theorem gives
\eqref{F-4.5}.
}%DIFDELCMD < \end{proof}
%DIFDELCMD < 

%DIFDELCMD < \begin{lemma}\label{L-4.2}
%DIFDELCMD < %%%
\DIFdel{Let $(\Me,\tau)$ }\DIFdelend and 
\DIFdelbegin \DIFdel{$\psi,\ffi$ be as above, and let $0<z\le z'$. Then
}\[
\DIFdel{\begin{cases}
D_{\alpha,z}(\psi\|\ffi)\le D_{\alpha,z'}(\psi\|\ffi), & \text{$0<\alpha<1$},\\
D_{\alpha,z}(\psi\|\ffi)\ge D_{\alpha,z'}(\psi\|\ffi), & \text{$\alpha>1$}.
\end{cases}
}\]%DIFAUXCMD
\DIFdelend \DIFaddbegin \[
\DIFadd{\|h^{\frac{1}{2p}}_\varphi a_n^*a_nh^{\frac{1}{2p}}_\varphi-y\|_p=\|(h^{\frac{1}{2p}}_\varphi
a_n^*-y^{\frac{1}{2}})a_nh^{\frac{1}{2p}}_\varphi+y^{\frac{1}{2}}(a_nh^{\frac{1}{2p}}_\varphi-y^{\frac{1}{2}})\|_p
}\]\DIFaddend 
\DIFdelbegin %DIFDELCMD < \end{lemma}
%DIFDELCMD < 

%DIFDELCMD < \begin{proof}
%DIFDELCMD < %%%
\DIFdel{The case $0<\alpha<1$ was shown in \mbox{%DIFAUXCMD
\cite[Theorem 1(x)]{kato2023onrenyi} }\hskip0pt%DIFAUXCMD
for general von Neumann algebras.
For the case $\alpha>1$, by Lemma \ref{L-4.1} it suffices to show that , for every $\eps>0$,
}\[
\DIFdel{\tau\Bigl(\Bigl(y_\eps^{\alpha-1\over2z}h_\psi^{\alpha/z}y_\eps^{\alpha-1\over2z}\Bigr)^z\Bigr)
\ge\tau\Bigl(\Bigl(y_\eps^{\alpha-1\over2z'}h_\psi^{\alpha/z'}y_\eps^{\alpha-1\over2z'}\Bigr)^z\Bigr),
}\]%DIFAUXCMD
\DIFdel{where $y_\eps:=(h_\ffi+\eps\1)^{-1}\in\Me_+$. The above is equivalently written as
}\[
\DIFdel{\tau\Bigl(\big|(h_\psi^{\alpha/2z'})^r(y^{(\alpha-1)/2z'})^r\big|^{2z}\Bigr)
\ge\tau\Bigl(\big|h_\psi^{\alpha/2z'}y^{(\alpha-1)/2z'}\big|^{2zr}\Bigr),
}\]%DIFAUXCMD
\DIFdel{where $r:=z'/z\ge1$. Hence the desired inequality follows from Kosaki's ALT inequality
\mbox{%DIFAUXCMD
\cite[Corollary 3]{kosaki1992aninequality}}\hskip0pt%DIFAUXCMD
.
}%DIFDELCMD < \end{proof}
%DIFDELCMD < 

%DIFDELCMD < %%%
\DIFdel{When $(\Me,\tau)$ and $\psi,\ffi$ are as in Lemma \ref{L-4.1}, one can define, thanks to Lemma \ref{L-4.2},
for any $\alpha\in(0,\infty)\setminus\{1\}$,
}\begin{align*}
\DIFdel{Q_{\alpha,\infty}(\psi\|\ffi)}&\DIFdel{:=\lim_{z\to\infty}Q_{\alpha,\infty}(\psi\|\ffi)
=\inf_{z>0}Q_{\alpha,z}(\psi\|\ffi), \nonumber}\\
\DIFdel{D_{\alpha,\infty}(\psi\|\ffi)}&\DIFdel{:=}{\DIFdel{1\over\alpha-1}}\DIFdel{\log{Q_{\alpha,\infty}(\psi\|\ffi)\over\psi(\1)} \nonumber}\\
&\DIFdel{\ =\lim_{z\to\infty}D_{\alpha,z}(\psi\|\ffi)
=\begin{cases}\sup_{z>0}D_{\alpha,z}(\psi\|\ffi), & \text{$0<\alpha<1$},\\
\inf_{z>0}D_{\alpha,z}(\psi\|\ffi), & \text{$\alpha>1$}.\end{cases} %DIFDELCMD < \label{F-4.9}%%%
}\end{align*}%DIFAUXCMD
\DIFdel{If $h_\psi,h_\ffi\in\Me^{++}$ (i. e., $\delta\tau\le\psi,\ffi\le\delta^{-1}\tau$ for some $\delta\in(0,1)$), then
the Lie--Trotter formula gives
}\begin{align*}\DIFdel{%DIFDELCMD < \label{F-4.10}%%%
Q_{\alpha,\infty}(\psi\|\ffi)=\tau\bigl(\exp(\alpha\log h_\psi+(1-\alpha)\log h_\ffi)\bigr).
}\end{align*}%DIFAUXCMD
%DIFDELCMD < 

%DIFDELCMD < \begin{lemma}\label{L-4.3}
%DIFDELCMD < %%%
\DIFdel{Let $(\Me,\tau)$ and $\psi,\ffi$ be as above. Then for any $z>0$, }\[
\DIFdel{\begin{cases}
D_{\alpha,z}(\psi\|\ffi)\le D_1(\psi\|\ffi), & \text{$0<\alpha<1$},\\
D_{\alpha,z}(\psi\|\ffi)\ge D_1(\psi\|\ffi), & \text{$\alpha>1$}.
\end{cases}
}\]%DIFAUXCMD
%DIFDELCMD < \end{lemma}
%DIFDELCMD < 

%DIFDELCMD < \begin{proof}
%DIFDELCMD < %%%
\DIFdel{First, assume that $h_\psi,h_\ffi\in\Me^{++}$. Set self-adjoint $H:=\log h_\psi$ and $K:=\log h_\ffi$ in $\Me$
and define $F(\alpha):=\log\tau\bigl(e^{\alpha H+(1-\alpha)K}\bigr)$ for $\alpha>0$.
Then by \eqref{F-4.10}, $F(\alpha)=\log Q_{\alpha,\infty}(\psi\|\ffi)$ for all $\alpha\in(0,\infty)\setminus\{1\}$,
and we compute
}\begin{align*}
\DIFdel{F'(\alpha)}&\DIFdel{=}{\DIFdel{\tau\bigl(e^{\alpha H+(1-\alpha)K}(H-K)\bigr)\over\tau\bigl(e^{\alpha H+(1-\alpha)K}\bigr)}}\DIFdel{, }\\
\DIFdel{F''(\alpha)}&\DIFdel{=}{\DIFdel{\bigl\{\tau\bigl(e^{\alpha H+(1-\alpha)K}(H-K)\bigr)\bigr\}^2
-\tau\bigl(e^{\alpha H+(1-\alpha)K}(H-K)^2\bigr)\over\bigl\{\tau\bigl(e^{\alpha H+(1-\alpha)K}\bigr)\bigr\}^2}}\DIFdel{.
}\end{align*}%DIFAUXCMD
\DIFdelend Since \DIFdelbegin \DIFdel{$F''(\alpha)\ge0$ on $(0,\infty)$ thanks to the Schwarz inequality, we see that $F(\alpha)$ is
convex on $(0,\infty)$ and hence }\[
\DIFdel{D_{\alpha,\infty}(\psi\|\ffi)=}{\DIFdel{F(\alpha)-F(1)\over\alpha-1}}
\DIFdel{}\]%DIFAUXCMD
\DIFdel{is increasing in $\alpha\in(0,\infty)$, where for $\alpha=1$ the above RHS is understood as
}\[
\DIFdel{F'(1)=}{\DIFdel{\tau(e^H(H-K))\over\tau(e^H)}}\DIFdel{=}{\DIFdel{\tau\bigl(h_\psi(\log h_\psi-\log h_\ffi)\bigr)\over\tau(h_\psi)}}
\DIFdel{=D_1(\psi\|\ffi).
}\]%DIFAUXCMD
\DIFdel{Hence by \eqref{F-4.9} the assertion holds when $h_\psi,h_\ffi\in\Me^{++}$. Below we extend it to general
$\psi,\ffi\in\Me_*^+$}\DIFdelend \DIFaddbegin \DIFadd{$\|\cdot\|_p$ is a (quasi)-norm, the above expression goes to 0 by the H\"older
inequality}\DIFaddend .



\DIFdelbegin %DIFDELCMD < {\it %%%
\DIFdel{Case $0<\alpha<1$}%DIFDELCMD < }%%%
\DIFdel{.}%DIFDELCMD < \enspace
%DIFDELCMD < %%%
\DIFdel{Let $\psi,\ffi\in\Me_*^+$ and $z>0$. From \mbox{%DIFAUXCMD
\cite[Theorem 1(iv)]{kato2023onrenyi} }\hskip0pt%DIFAUXCMD
and
\mbox{%DIFAUXCMD
\cite[Corollary 2.8(3)]{hiai2021quantum} }\hskip0pt%DIFAUXCMD
we have
}\begin{align*}
\DIFdel{D_{\alpha,z}(\psi\|\ffi)}&\DIFdel{=\lim_{\eps\searrow0}D_{\alpha,z}(\psi+\eps\tau\|\ffi+\eps\tau), }\\
\DIFdel{D_1(\psi\|\ffi)}&\DIFdel{=\lim_{\eps\searrow0}D_1(\psi+\eps\tau\|\ffi+\eps\tau),
}\end{align*}%DIFAUXCMD
\DIFdel{so that
we may assume that $\psi,\ffi\ge\eps\tau$ for some $\eps>0$. Take the spectral decompositions
$h_\psi=\int_0^\infty t\,de_t^\psi$ and $h_\ffi=\int_0^\infty t\,de_t^\ffi$, and define
$e_n:=e_n^\psi\wedge e_n^\ffi$ for each $n\in\bN$. Then $\tau(e_n^\perp)\le\tau((e_n^\psi)^\perp)+\tau((e_n^\ffi)^\perp)\to0\quad\mbox{as $n}%DIFDELCMD < \to%%%
\DIFdel{\infty$}$,  so that
$e_n\nearrow\1$. We set $\psi_n:=\psi(e_n\cdot e_n)$ and $\ffi_n:=\ffi(e_n\cdot e_n)$; then
$h_{\psi_n}=e_nh_\psi e_n$ and $h_{\ffi_n}=e_nh_\ffi e_n$ are in $(e_n\Me e_n)^{++}$. Note that
}\begin{align*}
\DIFdel{\|h_\psi-e_nh_\psi e_n\|_1}&\DIFdel{\le\|(\1-e_n)h_\psi\|_1+\|e_nh_\psi(\1-e_n)\|_1 }\\
&\DIFdel{\le\|(\1-e_n)h_\psi^{1/2}\|_2\|h_\psi^{1/2}\|_2+\|e_nh_\psi^{1/2}\|_2\|h_\psi^{1/2}(\1-e_n)\|_2 }\\
&\DIFdel{=\psi(\1-e_n)^{1/2}\psi(\1)^{1/2}+\psi(e_n)^{1/2}\psi(\1-e_n)^{1/2}\to0\quad\mbox{as $n\to\infty$},
}\end{align*}%DIFAUXCMD
\DIFdel{and similarly $\|h_\ffi-e_nh_\ffi e_n\|_1\to0$. Hence by \mbox{%DIFAUXCMD
\cite[Theorem 1(iv)]{kato2023onrenyi} }\hskip0pt%DIFAUXCMD
one has
$D_{\alpha,z}(e_n\psi e_n\|e_n\ffi e_n)\to D_{\alpha,z}(\psi\|\ffi)$. On the other hand, one has
$D_1(e_n\psi e_n\|e_n\ffi e_n)\to D_1(\psi\|\ffi)$ by \mbox{%DIFAUXCMD
\cite[Proposition 2.10]{hiai2021quantum}}\hskip0pt%DIFAUXCMD
. Since
$D_{\alpha,z}(e_n\psi e_n\|e_n\ffi e_n)\le D_1(e_n\psi e_n\|e_n\ffi e_n)$ holds by regarding
$e_n\psi e_n,e_n\ffi e_n$ as functionals on the reduced von Neumann algebra $e_n\Me e_n$, we obtain
the desired inequality for general $\psi,\ffi\in\Me_*^+$.
}%DIFDELCMD < 

%DIFDELCMD < {\it %%%
\DIFdel{Case $\alpha>1$}%DIFDELCMD < }%%%
\DIFdel{.}%DIFDELCMD < \enspace
%DIFDELCMD < %%%
\DIFdel{We show the extension to general $\psi,\ffi\in\Me_*^+$ by dividing four steps as follows, where
$h_\psi=\int_0^\infty t\,e_t^\psi$ and $h_\ffi=\int_0^\infty t\,de_t^\ffi$ are the spectral decompositions.
}%DIFDELCMD < 

%DIFDELCMD < %%%
\DIFdel{(1)}%DIFDELCMD < \enspace
%DIFDELCMD < %%%
\DIFdel{Assume that $h_\psi\in\Me^+$ and $h_\ffi\in\Me^{++}$. Set $\psi_n\in\Me_*^+$ by
$h_{\psi_n}=(1/n)e_{[0,1/n]}^\psi+\int_{(1/n,\infty)}t\,de_t^\psi$ ($\in\Me^{++}$). Since
$h_{\psi_n}^{\alpha/z}\searrow h_\psi^{\alpha/z}$ in the operator norm, we have by \eqref{F-4.4} and
\mbox{%DIFAUXCMD
\cite[Lemma 3.4]{fack1986generalized}
}\hskip0pt%DIFAUXCMD
}\begin{displaymath}\DIFdel{%DIFDELCMD < \label{F-4.11}%%%
\begin{aligned}
Q_{\alpha,z}(\psi\|\ffi)&=\int_0^\infty\mu_s\bigl((h_\ffi^{-1})^{\alpha-1\over2z}h_\psi^{\alpha/z}
(h_\ffi^{-1})^{\alpha-1\over2z}\bigr)^z\,ds \\
&=\lim_{n\to\infty}\int_0^\infty\mu_s\bigl((h_\ffi^{-1})^{\alpha-1\over2z}h_{\psi_n}^{\alpha/z}
(h_\ffi^{-1})^{\alpha-1\over2z}\bigr)^z\,ds
=\lim_{n\to\infty}Q_{\alpha,z}(\psi_n\|\ffi).
\end{aligned}
}\end{displaymath}%DIFAUXCMD
\DIFdel{From this and the lower semicontinuity of $D_1$ the extension holds in this case.
}%DIFDELCMD < 

%DIFDELCMD < %%%
\DIFdel{(2)}%DIFDELCMD < \enspace
%DIFDELCMD < %%%
\DIFdel{Assume that $h_\psi\in\Me^+$ and $h_\ffi\ge\delta\1$ for some $\delta>0$. Set $\ffi_n\in\Me_*^+$
by $h_{\ffi_n}=\int_{[\delta,n]}t\,de_t^\ffi+ne_{(n,\infty)}^\ffi$ ($\in\Me^{++}$). Since $h_{\ffi_n}^{-{\alpha-1\over z}}\searrow h_\ffi^{-{\alpha-1\over z}}$ in the operator norm, we have by
\eqref{F-4.4} and \mbox{%DIFAUXCMD
\cite[Lemma 3.4]{fack1986generalized} }\hskip0pt%DIFAUXCMD
again
}\begin{align*}
\DIFdel{Q_{\alpha,z}(\psi\|\ffi)}&\DIFdel{=\int_0^\infty\mu_s\bigl(h_\psi^{\alpha/2z}h_\ffi^{-{\alpha-1\over z}}
h_\psi^{\alpha/2z}\bigr)^z\,ds }\\
&\DIFdel{=\lim_{n\to\infty}\int_0^\infty\mu_s\bigl(h_\psi^{\alpha/2z}h_{\ffi_n}^{-{\alpha-1\over z}}
h_\psi^{\alpha/2z}\bigr)^z\,ds
=\lim_{n\to\infty}Q_{\alpha,z}(\psi,\ffi_n).
}\end{align*}%DIFAUXCMD
\DIFdel{From this and (1) above the extension holds in this case too.
}%DIFDELCMD < 

%DIFDELCMD < %%%
\DIFdel{(3)}%DIFDELCMD < \enspace
%DIFDELCMD < %%%
\DIFdel{Assume that $\psi$ is general and $\ffi\ge\delta\tau$ for some $\delta>0$. Set $\psi_n\in\Me_*^+$
by $h_{\psi_n}=\int_{[0,n]}t\,de_t^\psi+ne_{(n,\infty)}^\ffi$ ($\in\Me_+$). Since
$h_{\psi_n}^{\alpha/z}\nearrow h_\psi^{\alpha/z}$ in the measure topology, one can argue as in \eqref{F-4.11}
with use of the monotone convergence theorem to see from (2) that the extension holds in this case too.
}%DIFDELCMD < 

%DIFDELCMD < %%%
\DIFdel{(4)}%DIFDELCMD < \enspace
%DIFDELCMD < %%%
\DIFdel{Finally, from (3) with Lemma \ref{L-4.1} and \mbox{%DIFAUXCMD
\cite[Corollary 2.8(3)]{hiai2021quantum} }\hskip0pt%DIFAUXCMD
it follows that
the desired extension hods for general $\psi,\ffi\in\Me_*^+$.
}\DIFdelend \end{proof}

\DIFdelbegin \DIFdel{In the next proposition, we summarize inequalities for $D_{\alpha,z}$ obtained so far in Lemmas \ref{L-4.2}
and \ref{L-4.3}.
}%DIFDELCMD < 

%DIFDELCMD < \begin{prop}\label{P-4.4}
%DIFDELCMD < %%%
\DIFdel{Assume that $\Me$ is a finite von Neumann algebra with a faithful normal finite trace $\tau$.
}\DIFdelend \DIFaddbegin \begin{lemma}\label{lemma:order} \DIFaddend Let \DIFdelbegin \DIFdel{$\psi,\ffi\in\Me_*^+$, $\psi\ne0$. If $0<\alpha<1<\alpha'$ and $0<z\le z'\le\infty$, then
}\[
\DIFdel{D_{\alpha,z}(\psi\|\ffi)\le D_{\alpha,z'}(\psi\|\ffi)\le D_1(\psi\|\ffi)
\le D_{\alpha',z'}(\psi\|\ffi)\le D_{\alpha',z}(\psi\|\ffi).
}\]%DIFAUXCMD
%DIFDELCMD < \end{prop}
%DIFDELCMD < 

%DIFDELCMD < \begin{coro}\label{C-4.5}
%DIFDELCMD < %%%
\DIFdel{Let $(\Me,\tau)$ and $\psi,\ffi$ be as in Proposition \ref{P-4.4}. }\DIFdelend \DIFaddbegin \DIFadd{$\psi,\varphi\in \Me_*^+$ with $\psi\le \varphi$.
}\DIFaddend Then for any \DIFdelbegin \DIFdel{$z\in[1,\infty]$,
}\begin{align*}\DIFdel{%DIFDELCMD < \label{F-4.12}%%%
\lim_{\alpha\nearrow1}D_{\alpha,z}(\psi\|\ffi)=D_1(\psi\|\ffi).
}\end{align*}%DIFAUXCMD
\DIFdel{Moreover, if $D_{\alpha,\alpha}(\psi\|\ffi)<\infty$ for some $\alpha>1$ then for any $z\in(1,\infty]$,
}\begin{align*}\DIFdel{%DIFDELCMD < \label{F-4.13}%%%
\lim_{\alpha\searrow1}D_{\alpha,z}(\psi\|\ffi)=D_1(\psi\|\ffi).
}\end{align*}%DIFAUXCMD
%DIFDELCMD < \end{coro}
%DIFDELCMD < 

%DIFDELCMD < \begin{proof}
%DIFDELCMD < %%%
\DIFdel{Let $z\ge1$. For every $\alpha\in(0,1)$, Proposition \ref{P-4.4} gives
}\[
\DIFdel{D_{\alpha,1}(\psi\|\ffi)\le D_{\alpha,z}(\psi\|\ffi)\le D_1(\psi\|\ffi).
}\]%DIFAUXCMD
\DIFdel{Hence \eqref{F-4.12} follows since it holds for $D_{\alpha,1}$ \mbox{%DIFAUXCMD
\cite[Proposition 5.3(3)]{hiai2018quantum}}\hskip0pt%DIFAUXCMD
.
}%DIFDELCMD < 

%DIFDELCMD < %%%
\DIFdel{Next, assume that $D_{\alpha,\alpha}(\psi\|\ffi)<\infty$ for some $\alpha>1$. Let $z>1$. For every
$\alpha\in(1,z]$, Proposition \ref{P-4.4} gives
}\[
\DIFdel{D_1(\psi\|\ffi)\le D_{\alpha,z}(\psi\|\ffi)\le D_{\alpha,\alpha}(\psi\|\ffi).
}\]%DIFAUXCMD
\DIFdel{Hence \eqref{F-4.13} follows since it holds for $D_{\alpha,\alpha}$ \mbox{%DIFAUXCMD
\cite[Proposition 3.8(ii)]{jencova2018renyi}}\hskip0pt%DIFAUXCMD
.
}%DIFDELCMD < \end{proof}
%DIFDELCMD < 

%DIFDELCMD < \medskip
%DIFDELCMD < %%%
\DIFdel{In this subsection, in the specialized setting of finite von Neumann algebras, we have given monotonicity
of $D_{\alpha,z}$ in the parameter $z$ in an essentially similar way to the finite-dimensional case
\mbox{%DIFAUXCMD
\cite{mosonyi2023somecontinuity}}\hskip0pt%DIFAUXCMD
. In the next subsection we will extend it to general von Neumann algebras
under certain restrictions of $\alpha,z$.
}%DIFDELCMD < 

%DIFDELCMD < %%%
\subsection{\DIFdel{The general von Neumann algebra case}}
%DIFAUXCMD
\addtocounter{subsection}{-1}%DIFAUXCMD
%DIFDELCMD < 

%DIFDELCMD < \begin{theorem}
%DIFDELCMD < %%%
\DIFdel{For every $\psi,\ffi\in\Me_*^+$, $\psi\ne0$, and $0<\alpha<1$, we have:
}%DIFDELCMD < \begin{itemize}
%DIFDELCMD < \item[(1)] %%%
\DIFdel{If $0<\alpha<1$ }\DIFdelend \DIFaddbegin \DIFadd{$a\in \Me$ }\DIFaddend and \DIFdelbegin \DIFdel{$\max\{\alpha,1-\alpha\}\le z\le z'$, then
}\[
\DIFdel{D_{\alpha,z}(\psi\|\ffi)\le D_{\alpha,z'}(\psi\|\ffi)\le D_1(\psi\|\ffi).
}\]%DIFAUXCMD
\DIFdelend \DIFaddbegin \DIFadd{$p\in [1,\infty)$,
}\[
\DIFadd{\Tr\left( (a^*h_\psi^{\frac{1}p}a)^p\right)\le Tr\left( (a^*h_\varphi^{\frac{1}p}a)^p\right)
}\]\DIFaddend 
\DIFdelbegin %DIFDELCMD < \item[(2)] %%%
\DIFdel{If $\alpha>1$ and $\max\{\alpha/2,\alpha-1\}\le z\le z'\le\alpha$, then
}\[
\DIFdel{D_1(\psi\|\ffi)\le D_{\alpha,z'}(\psi\|\ffi)\le D_{\alpha,z}(\psi\|\ffi).
}\]%DIFAUXCMD
%DIFDELCMD < \end{itemize}
%DIFDELCMD < \end{theorem}
%DIFDELCMD < {\color{magenta}%%%
\DIFdel{Hiai (12/8/2023)}%DIFDELCMD < } {\color{blue}%%%
\DIFdel{In fact, (2) is improved in Theorem 6.}%DIFDELCMD < }
%DIFDELCMD < %%%
\DIFdelend 


\DIFdelbegin %DIFDELCMD < \begin{theorem}
%DIFDELCMD < %%%
\DIFdel{For every $\psi,\ffi\in\Me_*^+$, $\psi\ne0$, and
$\alpha>1$, the function $z\mapsto D_{\alpha,z}(\psi\|\varphi)$
is monotone decreasing on $[\alpha/2,\infty)$.
}%DIFDELCMD < \end{theorem}
%DIFDELCMD < {\color{magenta}%%%
\DIFdel{Anna (Jan.\ 23, 2024) }%DIFDELCMD < }
%DIFDELCMD < %%%
\DIFdelend \DIFaddbegin \end{lemma}
\DIFaddend 

\DIFdelbegin \section{\DIFdel{Monotonicity in the parameter $\alpha$}}
%DIFAUXCMD
\addtocounter{section}{-1}%DIFAUXCMD
%DIFDELCMD < 

%DIFDELCMD < %%%
\subsection{\DIFdel{The case $\alpha<1$ and all $z>0$}}
%DIFAUXCMD
\addtocounter{subsection}{-1}%DIFAUXCMD
%DIFDELCMD < 

%DIFDELCMD < \begin{theorem}
%DIFDELCMD < %%%
\DIFdel{Let $\psi,\ffi\in\Me_*^+$ and $z>0$. Then we have
}%DIFDELCMD < \begin{itemize}
\begin{itemize}%DIFAUXCMD
%DIFDELCMD < \item[(1)] %%%
\item[\DIFdel{(1)}]%DIFAUXCMD
\DIFdel{$\alpha\mapsto\log Q_{\alpha,z}(\psi\|\ffi)$ is convex on $(0,1)$,
}%DIFDELCMD < \item[(2)] %%%
\item[\DIFdel{(2)}]%DIFAUXCMD
\DIFdel{$\alpha\mapsto D_{\alpha,z}(\psi\|\ffi)$ is monotone increasing on $(0,1)$.
}
\end{itemize}%DIFAUXCMD
%DIFDELCMD < \end{itemize}
%DIFDELCMD < \end{theorem}
%DIFDELCMD < {\color{magenta}%%%
\DIFdel{Anna (Jan.\ 10, 2024), Hiai (1/16/2024)}%DIFDELCMD < }
%DIFDELCMD < 

%DIFDELCMD < %%%
\subsection{\DIFdel{The case $1<\alpha\le2z$}}
%DIFAUXCMD
\addtocounter{subsection}{-1}%DIFAUXCMD
%DIFDELCMD < 

%DIFDELCMD < \begin{theorem}
%DIFDELCMD < %%%
\DIFdel{Let $\psi,\ffi\in\Me_*^+$ and $z>1/2$. Then we have
}%DIFDELCMD < \begin{itemize}
\begin{itemize}%DIFAUXCMD
%DIFDELCMD < \item[(1)] %%%
\item[\DIFdel{(1)}]%DIFAUXCMD
\DIFdel{$\alpha\mapsto\log Q_{\alpha,z}(\psi\|\ffi)$ is convex on $(1,2z]$,
}%DIFDELCMD < \item[(2)] %%%
\item[\DIFdel{(2)}]%DIFAUXCMD
\DIFdel{$\alpha\mapsto D_{\alpha,z}(\psi\|\ffi)$ is monotone increasing on $(1,2z]$.
}
\end{itemize}%DIFAUXCMD
%DIFDELCMD < \end{itemize}
%DIFDELCMD < \end{theorem}
%DIFDELCMD < 

%DIFDELCMD < {\color{magenta}%%%
\DIFdel{Anna (Jan.\ 23, 2024), Hiai (12/31/2023)}%DIFDELCMD < }
%DIFDELCMD < 

%DIFDELCMD < %%%
\subsection{\DIFdel{Limits as $\alpha\nearrow1$ and $\alpha\searrow1$}}
%DIFAUXCMD
\addtocounter{subsection}{-1}%DIFAUXCMD
%DIFDELCMD < 

%DIFDELCMD < \begin{theorem}
%DIFDELCMD < %%%
\DIFdel{Let $\psi,\ffi\in\Me_*^+$, $\psi\ne0$. For every $z\in(0,1]$ we have
}\[
\DIFdel{\lim_{\alpha\nearrow1}D_{\alpha,z}(\psi\|\ffi)=D_1(\psi\|\ffi).
}\]%DIFAUXCMD
%DIFDELCMD < \end{theorem}
%DIFDELCMD < {\color{magenta}%%%
\DIFdel{Anna (Dec.\ 7, 2023)}%DIFDELCMD < }
%DIFDELCMD < 

%DIFDELCMD < \begin{theorem}
%DIFDELCMD < %%%
\DIFdel{Let $\psi,\ffi\in\Me_*^+$, $\psi\ne0$, and $z>1/2$. Assume that $D_{\alpha,z}(\psi\|\ffi)<\infty$ for some
$\alpha\in(1,2z]$. Then we have
}\[
\DIFdel{\lim_{\alpha\searrow1}D_{\alpha,z}(\psi\|\ffi)=D_1(\psi\|\ffi).
}\]%DIFAUXCMD
%DIFDELCMD < \end{theorem}
%DIFDELCMD < {\color{magenta}%%%
\DIFdel{Anna (Jan.\ 23, 2024)}%DIFDELCMD < }
%DIFDELCMD < 

%DIFDELCMD < %%%
%DIF < \bibliography{alphaz}
%DIF < \bibliographystyle{abbrvnat}
%DIFDELCMD < \begin{thebibliography}{99}
%DIFDELCMD < \bibitem{berta2018renyi}
\DIFdel{M. Berta, V. B. Scholz, and M. Tomamichel. R\'enyi divergences as
weighted noncommutative
vector valued $Lp$-spaces. Annales Henri Poincar\'e, 19:1843--1867, 2018.
doi:https://doi.org/10.48550/arXiv.1608.05317.
}%DIFDELCMD < 

%DIFDELCMD < \bibitem{choi1974aschwarz}
\DIFdel{M.-D. Choi. A Schwarz inequality for positive linear maps on $C^*$-algebras. Illinois Journal
of Mathematics, 18}\DIFdelend \DIFaddbegin \begin{proof} \DIFadd{Since $1/p\in (p,1]$, it follows }\DIFaddend (\DIFdelbegin \DIFdel{4):565--574, 1974. doi:10.1215/ijm/1256051007.
}%DIFDELCMD < 

%DIFDELCMD < \bibitem{fack1986generalized}
\DIFdel{T. Fack and H. Kosaki. Generalized $s$-numbers of }\DIFdelend \DIFaddbegin \DIFadd{see \mbox{%DIFAUXCMD
\cite[Lemma B.7]{hiai2021quantum} }\hskip0pt%DIFAUXCMD
and
\mbox{%DIFAUXCMD
\cite[Lemma 3.2]{hiai2021connections}}\hskip0pt%DIFAUXCMD
) that $h_\psi^{1/p}\le h_\varphi^{1/p}$ as
}\DIFaddend $\tau$-measurable operators \DIFdelbegin \DIFdel{. Pacific Journal
of Mathematics, 123(2):269--300, 1986.
}%DIFDELCMD < 

%DIFDELCMD < \bibitem{haagerup1979lpspaces}
\DIFdel{U. Haagerup. $L_p$-spaces associated with an arbitrary von Neumann algebra. In Algebres
d’op\'erateurs et leurs applications en physique mathématique (Proc. Colloq., Marseille, 1977),
volume 274, pages 175--184, 1979.
}%DIFDELCMD < 

%DIFDELCMD < \bibitem{hiai2018quantum}
\DIFdel{F. Hiai. Quantum $f$-divergences in
von Neumann algebras. I. Standard $f$-divergences. Journal
of Mathematical Physics, 59(10):102202, 2018.
}%DIFDELCMD < 

%DIFDELCMD < \bibitem{hiai2021quantum}
\DIFdel{F.  Hiai. Quantum $f$-Divergences in von Neumann Algebras: Reversibility of Quantum Operations.
Mathematical Physics Studies. Springer, Singapore, 2021. ISBN 9789813341999.
doi:10.1007/978-981-33-4199-9}\DIFdelend \DIFaddbegin \DIFadd{affiliated with $\Me \rtimes_{\sigma^\phi} \mathbb R$ (in
which $L_p(\Me)$ lives)}\DIFaddend .  \DIFaddbegin \DIFadd{Hence $a^*h_\psi^{1/p}a\le a^*h_\varphi^{1/p}a$
in the same sense. Therefore, by \mbox{%DIFAUXCMD
\cite[Lemma 2.5 (iii), Lemma 4.8]{fack1986generalized}}\hskip0pt%DIFAUXCMD
,
we have the statement.
}\DIFaddend 

\DIFdelbegin \bibitem{hiai2021lectures}
\DIFdel{F. Hiai. Lectures on Selected Topics in Von Neumann Algebras. EMS Series of Lectures in Mathematics.
EMS Press, Berlin, 2021.
}\DIFdelend \DIFaddbegin \end{proof}
\DIFaddend 


\DIFdelbegin \bibitem{jencova2018renyi}
\DIFdel{A. Jen\v cov\'a. R\'enyi relative entropies and noncommutative Lp-spaces. Annales Henri Poincar\'e,
19:2513--2542, 2018. doi:10.1007/s00023-018-0683-5.
}%DIFDELCMD < 

%DIFDELCMD < \bibitem{jencova2021renyi}
\DIFdel{A. Jen\v cov\'a. R\'enyi relative entropies and noncommutative $Lp$-spaces II. Annales Henri
Poincar\'e, 22:3235--3254, 2021. doi:10.1007/s00023-021-01074-9.
}%DIFDELCMD < 

%DIFDELCMD < \bibitem{junge2003noncommutative}
\DIFdel{M. Junge and Q. Xu. Noncommutative Burkholder/Rosenthal inequalities. The Annals of
Probability, 31(2):948--995, 2003.
}%DIFDELCMD < 

%DIFDELCMD < \bibitem{kato2023onrenyi}
\DIFdel{S. Kato. On $\alpha$-$z$-R\'enyi divergence in the von Neumann algebra setting. arXiv preprint
arXiv:2311.01748, 2023.
}%DIFDELCMD < 

%DIFDELCMD < \bibitem{kato2023aremark}
\DIFdel{S. Kato and Y. Ueda. A remark on non-commutative $Lp$-spaces. arXiv preprint
arXiv:2307.01790,
2023.
}%DIFDELCMD < 

%DIFDELCMD < \bibitem{kosaki1984applications}
\DIFdel{H. Kosaki. Applications of the complex interpolation method to a von Neumann algebra: Noncommutative
$L^p$-spaces.
J. Funct. Anal., 56:26--78, 1984. doi:https://doi.org/10.1016/0022-1236(84)90025-9.
}%DIFDELCMD < 

%DIFDELCMD < \bibitem{kosaki1992aninequality}
\DIFdel{H. Kosaki. An inequality of Araki--Lieb--Thirring (von Neumann algebra case).
Proc. Amer. Math. Soc., 114:477--481, 1992.
}%DIFDELCMD < 

%DIFDELCMD < \bibitem{mosonyi2023somecontinuity}
\DIFdel{M. Mosonyi and F. Hiai. Some continuity properties of quantum R\'enyi divergences.
IEEE Transactions on Information Theory, to appear, DOI 10.1109/TIT.2023.3324758.
}%DIFDELCMD < 

%DIFDELCMD < \bibitem{petz1985quasi}
\DIFdel{D. Petz. Quasi-entropies for states of a von Neumann algebra. Publications of the Research
Institute for Mathematical Sciences, 21(4):787--800, 1985. doi:10.2977/prims/1195178929.
}%DIFDELCMD < 

%DIFDELCMD < \bibitem{petz1986sufficient}
\DIFdel{D. Petz. Sufficient subalgebras and the relative entropy of states of a von Neumann algebra.
Communications in Mathematical Physics, 105(1):123--131, 1986. doi:10.1007/BF01212345.
}%DIFDELCMD < 

%DIFDELCMD < \bibitem{petz1988sufficiency}
\DIFdel{D. Petz. Sufficiency of channels over von Neumann algebras. The Quarterly Journal of
Mathematics, 39(1):97–108, 1988. doi:10.1093/qmath/39.1.97.8
}%DIFDELCMD < 

%DIFDELCMD < \bibitem{terp1981lpspaces}
\DIFdel{M. Terp. $L_p$-spaces associated with von Neumann algebras. Notes, Copenhagen University,
1981.
}%DIFDELCMD < \end{thebibliography}
%DIFDELCMD < %%%
\DIFdelend \DIFaddbegin \bibliography{alphaz}
\bibliographystyle{abbrvnat}
\DIFaddend 

\end{document}




