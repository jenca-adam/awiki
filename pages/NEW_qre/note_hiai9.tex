\documentclass[11pt,reqno]{article}
\usepackage{amssymb,amscd,amsmath,amsthm,color}
%\usepackage[dvipdfmx]{hyperref}
\textwidth=15.5cm
\textheight=23.5cm
\parindent=15pt
\parskip=4pt

\hoffset=-13mm
\voffset=-25mm

\newtheorem{thm}{Theorem}[section]
\newtheorem{prop}[thm]{Proposition}
\newtheorem{lemma}[thm]{Lemma}
\newtheorem{cor}[thm]{Corollary}
\newtheorem{definition}[thm]{Definition}
\newtheorem{remark}[thm]{Remark}
\newtheorem{example}[thm]{Example}
\newtheorem{problem}[thm]{Problem}
\numberwithin{equation}{section}

\def\cM{\mathcal{M}}
\def\bN{\mathbb{N}}
\def\bZ{\mathbb{Z}}
\def\bM{\mathbb{M}}
\def\bR{\mathbb{R}}
\def\bC{\mathbb{C}}
\def\cH{\mathcal{H}}
\def\cE{\mathcal{E}}
\def\cA{\mathcal{A}}
\def\cN{\mathcal{N}}
\def\cB{\mathcal{B}}
\def\cK{\mathcal{K}}
\def\Tr{\mathrm{Tr}\,}
\def\tr{\mathrm{tr}\,}
\def\Im{\mathrm{Im}\,}
\def\1{\mathbf{1}}
\def\ffi{\varphi}
\def\eps{\varepsilon}
\def\<{\langle}
\def\>{\rangle}

\begin{document}
%\baselineskip=15pt
\allowdisplaybreaks

\noindent (2/21/2024)\hfill F.\ Hiai

\medskip
\centerline{\LARGE Equality conditions of DPI}

\bigskip
Here we consider equality conditions of the DPI of $D_{\alpha,z}$ for $\alpha<1$. Assume that $0<\alpha<1$
and $z\ge\max\{\alpha,1-\alpha\}$. For simplicity, assume that $\lambda^{-1}\ffi\le\psi\le\lambda\ffi$ for some
$\lambda>0$. Let $\gamma:\cN\to\cM$ be a unital normal positive map, and set $\psi_0:=\psi\circ\gamma$ and
$\ffi_0:=\ffi\circ\gamma$. Moreover, set $e:=s(\psi)=s(\ffi)$ and $e_0:=s(\psi_0)=s(\ffi_0)$ (since
$\lambda\ffi_0\le\psi_0\le\lambda\ffi_0$). Consider $\gamma_{e_0,e}:e_0\cN e_0\to e\cM e$ defined by
$\gamma_{e_0,e}(y):=e\gamma(e_0ye_0)e$ for $y\in e_0\cN e_0$. Then for every $y\in e_0\cN e_0$,
\[
\psi\circ\gamma_{e_0,e}(y)=\psi(e\gamma(e_0ye_0)e)=\psi(\gamma(e_0ye_0)e)=\psi_0(e_0ye_0)=\psi_0(y),
\]
so that we have $\psi\circ\gamma_{e_0,e}=\psi_0|_{e_0\cN e_0}$ and similarly
$\ffi\circ\gamma_{e_0,e}=\ffi_0|_{e_0\cN e_0}$. Hence, by replacing $\gamma$ with $\gamma_{e_0,e}$ we may
assume that $\psi,\ffi,\psi_0,\ffi_0$ are all faithful.

One can define $b,c\in\cM$ in such a way that
\begin{align}\label{F-1}
h_\ffi^{1-\alpha\over2z}
=b\bigl(h_\ffi^{1-\alpha\over2z}h_\psi^{\alpha\over z}h_\ffi^{1-\alpha\over2z}\bigr)^{1-\alpha\over2},\qquad
\bigl(h_\ffi^{1-\alpha\over2z}h_\psi^{\alpha\over z}h_\ffi^{1-\alpha\over2z}\bigr)^{1-\alpha\over2}
=ch_\ffi^{1-\alpha\over2z}.
\end{align}
Then $a:=bb^*\in\cM^{++}$ and $a^{-1}=c^*c$. By \cite[Theorem 1\,(vi)]{Ka1} we have
\begin{align}\label{F-2}
Q_{\alpha,z}(\psi\|\ffi)
=\inf_{x\in\cM^{++}}\biggl\{\alpha\Big\|h_\psi^{\alpha\over2z}xh_\psi^{\alpha\over2z}\Big\|_{z/\alpha}^{z/\alpha}
+(1-\alpha)\Big\|h_\ffi^{1-\alpha\over2z}x^{-1}h_\ffi^{1-\alpha\over2z}\Big\|_{z/(1-\alpha)}^{z/(1-\alpha)}\biggr\},
\end{align}
and $a$ is a minimizer of the above infimum expression, so that
\begin{align}\label{F-3}
Q_{\alpha,z}(\psi\|\ffi)=\alpha\Big\|h_\psi^{\alpha\over2z}ah_\psi^{\alpha\over2z}\Big\|_{z/\alpha}^{z/\alpha}
+(1-\alpha)\Big\|h_\ffi^{1-\alpha\over2z}a^{-1}h_\ffi^{1-\alpha\over2z}\Big\|_{z/(1-\alpha)}^{z/(1-\alpha)}.
\end{align}

One can also define $b_0,c_0\in\cN$ in such a way that
\begin{align}\label{F-4}
h_{\ffi_0}^{1-\alpha\over2z}
=b_0\bigl(h_{\ffi_0}^{1-\alpha\over2z}h_{\psi_0}^{\alpha\over z}
h_{\ffi_0}^{1-\alpha\over2z}\bigr)^{1-\alpha\over2},\qquad
\bigl(h_{\ffi_0}^{1-\alpha\over2z}h_{\psi_0}^{\alpha\over z}h_{\ffi_0}^{1-\alpha\over2z}\bigr)^{1-\alpha\over2}
=c_0h_{\ffi_0}^{1-\alpha\over2z}.
\end{align}
Then $a_0:=b_0b_0^*\in\cN^{++}$ and $a_0^{-1}=c_0^*c_0$, and we have
\begin{align}
Q_{\alpha,z}(\psi_0\|\ffi_0)
&=\alpha\Big\|h_{\psi_0}^{\alpha\over2z}a_0h_{\psi_0}^{\alpha\over2z}\Big\|_{z/\alpha}^{z/\alpha}
+(1-\alpha)\Big\|h_{\ffi_0}^{1-\alpha\over2z}a_0^{-1}h_\ffi^{1-\alpha\over2z}\Big\|_{z/(1-\alpha)}^{z/(1-\alpha)}.
\label{F-5}
\end{align}

\begin{lemma}\label{L-1}
The operator $a\in\cM^{++}$ is uniquely determined by equality \eqref{F-3}. Similarly, $a_0\in\cN^{++}$ is
uniquely determined by equality \eqref{F-5}.
\end{lemma}

\begin{proof}
Suppose that $a_1,a_2\in\cM^{++}$ satisfy equality \eqref{F-3}. Let $a_0:=(a_1+a_2)/2$. Since
$k\in L^{z/\alpha}(\cM)\mapsto\|k\|_{z/\alpha}^{z/\alpha}$ and
$k\in L^{z/(1-\alpha)}(\cM)\mapsto\|k\|_{z/(1-\alpha)}^{z/(1-\alpha)}$ are convex and
$a_0^{-1}\le(a_1^{-1}+a_2^{-1})/2$, we have
\begin{align*}
\Big\|h_\psi^{\alpha\over2z}a_0h_\psi^{\alpha\over2z}\Big\|_{z/\alpha}^{z/\alpha}
&\le{1\over2}\biggl\{\Big\|h_\psi^{\alpha\over2z}a_1h_\psi^{\alpha\over2z}\Big\|_{z/\alpha}^{z/\alpha}
+\Big\|h_\psi^{\alpha\over2z}a_2h_\psi^{\alpha\over2z}\Big\|_{z/\alpha}^{z/\alpha}\biggr\}, \\
\Big\|h_\ffi^{1-\alpha\over2z}a_0^{-1}h_\ffi^{1-\alpha\over2z}\Big\|_{z/(1-\alpha)}^{z/(1-\alpha)}
&\le\Big\|h_\ffi^{1-\alpha\over2z}\biggl({a_1^{-1}+a_2^{-1}\over2}\biggr)
h_\ffi^{1-\alpha\over2z}\Big\|_{z/(1-\alpha)}^{z/(1-\alpha)} \\
&\le{1\over2}\biggl\{\Big\|h_\ffi^{1-\alpha\over2z}a_1^{-1}h_\ffi^{1-\alpha\over2z}\Big\|_{z/(1-\alpha)}^{z/(1-\alpha)}
+\Big\|h_\ffi^{1-\alpha\over2z}a_2^{-1}h_\ffi^{1-\alpha\over2z}\Big\|_{z/(1-\alpha)}^{z/(1-\alpha)}\biggr\}.
\end{align*}
Hence we have
\[
\Big\|h_\ffi^{1-\alpha\over2z}a_0^{-1}h_\ffi^{1-\alpha\over2z}\Big\|_{z/(1-\alpha)}
=\Big\|h_\ffi^{1-\alpha\over2z}\biggl({a_1^{-1}+a_2^{-1}\over2}\biggr)
h_\ffi^{1-\alpha\over2z}\Big\|_{z/(1-\alpha)},
\]
which implies that $a_0^{-1}={a_1^{-1}+a_2^{-1}\over2}$, as easily verified. From this we easily have $a_1=a_2$.
\end{proof}

Furthermore, one has
\begin{align}
h_\psi^{\alpha\over2z}ah_\psi^{\alpha\over2z}
&=\bigl(h_\psi^{\alpha\over2z}h_\ffi^{1-\alpha\over z}h_\psi^{\alpha\over2z}\bigr)^\alpha,
\label{F-6}\\
h_\ffi^{1-\alpha\over2z}a^{-1}h_\ffi^{1-\alpha\over2z}
&=\bigl(h_\ffi^{1-\alpha\over2z}h_\psi^{\alpha\over z}h_\ffi^{1-\alpha\over2z}\bigr)^{1-\alpha},
\label{F-7}\\
h_{\psi_0}^{\alpha\over2z}a_0h_{\psi_0}^{\alpha\over2z}
&=\bigl(h_{\psi_0}^{\alpha\over2z}h_{\ffi_0}^{1-\alpha\over z}h_{\psi_0}^{\alpha\over2z}\bigr)^\alpha,
\label{F-8}\\
h_{\ffi_0}^{1-\alpha\over2z}a_0^{-1}h_{\ffi_0}^{1-\alpha\over2z}
&=\bigl(h_{\ffi_0}^{1-\alpha\over2z}h_{\psi_0}^{\alpha\over z}h_{\ffi_0}^{1-\alpha\over2z}\bigr)^{1-\alpha}.
\label{F-9}
\end{align}
Indeed, \eqref{F-7} is obvious from the second equality in \eqref{F-1} and $a^{-1}=c^*c$. Since
$Q_{\alpha,z}(\psi\|\ffi)=Q_{1-\alpha,z}(\ffi\|\psi)$, we see in view of Lemma \ref{L-1} that the minimizer of the
infimum expression for $Q_{1-\alpha,z}(\ffi\|\psi)$ (instead of \eqref{F-2}) is $a^{-1}$ (instead of $a$). This
says that \eqref{F-6} is the equality corresponding to \eqref{F-7} when $\psi,\ffi,\alpha$ are replaced with
$\ffi,\psi,1-\alpha$, respectively. \eqref{F-8} and \eqref{F-9} are similar.

\begin{prop}\label{P-2}
In the above stated situation the following conditions are equivalent:
\begin{itemize}
\item[(i)] $D_{\alpha,z}(\psi_0\|\ffi_0)=D_{\alpha,z}(\psi\|\ffi)$, i.e.,
$Q_{\alpha,z}(\psi_0\|\ffi_0)=Q_{\alpha,z}(\psi\|\ffi)$.
\item[(ii)] $\gamma(a_0)=a$ and
$\big\|h_\psi^{\alpha\over2z}\gamma(a_0)h_\psi^{\alpha\over2z}\big\|_{z/\alpha}
=\big\|h_{\psi_0}^{\alpha\over2z}a_0h_{\psi_0}^{\alpha\over2z}\big\|_{z/\alpha}$.
\item[(iii)] $\big\|h_\psi^{\alpha\over2z}ah_\psi^{\alpha\over2z}\big\|_{z/\alpha}
=\big\|h_{\psi_0}^{\alpha\over2z}a_0h_{\psi_0}^{\alpha\over2z}\big\|_{z/\alpha}$.
\item[(iv)] $\gamma(a_0^{-1})=a^{-1}$ and
$\big\|h_\ffi^{1-\alpha\over2z}\gamma(a_0^{-1})h_\ffi^{1-\alpha\over2z}\big\|_{z/(1-\alpha)}
=\big\|h_{\ffi_0}^{1-\alpha\over2z}a_0^{-1}h_{\ffi_0}^{1-\alpha\over2z}\big\|_{z/(1-\alpha)}$.
\item[(v)] $\big\|h_\ffi^{1-\alpha\over2z}a^{-1}h_\ffi^{1-\alpha\over2z}\big\|_{z/(1-\alpha)}
=\big\|h_{\ffi_0}^{1-\alpha\over2z}a_0^{-1}h_{\ffi_0}^{1-\alpha\over2z}\big\|_{z/(1-\alpha)}$.
\end{itemize}
\end{prop}

\begin{proof}
(i)$\implies$(ii) \& (iv).\enspace
By \cite[(22)]{Ka1} one has
\begin{align}
\Big\|h_\psi^{\alpha\over2z}\gamma(a_0)h_\psi^{\alpha\over2z}\Big\|_{z/\alpha}
&\le\Big\|h_{\psi_0}^{\alpha\over2z}a_0h_{\psi_0}^{\alpha\over2z}\Big\|_{z/\alpha}, \label{F-10}\\
\Big\|h_\ffi^{1-\alpha\over2z}\gamma(a_0^{-1})h_\ffi^{1-\alpha\over2z}\Big\|_{z/(1-\alpha)}
&\le\Big\|h_{\ffi_0}^{1-\alpha\over2z}a_0^{-1}h_{\ffi_0}^{1-\alpha\over2z}\Big\|_{z/(1-\alpha)}. \label{F-11}
\end{align}
Moreover, since $\gamma(a_0^{-1})\ge\gamma(a_0)^{-1}$ due to Choi's inequality \cite[Corollary 2.3]{Ch},
one has
\begin{align}\label{F-12}
\Big\|h_\ffi^{1-\alpha\over2z}\gamma(a_0)^{-1}h_\ffi^{1-\alpha\over2z}\Big\|_{z/(1-\alpha)}
\le\Big\|h_\ffi^{1-\alpha\over2z}\gamma(a_0^{-1})h_\ffi^{1-\alpha\over2z}\Big\|_{z/(1-\alpha)}.
\end{align}
From \eqref{F-10}--\eqref{F-12} it follows that
\begin{align*}
&\alpha\big\|h_\psi^{\alpha\over2z}\gamma(a_0)h_\psi^{\alpha\over2z}\big\|_{z/\alpha}
+(1-\alpha)\big\|h_\ffi^{1-\alpha\over2z}\gamma(a_0)^{-1}h_\ffi^{1-\alpha\over2z}\big\|_{z/(1-\alpha)} \\
&\qquad\le\alpha\big\|h_{\psi_0}^{\alpha\over2z}a_0h_{\psi_0}^{\alpha\over2z}\big\|_{z/\alpha}
+(1-\alpha)\big\|h_{\ffi_0}^{1-\alpha\over2z}a_0^{-1}h_{\ffi_0}^{1-\alpha\over2z}\big\|_{z/(1-\alpha)} \\
&\qquad=Q_{\alpha,z}(\psi_0\|\ffi_0)=Q_{\alpha,z}(\psi\|\ffi).
\end{align*}
By Lemma \ref{L-1} we find that $\gamma(a_0)=a$ and all the inequalities in \eqref{F-10}--\eqref{F-12} must
become equalities. Since $\gamma(a_0^{-1})\ge\gamma(a_0)^{-1}$, we easily verify that the equality in
\eqref{F-12} yields $\gamma(a_0^{-1})=\gamma(a_0)^{-1}$ and hence $\gamma(a_0^{-1})=a^{-1}$. Therefore,
(ii) and (iv) hold.

(ii)$\implies$(iii) and (iv)$\implies$(v) are obvious.

(iii)$\implies$(i).\enspace
By (iii) with \eqref{F-6} and \eqref{F-8} we have
\begin{align*}
Q_{\alpha,z}(\psi\|\ffi)
&=\tr\bigl(h_\psi^{\alpha\over2z}h_\ffi^{1-\alpha\over z}h_\psi^{\alpha\over2z}\bigr)^z
=\tr\bigl(h_\psi^{\alpha\over2z}ah_\psi^{\alpha\over2z}\bigr)^{z/\alpha} \\
&=\tr\bigl(h_{\psi_0}^{\alpha\over2z}a_0h_{\psi_0}^{\alpha\over2z}\bigr)^{z/\alpha}
=\tr\bigl(h_{\psi_0}^{\alpha\over2z}h_{\ffi_0}^{1-\alpha\over z}h_{\psi_0}^{\alpha\over2z}\bigr)^z \\
&=Q_{\alpha,z}(\psi_0\|\ffi_0).
\end{align*}

(v)$\implies$(i).\enspace
By (iii) with \eqref{F-7} and \eqref{F-9} we have
\begin{align*}
Q_{\alpha,z}(\psi\|\ffi)
&=\tr\bigl(h_\ffi^{1-\alpha\over2z}h_\psi^{\alpha\over z}h_\ffi^{1-\alpha\over2z}\bigr)^z
=\tr\bigl(h_\ffi^{1-\alpha\over2z}a^{-1}h_\ffi^{1-\alpha\over2z}\bigr)^{z/(1-\alpha)} \\
&=\tr\bigl(h_{\ffi_0}^{1-\alpha\over2z}a_0^{-1}h_{\ffi_0}^{1-\alpha\over2z}\bigr)^{z/(1-\alpha)}
=\tr\bigl(h_{\ffi_0}^{1-\alpha\over2z}h_{\psi_0}^{\alpha\over z}h_{\ffi_0}^{1-\alpha\over2z}\bigr)^z \\
&=Q_{\alpha,z}(\psi_0\|\ffi_0).
\end{align*}
\end{proof}

\begin{remark}\rm
Assume that $\cM=\cB(\cH)$ and $\cN=\cB(\cK)$ with finite-dimensional Hilbert spaces $\cH,\cK$ and
$\gamma=\Phi^*$ with a trace-preserving positive map $\Phi:\cB(\cH)\to\cB(\cK)$. For $\psi=\rho$, $\ffi=\sigma$
we write
\[
a=\sigma^{1-\alpha\over2z}\bigl(\sigma^{1-\alpha\over2z}\rho^{\alpha\over z}
\sigma^{1-\alpha\over2z}\bigr)^{\alpha-1}\sigma^{1-\alpha\over2z}
=\rho^{-{\alpha\over2z}}\bigl(\rho^{\alpha\over2z}\sigma^{1-\alpha\over z}
\rho^{\alpha\over2z}\bigr)^\alpha\rho^{-{\alpha\over2z}}
\]
and similarly for $\rho_0:=\Phi(\rho)$, $\sigma_0:=\Phi(\sigma)$,
\[
a_0=\rho_0^{-{\alpha\over2z}}\bigl(\rho_0^{\alpha\over2z}\sigma_0^{1-\alpha\over z}
\rho_0^{\alpha\over2z}\bigr)^\alpha\rho_0^{-{\alpha\over2z}}.
\]
Consequently, the equality $\Phi^*(a_0)=a$ in (ii) coincides with \cite[Theorem I.2\,(2)]{Zh}. The conditions
given in \cite[Theorem I.2\,(3) and (4)]{Zh} are
\begin{align*}
\Phi\bigl(\bigl(a^{1/2}\rho^{\alpha\over z}a^{1/2}\bigr)^{z/\alpha}\bigr)
&=\bigl(a_0^{1/2}\rho_0^{\alpha\over z}a_0^{1/2}\bigr)^{z/\alpha}, \\
\Phi\bigl(\bigl(a^{-1/2}\sigma^{1-\alpha\over z}a^{-1/2}\bigr)^{z/(1-\alpha)}\bigr)
&=\bigl(a_0^{-1/2}\sigma_0^{1-\alpha\over z}a_0^{-1/2}\bigr)^{z/(1-\alpha)}.
\end{align*}
In the setting of Proposition \ref{P-2} these correspond to
\begin{align}
\gamma_*\bigl(\bigl(a^{1/2}h_\psi^{\alpha\over z}a^{1/2}\bigr)^{z/\alpha}\bigr)
&=\bigl(a_0h_{\psi_0}^{\alpha\over z}a_0^{1/2}\bigr)^{z/\alpha}, \label{F-13}\\
\gamma_*\bigl(\bigl(a^{-1/2}h_\ffi^{1-\alpha\over z}a^{-1/2}\bigr)^{z/(1-\alpha)}\bigr)
&=\bigl(a_0^{-1/2}h_{\ffi_0}^{1-\alpha\over z}a_0^{-1/2}\bigr)^{z/(1-\alpha)}. \label{F-14}
\end{align}
Since
\[
\tr\gamma_*\bigl(\bigl(a^{1/2}h_\psi^{\alpha\over z}a^{1/2}\bigr)^{z/\alpha}\bigr)
=\tr\bigl(a^{1/2}h_\psi^{\alpha\over z}a^{1/2}\bigr)^{z/\alpha}
=\tr\bigl(h_\psi^{\alpha\over2z}ah_\psi^{\alpha\over2z}\bigr)^{z/\alpha}
=\big\|h_\psi^{\alpha\over2z}ah_\psi^{\alpha\over2z}\big\|_{z/\alpha}^{z/\alpha},
\]
we note that \eqref{F-13} is a stronger version of Proposition \ref{P-2}\,(iii). Similarly, \eqref{F-14} is a
stronger version of Proposition \ref{P-2}\,(v). {\color{red}In view of \cite[Theorem I.2\,(iii) and (iv)]{Zh} we may
conjecture that Proposition \ref{P-2}\,(i)$\implies$\eqref{F-13} whenever $z\ne\alpha$, and that
Proposition \ref{P-2}\,(i)$\implies$\eqref{F-14} whenever $z\ne1-\alpha$, if $\gamma$ is a unital normal CP map.}
\end{remark}

\begin{thebibliography}{99}

\bibitem{Ch}
M.-D. Choi, A Schwarz inequality for positive linear maps on $C^*$-algebras,
{\it Illinois J. Math.} {\bf 18} (1974), 565--574.

\bibitem{Ka1}
S. Kato, On $\alpha$-$z$-R\'enyi divergence in the von Neumann algebra setting, Preprint, 2023.

\bibitem{Zh}
H. Zhang, Equality conditions of data processing inequality for $\alpha$-$z$ R\'enyi relative entropies,
{\it J. Math. Phys.} {\bf 61} (2020), 102201, 15 pp.
\end{thebibliography}

\end{document}
