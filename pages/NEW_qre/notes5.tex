\documentclass[12pt]{article}

\usepackage{hyperref}
\usepackage{amsmath, amssymb, amsthm}
\usepackage[sort&compress,numbers]{natbib}
\usepackage{doi}
\usepackage[margin=0.8in]{geometry}
%\textheight23cm \topmargin-20mm  
%\textwidth175mm  
%\oddsidemargin=0mm
%\evensidemargin=0mm
%

\usepackage{amsmath, amssymb, amsthm, mathtools}

\newtheorem{lemma}{Lemma}
\newtheorem{prop}{Proposition}
\newtheorem{theorem}{Theorem}
\newtheorem{coro}{Corollary}


\theoremstyle{definition}
\newtheorem{defi}{Definition}


\theoremstyle{remark}
\newtheorem{remark}{Remark}

\def\Me{\mathcal M}
\def\Ee{\mathcal E}
\def\Fe{\mathcal F}
\def\Ra{\mathcal R}
\def\Ne{\mathcal N}
\def \Tr{\mathrm{Tr}\,}
\def\Se {\mathcal S}
\def\supp{\mathrm{supp}}
\def\<{\langle\.}
\def\>{\.\rangle}

\title{Notes on monotonicity of  $\alpha\mapsto D_{\alpha,z}$}
\author{Anna Jen\v cov\'a}

\begin{document}

\maketitle

Assume that $z>1/2$ and let $p=2z$ and $q=\frac{2z-1}{2z}$ the dual parameter. 
Let $e:=s(\varphi)$ and $f:=s(\psi)$, and let $\sigma,\tau\in \Me_*^+$ be such that
$s(\sigma)=1-e$, $s(\tau)=1-f$. Put $\psi_0:=\psi+\tau$, $\varphi_0=\varphi+\sigma$, then
$\psi_0,\varphi_0$ are faithful positive normal functionals on $\Me$ and we have
$h_{\varphi}^\theta=eh_{\varphi_0}^\theta=h_{\varphi_0}^\theta e$ and
$h_{\psi}^\theta=fh_{\psi_0}^\theta=h_{\psi_0}^\theta f$ for any $\theta>0$. 
We will use the notations $L^p_L:=L^p(\Me;\varphi_0)_L$, $L^p_R:=L^p(\Me;\psi_0)_R$ and 
$L^p_\eta:=C_\eta(L^p_L,L^p_R)$.


\section{Remarks for the case $\alpha>1$}

\begin{enumerate}
\item Let $1<\alpha\le 2z$ and assume that $Q_{\alpha,z}(\psi\|\varphi)<\infty$, so that there
exists a (unique) $y\in L^p(\Me)e$ such that
$h_\psi^{\alpha/p}=yh_\varphi^{(\alpha-1)/p}$. Note that we may as well assume that $y\in
fL^p(\Me)e$, so that
\[
h_\psi=h_{\psi_0}^{\eta/q}yh_{\varphi_0}^{(1-\eta)/q}\in L^p_\eta,
\]
where $\eta=(2z-\alpha)/(2z-1)$ and
$Q_{\alpha,z}(\psi\|\varphi)=\|y\|_{2z}^{2z}=\|h_\psi\|_{p,\varphi_0,\psi_0,\eta}^p$. In
this way, we may use interpolation theory directly, without first assuming that $\varphi$
and $\psi$ are faithful. 

\item Note that we always have $h_\psi=h_{\psi_0}^{1/q}h_\psi^{1/p}\in L^p_R$ and
$\|h_\psi\|_{L^p_R}^p=\|h_\psi^{1/p}\|_p^p=\psi(1)$. 
Assume that $1<\alpha_1\le 2z$ is such that $Q_{\alpha_1,z}(\psi\|\varphi)<\infty$ and let
$1<\alpha<\alpha_1$. Then $\alpha=(1-\theta)\alpha_1+\theta$ for some 
$\theta\in (0,1)$. Using the reiteration theorem as in \cite{FHnote6}, we obtain
\[
Q_{\alpha,z}(\psi\|\varphi)\le Q_{\alpha_1,z}(\psi\|\varphi)^{1-\theta}\psi(1)^{\theta}.
\]
Taking the logarithm and noting that $\theta=(\alpha_1-\alpha)/(\alpha_1-1)$, we obtain
directly that 
\[
D_{\alpha,z}(\psi\|\varphi)\le D_{\alpha_1,z}(\psi\|\varphi).
\]

\item Note that elements of the form $xh_{\varphi_0}^{1/q}$ with $x\in L_p(\Me)$ an
analytic element with respect to $h_{\psi_0}^{it}\cdot h_{\varphi_0}^{-it}$ are contained
in $L^p_L\cap L^p_R$. Since the analytic elements are dense in $L_p(\Me)$ \cite[Lemma
10.4]{kosaki1984applications}, it follows that $L^p_L\cap L^p_R$ is dense in both $L^p_L$
and $L^p_R$. I am not sure if this implies that $L^p_L\cap L^p_R$ is dense in
$L^p_{\eta_1}\cap L^p_{\eta_2}$, though, as required by the usual form of the reiteration
theorem. In any case, the result by Cwikel can be used.

\end{enumerate}


\section{The case $\alpha\in (0,1)$}

We will show that we can use complex interpolation and Kosaki $L_p$-spaces  also in the case $\alpha\in (0,1)$ if
$z>1/2$. 
The proof is easier in the case $z\ge 1$, which we prove first.

\begin{prop}\label{prop:incr}
Assume that $z\ge 1$.  Then 
\begin{enumerate}
\item $\alpha\mapsto \log
Q_{\alpha,z}(\psi\|\varphi)$ is convex on $(0,1)$
\item $\alpha\mapsto D_{\alpha,z}(\psi\|\varphi)$ is monotone increasing on $(0,1)$.
\end{enumerate}

\end{prop}



\begin{proof}  Put $\xi:=h_\psi^{1/2}h_\varphi^{1/2}\in L_1(\Me)$. Let $\alpha\in (0,1)$ and put
$\eta:=\tfrac{z-\alpha}{2z-1}$, so that we have 
\[
0\le 1-\frac q2=\frac{z-1}{2z-1}<\eta< \frac{z}{2z-1}=\frac q2\le 1.
\]
Then 
\[
\xi=h_\psi^{\frac{\eta}q}(h_\psi^{\frac{\alpha}{2z}}h_\varphi^{\frac{1-\alpha}{2z}})h_\varphi^{\frac{1-\eta}q}=
h_{\psi_0}^{\frac{\eta}q}(h_\psi^{\frac{\alpha}{2z}}h_\varphi^{\frac{1-\alpha}{2z}})h_{\varphi_0}^{\frac{1-\eta}q}\in
L^p_\eta
\]
and $Q_{\alpha,z}(\psi\|\varphi)=\|\xi\|^p_{p,\psi_0,\varphi_0,\eta}$. The proof can be
finished by reiteration theorem,  similarly as in \cite[Prop. 0.1]{FHnote6} and the remark
2. in Section 1 above.




\end{proof}



Now we turn to the case $1/2<z<1$. Note that a similar strategy as in the above proof
works only for  restricted values of $\alpha$. We will need a bit more of the complex
interpolation method. Let us denote 
$\Sigma:=\Sigma(L^p_L,L^p_R)=L_L^p+L^p_R$ and let
$\Fe:=\Fe(L^p_L,L^p_R)$ be the set of functions $S:=\{w\in \mathbb C,\ \mathrm{Re}\,(w)\in
[0,1]\}\to \Sigma$ that are 
\begin{enumerate}
\item[(i)] bounded, continuous and analytic in the interior of $S$ (with respect
to the norm in $\Sigma$),
\item[(ii)] $f(it)\in L^p_L$, $f(1+it)\in L^p_R$, $t\in \mathbb R$,
\item[(iii)] the maps $t\mapsto f(it)\in L^p_L$ and $t\mapsto f(1+it)\in L^p_R$ are
continuous and 
\[
\max\{\sup_{t}\|f(it)\|_{p,\varphi_0,L},\sup_{t}\|f(1+it)\|_{p,\psi_0,R}\}<\infty.
\]
\end{enumerate}

%\begin{remark}\label{rem:dense} Note that the set elements of the form $xh_{\varphi_0}^{1/q}$,
%where $x\in L_p(\Me)$ is analytic with respect to $h_{\psi_0}^{it} \cdot
%h_{\varphi_0}^{-it}$ is contained in $L^p_L\cap L^p_R$. Since analytic elements are dense
%in $L_p(\Me)$ \cite[Lemma 10.4]{kosaki1984applications}, we see that $L^p_L\cap L^p_R$ is
%dense in $L^p_L$ and $L^p_R$.
%
%\end{remark}
%
%
We will use the following functions, defined on the strip $S$:
\begin{equation}\label{eq:f}
f(w)= h_\psi^{\frac wq+\frac{1-w}p}h_\varphi^{\frac{1-w}q+\frac wp},\qquad w\in S.
\end{equation}
Note that $f(w)$ is an element in $L_1(\Me)$. The next lemma shows that $f$ has values in
$\Sigma$.

\begin{lemma}\label{lemma:fw}  We have $f\in \Fe$ and  
for each $\eta\in (0,1)$, we have
\[
\|f(\eta+it)\|_{p,\varphi_0,\psi_0,\eta}^p=Q_{1-\eta,z}(\psi\|\varphi).
\]



\end{lemma}

\begin{proof} For $\eta\in [0,1]$ we have
\[
f(\eta+it)=h_\psi^{\frac{\eta}q}h_\psi^{i(\frac 1q-\frac
1p)t}h_\psi^{\frac{1-\eta}p}h_\varphi^{\frac{\eta}p}h_\varphi^{i(\frac 1p-\frac
1q)t}h_\varphi^{\frac{1-\eta}q}=h_{\psi_0}^{\frac{\eta}q}\bigl(h_{\psi_0}^{i(\frac 1q-\frac
1p)t}h_\psi^{\frac{1-\eta}p}h_\varphi^{\frac{\eta}p}h_{\varphi_0}^{i(\frac 1p-\frac
1q)t}\bigr)h_{\varphi_0}^{\frac{1-\eta}q}
\]
By \cite[Lemmas 10.1 and 10.2]{kosaki1984applications}, $h_{\psi_0}^{it} \cdot
h_{\varphi_0}^{-it}$ defines a strongly continuous group of isometries on $L_p(\Me)$ for
every $1\le p\le \infty$. This implies the property (iii) in the definition of $\Fe$.
Also for $\eta\in (0,1)$, we see that $f(\eta+it)\in L^p_\eta$ and
\[
\|f(\eta+it)\|_{p,\varphi_0,\psi_0,\eta}^p=\|h_{\psi_0}^{i(\frac 1q-\frac
1p)t}h_\psi^{\frac{1-\eta}p}h_\varphi^{\frac{\eta}p}h_{\varphi_0}^{i(\frac 1p-\frac
1q)t}\|_p^p=\|h_\psi^{\frac{1-\eta}p}h_\varphi^{\frac{\eta}p}\|_p^p=Q_{1-\eta,z}(\psi\|\varphi).
\]
Since $L^p_\eta$ for each $\eta$ is continuously
embedded in $\Sigma$, this implies that $f$ is $\Sigma$-valued. Since by H\"older
$\|h_\psi^{\frac{1-\eta}p}h_\varphi^{\frac{\eta}p}\|_p\le \psi(1)\varphi(1)$ for any $\eta$, $f$ is also
bounded. Note that as a function with values in $L_1(\Me)$, $f$ is bounded, continuous on
$S$ and analytic in the interior. We now prove that the continuity and analyticity  properties also hold in $\Sigma$
(maybe this is already obvious, but I will give an argument similar to that in \cite[Sec.
9.1,29.1]{calderon1964intermediate} just for the case). Let $\mu_0(w,t)$ and $\mu_1(w,t)$ be the Poisson
kernels associated with $S$. We then have
\[
f(w)=\int_{\mathbb R} f(it)\mu_0(w,t)dt+\int f(1+it)\mu_1(w,t)dt.
\]
The integrals are in $L_1(\Me)$, but since $t\mapsto f(it)\in L^p_L$ and $t\mapsto
f(1+it)\in L^p_R$ are continuous and bounded in the respective norms, we see that the integrals
also exist in $\Sigma$ and since $\Sigma$ is continuously embedded in $L_1(\Me)$, the
above equality holds.  This shows that $f:S\to \Sigma$ is continuous. Therefore, the
expressions 
\[
\frac1{2\pi i}\int_\Gamma \frac{f(\xi)}{\xi-w}d\xi
\]
for a suitable circle $\Gamma$ around a point $w$ in the interior of $S$ are defined in
$\Sigma$. Since $f$ is analytic in $L_1(\Me)$, this
expression is equal to $f(w)$, hence $f$ is analytic in the interior of $S$.

\end{proof}




\begin{prop}\label{prop:incr2}
Assume that $1/2<z< 1$.  Then 
\begin{enumerate}
\item $\alpha\mapsto \log
Q_{\alpha,z}(\psi\|\varphi)$ is convex on $(0,1)$
\item $\alpha\mapsto D_{\alpha,z}(\psi\|\varphi)$ is monotone increasing on $(0,1)$.
\end{enumerate}



\end{prop}



\begin{proof} Let $\alpha_1,\alpha_2\in (0,1)$ and let
$\alpha:=(1-\theta)\alpha_1+\theta\alpha_2$. Put $\eta_i=1-\alpha_i$, $i=1,2$ so that
$\eta:=1-\alpha=(1-\theta)\eta_1+\theta\eta_2$. By the reiteration theorem,
$L^p_\eta=C_\theta(L^p_{\eta_1},L^p_{\eta_2})$. Let $f$ be the function given by
\eqref{eq:f}. Then $f_1:w\mapsto f((1-w)\eta_1+w\eta_2)\in \Fe(L^p_{\eta1},L^p_{\eta_2})$ and
by usual arguments, we have
\[
\|f(\eta)\|_{p,\varphi_0,\psi_0,\eta}=\|f_1(\theta)\|_{C_\theta(L^p_{\eta_1},L^p_{\eta_2})}\le 
(\sup_t\|f_1(it)\|_{L^p_{\eta_1}})^{1-\theta}(\sup_t\|f_1(1+it)\|_{L^p_{\eta_1}})^\theta.
\]
Since $f_1(it)=f(\eta_1+i(\eta_2-\eta_1)t)$ and $f_1(1+it)=f(\eta_2+i(\eta_2-\eta_1)t)$,
we get from Lemma \ref{lemma:fw} that
\[
Q_{1-\eta,z}(\psi\|\varphi)\le
Q_{1-\eta_1,z}(\psi\|\varphi)^{1-\theta}Q_{1-\eta_2,z}(\psi,\varphi)^\theta.
\]
This implies 1. Further, since $f(it)\in L^p_L$ and $\|f(it)\|_{L^p_L}^p=\psi(1)$, we
obtain 2. similarly as in remark 2. of Section 1.


\end{proof}

\begin{thebibliography}{2}
\providecommand{\natexlab}[1]{#1}
\providecommand{\url}[1]{\texttt{#1}}
\expandafter\ifx\csname urlstyle\endcsname\relax
  \providecommand{\doi}[1]{doi: #1}\else
  \providecommand{\doi}{doi: \begingroup \urlstyle{rm}\Url}\fi

\bibitem[Calderón(1964)]{calderon1964intermediate}
A.~Calderón.
\newblock Intermediate spaces and interpolation, the complex method.
\newblock \emph{Studia Mathematica}, 24\penalty0 (2):\penalty0 113--190, 1964.
\bibitem{FHnote6} F. Hiai, Monotonicity of $\alpha\mapsto D_{\alpha,z}$, (12/31/2023)
notes.


\bibitem[Kosaki({1984})]{kosaki1984applications}
H.~Kosaki.
\newblock {Applications of the complex interpolation method to a von Neumann
  algebra: Non-commutative $L_p$-spaces}.
\newblock \emph{{J. Funct. Anal.}}, {56}:\penalty0 {26--78}, {1984}.
\newblock \doi{https://doi.org/10.1016/0022-1236(84)90025-9}.

\end{thebibliography}



%\bibliography{NEW_qre}
%\bibliographystyle{abbrvnat}

\end{document}

\begin{thebibliography}{5}
\providecommand{\natexlab}[1]{#1}
\providecommand{\url}[1]{\texttt{#1}}
\expandafter\ifx\csname urlstyle\endcsname\relax
  \providecommand{\doi}[1]{doi: #1}\else
  \providecommand{\doi}{doi: \begingroup \urlstyle{rm}\Url}\fi


\bibitem[Haagerup et~al.(2010)Haagerup, Junge, and Xu]{haagerup2010areduction}
U.~Haagerup, M.~Junge, and Q.~Xu.
\newblock {A reduction method for noncommutative $L_p$-spaces and
  applications}.
\newblock \emph{Transactions of the American Mathematical Society},
  362\penalty0 (4):\penalty0 2125--2165, 2010.
\newblock \doi{10.1090/S0002-9947-09-04935-6}.
\bibitem{hiai2021quantum}  F. Hiai, Quantum f-Divergences in von Neumann Algebras: Reversibility of Quantum Operations,  Mathematical Physics Studies, Springer Singapore, 2021

\bibitem{FHnote3} F. Hiai, Monotonicity of $z\mapsto D_{\alpha,z}(\psi\|\varphi)$, 
(12/3/2023, 12/8/2023), notes.

\bibitem{AJnote2} A. Jen\v cov\'a, Notes for $\alpha-z$-R\'enyi divergence, December 7,
2023, notes.
\bibitem[Junge and Xu(2003)]{junge2003noncommutative}
M.~Junge and Q.~Xu.
\newblock Noncommutative {B}urkholder/{R}osenthal inequalities.
\newblock \emph{The Annals of Probability}, 31\penalty0 (2):\penalty0 948--995,
  2003.

\bibitem[Takesaki(2003)]{takesaki2003theory2}
M.~Takesaki.
\newblock \emph{Theory of Operator Algebras. {II}}, volume 125 of
  \emph{Encyclopaedia of Mathematical Sciences}.
\newblock Springer-Verlag, Berlin, 2003.
\newblock ISBN 3-540-42914-X.
\newblock \doi{10.1007/978-3-662-10451-4}.


\end{thebibliography}



\end{document}

\begin{thebibliography}{5}
\providecommand{\natexlab}[1]{#1}
\providecommand{\url}[1]{\texttt{#1}}
\expandafter\ifx\csname urlstyle\endcsname\relax
  \providecommand{\doi}[1]{doi: #1}\else
  \providecommand{\doi}{doi: \begingroup \urlstyle{rm}\Url}\fi

\bibitem[Gu et~al.(2019)Gu, Yin, and Zhang]{gu2019interpolation}
J.~Gu, Z.~Yin, and H.~Zhang.
\newblock {Interpolation of quasi noncommutative $L_p$-spaces}.
\newblock \emph{arXiv:1905.08491}, 2019.

\bibitem{FHnotes} F. Hiai, Questions, note.

\bibitem{FHnote2} F. Hiai, Martingale convergence for $D_{\alpha,z}$, note.

\bibitem{AJnote} A. Jen{\v c}ov\'a, DPI for $\alpha-z$-R\'enyi divergence, note.

\bibitem[Kato(2023)]{kato2023onrenyi}
S.~Kato.
\newblock On $\alpha $-$ z $-{R}\'enyi divergence in the von
  {N}eumann algebra setting.
\newblock \emph{arXiv preprint arXiv:2311.01748}, 2023.

\bibitem{SKnote} S.~Kato, Variational expression for $\alpha>1$, note.

\bibitem[Kosaki({1984})]{kosaki1984applications}
H.~Kosaki.
\newblock {Applications of the complex interpolation method to a von Neumann
  algebra: Non-commutative $L_p$-spaces}.
\newblock \emph{{J. Funct. Anal.}}, {56}:\penalty0 {26--78}, {1984}.

\bibitem[Mosonyi(2023)]{mosonyi2023thestrong}
M.~Mosonyi.
\newblock The strong converse exponent of discriminating infinite-dimensional
  quantum states.
\newblock \emph{Communications in Mathematical Physics}, 400\penalty0
  (1):\penalty0 83--132, 2023.

\bibitem[Zhang(2020)]{zhang2020fromwyd}
H.~Zhang.
\newblock From Wigner-Yanase-Dyson conjecture to Carlen-Frank-Lieb conjecture.
\newblock \emph{Advances in Mathematics}, 365:\penalty0 107053, 2020.

\end{thebibliography}







\end{document}



