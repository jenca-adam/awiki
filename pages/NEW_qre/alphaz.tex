\documentclass[12pt]{article}

\usepackage{hyperref}
\usepackage{amsmath, amssymb, amsthm}
\usepackage{xcolor}
\usepackage[sort&compress,numbers]{natbib}
\usepackage{doi}
\usepackage[margin=0.8in]{geometry}
%\textheight23cm \topmargin-20mm  
%\textwidth175mm  
%\oddsidemargin=0mm
%\evensidemargin=0mm
%

\usepackage{amsmath, amssymb, amsthm, mathtools}

\newtheorem{lemma}{Lemma}
\newtheorem{theorem}{Theorem}
\newtheorem{coro}{Corollary}


\theoremstyle{definition}
\newtheorem{defi}{Definition}


\theoremstyle{remark}
\newtheorem{remark}{Remark}

\def\Me{\mathcal M}
\def\Ne{\mathcal N}
\def \Tr{\mathrm{Tr}\,}
\def\states {\mathfrak S}
\def\supp{\mathrm{supp}}
\def\<{\langle\.}
\def\>{\.\rangle}

\title{On the properties $\alpha-z$ R\'enyi divergences on general von Neumann algebras}
\author{Fumio Hiai and Anna Jen\v cov\'a}

\begin{document}

\maketitle


\section{Introduction}

\section{Preliminaries}

\subsection{Basic definitions}

Let $\Me$ be a von Neumann algebra  and let $\Me^+$ be the cone of positive elements in $\Me$. We denote the predual by $\Me_*$, its positive part by $\Me_*^+$ and the set of normal states by $\states_*(\Me)$. For $\psi\in \Me_*^+$, we will denote by $s(\psi)$ the support projection  of $\psi$.

For $0< p\le \infty$, let $L_p(\Me)$ be the Haagerup $L_p$-space over $\Me$ and let
$L_p(\Me)$ its positive cone, 
\cite{haagerup1979Lp}. We will use the identifications $\Me\simeq L_\infty(\Me)$, $\Me_*\ni \psi \leftrightarrow h_\psi\in L_1(\Me)$ and the notation $\Tr h_\psi=\psi(1)$ for the trace in $L_1(\Me)$. It this way,
 $\Me_*^+$  is identified with the positive cone $L_1(\Me)^+$ and $\states_*(\Me)$ with subset of elements in $L_1(\Me)^+$ 
 with unit trace. 
Precise definitions and further details on the spaces $L_p(\Me)$ can be found in the notes \cite{terp1981lpspaces}.
 


%Kosaki, complex interpolation. Generalized s-numbers. Haagerup reduction. Martingale
%convergence


\subsection{The $\alpha-z$-R\'enyi divergences}

In \cite{kato2023aremark, kato2023onrenyi}, the
$\alpha-z$-R\'enyi divergence for $\psi,\varphi\in \mathcal M_*^+$  was defined as
follows: 
\begin{defi}\label{defi:renyi} Let $\psi,\varphi\in \Me_*^+$, $\psi\ne 0$ and let
$\alpha,z>0$, $\alpha\ne 1$. The $\alpha-z$-R\'enyi divergence is defined as 
\[
D_{\alpha,z}(\psi\|\varphi):=\frac1{\alpha-1}\log
\frac{Q_{\alpha,z}(\psi\|\varphi)}{\psi(1)},
\]
where 
\[
Q_{\alpha,z}=\begin{dcases} \Tr
\left(h_\varphi^{(1-\alpha)/2z}h_\psi^{\alpha/z}h_\varphi^{(1-\alpha)/2z}\right)^z, &
\text{if } 0<\alpha<1\\[0.3em]
\|x\|_z^z, & \text{if } \alpha>1 \text{ and }\\
\ & h_\psi^{\alpha/z}=h_\varphi^{(\alpha-1)/2z}xh_\varphi^{(\alpha-1)/2z},\text{ with }
x\in s(\varphi)L_z(\Me)s(\varphi)\\[0.3em]
\infty& \text{otherwise}.
\end{dcases}
\]
\end{defi}

%It is easily checked that this definition coincides with \eqref{eq:fdrenyi} in the finitedimensional case.

In the case $\alpha>1$, the following alternative form will be useful.

\begin{lemma}\cite{kato2023onrenyi} \label{lemma:renyi_2z}
Let $\alpha>1$ and $\psi,\varphi\in \Me_*^+$. Then $Q_{\alpha,z}(\psi\|\varphi)<\infty$ if
and only if there is some $y\in L_{2z}(\Me)s(\varphi)$ such that 
\[
h_\psi^{\alpha/2z}=yh_\varphi^{(\alpha-1)/2z}.
\]
Moreover, in this case, such $y$ is unique and we have
$Q_{\alpha,z}(\psi\|\varphi)=\|y\|_{2z}^{2z}$. 
\end{lemma}

The standard R\'enyi divergence \cite{petz1985quasi, hiai2018quantum, hiai2021quantum} is
contained in this range as $D_\alpha(\psi\|\varphi)=D_{\alpha,1}(\psi\|\varphi)$. The
sandwiched R\'enyi divergence  is obtained as $\tilde
D_\alpha(\psi\|\varphi)=D_{\alpha,\alpha}(\psi\|\varphi)$, see
\cite{berta2018renyi,hiai2021quantum,jencova2018renyi, jencova2021renyi} for some
alternative definitions and properties of $\tilde D_\alpha$. The definition in
\cite{jencova2018renyi} and \cite{jencova2021renyi} is based on the Kosaki interpolation
spaces  $L_p(\Me,\varphi)$ with respect to a state \cite{kosaki1986applications}. These spaces and
complex interpolation method will be used frequently also in the present work. 


Many of the  properties of $D_{\alpha,z}(\psi\|\varphi)$ 
were extended from the finite dimensional case in \cite{kato2023onrenyi}. In particular,
the following variational expressions will be an important tool for our work.

\begin{theorem}\label{thm:variational} Let $\psi,\varphi\in \Me_*^+$, $\psi\ne 0$. 
\begin{enumerate}
\item[(i)] Let $0<\alpha<1$ and $\max\{\alpha,1-\alpha\}\le z$. Then
\[
Q_{\alpha,z}(\psi\|\varphi)=\inf_{a\in \Me^{++}}\left\{\alpha
\Tr\left((a^{1/2}h_\psi^{\alpha/z}a^{1/2})^{z/\alpha}\right)+(1-\alpha)
\Tr\left((a^{-1/2}h_\varphi^{(1-\alpha)/2z}a^{-1/2})^{z/(1-\alpha)}\right) \right\}.
\]
Moreover, if $\lambda^{-1}\varphi\le \psi\le \lambda \varphi$ for some $\lambda>0$, then
the infimum is attained at some $a_0\in \Me^{++}$ satisfying
\[
h_\varphi^{(1-\alpha)/2z}a_0^{-1}h_\varphi^{(1-\alpha)/2z}=\left(h_\varphi^{(1-\alpha)/2z}h_\psi^{\alpha/z}h_\varphi^{(1-\alpha)/2z}\right)^{1-\alpha}
\]
and
\[
\Tr\left((h_\psi^{\alpha/2z}a_0h_\psi^{\alpha/2z})^{z/\alpha}\right)=
\Tr\left((h_\varphi^{(1-\alpha)/2z}h_\psi^{\alpha/z}h_\varphi^{(1-\alpha)/2z})^{z}\right).
\]
\item[(ii)] Let $1<\alpha\le 2z$, then
\[
Q_{\alpha,z}(\psi\|\varphi)=\sup_{a\in \Me_+} \left\{\alpha
\Tr\left((a^{1/2}h_\psi^{\alpha/z}a^{1/2})^{z/\alpha}\right)-(\alpha-1)
\Tr\left((a^{1/2}h_\varphi^{(\alpha-1)/2z}a^{1/2})^{z/(\alpha-1)}\right) \right\}.
\]

\end{enumerate}


\end{theorem}


\begin{proof} For part (i) see \cite[Theorem 1 (vi)]{kato2023onrenyi} and its proof. The
inequality $\ge$ in part (ii) holds for all $\alpha$ and $z$ and was proved in
\cite[Theorem 2 (vi)]{kato2023onrenyi}. We now prove the opposite inequality. 

Assume first that $Q_{\alpha,z}(\psi\|\varphi)<\infty$, so that there is some $x\in
s(\varphi)L_z(\Me)^+s(\varphi)$ such that
$h_\psi^{\alpha/z}=h_\varphi^{(\alpha-1)/2z}xh_\varphi^{(\alpha-1)/2z}$. Plugging this
into the right hand side, we obtain
\begin{align*}
&\sup_{a\in \Me_+} \left\{\alpha
\Tr\left((a^{1/2}h_\psi^{\alpha/z}a^{1/2})^{z/\alpha}\right)-(\alpha-1)
\Tr\left((a^{1/2}h_\varphi^{(\alpha-1)/2z}a^{1/2})^{z/(\alpha-1)}\right) \right\}\\
&=\sup_{a\in \Me_+} \left\{\alpha
\Tr\left((a^{1/2}h_\varphi^{(\alpha-1)/2z}xh_\varphi^{(\alpha-1)/2z}    a^{1/2})^{z/\alpha}\right)-(\alpha-1)
\Tr\left((a^{1/2}h_\varphi^{(\alpha-1)/2z}a^{1/2})^{z/(\alpha-1)}\right) \right\}\\
&=\sup_{a\in \Me_+} \left\{\alpha
\Tr\left((x^{1/2}h_\varphi^{(\alpha-1)/2z}ah_\varphi^{(\alpha-1)/2z}    x^{1/2})^{z/\alpha}\right)-(\alpha-1)
\Tr\left((h_\varphi^{(\alpha-1)/2z}a h_\varphi^{(\alpha-1)/2z} )^{z/(\alpha-1)}\right)
\right\}\\
&=\sup_{w\in L_{z/(\alpha-1)}(\Me)^+} \left\{\alpha
\Tr\left((x^{1/2}wx^{1/2})^{z/\alpha}\right)-(\alpha-1)
\Tr\left(w^{z/(\alpha-1)}\right)
\right\},
\end{align*}
where we used the fact that $\Tr \left((a^*a)^p\right)=\Tr \left((aa^*)^p\right)$ for
$p>0$ and $a\in L_{p/2}(\Me)$ and the fact that the set of
elements of the form $h_\varphi^{(\alpha-1)/2z}a h_\varphi^{(\alpha-1)/2z}$ with $a \in
\Me^+$ is dense in the positive cone $L_{z/(\alpha-1)}(\Me)^+$. Putting $w=x^{\alpha-1}$ we
get
\[
\sup_{w\in L_{z/(\alpha-1)}(\Me)^+} \left\{\alpha
\Tr\left((x^{1/2}wx^{1/2})^{z/\alpha}\right)-(\alpha-1)
\Tr\left(w^{z/(\alpha-1)}\right)
\right\}\ge \Tr(x^z)=\|x\|_z^z= Q_{\alpha,z}(\psi\|\varphi).
\]
This finishes the proof in the case that $Q_{\alpha,z}(\psi\|\varphi)<\infty$.  Note that
this holds if $\psi\le \lambda\varphi$ for some $\lambda>0$. Indeed, since
$\alpha/2z\in (0,1]$ by the assumption, we then have 
\[
h_\psi^{\alpha/2z}\le \lambda^{\alpha/2z}h_\varphi^{\alpha/2z},
\]
hence by \cite[Lemma A.58]{hiai2021quantum} there is some $b\in \Me$ such that 
\[
h_\psi^{\alpha/2z}=bh_\varphi^{\alpha/2z}=yh_\varphi^{(\alpha-1)/2z},
\]
where $y=bh_\varphi^{1/2z}\in L_{2z}(\Me)$. By Lemma \ref{lemma:renyi_2z} we get 
$Q_{\alpha,z}(\psi\|\varphi)=\|y\|_{2z}^{2z}<\infty$. 

In the general case, note that lower semicontinuity \cite[]{kato2023onrenyi}, we have
\[
Q_{\alpha,z}(\psi\|\varphi)\le \liminf_{\epsilon\searrow 0}
Q_{\alpha,z}(\psi\|\varphi+\epsilon \psi) 
\]
and by the previous paragraph, the variational expression holds for
$Q_{\alpha,z}(\psi\|\varphi)$  for all $\epsilon>0$. The proof is finished 
by using norm continuity of the map $L_1(\Me)^+\ni h\mapsto h^{1/p}\in L_p(\Me)^+$ for
$p>1$. 
 

 {\color{red} I am not entirely sure about this proof, since the convergence of the
 expressions 
\[
 \Tr\left((a^{1/2}(h_\varphi+\epsilon
 h_\psi)^{(\alpha-1)/2z}a^{1/2})^{z/(\alpha-1)}\right)\to
 \Tr\left((a^{1/2}h_\varphi^{(\alpha-1)/2z}a^{1/2})^{z/(\alpha-1)}\right)\]
 also depends
 on $\|a\|$, which is not bounded. But probably I misinterpreted something, or I am just being
 stupid. }
 
\end{proof}

\section{Data processing inequality and reversibility of quantum channels}

Let  $\gamma: \Ne\to \Me$ be a normal positive unital map. Then the  predual of $\gamma$  defines a 
positive linear map $\gamma_*: L_1(\Me)\to L_1(\Ne)$ that preserves the trace. The support
of $\gamma$ will be denoted by $s(\gamma)$, recall that this is the largest projection
$p\in \Ne$ such that $\gamma(p)=1$. For any $\rho\in \Me_*^+$ we clearly have
$s(\rho\circ\gamma)\le s(\gamma)$, with equality if $\rho$ is faithful. 
It follows that $\gamma_*$ maps $L_1(\Me)$ to $s(\gamma)L_1(\Ne)s(\gamma)\equiv
L_1(s(\gamma)\Ne s(\gamma))$.  For any $\rho\in \Me_+^*$, $\rho\ne 0$, the map
\[
s(\gamma)\Ne s(\gamma)\to s(\rho)\Me s(\rho),\qquad a\mapsto s(\rho) \gamma(a)s(\rho)
\]
is a faithful normal positive unital map, so using such  restrictions we may always assume that both $\rho$ and $\rho\circ
\gamma$ are faithful.

The Petz dual  of $\gamma$ with respect to a faithful  $\rho\in \Me_*^+$
is a map $\gamma_\rho^*:\Me\to \Ne$,
introduced in \cite{petz...,petz1988sufficiency}. It was proved that it is again
normal, positive and unital, in addition, it is $n$-positive whenever $\gamma$ is. 
As explained in \cite{jencova2018renyi} $\gamma^*_\rho$ is determined by the equality
\begin{equation}\label{eq:petzdual}
(\gamma^*_\rho)_*(h_{\rho\circ\gamma}^{1/2}bh_{\rho\circ\gamma}^{1/2})=h_\rho^{1/2}\gamma(b)h_\rho^{1/2},
\end{equation}
for all $b\in \Ne^+$, here $(\gamma^*_\rho)_*$ is the predual map of $\gamma^*_\rho$. We
also have
\[
(\gamma^*_\rho)_*(h_{\rho\circ\gamma})=(\gamma^*_\rho)_*\circ \gamma_*(h_\rho)=h_\rho.
\]


\subsection{Data processing inequality}


In this paragraph we prove the data processing inequality (DPI) for $D_{\alpha,z}$ with respect to normal
positive unital maps. In the case of the sandwiched divergences $\tilde D_\alpha$ with
$1/2\le \alpha \ne 1$, DPI was proved in \cite{jencova2018renyi,
jencova2021renyi}, see also \cite{berta2018renyi} for an alternative proof in the case
when the maps are  also completely positive.

\begin{lemma}\label{lemma:dpi} Let $\gamma:\Ne\to \Me$ be a normal positive unital map and
let $\rho\in \Me_*^+$, $b\in \Ne^+$. 
\begin{enumerate}
\item[(i)]  If $p\in [1/2,1)$, then 
\[
\|h_{\rho\circ\gamma}^{\frac{1}{2p}}bh_{\rho\circ\gamma}^{\frac{1}{2p}}\|_p\le
\|h_{\rho}^{\frac{1}{2p}}\gamma(b)h_{\rho}^{\frac{1}{2p}}\|_p.
\]

\item[(ii)]  If $p\in [1,\infty]$, the inequality reverses.

\end{enumerate}


\end{lemma}

\begin{proof} Let us denote $\beta:=\gamma_\rho^*$ and let $\omega\in \Me_*^+$ be such
that 
$h_\omega:=h_{\rho\circ\gamma}^{1/2}bh_{\rho\circ\gamma}^{1/2}\in L_1(\Ne)^+$. Then
$\beta$ is a normal positive unital map and  we have 
\[
\beta_*(h_\omega)=h_\rho^{1/2}\gamma(b)h_\rho^{1/2},\qquad
\beta_*(h_{\rho\circ\gamma})=h_\rho.
\]
Let $p\in [1/2,1)$, then  
\begin{align*}
\|h_{\rho}^{\frac{1}{2p}}\gamma(b)h_{\rho}^{\frac{1}{2p}}\|^p_p&=
\|h_\rho^{\frac{1-p}{2p}}\beta_*(h_\omega)h_\rho^{\frac{1-p}{2p}}\|_p^p=
Q_{p,p}(\beta_*(h_\omega)\|h_\rho)=Q_{p,p}(\beta_*(h_\omega)\|\beta_*(h_{\rho\circ\gamma}))\\
&\ge  Q_{p,p}(h_\omega\|h_{\rho\circ\gamma})=\|h_{\rho\circ\gamma}^{\frac{1-p}{2p}}h_\omega
h_{\rho\circ\gamma}^{\frac{1-p}{2p}}\|_p^p=\|h_{\rho\circ\gamma}^{\frac{1}{2p}}bh_{\rho\circ\gamma}^{\frac{1}{2p}}\|^p_p.
\end{align*}
Here we have used the DPI for the sandwiched R\'enyi  divergence $D_{\alpha,\alpha}$ for
$\alpha\in [1/2,1)$, \cite[Theorem 4.1]{jencova2021renyi}.  This proves (i). 
The case (ii) was proved in \cite{kato2023onrenyi} (see Eq. (22) therein), using the
relation of the sandwiched R\'enyi divergence to the Kosaki $L_p$ norms as
\begin{align*}
\|h_\rho^{1/2p}\gamma(b)h_\rho^{1/2p}\|_p^p&=Q_{p,p}(h_\rho^{1/2}\gamma(b)h_\rho^{1/2}\|h_\rho)=Q_{p,p}(\beta_*(h_\omega)\|\beta_*(h_{\rho\circ\gamma}))\\
&\le
Q_{p,p}(h_\omega\|h_{\rho\circ\gamma})=\|h_{\rho\circ\gamma}^{\frac{1}{2p}}bh_{\rho\circ\gamma}^{\frac{1}{2p}}\|^p_p,
\end{align*}
here the inequality follows from the DPI for sandwiched R\'enyi divergence $D_{p,p}$ with
$p>1$, \cite[]{jencova2018renyi}.


\end{proof}




\begin{theorem}[DPI] \label{thm:dpi} Let $\psi,\varphi\in \Me_*^+$, $\psi\ne 0$ and let $\gamma:
\Ne\to \Me$ be a normal positive unital map. Assume either of the following conditions:
\begin{enumerate}
\item[(i)] $0<\alpha<1$, $\max\{\alpha,1-\alpha\}\le z$
\item[(ii)] $\alpha>1$, $\max\{\alpha/2,\alpha-1\}\le z\le \alpha$.
\end{enumerate}
Then we have
\[
D_{\alpha,z}(\psi\circ\gamma\|\varphi\circ\gamma)\le D_{\alpha,z}(\psi\|\varphi).
\]


\end{theorem}


\begin{proof} Under the conditions (i), the DPI was proved in \cite[Theorem 1
(viii)]{kato2023onrenyi}. There an additional assumption was used, namely  that $\Ne$ is
$\sigma$-finite, to construct a faithful map out of $\gamma$. This can be done using the
restriction described above, so the additional condition is not needed. 

We next prove that DPI holds under the condition (ii). 

\end{proof}












Let $\psi$ be a faithful normal state on a von Neumann algebra $\Me$. We will prove the following inequality:
\begin{equation}\label{eq:goal}
\|h_{\psi\circ\gamma}^{\frac{1}{2p}}bh_{\psi\circ\gamma}^{\frac{1}{2p}}\|_p\le
\|h_{\psi}^{\frac{1}{2p}}\gamma(b)h_{\psi}^{\frac{1}{2p}}\|_p
\end{equation}
for all $p\in [1/2,1]$, all $b\in \Ne^+$ and any unital positive  map $\gamma: \Ne\to
\Me$ (Eq. (19) in \cite{kato2023onrenyi}). This then implies DPI for the $\alpha-z$-R\'enyi divergence for
$\alpha/2,\alpha-1\le z\le \alpha$.

Let $\gamma_\psi^*$ be the Petz dual of $\gamma$ with respect to $\psi$, then 
its predual satisfies
\[
(\gamma^*_\psi)_*(h_{\psi\circ\gamma}^{1/2}bh_{\psi\circ\gamma}^{1/2})=h_\psi^{1/2}\gamma(b)h_\psi^{1/2}
\]
(this is eq. (21) in \cite{kato2023onrenyi}). Put
$h_\omega:=h_{\psi\circ\gamma}^{1/2}bh_{\psi\circ\gamma}^{1/2}\in L_1(\Ne)^+$. We then have, using Thm.
4.1 in \cite{jencova2021renyi}
\begin{align*}
\|h_{\psi}^{\frac{1}{2p}}\gamma(b)h_{\psi}^{\frac{1}{2p}}\|^p_p&=
\|h_\psi^{\frac{1-p}{2p}}(\gamma_\psi^*)_*(h_\omega)h_\psi^{\frac{1-p}{2p}}\|^p=\tilde
Q_p((\gamma_\psi^*)_*(h_\omega)\|(\gamma_\psi^*)_*(h_{\psi\circ\gamma}))\\
&\ge \tilde Q_p(h_\omega\|h_{\psi\circ\gamma})=\|
h_{\psi\circ\gamma}^{\frac{1-p}{2p}}h_\omega
h_{\psi\circ\gamma}^{\frac{1-p}{2p}}\|^p=\|h_{\psi\circ\gamma}^{\frac{1}{2p}}bh_{\psi\circ\gamma}^{\frac{1}{2p}}\|^p_p.
\end{align*}

\begin{thebibliography}{99}

\bibitem{jencova2021renyi} A. Jen\v cov\'a, Rényi relative entropies and noncommutative
$L_p$-spaces II, Ann. Henri Poincaré 22, 3235–3254 (2021)
\bibitem{kato2023onrenyi} Shinya Kato, On $\alpha-z$-R\'enyi divergence in the von Neumann
algebra setting, arXiv:2311.01748.

\end{thebibliography}



\end{document}


\|h_{\psi\circ\gamma}^{\frac{1}{2p}}bh_{\psi\circ\gamma}^{\frac{1}{2p}}\|^p_p=
\Tr(h_{\psi\circ\gamma}^{\frac{1-p}{2p}}
\end{align*}




\end{document}

We will use the results in \cite{gu2023interpolation} on interpolation of Haagerup
$L_p$-spaces. There, the space $L_p(\Me,\psi)$ for $0<p\le \infty$ is defined as
follows. For $x\in \Me$, define the norm
\[
\|x\|_{p,\psi}=\|h_\psi^{1/2p}xh_\psi^{1/2p}\|_p,
\]
the space $L_p(\Me,\psi)$ is the completion of $\Me$ under this norm. For $p\ge 1$, the
space can be identified with the Kosaki $L_p$-space. It is shown \cite[Thm.
4.1]{gu2023interpolation} that for any $0<p_0<p_1\le \infty$, the space
$L_{p_\theta}(\Me,\psi)$ is obtained by complex interpolation:
\[
L_{p_\theta}(\Me,\psi)=[L_{p_0}(\Me,\psi),L_{p_1}(\Me,\psi)]_\theta,\qquad
1/p_\theta=(1-\theta)/p_0+\theta/p_1.
\]

We need  to prove the inequality
\[
\|b\|_{p,\psi\circ\gamma}\le \|\gamma(b)\|_{p,\psi}
\]
for any $b\in \Ne^+$. By the complex interpolation result, it should be enough to prove the
inequality for the two extremal cases: $p=1/2$ and $p=1$. 

Let  $p=1/2$ and let $b\in \Ne$, $b=v|b|$ be the polar decomposition. Assume wlog that $\|b\|\le 1$.
Then by usin H\"older inequality,
\begin{align*}
\|b\|_{1/2,\psi\circ\gamma}&=\|h_{\psi\circ\gamma}v|b|h_{\psi\circ\gamma}\|_{1/2}\le \||b|^{1/2}h_{\psi\circ\gamma}\|_1=\Tr
u |b|^{1/2}h_{\psi\circ\gamma}=\Tr \gamma(u|b|^{1/2})h_\psi\\
&\le \|\gamma(u|b|^{1/2})h_\psi\|_1=\Tr h_\psi\gamma(u|b|^{1/2})^*\gamma(u|b|^{1/2})h_\psi\le
\Tr (h_\psi\gamma(|b|)h_\psi)^{1/2}=\|\gamma(|b|)\|_{1/2,\psi}.
\end{align*}

For $p=1$, we have
\begin{align*}
\|b\|_{1,\psi\circ\gamma}=\Tr h_{\psi\circ\gamma}^{1/2}bh_{\psi\circ\gamma}^{1/2}=\Tr
bh_{\psi\circ\gamma}=\Tr \gamma(b)h_\psi=\|\gamma(b)\|_{1,\psi}.
\end{align*}

The inequality for all $p\in [1/2,1]$ now should follow by complex interpolation: ale to
fakt neviem jak! Je to opacne!



\end{document}

