\documentclass[12pt]{article}

\usepackage{hyperref}
\usepackage{amsmath, amssymb, amsthm}
\usepackage[sort&compress,numbers]{natbib}
\usepackage{doi}
\usepackage[margin=0.8in]{geometry}
%\textheight23cm \topmargin-20mm  
%\textwidth175mm  
%\oddsidemargin=0mm
%\evensidemargin=0mm
%

\usepackage{amsmath, amssymb, amsthm, mathtools}

\newtheorem{lemma}{Lemma}
\newtheorem{theorem}{Theorem}
\newtheorem{coro}{Corollary}


\theoremstyle{definition}
\newtheorem{defi}{Definition}


\theoremstyle{remark}
\newtheorem{remark}{Remark}

\def\Me{\mathcal M}
\def\Ne{\mathcal N}
\def \Tr{\mathrm{Tr}\,}
\def\Se {\mathcal S}
\def\supp{\mathrm{supp}}
\def\<{\langle\.}
\def\>{\.\rangle}

\title{On the properties $\alpha-z$ R\'enyi divergences on general von Neumann algebras}
\author{Fumio Hiai and Anna Jen\v cov\'a}

\begin{document}

\maketitle


\section{Introduction}

\section{Preliminaries}

\subsection{Basic definitions}

Let $\mathcal M$ be a von Neumann algebra and let $\mathcal M_*$ be its predual. We will
denote the cone of positive operators in $\mathcal M$ by $\mathcal M^+$, similarly,
$\mathcal M_*^+$ will denote the cone of positive normal linear functionals on $\mathcal
M$.  

Haagerup $L_p$ spaces

Kosaki, complex interpolation. Generalized s-numbers. Haagerup reduction.


\subsection{The $\alpha-z$-R\'enyi divergences}

In \cite{kato2023onrenyi}, the
$\alpha-z$-R\'enyi divergence for $\psi,\varphi\in \mathcal M_*^+$  was defined as
follows: 




The following variational formulas will be an important tool for our work.

\begin{theorem}\label{thm:variational}
\begin{enumerate}
\item[(i)] $\alpha<1$, + attained
\item[(ii)] $\alpha>1$

\end{enumerate}


\end{theorem}


\begin{proof} \cite{kato2023onrenyi} for (i).


\end{proof}

\section{Data processing inequality and reversibility of quantum channels}


\end{document}


Let $\psi$ be a faithful normal state on a von Neumann algebra $\Me$. We will prove the following inequality:
\begin{equation}\label{eq:goal}
\|h_{\psi\circ\gamma}^{\frac{1}{2p}}bh_{\psi\circ\gamma}^{\frac{1}{2p}}\|_p\le
\|h_{\psi}^{\frac{1}{2p}}\gamma(b)h_{\psi}^{\frac{1}{2p}}\|_p
\end{equation}
for all $p\in [1/2,1]$, all $b\in \Ne^+$ and any unital positive  map $\gamma: \Ne\to
\Me$ (Eq. (19) in \cite{kato2023onrenyi}). This then implies DPI for the $\alpha-z$-R\'enyi divergence for
$\alpha/2,\alpha-1\le z\le \alpha$.

Let $\gamma_\psi^*$ be the Petz dual of $\gamma$ with respect to $\psi$, then 
its predual satisfies
\[
(\gamma^*_\psi)_*(h_{\psi\circ\gamma}^{1/2}bh_{\psi\circ\gamma}^{1/2})=h_\psi^{1/2}\gamma(b)h_\psi^{1/2}
\]
(this is eq. (21) in \cite{kato2023onrenyi}). Put
$h_\omega:=h_{\psi\circ\gamma}^{1/2}bh_{\psi\circ\gamma}^{1/2}\in L_1(\Ne)^+$. We then have, using Thm.
4.1 in \cite{jencova2021renyi}
\begin{align*}
\|h_{\psi}^{\frac{1}{2p}}\gamma(b)h_{\psi}^{\frac{1}{2p}}\|^p_p&=
\|h_\psi^{\frac{1-p}{2p}}(\gamma_\psi^*)_*(h_\omega)h_\psi^{\frac{1-p}{2p}}\|^p=\tilde
Q_p((\gamma_\psi^*)_*(h_\omega)\|(\gamma_\psi^*)_*(h_{\psi\circ\gamma}))\\
&\ge \tilde Q_p(h_\omega\|h_{\psi\circ\gamma})=\|
h_{\psi\circ\gamma}^{\frac{1-p}{2p}}h_\omega
h_{\psi\circ\gamma}^{\frac{1-p}{2p}}\|^p=\|h_{\psi\circ\gamma}^{\frac{1}{2p}}bh_{\psi\circ\gamma}^{\frac{1}{2p}}\|^p_p.
\end{align*}

\begin{thebibliography}{99}

\bibitem{jencova2021renyi} A. Jen\v cov\'a, Rényi relative entropies and noncommutative
$L_p$-spaces II, Ann. Henri Poincaré 22, 3235–3254 (2021)
\bibitem{kato2023onrenyi} Shinya Kato, On $\alpha-z$-R\'enyi divergence in the von Neumann
algebra setting, arXiv:2311.01748.

\end{thebibliography}



\end{document}


\|h_{\psi\circ\gamma}^{\frac{1}{2p}}bh_{\psi\circ\gamma}^{\frac{1}{2p}}\|^p_p=
\Tr(h_{\psi\circ\gamma}^{\frac{1-p}{2p}}
\end{align*}




\end{document}

We will use the results in \cite{gu2023interpolation} on interpolation of Haagerup
$L_p$-spaces. There, the space $L_p(\Me,\psi)$ for $0<p\le \infty$ is defined as
follows. For $x\in \Me$, define the norm
\[
\|x\|_{p,\psi}=\|h_\psi^{1/2p}xh_\psi^{1/2p}\|_p,
\]
the space $L_p(\Me,\psi)$ is the completion of $\Me$ under this norm. For $p\ge 1$, the
space can be identified with the Kosaki $L_p$-space. It is shown \cite[Thm.
4.1]{gu2023interpolation} that for any $0<p_0<p_1\le \infty$, the space
$L_{p_\theta}(\Me,\psi)$ is obtained by complex interpolation:
\[
L_{p_\theta}(\Me,\psi)=[L_{p_0}(\Me,\psi),L_{p_1}(\Me,\psi)]_\theta,\qquad
1/p_\theta=(1-\theta)/p_0+\theta/p_1.
\]

We need  to prove the inequality
\[
\|b\|_{p,\psi\circ\gamma}\le \|\gamma(b)\|_{p,\psi}
\]
for any $b\in \Ne^+$. By the complex interpolation result, it should be enough to prove the
inequality for the two extremal cases: $p=1/2$ and $p=1$. 

Let  $p=1/2$ and let $b\in \Ne$, $b=v|b|$ be the polar decomposition. Assume wlog that $\|b\|\le 1$.
Then by usin H\"older inequality,
\begin{align*}
\|b\|_{1/2,\psi\circ\gamma}&=\|h_{\psi\circ\gamma}v|b|h_{\psi\circ\gamma}\|_{1/2}\le \||b|^{1/2}h_{\psi\circ\gamma}\|_1=\Tr
u |b|^{1/2}h_{\psi\circ\gamma}=\Tr \gamma(u|b|^{1/2})h_\psi\\
&\le \|\gamma(u|b|^{1/2})h_\psi\|_1=\Tr h_\psi\gamma(u|b|^{1/2})^*\gamma(u|b|^{1/2})h_\psi\le
\Tr (h_\psi\gamma(|b|)h_\psi)^{1/2}=\|\gamma(|b|)\|_{1/2,\psi}.
\end{align*}

For $p=1$, we have
\begin{align*}
\|b\|_{1,\psi\circ\gamma}=\Tr h_{\psi\circ\gamma}^{1/2}bh_{\psi\circ\gamma}^{1/2}=\Tr
bh_{\psi\circ\gamma}=\Tr \gamma(b)h_\psi=\|\gamma(b)\|_{1,\psi}.
\end{align*}

The inequality for all $p\in [1/2,1]$ now should follow by complex interpolation: ale to
fakt neviem jak! Je to opacne!



\end{document}

