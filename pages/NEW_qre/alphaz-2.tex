\documentclass[12pt]{article}

\usepackage{hyperref}
%\usepackage[dvipdfmx]{hyperref}
\usepackage{amsmath, amssymb, amsthm}
\usepackage{xcolor}
\usepackage[sort&compress,numbers]{natbib}
\usepackage{doi}
\usepackage[margin=0.8in]{geometry}
%\textheight23cm \topmargin-20mm  
%\textwidth175mm  
%\oddsidemargin=0mm
%\evensidemargin=0mm
%

\usepackage{amsmath, amssymb, amsthm, mathtools}

\newtheorem{theorem}{Theorem}[section]
\newtheorem{lemma}[theorem]{Lemma}
\newtheorem{coro}[theorem]{Corollary}
\newtheorem{prop}[theorem]{Proposition}

\theoremstyle{definition}
\newtheorem{defi}[theorem]{Definition}

\theoremstyle{remark}
\newtheorem{remark}[theorem]{Remark}

\numberwithin{equation}{section}

\def\cE{\mathcal E}
\def\cM{\mathcal M}
\def\Me{\mathcal M}
\def\Ne{\mathcal N}
\def \Tr{\mathrm{Tr}\,}
\def\states {\mathfrak S}
\def\supp{\mathrm{supp}}
\def\<{\langle\.}
\def\>{\.\rangle}
\def\ffi{\varphi}
\def\1{\mathbf{1}}
\def\eps{\varepsilon}
\def\bN{\mathbb{N}}

\title{On the properties $\alpha-z$ R\'enyi divergences on general von Neumann algebras}
\author{Fumio Hiai and Anna Jen\v cov\'a}

\begin{document}

\maketitle


\section{Introduction}

\section{Preliminaries}

\subsection{Basic definitions}

Let $\Me$ be a von Neumann algebra  and let $\Me^+$ be the cone of positive elements in $\Me$. We denote the predual by $\Me_*$, its positive part by $\Me_*^+$ and the set of normal states by $\states_*(\Me)$. For $\psi\in \Me_*^+$, we will denote by $s(\psi)$ the support projection  of $\psi$.

For $0< p\le \infty$, let $L_p(\Me)$ be the Haagerup $L_p$-space over $\Me$ and let
$L_p(\Me)$ its positive cone, 
\cite{haagerup1979lpspaces}. We will use the identifications $\Me\simeq L_\infty(\Me)$, $\Me_*\ni \psi \leftrightarrow h_\psi\in L_1(\Me)$ and the notation $\Tr h_\psi=\psi(1)$ for the trace in $L_1(\Me)$. It this way,
 $\Me_*^+$  is identified with the positive cone $L_1(\Me)^+$ and $\states_*(\Me)$ with subset of elements in $L_1(\Me)^+$ 
 with unit trace. 
Precise definitions and further details on the spaces $L_p(\Me)$ can be found in the notes \cite{terp1981lpspaces}.
 


%Kosaki, complex interpolation. Generalized s-numbers. Haagerup reduction. Martingale
%convergence


\subsection{The $\alpha-z$-R\'enyi divergences}

In \cite{kato2023aremark, kato2023onrenyi}, the
$\alpha-z$-R\'enyi divergence for $\psi,\varphi\in \mathcal M_*^+$  was defined as
follows: 
\begin{defi}\label{defi:renyi} Let $\psi,\varphi\in \Me_*^+$, $\psi\ne 0$ and let
$\alpha,z>0$, $\alpha\ne 1$. The $\alpha-z$-R\'enyi divergence is defined as 
\[
D_{\alpha,z}(\psi\|\varphi):=\frac1{\alpha-1}\log
\frac{Q_{\alpha,z}(\psi\|\varphi)}{\psi(1)},
\]
where
\[
Q_{\alpha,z}=\begin{dcases} \Tr
\left(h_\varphi^{\frac{1-\alpha}{2z}}h_\psi^{\frac{\alpha}{z}}h_\varphi^{\frac{1-\alpha}{2z}}\right)^z, &
\text{if } 0<\alpha<1\\[0.3em]
\|x\|_z^z, & \text{if } \alpha>1 \text{ and }\\
\ &
h_\psi^{\frac{\alpha}{z}}=h_\varphi^{\frac{\alpha-1}{2z}}xh_\varphi^{\frac{\alpha-1}{2z}},\text{ with }
x\in s(\varphi)L_z(\Me)s(\varphi)\\[0.3em]
\infty& \text{otherwise}.
\end{dcases}
\]
\end{defi}

%It is easily checked that this definition coincides with \eqref{eq:fdrenyi} in the finitedimensional case.

In the case $\alpha>1$, the following alternative form will be useful.

\begin{lemma}\cite[Lemma 7]{kato2023onrenyi} \label{lemma:renyi_2z}
Let $\alpha>1$ and $\psi,\varphi\in \Me_*^+$. Then $Q_{\alpha,z}(\psi\|\varphi)<\infty$ if
and only if there is some $y\in L_{2z}(\Me)s(\varphi)$ such that 
\[
h_\psi^{\frac{\alpha}{2z}}=yh_\varphi^{\frac{\alpha-1}{2z}}.
\]
Moreover, in this case, such $y$ is unique and we have
$Q_{\alpha,z}(\psi\|\varphi)=\|y\|_{2z}^{2z}$. 
\end{lemma}

The standard R\'enyi divergence \cite{petz1985quasi, hiai2018quantum, hiai2021quantum} is
contained in this range as $D_\alpha(\psi\|\varphi)=D_{\alpha,1}(\psi\|\varphi)$. The
sandwiched R\'enyi divergence  is obtained as $\tilde
D_\alpha(\psi\|\varphi)=D_{\alpha,\alpha}(\psi\|\varphi)$, see
\cite{berta2018renyi,hiai2021quantum,jencova2018renyi, jencova2021renyi} for some
alternative definitions and properties of $\tilde D_\alpha$. The definition in
\cite{jencova2018renyi} and \cite{jencova2021renyi} is based on the Kosaki interpolation
spaces  $L_p(\Me,\varphi)$ with respect to a state \cite{kosaki1984applications}. These spaces and
complex interpolation method will be used frequently also in the present work. 


Many of the  properties of $D_{\alpha,z}(\psi\|\varphi)$ 
were extended from the finite dimensional case in \cite{kato2023onrenyi}. In particular,
the following variational expressions will be an important tool for our work.

\begin{theorem}[Variational expressions]\label{thm:variational} Let $\psi,\varphi\in \Me_*^+$, $\psi\ne 0$. 
\begin{enumerate}
\item[(i)] Let $0<\alpha<1$ and $\max\{\alpha,1-\alpha\}\le z$. Then
\[
Q_{\alpha,z}(\psi\|\varphi)=\inf_{a\in \Me^{++}}\left\{\alpha
\Tr\left((a^{\frac12}h_\psi^{\frac{\alpha}{z}}a^{\frac12})^{\frac{z}{\alpha}}\right)+(1-\alpha)
\Tr\left((a^{\frac12}h_\varphi^{\frac{1-\alpha}{z}}a^{\frac12})^{\frac{z}{1-\alpha}}\right) \right\}.
\]

\item[(ii)] Let $1<\alpha$, $\max\{\frac{\alpha}{2},\alpha-1\}\le z$. Then
\[
Q_{\alpha,z}(\psi\|\varphi)=\sup_{a\in \Me_+} \left\{\alpha
\Tr\left((a^{\frac12}h_\psi^{\frac{\alpha}{z}}a^{\frac12})^{\frac{z}{\alpha}}\right)-(\alpha-1)
\Tr\left((a^{\frac12}h_\varphi^{\frac{\alpha-1}{2z}}a^{\frac12})^{\frac{z}{\alpha-1}}\right) \right\}.
\]

\end{enumerate}


\end{theorem}


\begin{proof} For part (i) see \cite[Theorem 1 (vi)]{kato2023onrenyi}. The
inequality $\ge$ in part (ii) holds for all $\alpha$ and $z$ and was proved in
\cite[Theorem 2 (vi)]{kato2023onrenyi}. We now prove the opposite inequality. 

Assume first that $Q_{\alpha,z}(\psi\|\varphi)<\infty$, so that there is some $x\in
s(\varphi)L_z(\Me)^+s(\varphi)$ such that
$h_\psi^{\frac{\alpha}{z}}=h_\varphi^{\frac{\alpha-1}{2z}}xh_\varphi^{\frac{\alpha-1}{2z}}$. Plugging this
into the right hand side, we obtain
\begin{align*}
&\sup_{a\in \Me_+} \left\{\alpha
\Tr\left((a^{\frac12}h_\psi^{\frac{\alpha}{z}}a^{\frac12})^{\frac{z}{\alpha}}\right)-(\alpha-1)
\Tr\left((a^{\frac12}h_\varphi^{\frac{\alpha-1}{2z}}a^{\frac12})^{\frac{z}{\alpha-1}}\right) \right\}\\
&=\sup_{a\in \Me_+} \left\{\alpha
\Tr\left((a^{\frac12}h_\varphi^{\frac{\alpha-1}{2z}}xh_\varphi^{\frac{\alpha-1}{2z}}
a^{\frac12})^{\frac{z}{\alpha}}\right)-(\alpha-1)
\Tr\left((a^{\frac12}h_\varphi^{\frac{\alpha-1}{2z}}a^{\frac12})^{\frac{z}{\alpha-1}}\right) \right\}\\
&=\sup_{a\in \Me_+} \left\{\alpha
\Tr\left((x^{\frac12}h_\varphi^{\frac{\alpha-1}{2z}}ah_\varphi^{\frac{\alpha-1}{2z}}
x^{\frac12})^{\frac{z}{\alpha}}\right)-(\alpha-1)
\Tr\left((h_\varphi^{\frac{\alpha-1}{2z}}a h_\varphi^{\frac{\alpha-1}{2z}}
)^{\frac{z}{\alpha-1}}\right)
\right\}\\
&=\sup_{w\in L_{\frac{z}{\alpha-1}}(\Me)^+} \left\{\alpha
\Tr\left((x^{\frac12}wx^{\frac12})^{\frac{z}{\alpha}}\right)-(\alpha-1)
\Tr\left(w^{\frac{z}{\alpha-1}}\right)
\right\},
\end{align*}
where we used the fact that $\Tr \left((h^*h)^p\right)=\Tr \left((hh^*)^p\right)$ for
$p>0$ and $h\in L_{\frac{p}{2}}(\Me)$ and Lemma \ref{lemma:cone}.
 Putting $w=x^{\alpha-1}$ we
get
\[
\sup_{w\in L_{\frac{z}{\alpha-1}}(\Me)^+} \left\{\alpha
\Tr\left((x^{\frac12}wx^{\frac12})^{\frac{z}{\alpha}}\right)-(\alpha-1)
\Tr\left(w^{\frac{z}{\alpha-1}}\right)
\right\}\ge \Tr(x^z)=\|x\|_z^z= Q_{\alpha,z}(\psi\|\varphi).
\]
This finishes the proof in the case that $Q_{\alpha,z}(\psi\|\varphi)<\infty$.  Note that
this holds if $\psi\le \lambda\varphi$ for some $\lambda>0$. Indeed, since
$\frac{\alpha}{2z}\in (0,1]$ by the assumption, we then have 
\[
h_\psi^{\frac{\alpha}{2z}}\le \lambda^{\frac{\alpha}{2z}}h_\varphi^{\frac{\alpha}{2z}},
\]
hence by \cite[Lemma A.58]{hiai2021quantum} there is some $b\in \Me$ such that 
\[
h_\psi^{\frac{\alpha}{2z}}=bh_\varphi^{\frac{\alpha}{2z}}=yh_\varphi^{\frac{\alpha-1}{2z}},
\]
where $y=bh_\varphi^{\frac{1}{2z}}\in L_{2z}(\Me)$. By Lemma \ref{lemma:renyi_2z} we get 
$Q_{\alpha,z}(\psi\|\varphi)=\|y\|_{2z}^{2z}<\infty$. 

In the general case,  the variational expression holds for
$Q_{\alpha,z}(\psi\|\varphi+\epsilon\psi)$  for all $\epsilon>0$, so that we have
\begin{align*}
Q_{\alpha,z}(\psi\|\varphi+\epsilon\psi)&=\sup_{a\in \Me_+} \left\{\alpha
\Tr\left((a^{\frac12}h_\psi^{\frac{\alpha}{z}}a^{\frac12})^{\frac{z}{\alpha}}\right)-(\alpha-1)
\Tr\left((a^{\frac12}h_{\varphi+\epsilon \psi}^{\frac{\alpha-1}{2z}}a^{\frac12})^{\frac{z}{\alpha-1}}\right) \right\}\\
&\le\sup_{a\in \Me_+} \left\{\alpha
\Tr\left((a^{\frac12}h_\psi^{\frac{\alpha}{z}}a^{\frac12})^{\frac{z}{\alpha}}\right)-(\alpha-1)
\Tr\left((a^{\frac12}h_\varphi^{\frac{\alpha-1}{2z}}a^{\frac12})^{\frac{z}{\alpha-1}}\right)
\right\},
\end{align*}
where the inequality above follows by Lemma \ref{lemma:order}. Therefore, since lower
semicontinuity  \cite[Theorem 2 (iv)]{kato2023onrenyi} gives
\[
Q_{\alpha,z}(\psi\|\varphi)\le \liminf_{\epsilon\searrow 0}
Q_{\alpha,z}(\psi\|\varphi+\epsilon \psi) 
\]
the desired inequality follows.

 
\end{proof}

We finish this section by investigation of the properties of the variatonal expression for
$0<\alpha<1$, in the case when $\lambda^{-1}\ffi\le \psi\le \lambda \ffi$ for some
$\lambda>0$. This will be denoted as $\psi\sim \ffi$. 


\begin{lemma}\label{lemma:variational_majorized}  Assume that $\psi\sim\ffi$. Then
the infimum in the variational expression in Theorem \ref{thm:variational} (i) is attained at a unique element  $\bar a\in \Me^{++}$. This element  satisfies
\begin{align}
h_\psi^{\alpha\over2z}\bar ah_\psi^{\alpha\over2z}
&=\bigl(h_\psi^{\alpha\over2z}h_\ffi^{1-\alpha\over z}h_\psi^{\alpha\over2z}\bigr)^\alpha
\label{eq:minimizer1}\\
h_\ffi^{1-\alpha\over2z}\bar a^{-1}h_\ffi^{1-\alpha\over2z}
&=\bigl(h_\ffi^{1-\alpha\over2z}h_\psi^{\alpha\over
z}h_\ffi^{1-\alpha\over2z}\bigr)^{1-\alpha}.
\label{eq:minimizer2}
\end{align}
\end{lemma}



\begin{proof} We may assume that $\varphi$ and hence also $\psi$ is faithful. Following
 the proof of \cite[Theorem 1 (vi)]{kato2023onrenyi},  we may use the
assumptions and \cite[Lemma A.58]{hiai2021quantum} to show  that there are $b,c\in\cM$  such that
\begin{align}\label{eq:bc}
h_\ffi^{1-\alpha\over2z}
=b\bigl(h_\ffi^{1-\alpha\over2z}h_\psi^{\alpha\over z}h_\ffi^{1-\alpha\over2z}\bigr)^{1-\alpha\over2},\qquad
\bigl(h_\ffi^{1-\alpha\over2z}h_\psi^{\alpha\over z}h_\ffi^{1-\alpha\over2z}\bigr)^{1-\alpha\over2}
=ch_\ffi^{1-\alpha\over2z}.
\end{align}
With $\bar a:=bb^*\in\cM^{++}$ we have  $\bar a^{-1}=c^*c$ and $\bar a$ is indeed a
minimizer of 
\begin{align}\label{eq:infimum}
Q_{\alpha,z}(\psi\|\ffi)
=\inf_{a\in\cM^{++}}\biggl\{\alpha\Big\|h_\psi^{\alpha\over2z}ah_\psi^{\alpha\over2z}\Big\|_{z\over\alpha}^{z\over\alpha}
+(1-\alpha)\Big\|h_\ffi^{1-\alpha\over2z}a^{-1}h_\ffi^{1-\alpha\over2z}\Big\|_{z\over
1-\alpha}^{z\over 1-\alpha}\biggr\}.
\end{align}
We next observe that such  minimizer is unique. Indeed, suppose that the infimum is
attained  at some $a_1,a_2\in \Me^{++}$. Let $a_0:=(a_1+a_2)/2$. Since the map 
$L^{p}(\cM)\ni k\mapsto\|k\|_{p}^{p}$ is convex for any $p\ge 1$ and
$a_0^{-1}\le(a_1^{-1}+a_2^{-1})/2$, we have using Lemma \ref{lemma:order1} in the second
inequality that  
\begin{align*}
\Big\|h_\psi^{\alpha\over2z}a_0h_\psi^{\alpha\over2z}\Big\|_{z\over\alpha}^{z\over\alpha}
&\le{1\over2}\biggl\{\Big\|h_\psi^{\alpha\over2z}a_1h_\psi^{\alpha\over2z}\Big\|_{z\over\alpha}^{z\over\alpha}
+\Big\|h_\psi^{\alpha\over2z}a_2h_\psi^{\alpha\over2z}\Big\|_{z\over\alpha}^{z\over\alpha}\biggr\}, \\
\Big\|h_\ffi^{1-\alpha\over2z}a_0^{-1}h_\ffi^{1-\alpha\over2z}\Big\|_{z\over1-\alpha}^{z\over1-\alpha}
&\le\Big\|h_\ffi^{1-\alpha\over2z}\biggl({a_1^{-1}+a_2^{-1}\over2}\biggr)
h_\ffi^{1-\alpha\over2z}\Big\|_{z\over1-\alpha}^{z\over1-\alpha} \\
&\le{1\over2}\biggl\{\Big\|h_\ffi^{1-\alpha\over2z}a_1^{-1}h_\ffi^{1-\alpha\over2z}\Big\|_{z\over1-\alpha}^{z\over1-\alpha}
+\Big\|h_\ffi^{1-\alpha\over2z}a_2^{-1}h_\ffi^{1-\alpha\over2z}\Big\|_{z\over1-\alpha}^{z\over1-\alpha}\biggr\}.
\end{align*}
Hence we have
\[
\Big\|h_\ffi^{1-\alpha\over2z}a_0^{-1}h_\ffi^{1-\alpha\over2z}\Big\|_{z\over1-\alpha}
=\Big\|h_\ffi^{1-\alpha\over2z}\biggl({a_1^{-1}+a_2^{-1}\over2}\biggr)
h_\ffi^{1-\alpha\over2z}\Big\|_{z\over1-\alpha},
\]
which implies that $a_0^{-1}={a_1^{-1}+a_2^{-1}\over2}$, as easily verified by Lemma
\ref{lemma:order1}. From this we easily have $a_1=a_2$.

The equality  \eqref{eq:minimizer2} is obvious from the second equality in \eqref{eq:bc} and $\bar a^{-1}=c^*c$. Since
$Q_{\alpha,z}(\psi\|\ffi)=Q_{1-\alpha,z}(\ffi\|\psi)$, we see by uniqueness that the minimizer of the
infimum expression for $Q_{1-\alpha,z}(\ffi\|\psi)$ (instead of \eqref{eq:infimum}) is
$\bar a^{-1}$ (instead of $\bar a$). This
says that \eqref{eq:minimizer1} is the equality corresponding to \eqref{eq:minimizer2} when $\psi,\ffi,\alpha$ are replaced with
$\ffi,\psi,1-\alpha$, respectively. 



\end{proof}

In the next lemma, we will use the following notations:
\[
p:={z\over\alpha},\quad r:={z\over 1-\alpha},\quad  \xi_p(a):=h_\psi^{1\over
2p}ah_\psi^{1\over 2p},\quad \eta_r(a)=h_\ffi^{1\over
2r}a^{-1}h_\ffi^{1\over 2r}.
\]
We will also denote the function under the infimum in the variational expression in
Theorem \ref{thm:variational} (i) by $f$, that is,
\[
f(a)=\alpha\|\xi_p(a)\|_p^p
+(1-\alpha)\|\eta_r(a)\|_r^r,\qquad a\in \Me^{++}.
\]



\begin{lemma}\label{lemma:variational_majorized2} Assume that $\psi\sim\ffi$  and let $0<\alpha<1$,
$\max\{\alpha,1-\alpha\}\le z$. If $p>1$, then for every $C\ge Q_{\alpha,z}(\psi\|\ffi)$ and $\eps>0$ there is
some $\delta>0$ such that whenever $\|\xi_p(b)\|^p_p\le C$ and $\|\xi_p(b)-\xi_p(\bar
a)\|_p\ge \eps$, we have
\[
f(b)-Q_{\alpha,z}(\psi\|\ffi)\ge \delta.
\]
A similar statement holds if $r>1$.

\end{lemma}

\begin{proof} By assumptions, $p,r\ge 1$.  For
$a,b\in \Me^{++}$ and $s\in (1/2,0)$, we have
\begin{align*}
\|\xi_p(sb+(1-s)a)\|_p^p&=\|s\xi_p(b)+(1-s)\xi_p(a)\|_p^p=\|(1-2s)\xi_p(a)+2s\frac12(\xi_p(a)+\xi_p(b))\|_p^p\\
&\le (1-2s)\|\xi_p(a)\|_p^p+2s\|\frac12(\xi_p(a)+\xi_p(b))\|_p^p.
\end{align*}
Similarly,
\[
\|\eta_r(sb+(1-s)a)\|_r^r\le
(1-2s)\|\eta_r(a)\|_r^r+2s\|\frac12(\eta_r(a)+\eta_r(b))\|_r^r,
\]
here  we also used the fact that $(t a+(1-t)b)^{-1}\le t
a^{-1}+(1-t)b^{-1}$ for $t\in (0,1)$ and Lemma \ref{lemma:order1}. It follows
that 
\begin{align*}
\<\nabla f(a),b-a\>&=\lim_{s\to 0^+} s^{-1}[ f(sb+(1-s)a)-f(a)]\\
&\le 2\alpha\biggl(\|\frac12
(\xi_p(a)+\xi_p(b))\|_p^p-\|\xi_p(a)\|_p^p\biggr)+2(1-\alpha)\biggl(\|\frac12
(\eta_r(a)+\eta_r(b))\|_r^r-\|\eta_r(a)\|_r^r\biggr)\\
&= \begin{multlined}[t]f(b)-f(a)-2\biggl(\alpha\bigl(\frac12 \|\xi_p(a)\|_p^p+\frac12 \|\xi_p(b)\|_p^p-\|\frac12
(\xi_p(a)+\xi_p(b))\|_p^p\bigr)\\ +(1-\alpha)\bigl(\frac12 \|\eta_r(a)\|_r^r+\frac12 \|\eta_r(b)\|_r^r-\|\frac12
(\eta_r(a)+\eta_r(b))\|_r^r\bigr)\biggr).\end{multlined}
\end{align*}
Since $p,r\ge 1$, both terms in brackets
in  the last expression above are nonnegative. Assume now that $p>1$. 
Let $\bar a\in \Me^{++}$ be the minimizer as in Lemma \ref{lemma:variational_majorized},
then $f(\bar a)=Q_{\alpha,z}(\psi\|\varphi)$ and $\nabla f(\bar a)=0$, so that we get
\begin{align*}
f(b)-Q_{\alpha,z}(\psi\|\varphi)\ge 2\alpha\bigl(\frac12 \|\xi_p(a)\|_p^p+\frac12 \|\xi_p(b)\|_p^p-\|\frac12
(\xi_p(a)+\xi_p(b))\|_p^p\bigr).
\end{align*}
The space $L_p(\Me)$ is uniformly
convex, so that the function $h\mapsto \|h\|_p^p$ is uniformly convex on  each set where it is bounded (\cite[Thm. 3.7.7. and p.
288]{zalinescu2002convex}). Hence for each $C>0$ and $\epsilon>0$ there is some $\delta>0$ such
that for every $h,k$ with $\|h\|_p^p,\|k\|_p^p\le C$ and $\|h-k\|_p\ge \epsilon$, we have
\[
\frac12\|h\|_p^p+\frac12\|k\|_p^p-\|\frac12(h+k)\|_p^p\ge \delta,
\]
(\cite[Exercise 3.3]{zalinescu2002convex}).  The proof
in the case $r>1$ is similar. 







\end{proof}





\section{Data processing inequality and reversibility of channels}

Let  $\gamma: \Ne\to \Me$ be a normal positive unital map. Then the  predual of $\gamma$  defines a 
positive linear map $\gamma_*: L_1(\Me)\to L_1(\Ne)$ that preserves the trace, acting as
\[
L_1(\Me)\ni h_\rho\mapsto h_{\rho\circ\gamma} \in L_1(\Ne).
\]
The support
of $\gamma$ will be denoted by $s(\gamma)$, recall that this is defined as the smallest projection
$e\in \Ne$ such that $\gamma(e)=1$ and in this case, $\gamma(a)=\gamma(eae)$ for any $a\in
\Ne$. For any $\rho\in \Me_*^+$ we clearly have
$s(\rho\circ\gamma)\le s(\gamma)$, with equality if $\rho$ is faithful. 
It follows that $\gamma_*$ maps $L_1(\Me)$ to $s(\gamma)L_1(\Ne)s(\gamma)\equiv
L_1(s(\gamma)\Ne s(\gamma))$.  For any $\rho\in \Me_+^*$, $\rho\ne 0$, the map
\[
\gamma_0: s(\gamma)\Ne s(\gamma)\to s(\rho)\Me s(\rho),\qquad a\mapsto s(\rho) \gamma(a)s(\rho)
\]
is a faithful normal positive unital map. Moreover, for any $\sigma\in \Me_*^+$ such that
$s(\sigma)\le s(\rho)$, we have for any $a\in \Ne$,
\[
\sigma(\gamma_0(s(\gamma)as(\gamma)))=\sigma(s(\rho)\gamma(a)s(\rho))=\sigma(a).
\]
Replacing $\gamma$ by $\gamma_0$ and $\rho$ by the restriction $\rho|_{s(\rho)\Me
s(\rho)}$, we may  assume that both $\rho$ and $\rho\circ
\gamma$ are faithful.


The Petz dual  of $\gamma$ with respect to a faithful  $\rho\in \Me_*^+$
is a map $\gamma_\rho^*:\Me\to \Ne$,
introduced in \cite{petz1988sufficiency}. It was proved that it is again
normal, positive and unital, in addition, it is $n$-positive whenever $\gamma$ is. 
As explained in \cite{jencova2018renyi} $\gamma^*_\rho$ is determined by the equality
\begin{equation}\label{eq:petzdual}
(\gamma^*_\rho)_*(h_{\rho\circ\gamma}^{\frac12}bh_{\rho\circ\gamma}^{\frac12})=h_\rho^{\frac12}\gamma(b)h_\rho^{\frac12},
\end{equation}
for all $b\in \Ne^+$, here $(\gamma^*_\rho)_*$ is the predual map of $\gamma^*_\rho$. We
also have
\[
(\gamma^*_\rho)_*(h_{\rho\circ\gamma})=(\gamma^*_\rho)_*\circ \gamma_*(h_\rho)=h_\rho
\]
and $(\gamma_\rho^*)_{\rho\circ\gamma}^*=\gamma$. In the special case that $\gamma$ is the
inclusion map $\gamma: \Ne\hookrightarrow \Me$ for a subalgebra $\Ne\subseteq \Me$, the Petz dual is the generalized 
conditional expectation $\cE_{\Ne,\varphi}:\Me\to \Ne$, as introduced in \cite{accardi}; see e.g.
\cite[Proposition 6.5]{hiai2021quantum}. Hence  $\cE_{\Ne,\varphi}$ is a normal completely
positive unital with range in $\Ne$ and such that 
\[
\varphi\circ \cE_{\Ne,\varphi}=\varphi.
\]





\subsection{Data processing inequality}


In this paragraph we prove the data processing inequality (DPI) for $D_{\alpha,z}$ with respect to normal
positive unital maps. In the case of the sandwiched divergences $\tilde D_\alpha$ with
$1/2\le \alpha \ne 1$, DPI was proved in \cite{jencova2018renyi,
jencova2021renyi}, see also \cite{berta2018renyi} for an alternative proof in the case
when the maps are  also completely positive.

\begin{lemma}\label{lemma:dpi} Let $\gamma:\Ne\to \Me$ be a normal positive unital map and
let $\rho\in \Me_*^+$, $b\in \Ne^+$. 
\begin{enumerate}
\item[(i)]  If $p\in [1/2,1)$, then 
\[
\|h_{\rho\circ\gamma}^{\frac{1}{2p}}bh_{\rho\circ\gamma}^{\frac{1}{2p}}\|_p\le
\|h_{\rho}^{\frac{1}{2p}}\gamma(b)h_{\rho}^{\frac{1}{2p}}\|_p.
\]

\item[(ii)]  If $p\in [1,\infty]$, the inequality reverses.

\end{enumerate}


\end{lemma}

\begin{proof} Let us denote $\beta:=\gamma_\rho^*$ and let $\omega\in \Me_*^+$ be such
that 
$h_\omega:=h_{\rho\circ\gamma}^{\frac12}bh_{\rho\circ\gamma}^{\frac12}\in L_1(\Ne)^+$. Then
$\beta$ is a normal positive unital map and  we have 
\[
\beta_*(h_\omega)=h_\rho^{\frac12}\gamma(b)h_\rho^{\frac12},\qquad
\beta_*(h_{\rho\circ\gamma})=h_\rho.
\]
Let $p\in [1/2,1)$, then  
\begin{align*}
\|h_{\rho}^{\frac{1}{2p}}\gamma(b)h_{\rho}^{\frac{1}{2p}}\|^p_p&=
\|h_\rho^{\frac{1-p}{2p}}\beta_*(h_\omega)h_\rho^{\frac{1-p}{2p}}\|_p^p=
Q_{p,p}(\beta_*(h_\omega)\|h_\rho)=Q_{p,p}(\beta_*(h_\omega)\|\beta_*(h_{\rho\circ\gamma}))\\
&\ge  Q_{p,p}(h_\omega\|h_{\rho\circ\gamma})=\|h_{\rho\circ\gamma}^{\frac{1-p}{2p}}h_\omega
h_{\rho\circ\gamma}^{\frac{1-p}{2p}}\|_p^p=\|h_{\rho\circ\gamma}^{\frac{1}{2p}}bh_{\rho\circ\gamma}^{\frac{1}{2p}}\|^p_p.
\end{align*}
Here we have used the DPI for the sandwiched R\'enyi  divergence $D_{\alpha,\alpha}$ for
$\alpha\in [1/2,1)$, \cite[Theorem 4.1]{jencova2021renyi}.  This proves (i). 
The case (ii) was proved in \cite{kato2023onrenyi} (see Eq. (22) therein), using the
relation of the sandwiched R\'enyi divergence to the Kosaki $L_p$ norms. In our setting,
the proof can be written as 
\begin{align*}
\|h_\rho^{\frac1{2p}}\gamma(b)h_\rho^{\frac{1}{2p}}\|_p^p&=Q_{p,p}(h_\rho^{\frac12}\gamma(b)h_\rho^{\frac12}\|h_\rho)=Q_{p,p}(\beta_*(h_\omega)\|\beta_*(h_{\rho\circ\gamma}))\\
&\le
Q_{p,p}(h_\omega\|h_{\rho\circ\gamma})=\|h_{\rho\circ\gamma}^{\frac{1}{2p}}bh_{\rho\circ\gamma}^{\frac{1}{2p}}\|^p_p,
\end{align*}
here the inequality follows from the DPI for sandwiched R\'enyi divergence
$D_{\alpha,\alpha}$ with
$\alpha>1$, \cite[]{jencova2018renyi}.


\end{proof}

\begin{remark}\label{rem:pcontraction} Using the fact that $(\gamma^*_\rho)_*$ is a contraction for
the Kosaki $L_p$ spaces $L_p(\Ne,\rho\circ\gamma)\to L_p(\Me,\rho)$ for all $p\ge 1$ (see \cite[Thm.
]{jencova2018renyi} and properties of the Petz dual), we can see that the map
$\gamma_{\rho,p}: L_p(\Ne)\to L_p(\Me)$ determined by
\[
\gamma_{\rho,p}:h_{\rho\circ\gamma}^{1\over 2p}ah_{\rho\circ\gamma}^{1\over 2p}\mapsto h_{\rho}^{1\over
2p}\gamma(a)h_{\rho}^{1\over 2p},\qquad a\in \Ne
\]
is a contraction. Notice that $\gamma_{\rho,1}=(\gamma^*_\rho)_*$, 
$\gamma_{\rho,\infty}=\gamma$ and for any $p\ge 1$,
\[
(\gamma^*_\rho)_*(h_{\rho\circ\gamma}^{p-1\over 2p}w h_{\rho\circ\gamma}^{p-1\over
2p})=h_{\rho}^{p-1\over 2p}\gamma^*_{\rho,p}(w) h_{\rho}^{p-1\over 2p},\qquad w\in
L_p(\Ne).
\]


\end{remark}



\begin{theorem}[DPI] \label{thm:dpi} Let $\psi,\varphi\in \Me_*^+$, $\psi\ne 0$ and let $\gamma:
\Ne\to \Me$ be a normal positive unital map. Assume either of the following conditions:
\begin{enumerate}
\item[(i)] $0<\alpha<1$, $\max\{\alpha,1-\alpha\}\le z$
\item[(ii)] $\alpha>1$, $\max\{\alpha/2,\alpha-1\}\le z\le \alpha$.
\end{enumerate}
Then we have
\[
D_{\alpha,z}(\psi\circ\gamma\|\varphi\circ\gamma)\le D_{\alpha,z}(\psi\|\varphi).
\]


\end{theorem}


\begin{proof} Under the conditions (i), the DPI was proved in \cite[Theorem 1
(viii)]{kato2023onrenyi}.
 Since parts of this proof will be used below, we repeat it here.

 Assume the conditions in (i) and put $p:=\frac{z}{\alpha}$, $r:=\frac{z}{1-\alpha}$, so
 that  $p,r\ge 1$. 
For any   $b\in \Ne^{++}$, we have by  the Choi inequality \cite{choi1974aschwarz} 
that  $\gamma(b)^{-1}\le \gamma(b^{-1})$, so that  by Lemma \ref{lemma:order1} and
\ref{lemma:dpi} (ii), we have 
\begin{equation}\label{eq:ineq}
\|h_\varphi^{\frac{1}{2r}}\gamma(b)^{-1}\varphi^{\frac{1}{2r}}\|_r\le
\|h_\varphi^{\frac{1}{2r}}\gamma(b^{-1})\varphi^{\frac{1}{2r}}\|_r\le\|h_{\varphi\circ\gamma}^{\frac{1}{2r}}b^{-1}h_{\varphi\circ\gamma}^{\frac{1}{2r}}\|_r^r .
\end{equation}
Using the variational expression in Theorem \ref{thm:variational} (i), we have
\begin{align}\label{eq:dpif}
Q_{\alpha,z}(\psi\|\varphi)&\le \alpha\|h_\psi^{\frac{1}{2p}}\gamma(b)h_\psi^{\frac{1}{2p}}\|_p^p+
(1-\alpha)\|h_\varphi^{\frac{1}{2r}}\gamma(b)^{-1}h_\varphi^{\frac{1}{2r}}\|_r^r\\
&\le  \alpha\|h_\psi^{\frac{1}{2p}}\gamma(b)h_\psi^{\frac{1}{2p}}\|_p^p+
(1-\alpha)\|h_\varphi^{\frac{1}{2r}}\gamma(b^{-1})h_\varphi^{\frac{1}{2r}}\|_r^r\\
&\le  \alpha\|h_{\psi\circ\gamma}^{\frac{1}{2p}}bh_{\psi\circ\gamma}^{\frac{1}{2p}}\|_p^p+
(1-\alpha)\|h_{\varphi\circ\gamma}^{\frac{1}{2r}}b^{-1}h_{\varphi\circ\gamma}^{\frac{1}{2r}}\|_r^r.\label{eq:dpil}
\end{align}
 Since this holds for all
$b\in \Ne^{++}$, it follows that $Q_{\alpha,z}(\psi\|\varphi)\le
Q_{\alpha}(\psi\circ\gamma\|\varphi\circ\gamma)$, which proves the DPI in this case.


Assume next the condition (ii), and put $p:=\frac{z}{\alpha}$, $q:=\frac{z}{\alpha-1}$, so
that $p\in [1/2,1)$ and $q\ge 1$. Using Theorem
\ref{thm:variational} (ii), we get for any $b\in \Ne^+$,
\begin{align*}
Q_{\alpha,z}(\psi\|\varphi)&\ge
\alpha\|h_\psi^{\frac{1}{2p}}\gamma(b)h_\psi^{\frac{1}{2p}}\|_p^p-
(\alpha-1)\|h_\varphi^{\frac{1}{2q}}\gamma(b)h_\varphi^{\frac{1}{2q}}\|_q^q\\
&\ge \alpha\|h_{\psi\circ\gamma}^{\frac{1}{2p}}bh_{\psi\circ\gamma}^{\frac{1}{2p}}\|_p^p-
(\alpha-1)\|h_{\varphi\circ\gamma}^{\frac{1}{2r}}bh_{\varphi\circ\gamma}^{\frac{1}{2q}}\|_q^q,
\end{align*}
here we used both (i) and (ii) in Lemma \ref{lemma:dpi}. Again, since this holds for all
$b\in \Ne^+$, we get the desired inequality.



\end{proof}



\subsection{Martingale convergence}

An important consequence of DPI is the martingale convergence property that will be
proved in this paragraph.





Let $\cM$ be a $\sigma$-finite von Neumann algebra. Let $\{\cM_i\}$ be an increasing net of von Neumann
subalgebras of $\cM$ containing the unit of $\cM$ such that $\cM=\bigl(\bigcup_i\cM_i\bigr)''$.

\begin{theorem}\label{thm:martingale}
Assume that $\alpha$ and $z$ satisfy the DPI bounds (that is, conditions (i) or (ii) in Theorem
\ref{thm:dpi}). Then we have
\begin{align}\label{eq:martingale}
D_{\alpha,z}(\psi\|\ffi)=\lim_iD_{\alpha,z}(\psi|_{\cM_i}\|\ffi|_{\cM_i})
\quad\mbox{increasingly}.
\end{align}
\end{theorem}

\begin{proof}
Let $\ffi_i:=\ffi|_{\cM_i}$ and $\psi_i:=\psi|_{\cM_i}$. From Theorem
\ref{thm:dpi}, it
follows that $D_{\alpha,z}(\psi\|\ffi)\ge D_{\alpha,z}(\psi_i\|\ffi_i)$ for all $i$ and
$i\mapsto D_{\alpha,z}(\psi_i\|\ffi_i)$ is increasing. Hence, to show \eqref{eq:martingale}, it suffices to prove that
\begin{align}\label{eq:martingale1}
D_{\alpha,z}(\psi\|\ffi)\le\sup_iD_{\alpha,z}(\psi_i\|\ffi_i).
\end{align}
To do this, we may assume that $\ffi$ is faithful. Indeed, assume that
\eqref{eq:martingale1} has been sbown when $\ffi$ is
faithful. For general $\ffi\in\cM_*^+$, from the assumption of $\cM$ being $\sigma$-finite, there exists a
$\ffi_0\in\cM_*^+$ with $s(\ffi_0)=\1-s(\ffi)$. Let $\ffi^{(n)}:=\ffi+n^{-1}\ffi_0$ and $\ffi_i^{(n)}:=\ffi^{(n)}|_{\cM_i}$
for each $n\in\bN$. Thanks to the lower semi-continuity \cite[Theorem 1(iv) and Theorem
2(iv)]{kato2023onrenyi} and the order
relation \cite[Theorem 1(iii) and Theorem 2(iii)]{kato2023onrenyi} we have
\begin{align*}
D_{\alpha,z}(\psi\|\ffi)&\le\liminf_{n\to\infty}D_{\alpha,z}(\psi\|\ffi^{(n)}) \\
&\le\liminf_{n\to\infty}\sup_iD_{\alpha,z}(\psi_i\|\ffi_i^{(n)}) \\
&\le\sup_iD_{\alpha,z}(\psi_i\|\ffi_i),
\end{align*}
proving \eqref{eq:martingale1} for general $\ffi$. Below we assume the faithfulness of $\ffi$ and write $\cE_{\cM_i,\ffi}$
for the generalized conditional expectation from $\cM$ to $\cM_i$ with respect to $\ffi$. 
Then we note that we have by \cite[Theorem 3]{HT},  
\begin{align}\label{eq:martingaleHT}
\psi_i\circ\cE_{\cM_i,\ffi}=\psi\circ\cE_{\cM_i,\ffi}\to\psi\quad\mbox{ in the norm},
\end{align}
as well as
\begin{align}\label{eq:condexp}
\ffi_i\circ\cE_{\cM_i,\ffi}=\ffi\circ\cE_{\cM_i,\ffi}=\ffi.
\end{align}
Using lower semicontinuity and DPI, we obtain
\[
D_{\alpha,z}(\psi\|\ffi)\le \liminf_{i}
D_{\alpha,z}(\psi_i\circ\cE_{\cM_i,\varphi}\|\ffi)\le \liminf_i
D_{\alpha,z}(\psi_i\|\varphi)\le \sup_i D_{\alpha,z}(\psi_i\|\ffi).
\]
\end{proof}


\subsection{Equality in DPI and reversibility of channels}

In what follows, a channel is a normal 2-positive unital map $\gamma: \Ne\to \Me$.

\begin{defi} Let $\gamma:\Ne\to \Me$ be a channel and let $\mathcal S \subset
\Me_*^+$. We say that $\gamma$ is reversible (or sufficient) with respect to $\mathcal S$
if there exists a channel $\beta:\Me\to \Ne$ such that
\[
\rho\circ\gamma\circ\beta=\rho,\qquad \forall \rho\in \mathcal S.
\]

\end{defi}

The notion of sufficient channels was introduced by Petz
\cite{petz1986sufficient,petz1988sufficiency}, who also obtained a number of conditions
characterizing this situation. It particular, it was proved in \cite{petz1988sufficiency}
that sufficient channels can be characterized by equality in DPI for the relative entropy
$D(\psi\|\varphi)$: if $D(\psi\|\varphi)<\infty$, then a channel $\gamma$ is sufficient
with respect to $\{\psi,\varphi\}$ if and only if 
\[
D(\psi\circ\gamma\|\varphi\circ\gamma)=D(\psi\|\varphi). 
\]
This result has been proved for a number of other divergence measures, including the
standard R\'enyi divergences $D_{\alpha,1}$ with $0<\alpha<2$ (\cite{}) and the sandwiched
R\'enyi divergences $D_{\alpha,\alpha}$ for $\alpha>1/2$
(\cite{jencova2018renyi,jencova2021renyi}).
Our aim in this section is to prove that a similar statement holds for $D_{\alpha,z}$ for
values of the parameters strictly contained in the DPI bounds of Theorem \ref{thm:dpi}. 



Another important result of \cite{petz1988sufficiency} shows that the Petz dual $\gamma_\varphi^*$ is a universal
recovery map, in the sense given in the proposition below. 

\begin{prop}\label{prop:universal} Let $\gamma:\Ne\to \Me$ be a channel and let
$\varphi\in \Me_*^+$ be such that both $\ffi$ and $\ffi\circ\gamma$ are faithful. Then for any $\psi\in \Me_*^+$, $\gamma$ is
sufficient with respect to $\{\psi,\varphi\}$ if and only if $\psi\circ\gamma\circ
\gamma_\varphi^*=\psi$.

Consequently, there is a faithful normal conditional expectation $\mathcal E$ on $\Me$
such that $\varphi\circ \mathcal E=\varphi$ and $\gamma$ is sufficient with respect to
$\{\psi,\varphi\}$ if and only if also $\psi\circ\mathcal E=\psi$.

\end{prop}

Note that the range of the conditional expectation $\mathcal E$ in the above proposition
is the set  of fixed points of the channel $\gamma\circ\gamma_\varphi^*$. 

\subsubsection{The case $\alpha\in (0,1)$}



We first prove some equivalent conditions for equality in DPI, in the case $\psi\sim
\ffi$. We will use the notations $\psi_0:=\psi\circ\gamma$, $\ffi_0:=\ffi\circ\gamma$.
Note that these conditions do not require $\gamma$ to be 2-positive. 

\begin{prop}\label{prop:DPI_equality}
Let $0<\alpha<1$, $\max\{\alpha,1-\alpha\}\le z$ and assume that $\psi\sim \ffi$. Let
$\gamma:\Ne\to \Me$ be a normal positive unital map and put $\psi_0=\psi\circ\gamma$,
$\ffi_0=\ffi\circ\gamma$. Let 
$\bar a\in \Me^{++}$  
be the unique minimizer as in Lemma \ref{lemma:variational_majorized} for
$Q_{\alpha,z}(\psi\|\ffi)$ and let $\bar a_0\in \Ne^{++}$ be the minimizer for
$Q_{\alpha,z}(\psi_0\|\ffi_0)$. The following conditions are equivalent:
\begin{itemize}
\item[(i)] $D_{\alpha,z}(\psi_0\|\ffi_0)=D_{\alpha,z}(\psi\|\ffi)$, i.e.,
$Q_{\alpha,z}(\psi_0\|\ffi_0)=Q_{\alpha,z}(\psi\|\ffi)$.
\item[(ii)] $\gamma(\bar a_0)=\bar a$ and
$\big\|h_\psi^{\alpha\over2z}\gamma(\bar a_0)h_\psi^{\alpha\over2z}\big\|_{z/\alpha}
=\big\|h_{\psi_0}^{\alpha\over2z}\bar a_0h_{\psi_0}^{\alpha\over2z}\big\|_{z/\alpha}$.
\item[(iii)] $\big\|h_\psi^{\alpha\over2z}\bar ah_\psi^{\alpha\over2z}\big\|_{z/\alpha}
=\big\|h_{\psi_0}^{\alpha\over2z}\bar a_0h_{\psi_0}^{\alpha\over2z}\big\|_{z/\alpha}$.
\item[(iv)] $\gamma(\bar a_0^{-1})=\bar a^{-1}$ and
$\big\|h_\ffi^{1-\alpha\over2z}\gamma(\bar a_0^{-1})h_\ffi^{1-\alpha\over2z}\big\|_{z/(1-\alpha)}
=\big\|h_{\ffi_0}^{1-\alpha\over2z}\bar a_0^{-1}h_{\ffi_0}^{1-\alpha\over2z}\big\|_{z/(1-\alpha)}$.
\item[(v)] $\big\|h_\ffi^{1-\alpha\over2z}\bar a^{-1}h_\ffi^{1-\alpha\over2z}\big\|_{z/(1-\alpha)}
=\big\|h_{\ffi_0}^{1-\alpha\over2z}\bar a_0^{-1}h_{\ffi_0}^{1-\alpha\over2z}\big\|_{z/(1-\alpha)}$.
\end{itemize}
\end{prop}


\begin{proof}
By the assumptions, $s(\psi)=s(\ffi)$ and since also $\psi_0\sim \ffi_0$, we have
$s(\psi_0)=s(\ffi_0)$. Using restrictions, we may assume that all
$\psi,\ffi,\psi_0,\ffi_0$ are faithful.


(i)$\implies$(ii) \& (iv).\enspace
By Lemma \ref{lemma:dpi} (ii) 
\begin{align}
\Big\|h_\psi^{\alpha\over2z}\gamma(\bar a_0)h_\psi^{\alpha\over2z}\Big\|_{z/\alpha}
\le\Big\|h_{\psi_0}^{\alpha\over2z}\bar a_0h_{\psi_0}^{\alpha\over2z}\Big\|_{z/\alpha},
\label{eq:ineq2}
\end{align}
and by \eqref{eq:ineq} one has
\begin{align}
\Big\|h_\ffi^{1-\alpha\over2z}\gamma(\bar a_0)^{-1}h_\ffi^{1-\alpha\over2z}\Big\|_{z/(1-\alpha)}
\le\Big\|h_\ffi^{1-\alpha\over2z}\gamma(\bar a_0^{-1})h_\ffi^{1-\alpha\over2z}\Big\|_{z/(1-\alpha)}
\le\Big\|h_{\ffi_0}^{1-\alpha\over2z}\bar
a_0^{-1}h_{\ffi_0}^{1-\alpha\over2z}\Big\|_{z/(1-\alpha)}. \label{eq:ineq3}
\end{align}
From \eqref{eq:ineq2} and \eqref{eq:ineq3} it follows that
\begin{align*}
Q_{\alpha,z}(\psi\|\ffi)&\le \alpha\big\|h_\psi^{\alpha\over2z}\gamma(\bar a_0)h_\psi^{\alpha\over2z}\big\|_{z/\alpha}
+(1-\alpha)\big\|h_\ffi^{1-\alpha\over2z}\gamma(\bar a_0)^{-1}h_\ffi^{1-\alpha\over2z}\big\|_{z/(1-\alpha)} \\
&\le\alpha\big\|h_{\psi_0}^{\alpha\over2z}\bar a_0h_{\psi_0}^{\alpha\over2z}\big\|_{z/\alpha}
+(1-\alpha)\big\|h_{\ffi_0}^{1-\alpha\over2z}\bar a_0^{-1}h_{\ffi_0}^{1-\alpha\over2z}\big\|_{z/(1-\alpha)} =Q_{\alpha,z}(\psi_0\|\ffi_0)=Q_{\alpha,z}(\psi\|\ffi).
\end{align*}


By uniqueness in Lemma \ref{lemma:variational_majorized} we find that $\gamma(\bar
a_0)=\bar a$ and all the inequalities in \eqref{eq:ineq2} and \eqref{eq:ineq3} must
become equalities. Since $\gamma(\bar a_0^{-1})\ge\gamma(\bar a_0)^{-1}$, we  verify by
Lemma \ref{lemma:order1} that the equality in
\eqref{eq:ineq3} yields $\gamma(\bar a_0^{-1})=\gamma(\bar a_0)^{-1}=\bar a^{-1}$. Therefore,
(ii) and (iv) hold.

The implications (ii)$\implies$(iii) and (iv)$\implies$(v) are obvious.

(iii)$\implies$(i).\enspace
By (iii) with the equality \eqref{eq:minimizer1} in Lemma \ref{lemma:variational_majorized} we have
\begin{align*}
Q_{\alpha,z}(\psi\|\ffi)
&=\Tr\bigl(h_\psi^{\alpha\over2z}h_\ffi^{1-\alpha\over z}h_\psi^{\alpha\over2z}\bigr)^z
=\Tr\bigl(h_\psi^{\alpha\over2z}\bar ah_\psi^{\alpha\over2z}\bigr)^{z/\alpha} \\
&=\Tr\bigl(h_{\psi_0}^{\alpha\over2z}\bar a_0h_{\psi_0}^{\alpha\over2z}\bigr)^{z/\alpha}
=\Tr\bigl(h_{\psi_0}^{\alpha\over2z}h_{\ffi_0}^{1-\alpha\over z}h_{\psi_0}^{\alpha\over2z}\bigr)^z \\
&=Q_{\alpha,z}(\psi_0\|\ffi_0).
\end{align*}

(v)$\implies$(i).\enspace
Similarly, by (v) with the equality \eqref{eq:minimizer2} in Lemma
\ref{lemma:variational_majorized} we have
\begin{align*}
Q_{\alpha,z}(\psi\|\ffi)
&=\Tr\bigl(h_\ffi^{1-\alpha\over2z}h_\psi^{\alpha\over z}h_\ffi^{1-\alpha\over2z}\bigr)^z
=\Tr\bigl(h_\ffi^{1-\alpha\over2z}\bar a^{-1}h_\ffi^{1-\alpha\over2z}\bigr)^{z/(1-\alpha)} \\
&=\Tr\bigl(h_{\ffi_0}^{1-\alpha\over2z}\bar a_0^{-1}h_{\ffi_0}^{1-\alpha\over2z}\bigr)^{z/(1-\alpha)}
=\Tr\bigl(h_{\ffi_0}^{1-\alpha\over2z}h_{\psi_0}^{\alpha\over z}h_{\ffi_0}^{1-\alpha\over2z}\bigr)^z \\
&=Q_{\alpha,z}(\psi_0\|\ffi_0).
\end{align*}
\end{proof}


\begin{remark}\label{rem:conditions} Note that the above conditions extend the results
obtained in \cite{leditzky2017data} and \cite{zhang2020equality} in
the finite dimensional case.
Indeed, the condition (ii) with $\alpha=z$ is equivalent to the condition in \cite[Thm.
1]{laditzky2017data}, note here that in this case the second condition in (ii) is
automatic. Moreover, (ii) extends the necessary condition in \cite[Thm. 1.2
(2)]{zhang2020equality} to a necessary and sufficient one. While in both these
 works $\gamma$ was required to be completely positive, we have shown that only
positivity is enough. See also the related condition in Corollary \ref{coro:zhang} below.

\end{remark}





\begin{theorem}\label{thm:suffle1} Let $0<\alpha<1$ and $\alpha,1-\alpha\le
z$. Let $\psi,\varphi\in \Me_*^+$ and assume that $\alpha<z$ and $s(\ffi)\le
s(\psi)$ or $1-\alpha<z$ and $s(\psi)\le s(\ffi)$. 
Then $\gamma$ is reversible with respect to
$\{\psi,\varphi\}$ if and only if
\[
D_{\alpha,z}(\psi\|\varphi)=D_{\alpha,z}(\psi\circ\gamma\|\varphi\circ\gamma).
\]

\end{theorem}

\begin{proof} This proof is a modification of the proof of \cite[Thm.
5.1]{jencova2021renyi}. Let us denote $\psi_0:=\psi\circ\gamma$,
$\varphi_0:=\varphi\circ\gamma$ and put $p=\frac{z}{\alpha}$, $r=\frac{z}{1-\alpha}$.  
We will assume that $p>1$ and $s(\ffi)\le s(\psi)$,
otherwise we may exchange the role of $p,r$ and $\psi$, $\ffi$ by the equality
$Q_{\alpha,z}(\psi\|\varphi)=Q_{1-\alpha,z}(\varphi\|\psi)$. As before, we may assume that
both $\psi$ and $\psi_0$ are faithful.

The strategy of the proof is to use known results for the sandwiched R\'enyi divergence
$D_{p,p}$ with $p>1$, \cite[]{jencova2018renyi}. For this, notice that
\[
Q_{z,\alpha}(\psi\|\ffi)=Q_{p,p}(\omega\|\psi),
\]
where $\omega\in \Me_*^+$ is such that
\begin{equation}\label{eq:omega}
h_\omega=h_\psi^{p-1\over 2p}h_\mu^{\frac1p}h_\psi^{p-1\over 2p},\qquad
h_{\mu}=|h_{\ffi}^{1\over 2r}h_{\psi}^{1\over 2p}|^{2z}.
\end{equation}
Let $\omega_0,\mu_0\in \Ne_*^+$ be similar functionals obtained from $\psi_0,\ffi_0$. Then
we have the equality
\[
Q_{p,p}(\omega_0\|\psi_0)=Q_{\alpha,z}(\psi_0\|\ffi_0)=Q_{\alpha,z}(\psi\|\ffi)=Q_{p,p}(\omega\|\psi).
\]
Our first goal is to show that $\omega_0=\omega\circ\gamma$, which implies by
\cite[]{jencova2018renyi} that $\gamma$ is sufficient with respect to $\{\omega,\psi\}$. 

Let $\psi_n\to\psi$ and
$\varphi_n\to \varphi$ in $\Me_*^+$ be some sequences such that $\psi_n\sim \ffi_n$ for
all $n$. Then $\psi_n\circ\gamma\sim \ffi_n\circ\gamma$, $\psi_n\circ\gamma\to \psi_0$, $\ffi_n \circ \gamma\to \ffi_0$ and by joint
continuity of $Q_{\alpha,z}$ (\cite[Thm. 1 (iv)]{kato2023onrenyi}), we have
\[
\lim_n
Q_{\alpha,z}(\psi_n\circ\gamma\|\ffi_n\circ\gamma)=Q_{\alpha,z}(\psi_0\|\ffi_0)=Q_{\alpha,z}(\psi\|\ffi)=\lim_n
Q_{\alpha,z}(\psi_n\|\ffi_n).
\]
Let $\bar b_{n}\in \Ne^{++}$ be the minimizer for the variational expression for
$Q_{\alpha,z}(\psi_n\circ\gamma\|\ffi_n\circ\gamma)$. Let also $\bar a_n$ be the minimizer
for $Q_{\alpha,z}(\psi_n\|\ffi_n)$ and let $f_n:\Me^{++}\to \mathbb R^+$
be the function minimized in the expression for $Q_{\alpha,z}(\psi_n\|\ffi_n)$. 
We then have 
\begin{align*}
Q_{\alpha,z}(\psi_n\circ\gamma\|\ffi_n\circ\gamma)-f_n(\gamma(\bar
b_{n}))=&\alpha\left(\|h_{\psi_n\circ\gamma}^{\frac1{2p}}\bar b_{n}
h_{\psi_n\circ\gamma}^{\frac1{2p}}\|_p^p-\|h_{\psi_n}^{1\over 2p}\gamma(\bar
b_n)h_{\psi_n}^{1\over 2p}\|_p^p\right)\\
& + (1-\alpha)\left(\|h_{\ffi_n\circ\gamma}^{\frac1{2r}}\bar b_{n}^{-1}
h_{\ffi_n\circ\gamma}^{\frac1{2r}}\|_r^r-\|h_{\ffi_n}^{1\over 2r}\gamma(\bar
b_n)^{-1}h_{\ffi_n}^{1\over 2r}\|_r^r\right)\ge 0,
\end{align*}
where the inequality follows from Lemma \ref{lemma:dpi} (ii) and \eqref{eq:ineq}. We
obtain
\begin{equation}\label{eq:qfn}
Q_{\alpha,z}(\psi_n\circ\gamma\|\ffi_n\circ\gamma)-Q_{\alpha,z}(\psi_n\|\ffi_n)\ge f_n(\gamma(\bar
b_{n}))-Q_{\alpha,z}(\psi_n\|\ffi_n)\ge 0.
\end{equation}

Now let $\mu_{n,0}\in \Ne_*^+$ and $\mu_n\in \Me_*^+$ be such that by  \eqref{eq:minimizer1}
in Lemma \ref{lemma:variational_majorized}
\begin{align*}
h_{\mu_{n,0}}^{\frac 1p}= |h_{\ffi_n\circ\gamma}^{1\over 2r}h_{\psi_n\circ\gamma}^{1\over 2p}|^{2\alpha}=
h_{\psi_n\circ\gamma}^{\frac 1{2p}}\bar b_{n}
h_{\psi_n\circ\gamma}^{\frac1{2p}},\qquad
h_{\mu_{n}}^{\frac1p}=|h_{\ffi_n}^{1\over 2r}h_{\psi_n}^{1\over 2p}|^{2\alpha}= h_{\psi_n}^{\frac 1{2p}}\bar a_{n}
h_{\psi_n}^{\frac1{2p}}. 
\end{align*}
Then $h_{\mu_{n,0}}^{\frac1p}\to h_{\mu_0}^{\frac1p}$ in $L_p(\Ne)$, this follows by the H\"older
inequality and the fact that the map $L_{2z}(\Ne)\to L_p(\Ne)$, given as $h\mapsto
|h|^{2\alpha}$ is norm to norm continuous, \cite{kosaki1986aplicationsuc}. Similarly,
$h_{\mu_n}^{\frac1p}\to h_\mu^{\frac1p}$ in $L_p(\Me)$. 
Next, note that since
$Q_{\alpha,z}(\psi_n\circ\gamma\|\ffi_n\circ\gamma)$ and $Q_{\alpha,z}(\psi_n\|\ffi_n)$
have the same limit, we see from \eqref{eq:qfn} and Lemma
\ref{lemma:variational_majorized2} that 
$h_{\psi_n}^{\frac 1{2p}}\gamma(\bar b_{n})h_{\psi_n}^{1\over 2p}-h_{\mu_n}^{\frac1p}\to
0$ in $L_p(\Me)$. On the other hand, let $\gamma^*_{\psi_n,p}, \gamma^*_{\psi,p}:
L_p(\Ne)\to L_p(\Me)$ be the contractions as in Remark \ref{rem:pcontraction}. We
then have 
\[
h_{\psi_n}^{\frac 1{2p}}\gamma(\bar b_{n})h_{\psi_n}^{1\over
2p}=\gamma_{\psi_{n},p}^*(h_{\mu_{n,0}}^{\frac1p})
\]
and since $\gamma^*_{\psi_{n},p}(k)\to \gamma^*_{\psi,p}(k)$ in $L_p(\Me)$ for any
$k\in L_p(\Ne)$ by \cite[Lemma 4.3]{jencova2018renyi}, we have 
\[
\|\gamma^*_{\psi,p}(h_{\mu_0}^{\frac1p})-\gamma_{\psi_{n},p}^*(h_{\mu_{n,0}}^{\frac1p})\|_p\le
\|(\gamma^*_{\psi,p}-\gamma^*_{\psi_{n},p})(h_{\mu_0}^{\frac1p})\|_p+
\|\gamma^*_{\psi_{n},p}(h_{\mu_0}^{\frac1p}-h_{\mu_{n,0}}^{\frac1p})\|_p\to 0.
\]
Putting all together, we obtain that 
\[
h_\mu^{\frac1p}=\lim_n h_{\mu_n}^{\frac1p}=\lim_n
\gamma^*_{\psi_{n},p}(h_{\mu_{n,0}}^{\frac1p})=\gamma^*_{\psi,p}(h_{\mu_0}^{\frac1p}).
\]
It follows that 
\[
(\gamma^*_{\psi})_*(h_{\omega_0})=h_{\psi}^{\frac{p-1}{2p}}\gamma^*_{\psi,p}(h_{\mu_0}^{\frac1p})h_{\psi}^{\frac{p-1}{2p}}=h_{\psi}^{\frac{p-1}{2p}}h_\mu^{\frac1p}h_{\psi}^{\frac{p-1}{2p}}=h_\omega,
\]
so that we have
\[
Q_{p,p}(\omega_0\|\psi_0)=Q_{p,p}(\omega\|\psi)=Q_{p,p}(\omega_0\circ
\gamma^*_\psi\|\psi_0\circ\gamma^*_\psi).
\]

Let us remark here that in the situation of Proposition \ref{prop:DPI_equality}, we have
$h_\omega=h_\psi^{\frac12}\bar ah_\psi^{\frac12}$ and
$h_{\omega_0}=h_{\psi_0}^{\frac12}\bar a_0h_{\psi_0}^{\frac12}$, so that the equality
\[
(\gamma^*_{\psi})_*(h_{\omega_0})=h_\psi^{\frac12}\gamma(\bar
a_0)h_\psi^{\frac12}=h_\omega
\]
is immediate from the condition (iii) in Proposition \ref{prop:DPI_equality}.


By the properties of the sandwiched R\'enyi divergence $D_{p,p}$, it follows that
$\gamma^*_\psi$ is sufficient with respect to $\{\omega_0,\psi_0\}$. By Proposition \ref{prop:universal} and the fact that the
Petz dual  $(\gamma_\psi^*)_{\psi_0}^*$ is $\gamma$ itself, this implies 
\[
\omega\circ\gamma= \omega_0\circ \gamma_\psi^*\circ\gamma=\omega_0.
\]


Next, let $\mathcal E$ be the faithful  normal conditional expectation onto the
set of fixed points of $\gamma\circ\gamma^*_\psi$, as in Proposition \ref{prop:universal}. Then
$\mathcal E$ preserves both $\psi$ and $\omega$, which by
\cite{junge2003noncommutative} {\color{red}...!!}  implies that 
\[
h_\psi^{p-1\over 2p}h_\mu^{\frac1p}h_\psi^{p-1\over 2p}=h_\omega=\mathcal
E_*(h_\omega)=h_\psi^{p-1\over 2p}\cE_p(h_\mu^{\frac1p})h_\psi^{p-1\over 2p},
\]
so that $|h_\ffi^{1\over 2r}h_\psi^{1\over 2p}|^{2\alpha}= h_\mu^{\frac1p}\in
L_p(\cE(\Me))$ and consequently $|h_\varphi^{\frac1{2r}}h_\psi^{\frac1{2p}}|=h_\mu^{1\over
2z}\in L_{2z}(\mathcal E(\Me))$.
Note that by the assumptions $2z>1$, so that we may use the multiplicativity properties
of the extension of $\mathcal E$ \cite{junge2003noncommutative}. Let 
\[
h_\varphi^{\frac1{2r}}h_\psi^{\frac1{2p}}=u|h_\varphi^{\frac1{2r}}h_\psi^{\frac1{2p}}|
\]
be the polar decomposition in $L_{2z}(\Me)$, then we have 
\[
u^*h_\varphi^{\frac1{2r}}h_\psi^{\frac1{2p}}=\mathcal
E_{2z}(u^*h_\varphi^{\frac1{2r}}h_\psi^{\frac1{2p}})=\mathcal
E_{2r}(u^*h_\varphi^{\frac1{2r}})h_\psi^{\frac1{2p}},
\]
which implies that $u^*h_\varphi^{\frac1{2r}}\in L_{2r}(\cE(\Me))$. Since $\psi$ is
faithful, we have $uu^*=r(h_\varphi^{\frac1{2r}}h_\psi^{\frac1{2p}})=s(\ffi)$, so that by uniqueness of the polar decomposition in
$L_{2r}(\Me)$ and $L_{2r}(\cE(\Me))$, we
must have $h_\ffi^{1\over 2r}\in L_{2r}(\cE(\Me))$ and $u\in \cE(\Me)$.
Hence $\ffi\circ\mathcal E=\ffi$ and $\gamma$ is sufficient with respect to
$\{\psi,\varphi\}$. 

\end{proof}

We give further equality conditions related to those by Zhang (?) \cite{zhang2020equality} in
finite dimensions.


\begin{coro}\label{coro:zhang} Let $0<\alpha<1$, $\max\{\alpha,1-\alpha\}\le z$ and let
$\gamma$ be a normal positive unital map. 
\begin{enumerate}
\item[(i)]  Let $s(\ffi)\le s(\psi)$ and $p>1$. Then equality in DPI for $D_{\alpha,z}$ is equivalent to
\[
\gamma_*\bigl((h_\psi^{1\over 2p}h_\ffi^{1\over 2r}h_\psi^{1\over
2p})^z\bigr)=(h_{\psi_0}^{1\over 2p}h_{\ffi_0}^{1\over 2r}h_{\psi_0}^{1\over 2p})^z.
\]
\item[(ii)] Let $s(\psi)\le s(\ffi)$ and $r>1$. Then equality in DPI for $D_{\alpha,z}$ is equivalent to
\[
\gamma_*\bigl((h_\ffi^{1\over 2r}h_\psi^{1\over 2p}h_\ffi^{1\over
2r})^z\bigr)=(h_{\ffi_0}^{1\over 2r}h_{\psi_0}^{1\over 2p}h_{\ffi_0}^{1\over 2r})^z.
\]
\end{enumerate}

\end{coro}

{\color{red} 
Note that in the case $\psi\sim\ffi$, the condition in (i) can be written as 
\[
\gamma_*\bigl((h_\psi^{1\over 2p}\bar a h_\psi^{1\over 2p})^p\bigr)=(h_{\psi_0}^{1\over
2p}\bar a_0 h_{\psi_0}^{1\over 2p})^p
\]
and the condition in (ii) as 
\[
\gamma_*\bigl((h_\ffi^{1\over 2r}\bar a^{-1}h_\ffi^{1\over
2r})^r\bigr)=(h_{\ffi_0}^{1\over 2r}\bar a_0^{-1}h_{\ffi_0}^{1\over 2r})^r.
\]
This is similar to the condition by Zhang, but not the same, Zhang's condition is
\[
\gamma_*\bigl((\bar a^{\frac12}h_\psi^{1\over p}\bar a^{\frac12})^p\bigr)=(\bar
a_0^{\frac12}h_{\psi_0}^{1\over
p}\bar a_0^{\frac12})^p\qquad \text{or}\qquad \gamma_*\bigl((\bar
a^{-\frac12}h_\ffi^{1\over r}\bar a^{-\frac12})^r\bigr)=(\bar
a_0^{-\frac12}h_{\psi_0}^{1\over
r}\bar a_0^{-\frac12})^r.
\]
Note that by the proof of \cite[Thm. 1 (vi)]{kato2023onrenyi}, we have 
\[
b^*h_\psi^{1\over p}b=(h_\ffi^{1\over 2r}h_\psi^{1\over p}h_\ffi^{1\over 2r})^\alpha,
\]
where $bb^*=\bar a$, so that the first Zhang's equality above  seems to be related to part (ii)
rather than part (i), although the requirement for this in \cite[Thm. 1.2
(iii)]{zhang2020equality} is that $z\ne \alpha$, which is the condition in (i). I am a bit
confused about this.

}



\begin{proof} We prove (i), the statement (ii) is proved by
exchanging the roles of $\psi$ and $\ffi$ as before. As one of the steps in the proof of
Theorem \ref{thm:suffle1}, we have shown that if equality in DPI holds,  $\gamma$ is
sufficient with respect to $\{\omega,\psi\}$, where $\omega\in \Me_*^+$ is given by
\[
h_\omega=h_\psi^{p-1\over 2p}h_\mu^{\frac1p}h_\psi^{p-1\over 2p},\qquad
h_\mu=(h_\psi^{1\over 2p}h_\ffi^{1\over r} h_\psi^{1\over 2p})^z.
\]
We also proved that $\omega\circ \gamma =\omega_0$ where $\omega_0\in \Ne_*^+$ is
similarly obtained from $\psi_0,\ffi_0$. Notice that $h_\omega\in L_p(\Me,\psi)$ and we
have
\[
f_{\omega,p}(s)=\mu(1)^{\frac1p-s} h_\psi^{1-s\over 2}h_\mu^sh_\psi^{1-s\over 2},\qquad
s\in S,
\]
see Appendix \ref{app:kosaki_lp}. Similarly,
\[
f_{\omega_0,p}(s)=\mu_0(1)^{\frac1p-s} h_{\psi_0}^{1-s\over 2}h_{\mu_0}^sh_{\psi_0}^{1-s\over 2},\qquad
s\in S.
\]
Equality in DPI implies that $\mu(1)=\mu_0(1)$ and by the Hadamard three lines theorem and
the fact that $\gamma_*$ is a contraction on $L_q(\Me,\psi)$ for each $q\ge 1$,
\begin{align*}
\|h_{\omega_0}\|_{p,\psi_0}&=\|\gamma_*(f_{\omega,p}(\frac1p))\|_{p,\psi_0}\le
\left(\sup_t \|\gamma_*(f_{\omega,p}(it))\|_{\infty,\psi}\right)^{1-\frac1p}\left(\sup_t
\|\gamma_*(f_{\omega,p}(1+it))\|_{1}\right)^{\frac1p}\\
&\le \left(\sup_t \|f_{\omega,p}(it)\|_{\infty,\psi}\right)^{1-\frac1p}\left(\sup_t
\|f_{\omega,p}(1+it)\|_{1}\right)^{\frac1p}=\|h_\omega\|_{p,\psi}=\|h_{\omega_0}\|_{p,\psi_0}.
\end{align*}
We see that the function $S\ni s\mapsto \gamma_*(f_{\omega,p}(s))$ satisfies equality in
Hadamard three lines at $\theta=1/p\in (0,1)$, whence by \cite[Thm.
2.10]{jencova2018renyi} we must have
$\gamma_*(f_{\omega,p}(s))=f_{\omega_0,p}(s)M^{s-\frac1p}$ for all $s\in S$,  where $M=M_1/M_0$ with
$M_0= \sup_t \|f_{\omega,p}(it)\|_{\infty,\psi}$, $M_1=\sup_t
\|f_{\omega,p}(1+it)\|_{1}$. The equality above implies that we have
$M_0=M_1=\|h_\omega\|_{p,\psi}$, so that $M=1$.
Putting $s=1$ implies that $\gamma_*(h_\mu)=h_{\mu_0}$, which is the equality in
(i).  
The converse implication is clear.
\end{proof}





\subsubsection{The case $\alpha>1$}

We now turn to the case $\alpha>1$. We will put $p:=\frac z\alpha$ and
$q:=\frac z{\alpha-1}$, then within the DPI bounds, we have $p\in [1/2,1)$ and $q\ge 1$. Here we need to assume that
$D_{\alpha,z}(\psi\|\ffi)<\infty$, so that by Lemma \ref{lemma:renyi_2z} there is some
(unique)  $y\in L_{2z}(\Me)$ such that
\[
h_{\psi}^{1\over 2p}=y h_\ffi^{1\over 2q}.
\]
By the proof of Theorem \ref{thm:variational}, we have the following variational
expression
\begin{align}\label{eq:variationalq}
Q_{\alpha,z}(\psi\|\varphi) =\sup_{w\in
L_q(\Me)^+}\alpha\Tr\bigl((ywy^*)^p\bigr)-(\alpha-1)\Tr \bigl(w^q\bigr).
\end{align}
 The supremum is attained at a unique point $\bar
w=(y^*y)^{\alpha-1}\in L_q(\Me)^+$, uniqueness follows from strict concavity of the
function $w\mapsto \alpha\Tr\bigl((ywy^*)^p\bigr)-(\alpha-1)\Tr\bigl( w^q\bigr)$.


By DPI, we have $D_{\alpha,z}(\psi_0\|\varphi_0)\le D_{\alpha,z}(\psi\|\varphi)<\infty$,
so that there is some (unique) $y_0\in L_{2z}(\Ne)$ such that 
\[
h_{\psi_0}^{\frac1{2p}}=y_0h_{\varphi_0}^{\frac1{2q}}.
\]



\begin{lemma}\label{lemma:le} Let us assume that both $\ffi$ and $\ffi_0$ are faithful.
Let $\gamma^*_{\ffi,q}$ be the contraction as in Remark \ref{rem:pcontraction}. Keeping
the above assumptions and notations,   we  have for any $w_0\in
L_q(\Ne)^+$
\[
\Tr\bigl((y\gamma^*_{\ffi,q}(w_0)y^*)^p\bigr)\ge
\Tr \bigl((y_0w_0y_0^*)^p\bigr).
\]

\end{lemma}

\begin{proof} Let us first assume that
$w_0=h_{\varphi_0}^{\frac1{2q}}bh_{\varphi_0}^{\frac1{2q}}$ for some $b\in \Ne_+$. Then 
$\gamma^*_{\varphi,q}(w_0)=h_{\varphi}^{\frac1{2q}}\gamma(b)h_{\varphi}^{\frac1{2q}}$.
Therefore
\begin{align*}
\Tr\bigl((y\gamma^*_{\varphi,q}(w_0)y^*)^p\bigr)&=\Tr\bigl((yh_{\varphi}^{\frac1{2q}}\gamma(b)h_{\varphi}^{\frac1{2q}}y^*)^p\bigr)=
\Tr\bigl((h_\psi^{\frac1{2p}}\gamma(b)h_\psi^{\frac1{2p}})^p\bigr)\ge
\Tr\bigl((h_{\psi_0}^{\frac1{2p}}bh_{\psi_0}^{\frac1{2p}})^p\bigr)\\
&=
\Tr\bigl((y_0h_{\varphi_0}^{\frac1{2q}}bh_{\varphi_0}^{\frac1{2q}}y_0^*)^p\bigr)=\Tr\bigl((y_0w_0y_0^*)^p\bigr),
\end{align*}
here the inequality is from Lemma \ref{lemma:dpi} (i). The statement follows by Lemma
\ref{lemma:cone}.

\end{proof}

\begin{theorem}\label{thm:suffge1} Let $\gamma:\Ne\to \Me$ be a channel and let $\psi,\varphi\in \Me_*^+$ be
such that 
$s(\psi)\le s(\ffi)$ and 
$D_{\alpha,z}(\psi\|\varphi)<\infty$. Then
$D_{\alpha,z}(\psi\circ\gamma\|\ffi\circ\gamma)=D_{\alpha,z}(\psi\|\varphi)$ if and only if
$\gamma$ is sufficient with respect to $\{\psi,\ffi\}$.

\end{theorem}

\begin{proof} Supose that $D_{\alpha,z}(\psi_0\|\varphi_0)=D_{\alpha,z}(\psi\|\varphi)$.
As before, we may assume that both $\varphi$ and $\varphi_0$ are
faithful.  Let $\bar
w\in L_q(\Me)^+$ and $\bar w_0\in L_q(\Ne)^+$ be the unique elements such that the suprema
in the variational expression \eqref{eq:variationalq} for $D_{\alpha,z}(\psi\|\varphi)$ resp.
$D_{\alpha,z}(\psi_0\|\varphi_0)$ are attained. We have by Lemma \ref{lemma:le} and the
fact that $\gamma^*_{\ffi,q}$ is a contraction,
\begin{align*}
D_{\alpha,z}(\psi\|\varphi)&\ge \alpha\Tr\bigl((y\gamma^*_{\varphi,q}(\bar
w_0)y^*)^p\bigr)-(\alpha-1)\Tr
\bigl(\gamma^*_{\varphi,q}(\bar w_0)^q\bigr)\\
&\ge \alpha\Tr\bigr((y_0\bar w_0 y_0^*)^p\bigr)-(\alpha-1)\Tr \bigl(\bar
w_0^q\bigr)=D_{\alpha,z}(\psi_0\|\varphi_0)=D_{\alpha,z}(\psi\|\varphi),
\end{align*}
so that both inequalities must be equalities. This implies that in particular
\[
\Tr \bigl(\bar w_0^q\bigr)=\Tr \bigl(\gamma^*_{\varphi,q}(\bar w_0)^q\bigr).
\]
By uniqueness, we must also have  $\bar w=\gamma^*_{\varphi,q}(\bar w_0)$. Let now
$\omega\in \Me_*^+$, $\
\omega_0\in \Ne_*^+$ be given by
\begin{equation}\label{eq:omega}
h_{\omega}=h_\varphi^{q-1\over 2q}\bar wh_\varphi^{q-1\over 2q},\qquad 
h_{\omega_0}=h_{\varphi_0}^{q-1\over 2q}\bar w_0h_{\varphi_0}^{q-1\over 2q},
\end{equation}
then 
we get  $(\gamma^*_\varphi)_*(\omega_0)=\omega$ and also by
definition of the sandwiched R\'enyi divergence,
\[
D_{q,q}(\omega_0\|\varphi_0)=\Tr \bigl(\bar w_0^q\bigr)=\Tr \bigl(\gamma^*_{\varphi,q}(\bar
w_0)^q\bigr)=D_{q,q}(\omega_0\circ\gamma^*_\ffi\|\varphi_0\circ\gamma^*_\ffi).
\]
Similarly as in the proof of Theorem \ref{thm:suffge1}, this shows that $\gamma$ is sufficient with
respect to $\{\omega,\ffi\}$. Hence $\omega\circ \cE=\omega$, where $\cE$ is the
conditional expectation onto the fixed points of $\gamma\circ\gamma^*_\ffi$.  Using the
extensions of $\cE$ and their
properties, we get
\[
h_\varphi^{q-1\over2q}\bar
wh_\varphi^{q-1\over 2q}=h_\omega=\cE(h_\omega)=h_\varphi^{q-1\over 2q}\cE(\bar
w)h_\varphi^{q-1\over 2q},
\]
which implies that $\bar w=\cE(\bar w)\in L_q(\cE(\Me))^+$. But then also 
\[
|y|=\bar w^{\frac1{2(\alpha-1)}}=\bar w^{\frac{q}{2z}}\in L_{2z}(\cE(\Me))^+
\]
Let $y=u|y|$ be the polar decomposition of $y$, then we obtain from the definition of $y$ that
$uu^*=s(|y|)=s(\psi)$. Further,
\[
u^*h_\psi^{\frac1{2p}}=|y|h_\varphi^{\frac1{2q}}\in L_{2p}(\cE(\Me))
\]
and by uniqueness of the polar decomposition in $L_{2p}(\Me)$ and $L_{2p}(\cE(\Me))$, we
obtain that $h_{\psi}^{\frac1{2p}}\in L_{2p}(\cE(\Me))^+$, $u\in \cE(\Me)$. Hence we must
have $h_\psi\in L_1(\cE(\Me))$ so that $\psi\circ\cE=\psi$.


\end{proof}


\begin{coro}\label{coro:zhangge1} Let $\alpha> 1$ and $\max\{\alpha/2,\alpha-1\}\le
z\le \alpha$. Let $\psi,\ffi\in \Me_*^+$ be such that $s(\psi)\le s(\ffi)$. Let
$\gamma:\Ne\to \Me$ be a normal positive unital map. Assume that
$D_{\alpha,z}(\psi\|\ffi)<\infty$ and let $x\in L_z(\Me)^+$, $x_0\in L_z(\Ne)^+$ be such that 
\[
h_\psi^{\alpha\over z}=h_\ffi^{\alpha-1\over 2z}xh_\ffi^{\alpha-1\over 2z},\qquad
h_{\psi_0}^{\alpha\over z}=h_{\ffi_0}^{\alpha-1\over 2z}x_0h_{\ffi_0}^{\alpha-1\over 2z}.
\]
Then equality
in DPI holds if and only if $\gamma_*(x^z)=x_0^z$.


\end{coro}

\begin{proof} This can be proved the same way as Corollary \ref{coro:zhang}, using  $h_\omega$ and $h_{\omega_0}$ given by
\eqref{eq:omega}. Note that here $\bar w=(y^*y)^{\alpha-1}=x^{\alpha-1}$ and $\bar
w_0=(y^*_0y_0)^{\alpha-1}=x_0^{\alpha-1}$ and the equality $\omega\circ\gamma=\omega_0$ only uses positivity of
$\gamma$. 


\end{proof}

////////////////////////

\section{Monotonicity in the parameter $z$}

It is well known \cite{berta2018renyi,hiai2018quantum,jencova2018renyi} that the standard R\'enyi divergence
$D_{\alpha,1}(\psi\|\ffi)$ is monotone increasing in $\alpha\in(0,1)\cup(1,\infty)$ and the sandwiched R\'enyi
divergence $D_{\alpha,\alpha}(\psi\|\ffi)$ is monotone increasing in $\alpha\in[1/2,1)\cup(1,\infty)$. It is also
known \cite{berta2018renyi,hiai2018quantum,jencova2018renyi} that
\[
\lim_{\alpha\nearrow1}D_{\alpha,1}(\psi\|\ffi)=\lim_{\alpha\nearrow1}D_{\alpha,\alpha}(\psi\|\ffi)
=D_1(\psi\|\ffi),
\]
and if $D_{\alpha,1}(\psi\|\ffi)<\infty$ (resp., $D_{\alpha,\alpha}(\psi\|\ffi)<\infty$) for some $\alpha>1$, then
\[
\lim_{\alpha\searrow1}D_{\alpha,1}(\psi\|\ffi)=D_1(\psi\|\ffi)\quad
\Bigl(\mbox{resp.,}\ \lim_{\alpha\searrow1}D_{\alpha,\alpha}(\psi\|\ffi)=D_1(\psi\|\ffi)\Bigr).
\]
In the rest of the paper we will discuss similar monotonicity properties and limits for $D_{\alpha,z}(\psi\|\ffi)$.
We consider monotonicity in the parameter $z$ in Sec.~4 and monotonicity in the parameter $\alpha$ in Sec.~5.

\subsection{The finite von Neumann algebra case}

Assume that $(\Me,\tau)$ is a semi-finite von Neumann algebra $\Me$ with a faithful normal semi-finite trace
$\tau$. Then the Haagerup $L_p$-space $L_p(\Me)$ is identified with the $L_p$-space $L_p(\Me,\tau)$ with
respect to $\tau$ \cite[Example 9.11]{hiai2021lectures}. Hence one can define $Q_{\alpha,z}(\psi\|\ffi)$ for
$\psi,\ffi\in\Me_*^+$ by replacing, in Definition \ref{defi:renyi}, $L_p(\Me)$ with $L_p(\Me,\tau)$ and
$h_\psi\in L_1(\Me)_+$ with the Radon--Nikodym derivative $d\psi/d\tau\in L_1(\Me,\tau)^+$. Below we use the
symbol $h_\psi$ to denote $d\psi/d\tau$ as well. Note that $\tau$ on $\Me_+$ is naturally extended to the positive
part $\widetilde\Me^+$ of the space $\widetilde\Me$ of $\tau$-measurable operators. We then have 
\cite[Proposition 4.20]{hiai2021lectures}
\begin{align}\label{F-4.1}
\tau(a)=\int_0^\infty\mu_s(a)\,ds,\qquad a\in\widetilde\Me^+,
\end{align}
where $\mu_s(a)$ is the generalized $s$-number of $a$ \cite{fack1986generalized}.

Throughout this subsection we assume that $\Me$ is a finite von Neumann algebra with a faithful normal finite
trace $\tau$; then $\widetilde\Me^+$ consists of all positive self-adjoint operators affiliated with $\Me$.

\begin{lemma}\label{L-4.1}
For every $\psi,\ffi\in\Me_*^+$ with $\psi\ne0$ and for any $\alpha,z>0$ with $\alpha\ne1$,
\begin{align}\label{F-4.2}
D_{\alpha,z}(\psi\|\ffi)&=\lim_{\eps\searrow0}D_{\alpha,z}(\psi\|\ffi+\eps\tau)\quad\mbox{increasingly},
\end{align}
and hence $D_{\alpha,z}(\psi\|\ffi)=\sup_{\eps>0}D_{\alpha,z}(\psi\|\ffi+\eps\tau)$.
\end{lemma}

\begin{proof}
{\it Case $0<\alpha<1$}.\enspace
We need to prove that
\begin{align}\label{F-4.3}
Q_{\alpha,z}(\psi\|\ffi)&=\lim_{\eps\searrow0}Q_{\alpha,z}(\psi\|\ffi+\eps\tau)\quad\mbox{decreasingly}.
\end{align}
In the present setting we have by \eqref{F-4.1}
\begin{align}\label{F-4.4}
Q_{\alpha,z}(\psi\|\ffi)
=\tau\Bigl(\bigl(h_\psi^{\alpha/2z}h_\ffi^{1-\alpha\over z}h_\psi^{\alpha/2z}\bigr)^z\Bigr)
=\int_0^\infty\mu_s\bigl(h_\psi^{\alpha/2z}h_\ffi^{1-\alpha\over z}h_\psi^{\alpha/2z}\bigr)^z\,ds,
\end{align}
and similarly
\[
Q_{\alpha,z}(\psi\|\ffi+\eps\tau)
=\int_0^\infty\mu_s\bigl(h_\psi^{\alpha/2z}h_{\ffi+\eps\tau}^{1-\alpha\over z}h_\psi^{\alpha/2z}\bigr)^z\,ds.
\]
Since $h_{\ffi+\eps\tau}^{1-\alpha\over z}=(h_\ffi+\eps\1)^{1-\alpha\over z}$ decreases to
$h_\ffi^{1-\alpha\over z}$ in the measure topology as $\eps\searrow0$, it follows that
$h_\psi^{\alpha/2z}h_{\ffi+\eps\tau}^{1-\alpha\over z}h_\psi^{\alpha/2z}$ decreases to
$h_\psi^{\alpha/2z}h_\ffi^{1-\alpha\over z}h_\psi^{\alpha/2z}$ in the measure topology. Hence by
\cite[Lemma 3.4]{fack1986generalized} we have
$\mu_s\bigl(h_\psi^{\alpha/2z}h_{\ffi+\eps\tau}^{1-\alpha\over z}h_\psi^{\alpha/2z}\bigr)
\searrow\mu_s\bigl(h_\psi^{\alpha/2z}h_\ffi^{1-\alpha\over z}h_\psi^{\alpha/2z}\bigr)$
as $\eps\searrow0$ for almost every $s>0$. Since
$s\mapsto\mu_s\bigl(h_\psi^{\alpha/2z}h_{\ffi+\tau}^{1-\alpha\over z}h_\psi^{\alpha/2z}\bigr)$ is
integrable on $(0,\infty)$, the Lebesgue convergence theorem gives \eqref{F-4.3}.

{\it Case $\alpha>1$}.\enspace
We need to prove that
\begin{align}\label{F-4.5}
Q_{\alpha,z}(\psi\|\ffi)&=\lim_{\eps\searrow0}Q_{\alpha,z}(\psi\|\ffi+\eps\tau)\quad\mbox{increasingly}.
\end{align}
For any $\eps>0$, since $h_{\ffi+\eps\tau}=h_\psi+\eps\1$ has the bounded inverse
$h_{\ffi+\eps\tau}^{-1}=(h_\ffi+\eps\1)^{-1}\in\Me^+$, one can define
$x_\eps:=(h_\ffi+\eps\1)^{-{\alpha-1\over2z}}h_\psi^{\alpha/z}(h_\ffi+\eps\1)^{-{\alpha-1\over2z}}
\in\widetilde\Me^+$ so that
\[
h_\psi^{\alpha/z}=(h_\ffi+\eps\1)^{\alpha-1\over2z}x_\eps(h_\ffi+\eps\1)^{\alpha-1\over2z}.
\]
In the present setting one can write by \eqref{F-4.1}
\begin{align}\label{F-4.6}
Q_{\alpha,z}(\psi\|\ffi+\eps\tau)=\tau(x_\eps^z)=\int_0^\infty\mu_s(x_\eps)^z\,ds\ (\in[0,\infty]).
\end{align}
Let $0<\eps\le\eps'$. Since $(h_\ffi+\eps\1)^{-{\alpha-1\over z}}\ge(h_\ffi+\eps'\1)^{-{\alpha-1\over z}}$,
one has $\mu_s(x_\eps)\ge\mu_s(x_{\eps'})$ for all $s>0$, so that
\[
Q_{\alpha,z}(\psi\|\ffi+\eps\tau)\ge Q_{\alpha,z}(\psi\|\ffi+\eps'\tau).
\]
Hence $\eps>0\mapsto D_{\alpha,z}(\psi\|\ffi+\eps\tau)$ is decreasing.

First, assume that $s(\psi)\not\le s(\ffi)$. Then
$\mu_{s_0}(h_\psi^{\alpha/2z}s(\ffi)^\perp h_\psi^{\alpha/2z})>0$ for some $s_0>0$; indeed, otherwise,
$h_\psi^{\alpha/2z}s(\ffi)^\perp h_\psi^{\alpha/2z}=0$ so that $s(\psi)\le s(\ffi)$. Hence we have
\[
\mu_s(x_\eps)=\mu_s\bigl(h_\psi^{\alpha/2z}(h_\ffi+\eps\1)^{-{\alpha-1\over z}}h_\psi^{\alpha/2z}\bigr)
\ge\eps^{-{\alpha-1\over z}}\mu_s(h_\psi^{\alpha/2z}s(\ffi)^\perp h_\psi^{\alpha/2z})
\nearrow\infty\quad\mbox{as $\eps\searrow0$}
\]
for all $s\in(0,s_0]$. Therefore, it follows from \eqref{F-4.6} that
$Q_{\alpha,z}(\psi\|\ffi+\eps\tau)\nearrow\infty=Q_{\alpha,z}(\psi\|\ffi)$.

Next, assume that $s(\psi)\le s(\ffi)$. Take the spectral decomposition $h_\ffi=\int_0^\infty t\,de_t$ and
define $y,x\in\widetilde\Me_+$ by
\[
y:=h_\ffi^{-{\alpha-1\over z}}s(\ffi)=\int_{(0,\infty)}t^{-{\alpha-1\over z}}\,de_t,
\qquad x:=y^{1/2}h_\psi^{\alpha/z}y^{1/2}.
\]
Since
\[
h_\psi^{\alpha/z}=s(\ffi)h_\psi^{\alpha/z}s(\ffi)
=h_\ffi^{\alpha-1\over2z}y^{1/2}h_\psi^{\alpha/z}y^{1/2}h_\ffi^{\alpha-1\over2z}
=h_\ffi^{\alpha-1\over2z}xh_\ffi^{\alpha-1\over2z},
\]
one has, similarly to \ref{F-4.6},
\begin{align}\label{F-4.7}
Q_{\alpha,z}(\psi\|\ffi)=\tau(x^z)=\int_0^\infty\mu_s(x)^z\,ds.
\end{align}
We write $(h_\ffi+\eps\1)^{-{\alpha-1\over z}}s(\ffi)=\int_{(0,\infty)}(t+\eps)^{-{\alpha-1\over z}}\,de_t$,
and for any $\delta>0$ choose a $t_0>0$ such that $\tau(e_{(0,t_0)})<\delta$. Then, since
$\int_{[t_0,\infty)}(t+\eps)^{-{\alpha-1\over z}}\,de_t\to\int_{[t_0,\infty)}t^{-{\alpha-1\over z}}\,de_t$
in the operator norm as $\eps\searrow0$, we obtain $(h_\ffi+\eps\1)^{-{\alpha-1\over z}}s(\ffi)\nearrow y$
in the measure topology (see \cite[1.5]{fack1986generalized}), so that
$h_\psi^{\alpha/2z}(h_\ffi+\eps\1)^{-{\alpha-1\over z}}h_\psi^{\alpha/2z}
\nearrow h_\psi^{\alpha/2z}yh_\psi^{\alpha/2z}$ in the measure topology as $\eps\searrow0$. Hence
we have by \cite[Lemma 3.4]{fack1986generalized}
\begin{align}\label{F-4.8}
\mu_s(x_\eps)=\mu_s\bigl(h_\psi^{\alpha/2z}(h_\ffi+\eps\1)^{-{\alpha-1\over z}}h_\psi^{\alpha/2z}\bigr)
\nearrow\mu_s(h_\psi^{\alpha/2z}yh_\psi^{\alpha/2z})=\mu_s(x)
\end{align}
for all $s>0$. Therefore, by \eqref{F-4.6} and \eqref{F-4.7} the monotone convergence theorem gives
\eqref{F-4.5}.
\end{proof}

\begin{lemma}\label{L-4.2}
Let $(\Me,\tau)$ and $\psi,\ffi$ be as above, and let $0<z\le z'$. Then
\[
\begin{cases}
D_{\alpha,z}(\psi\|\ffi)\le D_{\alpha,z'}(\psi\|\ffi), & \text{$0<\alpha<1$},\\
D_{\alpha,z}(\psi\|\ffi)\ge D_{\alpha,z'}(\psi\|\ffi), & \text{$\alpha>1$}.
\end{cases}
\]
\end{lemma}

\begin{proof}
The case $0<\alpha<1$ was shown in \cite[Theorem 1(x)]{kato2023onrenyi} for general von Neumann algebras.
For the case $\alpha>1$, by Lemma \ref{L-4.1} it suffices to show that, for every $\eps>0$,
\[
\tau\Bigl(\Bigl(y_\eps^{\alpha-1\over2z}h_\psi^{\alpha/z}y_\eps^{\alpha-1\over2z}\Bigr)^z\Bigr)
\ge\tau\Bigl(\Bigl(y_\eps^{\alpha-1\over2z'}h_\psi^{\alpha/z'}y_\eps^{\alpha-1\over2z'}\Bigr)^z\Bigr),
\]
where $y_\eps:=(h_\ffi+\eps\1)^{-1}\in\Me_+$. The above is equivalently written as
\[
\tau\Bigl(\big|(h_\psi^{\alpha/2z'})^r(y^{(\alpha-1)/2z'})^r\big|^{2z}\Bigr)
\ge\tau\Bigl(\big|h_\psi^{\alpha/2z'}y^{(\alpha-1)/2z'}\big|^{2zr}\Bigr),
\]
where $r:=z'/z\ge1$. Hence the desired inequality follows from Kosaki's ALT inequality
\cite[Corollary 3]{kosaki1992aninequality}.
\end{proof}

When $(\Me,\tau)$ and $\psi,\ffi$ are as in Lemma \ref{L-4.1}, one can define, thanks to Lemma \ref{L-4.2},
for any $\alpha\in(0,\infty)\setminus\{1\}$,
\begin{align}
Q_{\alpha,\infty}(\psi\|\ffi)&:=\lim_{z\to\infty}Q_{\alpha,\infty}(\psi\|\ffi)
=\inf_{z>0}Q_{\alpha,z}(\psi\|\ffi), \nonumber\\
D_{\alpha,\infty}(\psi\|\ffi)&:={1\over\alpha-1}\log{Q_{\alpha,\infty}(\psi\|\ffi)\over\psi(\1)} \nonumber\\
&\ =\lim_{z\to\infty}D_{\alpha,z}(\psi\|\ffi)
=\begin{cases}\sup_{z>0}D_{\alpha,z}(\psi\|\ffi), & \text{$0<\alpha<1$},\\
\inf_{z>0}D_{\alpha,z}(\psi\|\ffi), & \text{$\alpha>1$}.\end{cases} \label{F-4.9}
\end{align}
If $h_\psi,h_\ffi\in\Me^{++}$ (i.e., $\delta\tau\le\psi,\ffi\le\delta^{-1}\tau$ for some $\delta\in(0,1)$), then
the Lie--Trotter formula gives
\begin{align}\label{F-4.10}
Q_{\alpha,\infty}(\psi\|\ffi)=\tau\bigl(\exp(\alpha\log h_\psi+(1-\alpha)\log h_\ffi)\bigr).
\end{align}

\begin{lemma}\label{L-4.3}
Let $(\Me,\tau)$ and $\psi,\ffi$ be as above. Then for any $z>0$,
\[
\begin{cases}
D_{\alpha,z}(\psi\|\ffi)\le D_1(\psi\|\ffi), & \text{$0<\alpha<1$},\\
D_{\alpha,z}(\psi\|\ffi)\ge D_1(\psi\|\ffi), & \text{$\alpha>1$}.
\end{cases}
\]
\end{lemma}

\begin{proof}
First, assume that $h_\psi,h_\ffi\in\Me^{++}$. Set self-adjoint $H:=\log h_\psi$ and $K:=\log h_\ffi$ in $\Me$
and define $F(\alpha):=\log\tau\bigl(e^{\alpha H+(1-\alpha)K}\bigr)$ for $\alpha>0$.
Then by \eqref{F-4.10}, $F(\alpha)=\log Q_{\alpha,\infty}(\psi\|\ffi)$ for all $\alpha\in(0,\infty)\setminus\{1\}$,
and we compute
\begin{align*}
F'(\alpha)&={\tau\bigl(e^{\alpha H+(1-\alpha)K}(H-K)\bigr)\over\tau\bigl(e^{\alpha H+(1-\alpha)K}\bigr)}, \\
F''(\alpha)&={\bigl\{\tau\bigl(e^{\alpha H+(1-\alpha)K}(H-K)\bigr)\bigr\}^2
-\tau\bigl(e^{\alpha H+(1-\alpha)K}(H-K)^2\bigr)\over\bigl\{\tau\bigl(e^{\alpha H+(1-\alpha)K}\bigr)\bigr\}^2}.
\end{align*}
Since $F''(\alpha)\ge0$ on $(0,\infty)$ thanks to the Schwarz inequality, we see that $F(\alpha)$ is
convex on $(0,\infty)$ and hence
\[
D_{\alpha,\infty}(\psi\|\ffi)={F(\alpha)-F(1)\over\alpha-1}
\]
is increasing in $\alpha\in(0,\infty)$, where for $\alpha=1$ the above RHS is understood as
\[
F'(1)={\tau(e^H(H-K))\over\tau(e^H)}={\tau\bigl(h_\psi(\log h_\psi-\log h_\ffi)\bigr)\over\tau(h_\psi)}
=D_1(\psi\|\ffi).
\]
Hence by \eqref{F-4.9} the assertion holds when $h_\psi,h_\ffi\in\Me^{++}$. Below we extend it to general
$\psi,\ffi\in\Me_*^+$.

{\it Case $0<\alpha<1$}.\enspace
Let $\psi,\ffi\in\Me_*^+$ and $z>0$. From \cite[Theorem 1(iv)]{kato2023onrenyi} and
\cite[Corollary 2.8(3)]{hiai2021quantum} we have
\begin{align*}
D_{\alpha,z}(\psi\|\ffi)&=\lim_{\eps\searrow0}D_{\alpha,z}(\psi+\eps\tau\|\ffi+\eps\tau), \\
D_1(\psi\|\ffi)&=\lim_{\eps\searrow0}D_1(\psi+\eps\tau\|\ffi+\eps\tau),
\end{align*}
so that we may assume that $\psi,\ffi\ge\eps\tau$ for some $\eps>0$. Take the spectral decompositions
$h_\psi=\int_0^\infty t\,de_t^\psi$ and $h_\ffi=\int_0^\infty t\,de_t^\ffi$, and define
$e_n:=e_n^\psi\wedge e_n^\ffi$ for each $n\in\bN$. Then
$\tau(e_n^\perp)\le\tau((e_n^\psi)^\perp)+\tau((e_n^\ffi)^\perp)\to0\quad\mbox{as $n\to\infty$}$,  so that
$e_n\nearrow\1$. We set $\psi_n:=\psi(e_n\cdot e_n)$ and $\ffi_n:=\ffi(e_n\cdot e_n)$; then
$h_{\psi_n}=e_nh_\psi e_n$ and $h_{\ffi_n}=e_nh_\ffi e_n$ are in $(e_n\Me e_n)^{++}$. Note that
\begin{align*}
\|h_\psi-e_nh_\psi e_n\|_1&\le\|(\1-e_n)h_\psi\|_1+\|e_nh_\psi(\1-e_n)\|_1 \\
&\le\|(\1-e_n)h_\psi^{1/2}\|_2\|h_\psi^{1/2}\|_2+\|e_nh_\psi^{1/2}\|_2\|h_\psi^{1/2}(\1-e_n)\|_2 \\
&=\psi(\1-e_n)^{1/2}\psi(\1)^{1/2}+\psi(e_n)^{1/2}\psi(\1-e_n)^{1/2}\to0\quad\mbox{as $n\to\infty$},
\end{align*}
and similarly $\|h_\ffi-e_nh_\ffi e_n\|_1\to0$. Hence by \cite[Theorem 1(iv)]{kato2023onrenyi} one has
$D_{\alpha,z}(e_n\psi e_n\|e_n\ffi e_n)\to D_{\alpha,z}(\psi\|\ffi)$. On the other hand, one has
$D_1(e_n\psi e_n\|e_n\ffi e_n)\to D_1(\psi\|\ffi)$ by \cite[Proposition 2.10]{hiai2021quantum}. Since
$D_{\alpha,z}(e_n\psi e_n\|e_n\ffi e_n)\le D_1(e_n\psi e_n\|e_n\ffi e_n)$ holds by regarding
$e_n\psi e_n,e_n\ffi e_n$ as functionals on the reduced von Neumann algebra $e_n\Me e_n$, we obtain
the desired inequality for general $\psi,\ffi\in\Me_*^+$.

{\it Case $\alpha>1$}.\enspace
We show the extension to general $\psi,\ffi\in\Me_*^+$ by dividing four steps as follows, where
$h_\psi=\int_0^\infty t\,e_t^\psi$ and $h_\ffi=\int_0^\infty t\,de_t^\ffi$ are the spectral decompositions.

(1)\enspace
Assume that $h_\psi\in\Me^+$ and $h_\ffi\in\Me^{++}$. Set $\psi_n\in\Me_*^+$ by
$h_{\psi_n}=(1/n)e_{[0,1/n]}^\psi+\int_{(1/n,\infty)}t\,de_t^\psi$ ($\in\Me^{++}$). Since
$h_{\psi_n}^{\alpha/z}\searrow h_\psi^{\alpha/z}$ in the operator norm, we have by \eqref{F-4.4} and
\cite[Lemma 3.4]{fack1986generalized}
\begin{equation}\label{F-4.11}
\begin{aligned}
Q_{\alpha,z}(\psi\|\ffi)&=\int_0^\infty\mu_s\bigl((h_\ffi^{-1})^{\alpha-1\over2z}h_\psi^{\alpha/z}
(h_\ffi^{-1})^{\alpha-1\over2z}\bigr)^z\,ds \\
&=\lim_{n\to\infty}\int_0^\infty\mu_s\bigl((h_\ffi^{-1})^{\alpha-1\over2z}h_{\psi_n}^{\alpha/z}
(h_\ffi^{-1})^{\alpha-1\over2z}\bigr)^z\,ds
=\lim_{n\to\infty}Q_{\alpha,z}(\psi_n\|\ffi).
\end{aligned}
\end{equation}
From this and the lower semicontinuity of $D_1$ the extension holds in this case.

(2)\enspace
Assume that $h_\psi\in\Me^+$ and $h_\ffi\ge\delta\1$ for some $\delta>0$. Set $\ffi_n\in\Me_*^+$
by $h_{\ffi_n}=\int_{[\delta,n]}t\,de_t^\ffi+ne_{(n,\infty)}^\ffi$ ($\in\Me^{++}$). Since
$h_{\ffi_n}^{-{\alpha-1\over z}}\searrow h_\ffi^{-{\alpha-1\over z}}$ in the operator norm, we have by
\eqref{F-4.4} and \cite[Lemma 3.4]{fack1986generalized} again
\begin{align*}
Q_{\alpha,z}(\psi\|\ffi)&=\int_0^\infty\mu_s\bigl(h_\psi^{\alpha/2z}h_\ffi^{-{\alpha-1\over z}}
h_\psi^{\alpha/2z}\bigr)^z\,ds \\
&=\lim_{n\to\infty}\int_0^\infty\mu_s\bigl(h_\psi^{\alpha/2z}h_{\ffi_n}^{-{\alpha-1\over z}}
h_\psi^{\alpha/2z}\bigr)^z\,ds
=\lim_{n\to\infty}Q_{\alpha,z}(\psi,\ffi_n).
\end{align*}
From this and (1) above the extension holds in this case too.

(3)\enspace
Assume that $\psi$ is general and $\ffi\ge\delta\tau$ for some $\delta>0$. Set $\psi_n\in\Me_*^+$
by $h_{\psi_n}=\int_{[0,n]}t\,de_t^\psi+ne_{(n,\infty)}^\ffi$ ($\in\Me_+$). Since
$h_{\psi_n}^{\alpha/z}\nearrow h_\psi^{\alpha/z}$ in the measure topology, one can argue as in \eqref{F-4.11}
with use of the monotone convergence theorem to see from (2) that the extension holds in this case too.

(4)\enspace
Finally, from (3) with Lemma \ref{L-4.1} and \cite[Corollary 2.8(3)]{hiai2021quantum} it follows that
the desired extension hods for general $\psi,\ffi\in\Me_*^+$.
\end{proof}

In the next proposition, we summarize inequalities for $D_{\alpha,z}$ obtained so far in Lemmas \ref{L-4.2}
and \ref{L-4.3}.

\begin{prop}\label{P-4.4}
Assume that $\Me$ is a finite von Neumann algebra with a faithful normal finite trace $\tau$. Let
$\psi,\ffi\in\Me_*^+$, $\psi\ne0$. If $0<\alpha<1<\alpha'$ and $0<z\le z'\le\infty$, then
\[
D_{\alpha,z}(\psi\|\ffi)\le D_{\alpha,z'}(\psi\|\ffi)\le D_1(\psi\|\ffi)
\le D_{\alpha',z'}(\psi\|\ffi)\le D_{\alpha',z}(\psi\|\ffi).
\]
\end{prop}

\begin{coro}\label{C-4.5}
Let $(\Me,\tau)$ and $\psi,\ffi$ be as in Proposition \ref{P-4.4}. Then for any $z\in[1,\infty]$,
\begin{align}\label{F-4.12}
\lim_{\alpha\nearrow1}D_{\alpha,z}(\psi\|\ffi)=D_1(\psi\|\ffi).
\end{align}
Moreover, if $D_{\alpha,\alpha}(\psi\|\ffi)<\infty$ for some $\alpha>1$ then for any $z\in(1,\infty]$,
\begin{align}\label{F-4.13}
\lim_{\alpha\searrow1}D_{\alpha,z}(\psi\|\ffi)=D_1(\psi\|\ffi).
\end{align}
\end{coro}

\begin{proof}
Let $z\ge1$. For every $\alpha\in(0,1)$, Proposition \ref{P-4.4} gives
\[
D_{\alpha,1}(\psi\|\ffi)\le D_{\alpha,z}(\psi\|\ffi)\le D_1(\psi\|\ffi).
\]
Hence \eqref{F-4.12} follows since it holds for $D_{\alpha,1}$ \cite[Proposition 5.3(3)]{hiai2018quantum}.

Next, assume that $D_{\alpha,\alpha}(\psi\|\ffi)<\infty$ for some $\alpha>1$. Let $z>1$. For every
$\alpha\in(1,z]$, Proposition \ref{P-4.4} gives
\[
D_1(\psi\|\ffi)\le D_{\alpha,z}(\psi\|\ffi)\le D_{\alpha,\alpha}(\psi\|\ffi).
\]
Hence \eqref{F-4.13} follows since it holds for $D_{\alpha,\alpha}$ \cite[Proposition 3.8(ii)]{jencova2018renyi}.
\end{proof}

\medskip
In this subsection, in the specialized setting of finite von Neumann algebras, we have given monotonicity
of $D_{\alpha,z}$ in the parameter $z$ in an essentially similar way to the finite-dimensional case
\cite{mosonyi2023somecontinuity}. In the next subsection we will extend it to general von Neumann algebras
under certain restrictions of $\alpha,z$.

\subsection{The general von Neumann algebra case}

\begin{theorem}
For every $\psi,\ffi\in\Me_*^+$, $\psi\ne0$, and $0<\alpha<1$, we have:
\begin{itemize}
\item[(1)] If $0<\alpha<1$ and $\max\{\alpha,1-\alpha\}\le z\le z'$, then
\[
D_{\alpha,z}(\psi\|\ffi)\le D_{\alpha,z'}(\psi\|\ffi)\le D_1(\psi\|\ffi).
\]
\item[(2)] If $\alpha>1$ and $\max\{\alpha/2,\alpha-1\}\le z\le z'\le\alpha$, then
\[
D_1(\psi\|\ffi)\le D_{\alpha,z'}(\psi\|\ffi)\le D_{\alpha,z}(\psi\|\ffi).
\]
\end{itemize}
\end{theorem}
{\color{magenta}Hiai (12/8/2023)} {\color{blue}In fact, (2) is improved in Theorem 6.}

\begin{theorem}
For every $\psi,\ffi\in\Me_*^+$, $\psi\ne0$, and $\alpha>1$, the function $z\mapsto D_{\alpha,z}(\psi\|\varphi)$
is monotone decreasing on $[\alpha/2,\infty)$.
\end{theorem}
{\color{magenta}Anna (Jan.\ 23, 2024)}


\section{Monotonicity in the parameter $\alpha$}

\subsection{The case $\alpha<1$ and all $z>0$}

\begin{theorem}
Let $\psi,\ffi\in\Me_*^+$ and $z>0$. Then we have
\begin{itemize}
\item[(1)] $\alpha\mapsto\log Q_{\alpha,z}(\psi\|\ffi)$ is convex on $(0,1)$,
\item[(2)] $\alpha\mapsto D_{\alpha,z}(\psi\|\ffi)$ is monotone increasing on $(0,1)$.
\end{itemize}
\end{theorem}
{\color{magenta}Anna (Jan.\ 10, 2024), Hiai (1/16/2024)}

\subsection{The case $1<\alpha\le2z$}

\begin{theorem}
Let $\psi,\ffi\in\Me_*^+$ and $z>1/2$. Then we have
\begin{itemize}
\item[(1)] $\alpha\mapsto\log Q_{\alpha,z}(\psi\|\ffi)$ is convex on $(1,2z]$,
\item[(2)] $\alpha\mapsto D_{\alpha,z}(\psi\|\ffi)$ is monotone increasing on $(1,2z]$.
\end{itemize}
\end{theorem}

{\color{magenta}Anna (Jan.\ 23, 2024), Hiai (12/31/2023)}

\subsection{Limits as $\alpha\nearrow1$ and $\alpha\searrow1$}

\begin{theorem}
Let $\psi,\ffi\in\Me_*^+$, $\psi\ne0$. For every $z\in(0,1]$ we have
\[
\lim_{\alpha\nearrow1}D_{\alpha,z}(\psi\|\ffi)=D_1(\psi\|\ffi).
\]
\end{theorem}
{\color{magenta}Anna (Dec.\ 7, 2023)}

\begin{theorem}
Let $\psi,\ffi\in\Me_*^+$, $\psi\ne0$, and $z>1/2$. Assume that $D_{\alpha,z}(\psi\|\ffi)<\infty$ for some
$\alpha\in(1,2z]$. Then we have
\[
\lim_{\alpha\searrow1}D_{\alpha,z}(\psi\|\ffi)=D_1(\psi\|\ffi).
\]
\end{theorem}
{\color{magenta}Anna (Jan.\ 23, 2024)}


\appendix

\section{Haagerup  $L_p$-spaces}


The following lemmas are well known, proofs are given for completeness.

\begin{lemma}\label{lemma:cone} For any $0<p<\infty$ and $\varphi\in \Me_*^+$, 
$h_\varphi^{\frac1{2p}}\Me^+h_\varphi^{\frac1{2p}}$ is dense in $L_p(\Me)^+$ with respect
to the (quasi)-norm $\|\cdot\|_p$.

\end{lemma}

\begin{proof} We may assume that $\varphi$ is faithful. By \cite[Lemma 1.1]{junge2003noncommutative}, $\Me
h_\varphi^{\frac1{2p}}$ is dense in $L_{2p}(\Me)$ for any $0<p<\infty$. Let $y\in L_p(\Me)^+$, then
$y^{\frac12}\in L_{2p}(\Me)$, hence there is a sequence $a_n\in \Me$ such that
$\|a_nh^{\frac1{2p}}_\varphi-y^{\frac12}\|_{2p}\to 0$. Then also 
\[
\|h^{\frac1{2p}}_\varphi
a_n^*-y^{\frac12}\|_p=\|(a_nh^{\frac1{2p}}_\varphi-y^{\frac12})^*\|_p=\|a_nh^{\frac1{2p}}_\varphi-y^{\frac12}\|_p\to 0
\]
and 
\[
\|h^{\frac1{2p}}_\varphi a_n^*a_nh^{\frac1{2p}}_\varphi-y\|_p=\|(h^{\frac1{2p}}_\varphi
a_n^*-y^{\frac12})a_nh^{\frac1{2p}}_\varphi+y^{\frac12}(a_nh^{\frac1{2p}}_\varphi-y^{\frac12})\|_p
\]
Since $\|\cdot\|_p$ is a (quasi)-norm, the above expression goes to 0 by the H\"older
inequality.



\end{proof}


\begin{lemma}\label{lemma:order1} Let $0<p\le \infty$ and let $h,k\in L_p(\Me)^+$ be such
that $h\le k$. Then 
$\|h\|_p\le \|k\|_p$. Moreover, if $1\le p<\infty$, then 
\[
\|k-h\|_p^p\le \|k\|_p^p-\|h\|_p^p.
\]

\end{lemma}

\begin{proof} The first statement follows from \cite[Lemma 2.5 (iii) and  Lemma
4.8]{fack1986generalized}. The second statement is from \cite[Lemma
5.1]{fack1986generalized}.


\end{proof}


\begin{lemma}\label{lemma:order} Let $\psi,\varphi\in \Me_*^+$ with $\psi\le \varphi$.
Then for any $a\in \Me$ and $p\in [1,\infty)$,
\[
\Tr\left( (a^*h_\psi^{\frac1p}a)^p\right)\le Tr\left( (a^*h_\varphi^{\frac1p}a)^p\right)
\]


\end{lemma}

\begin{proof} Since $1/p\in (p,1]$, it follows (see \cite[Lemma B.7]{hiai2021quantum} and
\cite[Lemma 3.2]{hiai2021connections}) that $h_\psi^{1/p}\le h_\varphi^{1/p}$.  Hence $a^*h_\psi^{1/p}a\le a^*h_\varphi^{1/p}a$. 
Therefore, by Lemma \ref{lemma:order1}, 
we have the statement.

\end{proof}




%\bibliography{alphaz}
%\bibliographystyle{abbrvnat}
\begin{thebibliography}{99}
\bibitem{berta2018renyi}
M. Berta, V. B. Scholz, and M. Tomamichel. R\'enyi divergences as weighted noncommutative
vector valued $Lp$-spaces. Annales Henri Poincar\'e, 19:1843--1867, 2018.
doi:https://doi.org/10.48550/arXiv.1608.05317.

\bibitem{choi1974aschwarz}
M.-D. Choi. A Schwarz inequality for positive linear maps on $C^*$-algebras. Illinois Journal
of Mathematics, 18(4):565--574, 1974. doi:10.1215/ijm/1256051007.

\bibitem{fack1986generalized}
T. Fack and H. Kosaki. Generalized $s$-numbers of $\tau$-measurable operators. Pacific Journal
of Mathematics, 123(2):269--300, 1986.

\bibitem{haagerup1979lpspaces}
U. Haagerup. $L_p$-spaces associated with an arbitrary von Neumann algebra. In Algebres
d’op\'erateurs et leurs applications en physique mathématique (Proc. Colloq., Marseille, 1977),
volume 274, pages 175--184, 1979.

\bibitem{hiai2018quantum}
F. Hiai. Quantum $f$-divergences in von Neumann algebras. I. Standard $f$-divergences. Journal
of Mathematical Physics, 59(10):102202, 2018.

\bibitem{hiai2021quantum}
F. Hiai. Quantum $f$-Divergences in von Neumann Algebras: Reversibility of Quantum Operations.
Mathematical Physics Studies. Springer, Singapore, 2021. ISBN 9789813341999.
doi:10.1007/978-981-33-4199-9.

\bibitem{hiai2021lectures}
F. Hiai. Lectures on Selected Topics in Von Neumann Algebras. EMS Series of Lectures in Mathematics.
EMS Press, Berlin, 2021.

\bibitem{jencova2018renyi}
A. Jen\v cov\'a. R\'enyi relative entropies and noncommutative Lp-spaces. Annales Henri Poincar\'e,
19:2513--2542, 2018. doi:10.1007/s00023-018-0683-5.

\bibitem{jencova2021renyi}
A. Jen\v cov\'a. R\'enyi relative entropies and noncommutative $Lp$-spaces II. Annales Henri
Poincar\'e, 22:3235--3254, 2021. doi:10.1007/s00023-021-01074-9.

\bibitem{junge2003noncommutative}
M. Junge and Q. Xu. Noncommutative Burkholder/Rosenthal inequalities. The Annals of
Probability, 31(2):948--995, 2003.

\bibitem{kato2023onrenyi}
S. Kato. On $\alpha$-$z$-R\'enyi divergence in the von Neumann algebra setting. arXiv preprint
arXiv:2311.01748, 2023.

\bibitem{kato2023aremark}
S. Kato and Y. Ueda. A remark on non-commutative $Lp$-spaces. arXiv preprint
arXiv:2307.01790, 2023.

\bibitem{kosaki1984applications}
H. Kosaki. Applications of the complex interpolation method to a von Neumann algebra: Noncommutative
$L^p$-spaces. J. Funct. Anal., 56:26--78, 1984. doi:https://doi.org/10.1016/0022-1236(84)90025-9.

\bibitem{kosaki1992aninequality}
H. Kosaki. An inequality of Araki--Lieb--Thirring (von Neumann algebra case).
Proc. Amer. Math. Soc., 114:477--481, 1992.

\bibitem{mosonyi2023somecontinuity}
M. Mosonyi and F. Hiai. Some continuity properties of quantum R\'enyi divergences.
IEEE Transactions on Information Theory, to appear, DOI 10.1109/TIT.2023.3324758.

\bibitem{petz1985quasi}
D. Petz. Quasi-entropies for states of a von Neumann algebra. Publications of the Research
Institute for Mathematical Sciences, 21(4):787--800, 1985. doi:10.2977/prims/1195178929.

\bibitem{petz1986sufficient}
D. Petz. Sufficient subalgebras and the relative entropy of states of a von Neumann algebra.
Communications in Mathematical Physics, 105(1):123--131, 1986. doi:10.1007/BF01212345.

\bibitem{petz1988sufficiency}
D. Petz. Sufficiency of channels over von Neumann algebras. The Quarterly Journal of
Mathematics, 39(1):97–108, 1988. doi:10.1093/qmath/39.1.97.8

\bibitem{terp1981lpspaces}
M. Terp. $L_p$-spaces associated with von Neumann algebras. Notes, Copenhagen University,
1981.
\end{thebibliography}

\end{document}
