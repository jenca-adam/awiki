\documentclass[12pt]{article}
\usepackage{geometry}
\usepackage{amsfonts}
\geometry{total={210mm,290mm},
 left=23mm,right=23mm,%
 bindingoffset=0mm, top=20mm,bottom=20mm}





\begin{document}
\begin{center}
{\large Aina Mayumi, Gen Kimura, Hiromichi Ohno, and Dariusz Chruscinski. Uncertainty relations based on state-dependent norm of commutator
 }

\end{center}
\medskip

\centerline{Referee report}

\bigskip

In the paper, two uncertainty relations are proposed, based on two generalizations of the
B\"ottcher-Wenzel (BW) inequality. The BW inequality holds for the Frobenius norm of the
commutator of any pair of operators and the generalization deals with a state-dependent
version of this norm.  The new uncertainty relations  are compared to the known
uncertainty relations (the Robertson, Schr\"odinger, and the  Luo-Park uncertainty
relations), averaging over all pairs of observables on a qubit system, and over all pairs
of mutually unbiased observables on a qudit system. It is proved that the two new
relations outperform the both Robertson and Schr\"odinger bounds. The Luo-Park bound
is better than both new bounds in the qubit case, but the second of the new bounds
outperforms the Luo-Park bound in the case of mutually unbiased observables. 

The first of the generalized BW inequalities was proved in a previous paper of the
authors' [62], where the second inequality is conjectured. This second inequality is shown
to be tight, but there is no proof that it actually holds, the authors claim ''numerical
evidence'' for the inequality. 

These inequalities and the corresponding uncertainty relations, especially the conjectured
one, may well be worth further investigation. The present paper does contain some results that
would support such a study.  However,  I do not think that this paper brings enough new information. In
my view, it can be seen as some example extending or supporting a part of the previous
paper [62], which is not enough for a separate publication.  Also the computation methods are quite standard and not particularly
insightful. 


\medskip

\noindent
\textbf{Some further comments}

\begin{enumerate}
\item p. 2: ''Observing that the bounds in (7) and (8) are composed of
the commutator between observables, thus, our relations
are detecting a trade-off of non-commutative observables
that has been unidentified by conventional uncertainty
relations.'' I am not sure that I understand this remark, since all the uncertainty
relations are based on the commutator between observables, so this is no special feature
of the new bounds.

\item p. 3, paragraph 2:''...(not necessarily normalized) density matrix'' - this seems strange. Such a matrix is simply a
psd matrix. It might be better to work with a density matrix, and then remark that
(obviously) everything  holds also for any psd matrix. 

\item p.3, column 1, paragraph 3: descending -> ascending

\item p. 4, paragraph under Eq. (19): $\epsilon_{ijk}$ is not defined

\item Appendix B, the title - proof of *generalized* BW inequality 

\end{enumerate}


\end{document}

