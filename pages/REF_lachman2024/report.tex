\documentclass[12pt]{article}
\usepackage{geometry}
\usepackage{amsfonts}
\geometry{total={210mm,290mm},
 left=23mm,right=23mm,%
 bindingoffset=0mm, top=20mm,bottom=20mm}





\begin{document}
\begin{center}
{\large D. Lachman:  Tensor products in the category of effect algebras}

\end{center}
\medskip

\centerline{Referee report}

\bigskip

The paper elucidates some aspects of the tensor product in the category of effect algebras
and its relation to the tensor product in the category of partially ordered abelian group
with order unit. It is shown that the construction of the universal group preserves the
tensor product and the corresponding functor is strong monoidal. This is used to show that
the tensor product of effect algebras does not preserve RDP, which was a question
discussed and left open in previous works. Moreover, the colimits preserved by the tensor
products are characterized.

The results of the paper give a significant insight into the structure of the tensor
product in effect algebras, which is generally found rather complicated. The proofs are
based on several clever ideas by the author, which are very interesting in themselves an likely
to inspire further research on related subjects. The paper is generally well organized and fun to read, 
the ideas of the proofs are clear and  easy to follow. 

The only criticism I have pertains
some language issues and the relatively high number of typos, especially in the
introductory parts. (This becomes much better in the main part of the paper.) These issues
do not decrease readability of the paper very much, but they are annoying and, in my
opinion, degrade an otherwise excellent paper. I give a rather long list below, but 
perhaps the author should use some means of a language check. The list also gives a
few suggestions where some parts might be a bit expanded to increase accessibility. 

\bigskip

\noindent
\textbf{Comments, suggestions and typos}

\begin{itemize}
\item page 1, last line: operator -> operators. Also, the sentence starting with ''Whereas...'' would be better formulated as ''Whereas the spectrum of a projection is a subset of $\{0,1\}$, the spectrum of an effect
is a subset of the real interval $[0,1]$.''

\item page 2, line 4: measurement -> measurements

\item  page 2, line 8: ''yield attention'' perhaps replace by ''gained attention''

\item page 2, line 9: Frobenious -> Frobenius

\item page 2, line 15: ''...concept of tensor product...'' -> ''concept of a tensor product''

\item page 2, line 16: smalles -> smallest. 

\item page 2, end of second paragraph: ''establishes the whole existence...'' maybe replace
''whole'' by ''mere'', or just remove it

\item page 2, line 26: ''best sound'' -> perhaps ''most sound'', or ''soundest''

\item page 2, last paragraph: ''the tensor product of effect algebras does not preserve
ordinal sums'' This may be some folklore, but it would be better to give a reference,
or sketch a counterexample.

\item page 2, line 4 from below: ''not closed monoidal'' -> ''not a closed monoidal''

\item page 3, line 8: sentence starting with ''So far,...'' perhaps better rewrite as
''So far, thanks to various researchers, we have...'' (or maybe skip the ''thanks...'').

\item  page 3, first sentence of the last paragraph: whenever -> whether

\item page 4, last sentence of Sec. 1: Something seems to be missing, better reformulate the
sentence.

\item page 4, Definition 2.1: Where -> Here

\item page 4, line 8 from below: omomorphism -> homomorphism. Also algebra -> algebras

\item page 5, Lemma 2.3: $\ominus$ is not defined

\item page 5, Definition 2.5: ''its tensor product'' -> ''their tensor product''. Also 
''satisfying following'' -> ''satisfying the following''

\item page 5, Theorem 2.6: ''yields tensor product'' this does not seem to be a good word
here. Perhaps ''has a tensor product'', or better ''...the tensor product exists for each
pair...''

\item page 5, second line below Theorem 2.6: ''as a tensor product'' -> ''as the tensor
product''. Also the next sentence: ''...the tensor product behaves functorially'' (or ''is
functorial'')

\item page 5, line below Eq. (2.2): ''implies...'' -> ''implies that...''. Also next sentence:
''for fixed'' -> ''for a fixed''

\item page 6, Definition 2.7: add ''consisting'' between the tuplet and ''of a category
$\mathcal C$...''

\item page 6, line 7 from below: po-groups -> po-group

\item page 6, line 5 from below: $x\le n\cdot u$

\item page 7, second diagram of (2.6): $A\otimes_{\mathcal C} F(I)$ -> $F(A\otimes_{\mathcal
C} I)$

\item page 7, Lemma 2.13: ''A category $\mathcal C$...'' -> ''The category $\mathcal E$...''

\item page 10, line 16 of the proof of Theorem 3.4: $d\downarrow \mathcal D^\bullet$ ->
$d\downarrow \mathcal D$

\item page 10, line above the last displayed equation: Those -> Thus

\item page 11, beginning of Section 4: ''well known adjunction...'' the construction is
indeed
well known, but it would be perhaps better
either give a reference where it is formulated as an adjunction, or elaborate on this a
bit more

\item page 11, line above Lemma 4.1: ''is an epimorphism''

\item page 12, Definition 4.5: ''its tensor product'' -> their

\item page 13, proof of Corollary 4.8: ''very same'' -> ''the same'' is appropriate here

\item page 13, Eq. (4.3) in the first brackets $[,]$, replace $u$ by $v$

\item page 14, line below Eq. (4.4): it is better to write ''As $\Gamma(g)$ is just a
restriction of $g$...''

\item page 14, line 5 in Section 5: $Gr(E)$ -> $Gr(F)$

\item page 16, line line 6 above Proposition 6.1: ''the elements of form...'' -> ''elements of
the form...''

\item page 17: ''To achieve the naturality...''-> ''To prove naturality...'' would be
better

\item page 18, paragraph above Theorem 6.2: ''we subtle'' -> perhaps ''we refine''(?)

\item page 19, Theorem 6.5: ''Where..''-> ''Here...''
\end{itemize}







\end{document}

