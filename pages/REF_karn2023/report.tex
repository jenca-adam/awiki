\documentclass[12pt]{article}
\usepackage{geometry}
\usepackage{amsfonts}
\geometry{total={210mm,290mm},
 left=23mm,right=23mm,%
 bindingoffset=0mm, top=20mm,bottom=20mm}





\begin{document}
\begin{center}
{\large A. K. Karn,  On the geometry of an order unit space
 }

\end{center}
\medskip

\centerline{Referee report}

\bigskip

The aim of the paper is to give a geometric characterization of order unit spaces, similar
to the characterization of a base of the base-normed space as a radially compact convex
set. Here, a subset of a real vector space, called a skeleton with a head, is described by
its geometric properties, such that it generates an order unit space $(V,V^+, e)$, in which it
corresponds precisely to the set of elements  $\{u\in V, \|u\|=\|e-u\|=1\}\cup\{0,e\}$. 
The author also discusses some conditions under which an order unit space contains a copy
of $\ell_\infty^n$. 

\medskip
\noindent
\textbf{Overall evaluation}
\medskip

The results of the paper seem rather interesting to me (though I am not entirely sure about their
novelty). Recently, there is an increased interest  in  order unit spaces and their
geometric structure coming from their relation to physics, since  they serve as a basis for a description of general
probabilistic, or convex operational, theories. The description obtained in the present
paper might prove useful also in that context. However, in the present form, the paper is not written well
enough to be published. 

\medskip

First of all, the paper introduces a lot of  terminology: ''peripheral element'',
''skeleton with a head'', ''periphery'', ''canopy'' and its ''summit'', ''lead point'',
''periphery'', ''semi-peripheral''. This does not seem standard to me, and I do not think
that all of these notions are necessary or useful. There are also unclear formulations,
not properly defined or inconsistent notations, obscure or even plainly wrong arguments in some of the proofs
(though the main results seem to be correct). I also found a relatively large number of
typos. Some of these problems are listed in the comments below.





\medskip

\noindent
\textbf{Specific comments}


\begin{enumerate}

\item absolutely $\infty$-orthogonal - does not seem to be used anywhere 

\item p. 5, last line of the proof of Prop. 2.3: ''$K(e-u)=K(u)$'' should be
$K(e-u)=e-K(u)$ 

\item p. 6, line 5: why is $\kappa\le \frac12$? 

\item p. 6: What is the meaning of ''$K$ has a representation in $C$''?

\item p. 7, line 6: ''Thus by Lemma 2.1 (3)...'' how does the inequality follow from that
statement?

\item p. 8, line 9 from below: ''...by the definition of a canopy...''  
it is not clear how $\alpha,\beta\le 1$ follows from the definition given here.
\item p. 9: ''Thus by Proposition 2.7...'' it is not clear how the statement follows from
Prop. 2.7 
\item p. 9, line 2 from below: $\alpha=1=\beta$ also follows from Lemma 2.1 directly.
\item p. 10: statement (1)(b) is immediate from the definition of the order unit norm
\item p. 11, Lemma 3.3: better remind the definition of $(S_V)_0$
\item p. 11, Lemma 3.3 (3): ''$u$ has an $\infty$-orthogonal pair'': A pair always consists of two
things. Better write: there is an element in $C_V$ which is $\infty$-orthogonal to $u$
\item p. 11. line 4 from below: better repeat the reference to Ref. [11]
\item p. 12: is it clear that $S_V$ generates $(V,V^+,e)$?
\item Sec. 4: ''$[0,e]=\{v\in V^+: \|v\|\le 1\}$ this notation interferes with previously
used notation of $[u,v]$ as a line segment
\item From Sec. 4 onward, the notation $R_V$ is used instead of $(S_V)_0$. This should be
unified.
\item p. 13, proof of Thm. 4.1 does not work: a weak*-compact subset is not necessarily sequentially
compact. The dual unit ball of $\ell_\infty$, with the sequence of states
$f_n(x_1,x_2,\dots)=x_n$, is a notorious counterexample. 
\item Corollary 4.8: This is quite obvious.
\item Remark 5.2: Since $\chi$ is a unital order isomorphism, it is obvious that it must
be an isometry.
\item Theorem 5.3: what is the meaning of '' $\perp_\infty$ is additive''?
\item Prop. 5.5: This is the same as Coro. 2.4.
\item Remark 5.6: I do not understand this remark at all.
\end{enumerate}

\noindent
\textbf{Typos:}

\begin{enumerate}
\item p.1: ''in stead'' instead
\item p. 2: ''Kadison prove'' proved
\item p. 3: ''role model for a non-commutative ordered spaces'' either skip ''a'' or
''spaces''-> space. Also better skip ''role''.
\item p. 6, line 8 from below: ''$1-\alpha\gamma\ge 1$'' should be probably $\ge 0$


\item p. 7: ''and and''
\item p.7, line 4 from below: $e$ is missing in the equality after ''Now''
\item p. 8, last displayed equation: ''$K(w)$'' should be $K(e-w)$ (?)
\item p. 9:  ''$\alpha_n+\beta$'' -> $\alpha_n+\beta_n$, 
''$u_1:=u$'' -> $u:=u_1$
\item p. 9, line 3 from below: $R$ should probably be $S$ (?)
\item p. 13: ''Then Then''
\item p.14, Lemma 4.6: $V_!$, also ''$e_2$'' should be $e_1$ at several places in the
lemma and its proof.
\item p. 17, last line of the second displayed equation: should be $\sum_i \alpha_i u_i$.




\end{enumerate}

\noindent
\textbf{Notations used without definition:}

\begin{enumerate}
\item $[u,v]$ or $(u,v)$: line segment connecting $u$ and $v$
\item $S(V)$: set of states of the order unit space
\item $[0,e]_o$ order interval(?)
\end{enumerate}





\end{document}

