\documentclass[12pt]{article}


\usepackage{hyperref}
\usepackage{amsmath, amssymb, amsthm}
\usepackage[sort&compress,numbers]{natbib}
\usepackage{doi}
\usepackage[margin=0.8in]{geometry}
%\textheight23cm \topmargin-20mm  
%\textwidth175mm  
%\oddsidemargin=0mm
%\evensidemargin=0mm
%

\usepackage{amsmath, amssymb, amsthm, mathtools}

\newtheorem{lemma}{Lemma}
\newtheorem{theorem}{Theorem}
\newtheorem{coro}{Corollary}


\theoremstyle{definition}
\newtheorem{defi}{Definition}


\theoremstyle{remark}
\newtheorem{remark}{Remark}

\def\Ha{\mathcal H}
\def\Me{\mathcal M}
\def\Ee{\mathcal E}
\def\Ne{\mathcal N}
\def\Te{\mathcal T}
\def\Ka{\mathcal K}
\def \Tr{\mathrm{Tr}\,}
\def\Se {\mathcal S}
\def\supp{\mathrm{supp}}
\def\spec{\mathrm{spec}}
\def\<{\langle\,}
\def\>{\,\rangle}

\title{Notes on asymptotics of quantum hypothesis testing}
\author{Anna Jen\v cov\'a\footnote{Mathematical Institute, Slovak Academy of Sciences, \v
Stef\'anikova 49, 814 73 Bratislava, Slovakia, jenca@mat.savba.sk}}
\date{}
\begin{document}

\maketitle


\section{Preliminaries}

Let $\Ha$ be a finite dimensional Hilbert space.

\subsection{Pinching}

Let $A\in B(\Ha)$ be self-djoint, with spectral decomposition  $A=\sum_i \lambda_i P_i$.
We will need the pinching map $B(\Ha)\to B(\Ha)$, defined as 
\[
\Ee_A(X)=\sum_i P_iXP_i.
\]
Then $A$ is a cp unital map. Moreover, $\Ee_A(X)$ commutes with $X$ and we have the
pinching inequality \cite{hayashi...optimal}
\begin{equation}\label{eq:pinching}
\Ee_A(X)\le |\spec(A)|X,\qquad X\ge 0.
\end{equation}


\subsection{Relative entropies}

Let $\rho$ and $\sigma$ be density operators. 
The (Umegaki) relative entropy is defined as 
\[
D(\rho\|\sigma):=\begin{dcases} \Tr[\rho(\log \rho-\log\sigma)], &
\supp(\rho)\le\supp(\sigma)\\
\infty, & \mathrm{otherwise}.
\end{dcases}
\]
The standard R\'enyi relative entropy for $\alpha\in [0,1]\setminus\{1\}$ is defined as
\[
D_\alpha(\rho\|\sigma):=\begin{dcases}
\frac1{\alpha-1}\log\Tr[\rho^\alpha\sigma^{1-\alpha}],& \supp(\rho)\le \supp(\sigma)
\text{ or } \alpha\in (0,1)\\
\infty,& \text{otherwise}.
\end{dcases}
\]
The sandwiched R\'enyi relative entropy for $\alpha\in [1/2,\infty]\setminus\{1\}$ is
defined as
\[
\hat D_\alpha(\rho\|\sigma):=\begin{dcases}
\frac1{\alpha-1}\log\Tr[\sigma^{\frac{1-\alpha}{2\alpha}}\rho\sigma^{\frac{1-\alpha}{2\alpha}}],& \supp(\rho)\le \supp(\sigma)
\text{ or } \alpha\in [1/2,1)\\
\infty,& \text{otherwise}.
\end{dcases}
\]

\subsection{The function $\phi$}

\textbf{We assume $\supp(\rho)\le \supp(\sigma)$, so that $D(\rho\|\sigma)<\infty$.}

Let us define 
\[
\phi(s)=\log \Tr[\rho^{1-s}\sigma^s],\qquad s\in \mathbb R.
\]
Then $\phi$ is a convex and smooth function, with derivative
\[
\phi'(s)=(\Tr[\rho^{1-s}\sigma^s])^{-1}\Tr[\rho^{1-s}\sigma^s(\log\sigma-\log\rho),
\]
\cite[Exercise 3.5]{hayashi2017quantum}
In particular,  $\phi'(0)=-D(\rho\|\sigma)$ and $\phi'(1)=D(\sigma\|\rho)$.

\begin{lemma}\label{lemma:phipsi}
Put
\[
\psi(\lambda)=\inf_{s\in[0,1]} \lambda s+\phi(s),\qquad \lambda\in \mathbb R.
\]
Then 
\[
\psi(\lambda)=\begin{dcases} 0 & \lambda\ge -\phi'(0)=D(\rho\|\sigma)\\
<0 & \lambda <D(\rho\|\sigma)\\
\lambda & \lambda \le -\phi'(1)=-D(\sigma\|\rho).
\end{dcases}
\]

\end{lemma}

\begin{proof} By convexity, the derivative $\phi'(s)$ is nondecreasing.  It follows that
if $\lambda\ge -\phi'(0)$, then 
\[
\frac{d}{ds} (\lambda s+\phi(s))=\lambda+\phi'(s)\ge -\phi'(0)+\phi'(s)\ge 0,
\]
so that the function $s\mapsto \lambda s+\phi(s)$ is nondecreasing, so that the infimum is
attained at $s=0$. Similarly, if $\lambda\le -\phi'(1)$, then the infimum is attained at
$s=1$ and hence $\psi(\lambda)=\lambda$. Let now $\lambda<-\phi'(0)$, then we see that
the function $s\mapsto \lambda s+\phi(s)$ is strictly decreasing at $s=0$, so that we must
have $\psi(\lambda)<0$.

\end{proof}

We define
\[
\psi^*(\lambda)=\inf_{t\in [-1,0]} t\lambda+\phi(t).
\]
Again, if $\lambda> D(\rho\|\sigma)=-\phi'(0)$, then $t\mapsto t\lambda +\phi(t)$ is
strictly increasing at $t=0$, which implies that $\phi^*(\lambda)<0$. 

\subsection{Inequalities}

We have two basic inequalities. For $A,B\ge 0$, let  $\{A\ge B\}$ be the sum of
eigenprojections of $A-B$ corresponding to nonnegative eigenvalues, similarly $\{A\le
B\}$, $\{A>B\}$ etc. Then 
\begin{lemma}[Quantum Neyman-Pearson]\label{lemma:qnp} We have 
\[
\min_{0\le T\le I} \Tr[A(I-T)]+\Tr[BT]=\Tr[A\{A\le B\}]+\Tr[B\{A>B\}].
\]

\end{lemma}

\begin{lemma}[Audenaert et al]\label{lemma:audenaert} We have for any $s\in [0,1]$, 
\[
\Tr[A\{A\le B\}]+\Tr[B\{A>B\}]\le \Tr[A^{1-s}B^s].
\]

\end{lemma}

These statements hold in the von Neumann algebra case as well.

\section{QHT}


Let $\rho,\sigma$ be a pair of density matrices. We test the hypothesis $H_0=\rho$ against
the alternative $H_1=\sigma$. A test is given by an operator $0\le T\le I$, corresponding
to accepring $H_0$. The two error probabilities are
\[
\alpha(T)=\Tr[(I-T)\rho],\qquad \beta(T)=\Tr[T\sigma].
\]
We will consider the asymptotic behaviour of the error probabilities 
\[
\alpha_n(T_n)=\Tr[(I-T_n)\rho_n],\qquad \beta_n(T_n)=\Tr[T_n\sigma_n]
\]
in testing $H_0=\rho_n:=\rho^{\otimes n}$ against $H_1=\sigma_n:=\sigma^{\otimes n}$. 

\subsection{Quantum Stein's lemma}


\textbf{We assume $\supp(\rho)\le \supp(\sigma)$, so that $D(\rho\|\sigma)<\infty$.}

\bigskip

Let $\lambda\in \mathbb R$ and let $S_n:=\{\rho^{\otimes n}>e^{n\lambda}\sigma^{\otimes
n}\}$. Then using Lemma \ref{lemma:audenaert} (Audenaert) with $A=\rho^{\otimes n}$ and
$B=e^{\lambda n}\sigma^{\otimes n}$, we get  for any $s\in [0,1]$
\begin{align}
\alpha_n(S_n)+e^{n\lambda}\beta_n(S_n)\le \Tr e^{n\lambda s}[(\rho^{\otimes n})^{1-s}(\sigma^{\otimes
n})^s]=e^{n\lambda s}(\Tr[\rho^{1-s}\sigma^s])^n=e^{n(\lambda s+\phi(s))}.
\end{align}
Hence by taking the infimum over $s\in [0,1]$,
\begin{align}
\alpha_n(S_n)&\le e^{n\psi(\lambda)},\qquad\beta_n(S_n)\le e^{n(-\lambda
+\psi(\lambda))}\label{eq:direct}
\end{align}
On the other hand, put $p_n=\Tr[\rho^{\otimes n}S_n]$ and $q_n=\Tr[\sigma^{\otimes
n}S_n]$. Then $p_n \ge e^{n\lambda}q_n$ and therefore $p_n^t\le e^{n\lambda t}q_n^t$ for 
for any $t\in [-1,0]$. We get
\begin{align*}
1-\alpha_n(S_n)&=p_n\le e^{n\lambda t}p_n^{1-t}q_n^{t}\le e^{n\lambda
t}(p_n^{1-t}q_n^{t}+(1-p_n)^{1-t}(1-q_n)^{t})\le e^{n\lambda t}\Tr[(\rho^{\otimes
n})^{1-t}(\sigma^{\otimes n})^t]\\
&=e^{n(\lambda t+\phi(t))}
\end{align*}
for all $t\in [-1,0]$. It follows that for any test $T_n$, we have
\begin{align*}
1-\alpha_n(T_n)&=\Tr[\rho^{\otimes n} T_n]=\Tr[(\rho^{\otimes n}-e^{\lambda
n}\sigma^{\otimes n})T_n]+e^{\lambda n}\beta_n(T_n)\le \Tr[(\rho^{\otimes n}-e^{\lambda
n}\sigma^{\otimes n})S_n]+e^{\lambda n}\beta_n(T_n)\\
&\le 1-\alpha_n(S_n)+e^{\lambda n}\beta_n(T_n)\le e^{n(\lambda t+\phi(t))}+e^{\lambda n}\beta_n(T_n)
\end{align*}
and hence
\begin{equation}\label{eq:converse}
\beta_n(T_n)\ge e^{-n\lambda}(1-\alpha_n(T_n)-e^{n\psi^*(\lambda)})
\end{equation}


\begin{lemma}[Quantum Stein's lemma]\label{lemma:stein}\cite{hiai1991theproper,ogawa2000strong} For all $\epsilon\in (0,1)$, we have
\[
\lim_{n\to\infty} -\frac1n \log \beta_n(\epsilon)= D(\rho\|\sigma).
\]
\end{lemma}
\begin{proof} Let $\lambda<D(\rho\|\sigma)$, then by Lemma \ref{lemma:phipsi},
$\psi(\lambda)<0$, so that in this case \eqref{eq:direct}, $\alpha_n(S_n)\to 0$ and 
\[
-\frac1n\log \beta_n(S_n)\ge \lambda -\psi(\lambda)>\lambda.
\]
For $\epsilon\in (0,1)$ we have $\alpha_n(S_n)\le \epsilon$ for large enough $n$, so that 
$\beta_n(\epsilon)\le \beta_n(S_n)$. It follows that
\[
\liminf_n -\frac1n\log\beta_n(\epsilon)\ge -\frac1n\log \beta_n(S_n)\ge \lambda.
\]
Conversely, by \eqref{eq:converse} we have for any sequence of
tests such that $\alpha_n(T_n)\le \epsilon$ that
\[
\beta_n(T_n)\ge e^{-n\lambda}(1-\epsilon-e^{n\psi^*(\lambda)}).
\]
Since $\psi^*(\lambda)<0$ for  $\lambda>D(\rho\|\sigma)$, this implies that, for such
$\lambda$,
\[
\limsup_n -\frac1n\beta_n(\epsilon)\le \lambda.
\]
Chooseng any $\delta>0$, we obtain
\[
D(\rho\|\sigma)-\delta\le \liminf_n-\frac1n\beta_n(\epsilon)\le
\limsup_n-\frac1n\beta_n(\epsilon)\le D(\rho\|\sigma)+\delta.
\]
Since $\delta$ was arbitrary, this implies the statement.

\end{proof}


The following quantum Stein's
lemma describes the situation when the error of the first kind is constrained by some
$\epsilon>0$. In this case, the optimal value of the second kind error
 \[
\beta_n(\epsilon):=\min_{0\le T_n\le I}\{\beta_n(T_n)\ | \ \alpha_n(T_n)\le
\epsilon\},\qquad \epsilon>0
\]
converges to zero exponetially fast, that is, $\beta_n(\epsilon)\sim e^{-nr}$. The lemma
shows that for any $\epsilon>0$, the error exponent $r$ is given by the relative entropy










\end{document}

Let $\Me$ be a von Neumann algebra and let $\rho,\sigma$ be a pair of normal states. 
We consider the problem of testing the
hypothesis $H_0=\rho$ againts alternative $H_1=\sigma$. Any test is represented by an
operator $M\in \Me$, $0\le M\le I$, such that the value $\omega(M)$ is interpreted as the
probability of accepting the hypothesis $H_0$ in the state $\omega$.

The two error probabilities related to a test $M$ are given as
\[
\alpha(M)=\rho(1-M),\qquad \beta(M)=\sigma(M).
\]
In the Bayes approach, we choose a prior probability $\pi\in (0,1)$ and minimize the
average error probability over all tests, obtaining the optimal value
\begin{align*}
b_\pi(\rho\|\sigma)&:=\inf_{M} \pi\alpha(M)+(1-\pi)\beta(M)\\
&=\inf_M\pi\rho(1-M)+(1-\pi)\sigma(M)\\
&= \pi -\sup_M (\pi\rho-(1-\pi)\sigma)(M)=\pi(1-(\rho-s\sigma)_+(1))\\
&= \frac12(1-\pi\|\rho-s\sigma\|_1),\qquad
s=\frac{1-\pi}{\pi}.
\end{align*}
The optimal test for $\pi\in (0,1)$ is the quantum Neyman-Pearson test $M_\pi$, which is of the form
\[
M_\pi=\{\rho>s\sigma\}+ X,\qquad 0\le X\le \{\rho=s\sigma\},\qquad s=\frac{1-\pi}{\pi},
\]
here for $\varphi,\psi\in \Me^+_*$ we define $\{\varphi>\psi\}$ as the projection ionto
the support of $(\varphi-\psi)_+$, similarly $\{\varphi<\psi\}$ is the projectio onto the
support of $(\psi-\varphi)_+=(\varphi-\psi)_-$. We then have
$\{\varphi\le\psi\}=1-\{\varphi>\psi\}$, $\{\varphi\ge \psi\}=1-\{\varphi<\psi\}$ and 
$\{\varphi=\psi\}=1-\{\varphi<\psi\}-\{\varphi>\psi\}$.


It is clear from
these expressions that the function $\pi\mapsto b_\pi(\rho\|\sigma)$ is continuous.  
Furthermore, let us look at the error probabilities for the QNP test $\{\rho>s\sigma\}$. 
Put $\alpha(s):=\rho(\{\rho\le s\sigma\})$ and $\beta(s):=\sigma(\{\rho>s\sigma\})$. Then 
By the QNP lemma, we have for any $s,s'\in (0,\infty)$ that 
\begin{equation}\label{eq:qnp}
\alpha(s)+s\beta(s)\le \alpha(s')+s\beta(s')
\end{equation}
hence 
\[
s(\beta(s)-\beta(s'))\le \alpha(s')-\alpha(s)\le s'(\beta(s)-\beta(s')),
\]
the second inequality is obtained by exchanging $s$ and $s'$ in \eqref{eq:qnp}. It follows
that if $s'>s$, then $\beta(s)\ge \beta(s')$ and therefore also $\alpha(s)\le \alpha(s')$. 



In the asymmetric QHT approach, we fix the maximum acceptable value $\epsilon$ of the error
$\alpha(M)\le \epsilon$ and minimize the error $\beta(M)$:
\[
d_H^\epsilon(\rho\|\sigma):=\inf\{\sigma(M),\ 0\le M\le I,\ \rho(1-M)\le
\epsilon\}.
\]
The following result is proved similarly as in the classical case (cf. \cite{liese...}).

\begin{lemma}\label{lemma:btodh} We have
\[
b_\pi(\rho\|\sigma)=\inf_{0<\epsilon<1}[
\pi\epsilon+(1-\pi)d_H^\epsilon(\rho\|\sigma)],\qquad \pi\in (0,1)
\]


\end{lemma}

\begin{proof}
Let $\pi,\epsilon \in (0,1)$ and let $M$ be any test such that $\rho(1-M)\le
\epsilon$, then we have
\[
b_\pi(\rho\|\sigma)\le \pi\rho(1-M)+(1-\pi)\sigma(M)\le
\pi\epsilon+(1-\pi)\sigma(M).
\]
Taking infimum over all such tests, we obtain
\begin{equation}\label{eq:bled}
b_\pi(\rho\|\sigma)\le \pi\epsilon +(1-\pi)d_H^\epsilon(\rho\|\sigma),\qquad \forall
\epsilon,\pi\in (0,1).
\end{equation}
For a fixed $\pi\in (0,1)$ and a Neyman-Pearson test $M_\pi$, let
$\epsilon_\pi=\rho(1-M_\pi)$, then
\[
b_\pi(\rho\|\sigma)=\pi\epsilon_\pi+(1-\pi)\sigma(M_\pi)\ge
\pi\epsilon_\pi+(1-\pi)d_H^{\epsilon_\pi}(\rho\|\sigma),
\]
this proves equality. 

\end{proof}

\subsection{In finite dimensions...}


\begin{lemma}\label{lemma:NPepsilon} Let $s(\rho)\le s(\sigma)$. Then for any $\epsilon\in
(0,1)$ there is some $\pi_\epsilon\in (0,1)$ and a Neyman-Pearson test
$M_\epsilon=M_{\pi_\epsilon}$ such that $\rho(1- M_\epsilon)=\epsilon$.

\end{lemma}

\begin{proof} Let 
\[
s_\epsilon:=\sup\{s\ge0, \Tr[\rho P_{s,-}]< \epsilon\}.
\]
Under the assumption, there is some
$\lambda>0$ such that $\rho\le \lambda \sigma$, and  then $P_{s,-}=I$ for all $s\ge
\lambda$. It follows that we must have $c_\epsilon<\lambda$, in particular, $c_\epsilon$
is finite. Further, by  \cite[Lemma ]{jencova2012reversibility}, we have for $s\in [0,\infty)$:
\[
\lim_{t\to s^-} P_{t,-}=P_{s,-},\qquad \lim_{t\to s+}P_{t,-}=P_{s,-}+P_{s,0}
\]
and $P_{s,0}\ne 0$ for a finite number of values of $s$. If follows that
\[
\Tr[\rho
P_{c_\epsilon,-}]=\lim_{t\to c_{\epsilon-}}\Tr[\rho P_{t,-}]\le \epsilon \le \lim_{t\to c_\epsilon+}\Tr[\rho
P_{t,-}]=\Tr[\rho(P_{c_\epsilon,-}+P_{c_\epsilon,0})]
\]
and 
\[
M_\epsilon=P_{c_\epsilon,-}+\frac{\epsilon- \Tr[\rho P_{c_\epsilon,-}]}{\Tr[\rho
P_{c_\epsilon,0}]}P_{c_\epsilon,0}
\]
is a Neyman-Pearson test for $\pi_\epsilon:=(c_\epsilon+1)^{-1}$ such that $\Tr[\rho
M_\epsilon]=\epsilon$.



\end{proof}


\begin{lemma}\label{lemma:btodh_fd} 
If $s(\rho)\le s(\sigma)$ then also
\[
d_H^\epsilon(\rho\|\sigma)=\sup_{0<\pi<1}\frac{1}{1-\pi}
[b_\pi(\rho\|\sigma)-\pi\epsilon],\qquad \epsilon\in (0,1).
\]


\end{lemma}

\begin{proof}
We see that \eqref{eq:bled}
implies 
\[
d_H^\epsilon(\rho\|\sigma)\ge \frac{1}{1-\pi}[b_\pi(\rho\|\sigma)-\pi\epsilon]
\]
for all $\epsilon$ and $\pi$. If $s(\rho)\le s(\sigma)$, then the second equality follows
by Lemma \ref{lemma:btodh}.


\end{proof}


\section{Measured R\'enyi relative entropy}

Let $\Me$ be a von Neumann algebra. A measurement on $\Me$ (with values in a finite set
$\Omega$) is defined as a positive linear map $M:L_\infty(\Omega)\to \Me$ such that
$M(1)=1$. Equivalently, a measurement is given by a set of positive operators
$\{a_\omega\}_{\omega\in \Omega}$ such that $\sum_\omega a_\omega=1$. More generally, a
measurement can be defined as a unital positive map from a commutative von Neumann algebra
to $\Me$. 

Let $\rho,\sigma$ be normal states on $\Me$. The measured R\'enyi relative entropy is defined as
\begin{align*}
D_\alpha^M(\rho\|\sigma)=\sup\{D_\alpha(\rho\circ M\|\sigma\circ M),\ M \text{ is a
measurement}\}.
\end{align*}
By \cite[Thm. 5.2]{hiai2021quantum}, we may restrict to finite valued measurements. 

It is easily seen that $D_\alpha^M$ is monotone under positive unital normal maps. Indeed, let
$\Phi:\Ne\to \Me$ be such a map and let $M:L_\infty(\Omega)\to \Ne$ be a measurement on
$\Ne$, then $\Phi\circ M$ is a measurement on $\Me$ and we have
\begin{align*}
D^M_\alpha(\rho\circ\Phi\|\sigma\circ\Phi)&=\sup\{D_\alpha(\rho\circ \Phi\circ
M\|\sigma\circ\Phi\circ M),\ M \text{ is a measurement on } \Ne\}\\
&\le
\sup\{D_\alpha(\rho\circ M\|\sigma\circ M),\ M \text{ is a measurement on
}\Me\}\\
&=D^M_\alpha(\rho\|\sigma).
\end{align*}

Further, by monotonicity of the sandwiched R\'enyi relative entropy $\tilde D_\alpha$ for
$\alpha\in [1/2,1)\cup(1,\infty]$, and by
additivity with respect to tensor products, we have
\[
\frac 1n D_\alpha^M(\rho^{\otimes n}\|\sigma^{\otimes n})\le \frac 1n \tilde
D_\alpha(\rho^{\otimes n}\|\sigma^{\otimes n})= \tilde D_\alpha(\rho\|\sigma).
\]
We are interested in the  equality 
\begin{equation}\label{eq:regmeasured}
\lim_n\frac1nD^M_\alpha(\rho^{\otimes n}\|\sigma^{\otimes n})=\tilde D_\alpha(\rho\|\sigma),\qquad \alpha>1.
\end{equation}
This equality was proved in \cite{mosonyi..} in the finite dimensional case,  in 
\cite{Hiai...} in the approximately finite (injective) case and in \cite{fawzi} it was
extended to semifinite von Neumann algebras. We will use the Haagerup reduction to show
that this equality holds for all von Neumann algebras.

\subsection{Haagerup reduction}

We just give an outline: 
\begin{itemize}
\item There exists a von Neumann algebra $\hat \Me$ (a crossed product) such that we may
identify $\Me$ as a subalgebra in $\hat \Me$ and there is a canonical normal conditional
expectation $E_\Me$ of $\hat \Me$ onto $\Me$,
\item There is an increasing family of subalgebras $\Me_n\subseteq \hat \Me$ such that
\begin{itemize}
\item Each $\Me_n$ is finite (equipped with a faithful normal tracial state $\tau_n$),
\item $\bigcup_{n\ge 1} \Me_n$ is weak*-dense in $\hat\Me$
\item For each $n$, there exists a faithful normal conditional expectation $E_n$ of $\hat
\Me$ onto $\Me_n$ and for all $x\in \hat\Me$, $E_n(x)\to x$ in the $\sigma$-strong
topology.

\end{itemize}
\item Any normal state $\rho$ on $\Me$ extends to a normal state on $\hat \Me$ as
$\hat\rho:=\rho\circ E_\Me$. Put $\rho_n:=\hat\rho|_{\Me_n}$, then $\rho_n$ is a normal
state on $\Me_n$. Using the conditional expectation $E_n$, $\rho_n$ extends to a state
$\hat \rho_n=\rho_n\circ E_n=\hat\rho\circ E_n$. We have $\|\hat\rho_n-\hat\rho\|_1\to 0$.


\end{itemize}

We will use this result to prove \eqref{eq:regmeasured} for any von Neumann algebra $\Me$. 
First, note that since $\rho=\hat\rho|_\Me$ and $\hat\rho=\rho\circ E_\Me$, we have by DPI
that 
\[
\tilde D_\alpha(\rho\|\sigma)=\tilde D_\alpha(\hat\rho\|\hat\sigma),\qquad
D^M_\alpha(\rho^{\otimes n}\|\sigma^{\otimes n})=D^M_\alpha(\hat\rho^{\otimes
n}\|\hat\sigma^{\otimes n}).
\]
Hence it is enough to work with $\hat\rho$ and $\hat\sigma$. Notice that by th DPI for
$D^M_\alpha$, we have for any $m$ and $\alpha>1$ 
\[
\lim_n \frac 1n D^M_\alpha(\hat \rho^{\otimes n}\|\sigma^{\otimes n})\ge
\lim_n\frac1nD_\alpha^M(\hat\rho_m^{\otimes n}\|\hat\sigma_m^{\otimes n})=\tilde
D_\alpha(\hat\rho_m\|\hat\sigma_m),
\]
since $\Me_m$ is finite. We therefore have 
\[
\lim_n \frac 1n D^M_\alpha(\hat \rho^{\otimes n}\|\hat \sigma^{\otimes n})\ge \lim_m \tilde
D_\alpha (\hat\rho_m\|\hat\sigma_m)=\tilde
D_\alpha (\hat\rho\|\hat\sigma).
\]
The last equality holds by DPI and lower semicontinuity of $\tilde D_\alpha$. Since the
opposite inequality always holds, this proves \eqref{eq:regmeasured}.

\subsection{Strong converse exponents} 









\end{document}

