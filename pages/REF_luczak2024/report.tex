\documentclass[12pt]{article}
\usepackage{geometry}
\usepackage{amsfonts}
\geometry{total={210mm,290mm},
 left=23mm,right=23mm,%
 bindingoffset=0mm, top=20mm,bottom=20mm}





\begin{document}
\begin{center}
{\large Andrzej {\L}uczak, Hanna Podsedkowska, and Rafa{\l} Wieczorek: Mappings preserving quantum Renyi’s entropies in von Neumann algebras}

\end{center}
\medskip

\centerline{Referee report}

\bigskip

The aim of this paper is to characterize normal positive unital maps that preserve the
R\'enyi entropy of a state in a semifinite von Neumann algebra. The main result, stated as
the Main Theorem, shows that if such a map is also trace preserving, then it preserves the
R\'enyi entropy of a state $\rho$ if and only if its restriction to the subalgebra
generated by the density of $\rho$ is a *-isomorphism. 

This result is proved for any value of the parameter $\alpha$ in $(0,1)\cup (1,+\infty)$,
the proof is divided into the cases $\alpha\in (0,1)$, $\alpha\in (1,2]$, and $\alpha>2$
where an additional assumption is required. The proof in the first case is  based on
integral repsresentation of the function $t\mapsto t^\alpha$ and  the previous result of
some of the authors [2], where a similar statement is proved for the Segal entropy.
The other cases are reduced to this case by extensions of the known Jensen
operator or trace inequalities for bounded operators  to the case of measurable
operators, this is done  by standard approximation techniques. 



\medskip

\noindent
\textbf{Overall evaluation}

\medskip

The results of the paper are not particluarly surprising. It is basically an extension of
the known results on the equality in Jensen's inequality from the case of bounded
operators [4]. The proofs are not very sofisticated and it seems that they could be made
more effective by using known properties of the measurable operators. For example,
Proposition 1 follows easily from operator monotonicity of the function $t\mapsto
t^\alpha$ for $\alpha\in (0,1)$, the properties of the singular numbers of measurable
operators (Def. 2.1 in [T. Fack and H. Kosaki, Generalized $s$-numbers of
$\tau$-measurable operators, Pacific J. Math. 123(2): 269-300 (1986)]) and some
convergence theorems (see Lemmas 2.5, 3.4 and Corollary 2.8 of the Fack-Kosaki paper
above). 

The paper is also not very well written. I would be more inclined towards publication if
it was more self-contained, giving a more precise exposition about the main ingredients of
the paper. For example, a more precise definition of the space $L_1(M,\tau)$ and the
extension of the map $\Phi$ to it could be given  without taking up too much space. Also,
the final argument of the proof in case 1 could be included at least in a concise form,
instead of just referring to [2]. The authors also should give precise formulations and 
references to the (bounded operator case) Jensen operator and trace inequalities they are
using throughout the paper. 

The last paragraph contains the following statement:

''The above reasoning shows that some results which
state e.g. that a map leaving the relative entropy, or the $H$ as in (18),
invariant must be of the form $\Phi(x) = uxu^*$ where $u$ is a unitary or
antiunitary or an isometry are incorrect (a good example of such a
Jordan *-isomorphism which does not change the entropy but is not
of the form as above, is transposition in $B(H)$ with respect to a given
orthonormal basis).''

This is quite puzzling, since by the characterization by Kadison, any Jordan *-isomorphism
of a factor is precisely of the stated form, for a unitary or antiunitary operator $u$. I
particular, transposition coincodes with complex conjugation on positive operators and is
therefore given by an antiunitary transformation. 


\medskip

\noindent
\textbf{Some specific comments}

The proof of Proposition 1 can be simplified using known results on measurable operators
[T. Fack and H. Kosaki, Generalized $s$-numbers of $\tau$-measurable operators, Pacific J.
Math. 123(2): 269-300 (1986)] (all the references in this paragraph are to this paper). 
Indeed, assume that $0\le x_n\nearrow x$ in $L^1(M, \tau)$ and let $0<\alpha<1$. Then by operator monotonicity
of $t\mapsto t^\alpha$ we have $0\le x_n^\alpha\le x^\alpha$.
Further, by Theorem 3.7, $x_n\to x$ in measure topology. Using Lemmas 2.5
(iii) and 3.4 (ii), we get $\mu_t(x_n)\nearrow \mu_t(x)$ for all
$t>0$, where $\mu_t(T)$ is the $t$-th singular number of the measurable operator $T$ (Def.
2.1). Hence $\mu_t(x_n)^\alpha\nearrow \mu_t(x)^\alpha$ for all $0<\alpha<1$, so by the
Lebesgue monotone convergence theorem and Corollary 2.8, 
\[
\tau(x_n^\alpha)= \int \mu_t(x_n)^\alpha dt\nearrow \int \mu_t(x)^\alpha dt = \tau(x^\alpha).
\]






\end{document}

