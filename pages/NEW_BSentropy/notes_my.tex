\documentclass[12pt]{article}

\usepackage{hyperref}
\usepackage{amsmath, amssymb, amsthm}
\usepackage[sort&compress,numbers]{natbib}
\usepackage{doi}
\usepackage[margin=0.8in]{geometry}
%\textheight23cm \topmargin-20mm  
%\textwidth175mm  
%\oddsidemargin=0mm
%\evensidemargin=0mm
%

\usepackage{amsmath, amssymb, amsthm, mathtools}

\newtheorem{lemma}{Lemma}
\newtheorem{theorem}{Theorem}
\newtheorem{coro}{Corollary}
\newtheorem{prop}{Proposition}


\theoremstyle{definition}
\newtheorem{defi}{Definition}


\theoremstyle{remark}
\newtheorem{remark}{Remark}

\def\Be{\mathcal B}
\def\Me{\mathcal M}
\def\Fe{\mathcal F}
\def\Te{\mathcal T}
\def\Rr{\mathcal R}
\def\Ha{\mathcal H}
\def\Ka{\mathcal K}
\def\Ne{\mathcal N}
\def \Tr{\mathrm{Tr}\,}
\def\Se {\mathcal S}
\def\supp{\mathrm{supp}}
\def\<{\langle\.}
\def\>{\.\rangle}

\title{Another note on equality in DPI for the BS relative entropy}
\author{Anna Jen\v cov\'a}

\begin{document}

\maketitle

\section{Equality conditions in QRE and BS-RE}

Let $\Te$ be a channel and let $\rho,\sigma$ be states, $\sigma$ invertible.
According to \cite{hiai2017different,hiai2023equality}, we have the following eqivalen
conditions for equality in DPI.

\begin{center}
\begin{tabular}{l|l}
QRE & BS-RE\\[1em]
\hline\\
$\sigma^{1/2}\Te^*(\Te(\sigma)^{-1/2}\Te(\rho)\Te(\sigma)^{1/2})\sigma^{1/2}=\rho$ &
$\sigma\Te^*(\Te(\sigma)^{-1}\Te(\rho))=\rho$ \\[1em]
\ & $\sigma\Te^*(\Te(\sigma)^{-1}\Te(\rho)^2\Te(\sigma)^{-1})\sigma=\rho^2$\\[1em]
$\Tr \Te(\rho)^{1/2}\Te(\sigma)^{1/2}=\Tr\rho^{1/2}\sigma^{1/2}$ & $\Tr
\Te(\rho)^2\Te(\sigma)^{-1}=\Tr \rho^2\sigma^{-1}$\\[1em]
$\Te^*(\Te(\sigma)^{-1/2}\Te(\rho)\Te(\sigma)^{1/2})=\sigma^{-1/2}\rho\sigma^{-1/2}$ &
$\Te(\rho)\Te(\sigma)^{-1}\Te(\rho)=\rho\sigma^{-1}\rho$\\[1em]
$\sigma^{-1/2}\rho\sigma^{-1}\in \Fe_{(\Te_\sigma\circ\Te)^*}$ &
$\sigma^{-1/2}\rho\sigma^{-1/2}\in \Me_{\Te^*_\sigma}$\\[1em]
$\sigma^{it-1/2}\rho\sigma^{-it-1/2}\in \Me_{\Te^*_\sigma}$, $\forall t\in \mathbb R$ & \ \\[1em]

\hline
\end{tabular}


\end{center}



\begin{prop}\label{prop:BS_QMS} Assume that $\rho_{ABC}$ is such that $\rho_{AB}$ is
invertible. Put $\eta_{AB}=\rho_B^{-1/2}\rho_{AB}\rho_B^{-1}$,
$\eta_{BC}=\rho_B^{-1/2}\rho_{BC}\rho_B^{-1/2}$. The following are equivalent.

\begin{enumerate}
\item[(i)] $\rho_{ABC}$ is a BS-QMC.
\item[(ii)] $\rho_{ABC}=\rho_{AB}\rho_B^{-1}\rho_{BC}$.
\item[(iii)] $\eta_{AB}$ and $\eta_{BC}$ commute, and
$\rho_{ABC}=\rho_B^{1/2}\eta_{AB}\eta_{BC}\rho_{B}^{1/2}$.
\item[(iv)] There is a decomposition and a unitary $U_B:\Ha_B\to \oplus_n
\Ha_{B_L^n}\otimes \Ha_{B_R^N}$ such that 
\[
\rho_{ABC}=\rho_B^{1/2}U^*_B\left(\oplus_n \eta_{AB_{L}^n}\otimes
\eta_{B_R^nC}\right)U_B\rho_B^{1/2}
\]
for some $\eta_{AB_L^n}\in B(\Ha_{AB_L^n})^+$, $\eta_{B_R^nC}\in B(\Ha_{B_R^nC})^+$.

\end{enumerate}
Moreover, a BS-QMC $\rho_{ABC}$ is a QMC if and only if $\rho_B^{it}\eta_{AB}\rho_B^{-it}$
commutes with $\eta_{BC}$ for all $t\in \mathbb R$.




\end{prop}


\begin{proof} The equivalence (i) $\iff$ (ii) was proved in \cite{}. If (ii) holds, then 
clearly $\rho_{ABC}=\rho_B^{1/2}\eta_{AB}\eta_{BC}\rho_B^{1/2}=\rho_{ABC}^*$.
Since $\rho_B$ is invertible, 
$[\eta_{AB},\eta_{BC}]=0$. 

Assume (iii). Then $\eta_{BC}$ commutes with all
elements of the form 
\[
\eta_{AB}^{1/2}X_A\eta_{AB}^{1/2}, \qquad X_A\in B(\Ha_A).
\]
Let $\Gamma(X_{A})=\eta_{AB}^{1/2}X_A\eta_{AB}^{1/2}$, then $\eta_{BC}$ must be in the
commutant of $(\Gamma(B(\Ha_A)))$ in $B(\Ha_{ABC})$, which is equal to
$\Gamma(B(\Ha_A))'\otimes B(\Ha_C)$. Since $\Gamma$ defines a completely positive map
$B(\Ha_A)\to B(\Ha_{AB})$, it follows by the Arveson commutant lifting theorem
\cite[1.3.1]{arveson1969subalgebras} that any element  $T_{AB}\in \Gamma(B(\Ha_A))'$ must
commute with $\eta_{AB}$ and be of the form $T_{AB}=I_A\otimes T_B$. Put
\begin{equation}\label{eq:decomp}
\Be:=\{T_B\in B(\Ha_B),\ T_B \text{ commutes with } \eta_{AB}\},
\end{equation}
then $\Be$ is a *-subalgebra in $B(\Ha_B)$ and we must have $\eta_{BC}\in \Be\otimes
B(\Ha_C)$. It is also clear from the definition of $\Be$ that $\eta_{AB}\in (I_A\otimes
\Be)'=B(\Ha_A)\otimes \Be'$. 

For any subalgebra $\Be\subseteq B(\Ha_B)$, there is a decomposition and a unitary $U_B$
as in (iv) such that
\[
\Be=U_B\left(\oplus_n I_{B_L^n}\otimes B(\Ha_{B_R^n})\right)U_B^*,\qquad \Be'=U_B\left(\oplus_n
B(\Ha_{B_L^n})\otimes I_{B_R^n}\right)U_B^*.
\]
Since 
\[
\eta_{BC}\in (\Be\otimes B(\Ha_C))^+=U_B^*\left(\oplus_n I_{B_L^n}\otimes
B(\Ha_{B_R^nC})^+\right)U_B,
\]
we must have $\eta_{BC}=U_B^*\left(\oplus_n I_{B_L^n}\otimes\eta_{B_R^nC}\right)U_B$ for
some $\eta_{B_R^nC}\in B(\Ha_{B_R^nC})^+$. Similarly, $\eta_{AB}=U_B^*\left(\oplus_n
\eta_{AB_L^n}\otimes I_{B_R^n}\right)U_B$ for some $\eta_{AB_L^n}\in B(\Ha_{AB_L^n})^+$. 
The statement (iv) now follows from $\rho_{ABC}=\rho_B^{1/2}\eta_{AB}\eta_{BC}\rho_B^{1/2}$.


Suppose (iv) holds, then from
\[
I_B=\Tr_{AC}\rho_B^{-1/2}\rho_{ABC}\rho_B^{-1/2}=U_B^*\left(\oplus_n \eta_{B_L^n}\otimes
\eta_{B_R^n}\right)U_B
\]
we infer that $\eta_{B_L^n}=I_{B_L^n}$ and $\eta_{B_R^n}=I_{B_R^n}$. It follows that
$\rho_{AB}=\rho_B^{1/2}U^*_B\left(\oplus_n \eta_{AB_L^n}\otimes
I_{B_R^n}\right)U_B\rho_B^{1/2}$ and
similarly $\rho_{BC}=\rho_B^{1/2}U^*_B\left(\oplus_n I_{B_L^n}\otimes
\eta_{B_R^nC}\right)U_B\rho_B^{1/2}$. The condition (ii) is immediate from this.


Assume now that $\rho_{ABC}$ is a QMC. By \cite{hayden2004structure}, there is a
decomposition and unitary $U_B:\Ha_B\to \oplus_n \Ha_{B_L^n}\otimes \Ha_{B_R^n}$, such
that 
\begin{equation}\label{eq:QMC}
\rho_{ABC}=U_B^*\left(\oplus_n p_n \rho_{AB_L^n}\otimes \rho_{B_R^nC}\right)U_B,
\end{equation}
where $\rho_{AB_L^n}\in B(\Ha_{AB_L^n})$ and $\rho_{B_R^nC}\in B(\Ha_{B_R^nC})$ are states
and $\{p_n\}_n$ is a probability distribution. It follows from this that 
\begin{equation}\label{eq:invariant}
\rho_B=U_B^*\left(\oplus_n p_n \rho_{B_L^n}\otimes \rho_{B_R^n}\right)U_B,
\end{equation}
and $\eta_{AB}=U_B^*\left(\oplus_n
\rho_{B_L^n}^{-1/2}\rho_{AB_L^n}\rho_{B_L^n}^{-1/2}\otimes I_{B_R^n}\right)U_B$, 
$\eta_{BC}=U_B^*\left(\oplus_nI_{B_L^n}\otimes
\rho_{B_R^n}^{-1/2}\rho_{B_R^nC}\rho_{B_R^n}^{-1/2}\right)U_B$. It is clear from this that
$\rho_{ABC}$ is a BS-QMC and that $\rho_B^{it}\eta_{AB}\rho_B^{-it}$ commutes with
$\eta_{BC}$ for all $t\in \mathbb R$. 

For the converse, note that the condition implies that $\eta_{BC}\in \tilde \Be\otimes
B(\Ha_C)$, where
\[
\tilde \Be:=\{T_B\in B(\Ha_B),\ T_B \text{ commutes with }
\rho_B^{it}\eta_{AB}\rho_B^{-it},\ \forall t\}.
\]
Then $\tilde \Be$ is a subalgebra invariant under $\rho_B^{it}\cdot \rho_B^{-it}$. It also
follows that $\eta_{AB}\in B(\Ha_A)\otimes \tilde \Be'$, where the commutant $\tilde \Be'$
is also invariant under $\rho_B^{it}\cdot \rho_B^{-it}$. Assume that  $\tilde \Be$ has a
decomposition as in \eqref{eq:decomp}, then $\rho_{ABC}$ has the form given in the
statement  (iv), but the invariance condition implies that $\rho_B$ has the form
\eqref{eq:invariant}. It folows that $\rho_{ABC}$ has the form \eqref{eq:QMC}, so that it
sis a QMC.


\end{proof}



\end{document}

Let $\rho, \sigma\in B(\Ha)^+$. The Belavkin-Staszewski relative entropy is defined as
\[
\hat D(\rho\|\sigma):=\Tr\rho\log(\rho^{1/2}\sigma^{-1}\rho^{1/2})=\Tr\sigma
f(\sigma^{-1/2}\rho\sigma^{-1/2}),
\]
with $f(t)=t\log t$. By \cite[Cor. 3.31]{hiai2017different}, $\hat D$ is nonincreasing
under positive trace preserving maps $\Phi:B(\Ha)\to B(\Ka)$, and by \cite[Thm. 3.34
(h)]{hiai2017different}, the equality
\begin{equation}\label{eq:eq}
\hat D(\Phi(\rho)\|\Phi(\sigma))=\hat D(\rho\|\sigma)
\end{equation}
holds if and only if $R:=\sigma^{-1/2}\rho\sigma^{-1/2}$ satisfies
$\Phi_\sigma(R^2)=\Phi_\sigma(R)^2$, where
\[
\Phi_\sigma(X)=\Phi(\sigma)^{-1/2}\Phi(\sigma^{1/2}X\sigma^{1/2})\Phi(\sigma)^{-1/2},\qquad
X\in B(\Ha)
\]
is the Petz dual of $\Phi$ with respect to $\sigma$. Note that $\Phi_\sigma$ is  positive
and unital and the equality condition means that $R$ is in the multiplicative domain of
$\Phi_\sigma$. If $\Phi$ is completely positive, we may use the following fact.

\begin{lemma}\label{lemma:multiplicative} Let $\Psi:B(\Ha)\to B(\Ka)$ be a completely positive unital map with Kraus
representation $\Psi(\cdot)=\sum_i K_i^* (\cdot) K_i$. Then the multiplicative domain of
$\Psi$ has the form
\[
\Me_\Psi=\{K_iK_j^*,\ i,j\}',
\]
(here $C'$ denotes the commutant of a subset $C\subseteq B(\Ha)$).
\end{lemma}

 Assume that $\Phi: B(\Ha)\to B(\Ka)$ has the form
$\Phi(\cdot)=\sum_{i=1}^n L_i^*(\cdot)L_i$, for some $L_i:\Ka\to \Ha$ such that
$\sum_iL_iL_i^*=I_\Ha$. Then the equality \eqref{eq:eq} holds if and only if $R$ commutes
with all elements of the form 
\[
\sigma^{1/2}L_i\Phi(\sigma)^{-1}L_j^*\sigma^{1/2},\qquad i,j=1,\dots,n.
\]

We will  apply this in the special case when $\rho=\rho_{ABC}\in B(\Ha_{ABC})^+$, $\sigma=\rho_{AB}\otimes
\tau_C$ and $\Phi=\Tr_A$, here $\tau_C=\dim(\Ha_C)^{-1}I_C$ is the maximally mixed state.




\begin{prop} Let $\rho_{ABC}$ be a state (such that $\rho_{AB}$ is invertible). The equality
\begin{equation}\label{eq:eqm}
\hat D(\rho_{ABC}\|\rho_{AB}\otimes \tau_C)=\hat D(\rho_{BC}\|\rho_B\otimes  \tau_C)
\end{equation}
holds if and only if there are:
\begin{enumerate}
\item [(i)]Hilbert spaces $\Ha_{B^L_n}$, $\Ha_{B^R_n}$ such that 
$\Ha_B\simeq \oplus_n(\Ha_{B^L_n}\otimes \Ha_{B^R_n})$,
\item[(ii)] positive (invertible) elements $M_n\in B(\Ha_A\otimes \Ha_{B^L_n})$ such that
$\Tr_A M_n=I_{B^L_n}$,
\item [(iii)] positive elements $N_n\in B(\Ha_{B^R_n}\otimes \Ha_C)$ such that $\Tr_C
N_n=I_{B^R_n}$,
\item[(iv)] an (invertible) operator $S_B: \oplus_n (\Ha_{B^L_n}\otimes \Ha_{B^R_n})\to \Ha_B$ such
that $\Tr[S_BS_B^*]=1$
\end{enumerate}
such that
\[
\rho_{ABC}=(I_A\otimes S_B\otimes I_C)\left(\oplus_n M_n\otimes N_n \right)(I_A\otimes
S_B^*\otimes I_C)
\]

\end{prop}

\begin{proof} Assume that $\rho_{ABC}$ has this form. Let us denote $M:= \oplus_n
M_n\otimes I_{B^R_n}$, $N:=\oplus_n I_{B^L_n}\otimes N_n$, then $M\in B(\Ha_{AB})^+$ (is
invertible), $N\in B(\Ha_{BC})^+$ are such that $\Tr_A[M]=I_B=\Tr_C[N]$ and $M\otimes
I_C$ commutes with $I_A\otimes N$. We have
\[
\rho_{ABC}=(I_A\otimes S_B\otimes I_C)(M\otimes I_C)(I_A\otimes N)(I_A\otimes
S_B^*\otimes I_C)
\]
and  
\[
\rho_{AB}=\Tr_C \rho_{ABC}= (I_A\otimes S_B)M(I_A\otimes S_B^*),\qquad \rho_B=S_BS_B^*.
\]
Using polar decompositions,  there is some unitary $W\in B(\Ha_{AB})$ such that 
\[
(I_A\otimes S_B)M^{1/2}W^*=\rho_{AB}^{1/2}=WM^{1/2}(I_A\otimes S_B^*).
\]
It follows that 
\[
(\rho_{AB}^{-1/2}\otimes I_C)\rho_{ABC}(\rho_{AB}^{-1/2}\otimes I_C)=(W\otimes
I_C)(I_A\otimes N)(W^*\otimes I_C)
\]
and 
\[
\rho_{AB}=WM^{1/2}(I_A\otimes S_B^*S_B)M^{1/2}W^*.
\]
We may clearly replace $\tau_C$ by $I_C$ in the equality \eqref{eq:eqm}, since this only
adds a constant to both sides. We get
\begin{align*}
\hat D(\rho_{ABC}\|\rho_{AB}\otimes I_C)&=\Tr (\rho_{AB}\otimes I_C)f((W\otimes
I_C)(I_A\otimes N)(W^*\otimes I_C))\\
&=\Tr [(M^{1/2}(I_A\otimes S_B^*S_B)M^{1/2}\otimes I_C) f(I_A\otimes N)]\\
&=\Tr [(M(I_A\otimes S_B^*S_B)\otimes I_C)f(I_A\otimes N)]=\Tr [(S_B^*S_B\otimes I_C) f(N)],
\end{align*}
here $f(t)=t\log t$ and we have used the fact that $M\otimes I_C$ commutes with
$I_A\otimes N$. 

We also have
\[
\rho_{BC}=(S_B\otimes I_C)N(S_B^*\otimes I_C)
\]
and with the polar decomposition $S_B=\rho_B^{1/2}U_B$, we get 
\[
(\rho_B^{-1/2}\otimes I_C)\rho_{BC}(\rho_B^{-1/2}\otimes I_C)=(U_B\otimes
I_C)N(U_B^*\otimes I_C).
\]
It follows that
\[
\hat D(\rho_{BC}\|\rho_B\otimes I_C)=\Tr [(\rho_B\otimes I_C)f((U_B\otimes
I_C)N(U_B^*\otimes I_C))]=\Tr [(S_B^*S_B\otimes I_C)f(N)]=\hat
D(\rho_{ABC}\|\rho_{AB}\otimes I_C).
\]

For the converse, assume that \eqref{eq:eqm} holds. Put $R:=(\rho_{AB}^{-1/2}\otimes
I_C)\rho_{ABC}(\rho_{AB}^{-1/2}\otimes I_C)$, so that $R\ge 0$ and $\Tr_C[R]=I_{AB}$.
Moreover,  $R$ must be in the multiplicative
domain of the map
\[
\Phi_\sigma(X_{ABC})=(\rho_B^{-1/2}\otimes
I_C)\Tr_A[(\rho_{AB}^{1/2}\otimes I_C)X(\rho_{AB}^{1/2}\otimes I_C)](\rho_B^{-1/2}\otimes
I_C)=\sum_i L_i^*XL_i,
\]
where the Kraus operators have the form
\[
L_i=(\rho_{AB}^{1/2}(|i\>_A\otimes I_B)\rho_B^{-1/2})\otimes I_C.
\]
By Lemma \ref{lemma:multiplicative}, the operator $R$ must commute with all elements of
the form 
\[
\rho_{AB}^{1/2}(|i\>\<j|_A\otimes \rho_B^{-1})\rho_{AB}^{1/2}\otimes I_C,\qquad
i,j=1,\dots\dim(\Ha_A).
\]
This means that  
\[
R\in \Rr\otimes B(\Ha_C),
\]
where $\Rr= \Gamma(B(\Ha_A))'$, with $\Gamma: B(\Ha_A)\to B(\Ha_{AB})$ is a completely
positive map given as 
\[
\Gamma(X_A)=V^*(X_A\otimes I_B)V,\qquad X_A\in B(\Ha_A),
\]
with $V:=(I_A\otimes \rho_B^{-1/2})\rho_{AB}^{1/2}$. Since  $\rho_{AB}$ is invertible by
the assumption,  Arveson's commutant lifting
theorem \cite[Thm. 1.3.1]{arveson1969subalgebras} says that  for every $T\in \Rr$ there is a unique
$T_1\in B(\Ha_B)$ such that $(I_A\otimes T_1)V=VT$ and the map $T\mapsto T_1$ is a
*-isomorphism of $\Rr$ onto the subalgebra $\Rr_1\subseteq B(\Ha_B)$ given by
\[
(I_A\otimes B(\Ha_B))\cap \{VV^*\}'=I_A\otimes \Rr_1.
\]
Note that $M:=VV^*=(I_A\otimes \rho_B^{-1/2})\rho_{AB}(I_A\otimes \rho_B^{-1/2})$
satisfies $\Tr_A[M]=I_B$, so that this *-isomorphism is defined by
\[
\Tr_A [VTV^*]=\Tr_A[(I_A\otimes T_1)VV^*]=T_1\Tr_A [M]=T_1.
\]
The inverse map $\Rr_1\to \Rr$ is obtained from the polar decomposition  $V=M^{1/2}W$,
where  $W$ is a unitary. For any $T_1\in \Rr_1$, $I_A\otimes T_1$ commutes with $M^{1/2}$ and we have
\[
VW^*(I_A\otimes T_1)W=M^{1/2}(I_A\otimes T_1)W=(I_A\otimes T_1)M^{1/2}W=(I_A\otimes T_1)V,
\]
so that $T=W^*(I_A\otimes T_1)W$. It follows that $\Rr=W^*(I_A\otimes \Rr_1)W$ and hence
\[
R\in \Rr\otimes B(\Ha_C)=(W^*\otimes I_C)(I_A\otimes \Rr_1\otimes B(\Ha_C))(W\otimes I_C). 
\]
Therefore  there is some positive element $N\in \Rr_1\otimes B(\Ha_C)$ such that 
\begin{equation}\label{eq:R}
R=(W^*\otimes I_C)(I_A\otimes N)(W\otimes I_C).
\end{equation}
Moreover, since $\Tr_C[R]=I_{AB}$, we must have
$\Tr_C[N]=I_B$. 
 Note also that 
\[
 M\otimes I_C\in (I_A\otimes \Rr_1)'\otimes I_C=(I_A\otimes \Rr_1\otimes B(\Ha_C))',
\]
so that $M\otimes I_C$ commutes with $I_A\otimes N$. To finish the proof, we write
\[
\rho_{ABC}=(\rho_{AB}\otimes I_C)R(\rho_{AB}\otimes I_C)
\]
and 
\[
\rho_{AB}=(I_A\otimes \rho_B^{1/2})V=(I_A\otimes \rho_B^{1/2})M^{1/2}W.
\]
Combining this with \eqref{eq:R}, we obtain
\[
\rho_{ABC}=(I_A\otimes \rho_B^{1/2})(M\otimes I_C)(I_A\otimes N)(I_A\otimes \rho_B^{1/2}).
\]
Since $\Rr_1\subseteq B(\Ha_B)$ is a subalgebra, there are Hilbert spaces as in (i) and a
unitary $U_B:\Ha_B\to \oplus_n \Ha_{B^L_n}\otimes \Ha_{B^R_n}$ such that
\[
\Rr_1=U^*_B\left(\bigoplus_n I_{B^L_n}\otimes B(\Ha_{B^R_n})\right) U_B,
\]
Using this decomposition, we see that there are elements $M_n$ as in (ii) such that 
$M=(I_A\otimes U^*_B)(\oplus_n M_n\otimes I_{B^R_n})(I_A\otimes U_B)$ and similarly, there
are elements $N_n$ as in (iii) such that $N=(U_B^*\otimes I_C)(\oplus_n I_{B^L_n}\otimes
N_n)(U_B\otimes I_C)$. Now we see that $\rho_{ABC}$ has the required form, with
$S_B=\rho^{1/2}_BU^*_B$.


\end{proof}


\bibliography{IDEAS}
\bibliographystyle{abbrvnat}



\end{document}


\begin{thebibliography}{99}

\bibitem{jencova2021renyi} A. Jen\v cov\'a, Rényi relative entropies and noncommutative
$L_p$-spaces II, Ann. Henri Poincaré 22, 3235–3254 (2021)
\bibitem{kato2023onrenyi} Shinya Kato, On $\alpha-z$-R\'enyi divergence in the von Neumann
algebra setting, arXiv:2311.01748.

\end{thebibliography}



