\documentclass[12pt]{article}
\usepackage{geometry}
\usepackage{amsfonts, amsmath}
\geometry{total={210mm,290mm},
 left=23mm,right=23mm,%
 bindingoffset=0mm, top=20mm,bottom=20mm}





\begin{document}
\begin{center}
{\large  Chengkai Zhu, Zhiping Liu, Chenghong Zhu, and Xin Wang, Limitations of classically-simulable measurements for quantum state discrimination}

\end{center}
\medskip

\centerline{Referee report}

\bigskip

The topic of the paper is quantum state discrimination by using POVMs with nonnegative
Wigner functions (PWF). The importance of this problem  lies in the fact that such POVMs
 can be effectively simulated classically. The authors show that there are pairs of
 mutually orthogonal states  that cannot
 be unambiguously distinguished by such POVMs, even if  an  arbitrary number of copies of
 the states is given. This is done using a property the authors call (strong) PWF
 unextendability of certain subspaces. Further results are proved on the minimum error
 discrimination of the strange state and its orthogonal complement, and relations to data
 hiding are suggested. This situation is somewhat similar to entanglement and
 discrimination of states by LOCC measurements, but there are distinctions that are
 also discussed and illustrated  in the paper. 
\textbf{Overall evaluation}


\medskip

The results of the paper are not particularly surprising and the techniques applied are
not really involved. On the other hand, the question of limitations of classically
simulable measurements is surely important and  it seems that the problem of their use in
state discrimination was not considered before. The similarities and relations to
entanglement and LOCC shown in the paper, as well as the PWF robustness and the connection
to data hiding, are intriguing and surely deserve investigation. 



\medskip
 
The paper is quite well written, there are only a few small remarks and suggestions to make:

\begin{enumerate}
\item for the sake of the general readers, it would be good to define the magic states
\item page 2, column 1, line 22: ''the unitary ...operators... is defined...''  are
defined
\item in the same sentence as above: it would be better to indicate that $\oplus$ is the
addition in $\mathbb Z_d$
\item There is some confusion concerning the use of the expression ''POVM'' throughout
the paper and also the Supplementary material. There are sentences like ''Let $E$ be a
POVM...'', but actually $E$ is an operator $0\le E\le I$, such operator is usually called an
effect. A POVM is a collection of effects summing up to identity, used in description of
measurements.  Any effect defines a 
two-valued POVM $\{E, I-E\}$. So an effect $E$ is PWF if $W(E|\mathbf{u})\ge 0$, but the
corresponding POVM is PWF if also $I-E$ has this property, so that $0\le W(E|\mathbf{u})\le
1$. It would be better to clarify this, to avoid confusion.

\item p.3, paragraph below Theorem 4: ''Theorem 4 is broad applicability..,'' perhaps
''of'' is missing

\item p. 3, paragraph above prop. 5: ''We remain the dual SDP...'' better replace
''remain'' with some other verb 

\item Supplementary material, proof of Lemma 2: better give some arguments showing that
$\mathrm{Tr}_1 \sigma$ and $\sigma'$ are PWF (like the property 5. of the operators
$\{A_{\mathbf{u}}\}$)

\item Eq. (S12): the factor $\frac1d$ seems missing.
\item SM, proof of Prop. 3: ''...that contains only magic states.'' perhaps ''supports''
would be better instead of ''contains''
\item Main text and SM: both  notations $A_0$ and $A_{\textbf{0}}$  are used, better pick
one.
\item The arguments between Eq.(S28c) and (S28d) are  difficult to understand
\item  Eq. (S35): Perhaps the factor $\frac1d$ is missing here? Or should there be
$W(P_{\mathcal S}^\perp|\mathbf{u})$? These notations are somewhat inconsistent.

\item Better explain (or cite) the equality in (S44).
\item Eq. (S29) $a_{\mathbf{u}}$ or $\hat a_{\mathbf{u}}$? 
\end{enumerate}





\







\end{document}

