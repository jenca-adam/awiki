\documentclass[12pt]{article}

\usepackage{hyperref}
\usepackage{amsmath, amssymb, amsthm}
\usepackage[sort&compress,numbers]{natbib}
\usepackage{doi}
\usepackage[margin=0.8in]{geometry}
%\textheight23cm \topmargin-20mm  
%\textwidth175mm  
%\oddsidemargin=0mm
%\evensidemargin=0mm
%

\usepackage{amsmath, amssymb, amsthm, mathtools}

\newtheorem{lemma}{Lemma}
\newtheorem{theorem}{Theorem}
\newtheorem{coro}{Corollary}
\newtheorem{prop}{Proposition}


\theoremstyle{definition}
\newtheorem{defi}{Definition}


\theoremstyle{remark}
\newtheorem{remark}{Remark}

\def\Me{\mathcal M}
\def\Te{\mathcal T}
\def\Rr{\mathcal R}
\def\Ha{\mathcal H}
\def\Ka{\mathcal K}
\def\Ne{\mathcal N}
\def \Tr{\mathrm{Tr}\,}
\def\Se {\mathcal S}
\def\supp{\mathrm{supp}}
\def\<{\langle\.}
\def\>{\.\rangle}

\title{A note on equality in DPI for the BS relative entropy}
\author{Anna Jen\v cov\'a}

\begin{document}

\maketitle

Let $\rho, \sigma\in B(\Ha)^+$. The Belavkin-Staszewski relative entropy is defined as
\[
\hat D(\rho\|\sigma):=\Tr\rho\log(\rho^{1/2}\sigma^{-1}\rho^{1/2})=\Tr\sigma
f(\sigma^{-1/2}\rho\sigma^{-1/2}),
\]
with $f(t)=t\log t$. By \cite[Cor. 3.31]{hiai2017different}, $\hat D$ is nonincreasing
under positive trace preserving maps $\Phi:B(\Ha)\to B(\Ka)$, and by \cite[Thm. 3.34
(h)]{hiai2017different}, the equality
\begin{equation}\label{eq:eq}
\hat D(\Phi(\rho)\|\Phi(\sigma))=\hat D(\rho\|\sigma)
\end{equation}
holds if and only if $d:=\sigma^{-1/2}\rho\sigma^{-1/2}$ satisfies
$\Phi_\sigma(d^2)=\Phi_\sigma(d)^2$, where
\[
\Phi_\sigma(X)=\Phi(\sigma)^{-1/2}\Phi(\sigma^{1/2}X\sigma^{1/2})\Phi(\sigma)^{-1/2},\qquad
X\in B(\Ha)
\]
is the Petz dual of $\Phi$ with respect to $\sigma$. Note that $\Phi_\sigma$ is  positive
and unital. If $\Phi$ is completely positive, we may use the following fact.

\begin{lemma}\label{lemma:multiplicative} Let $\Psi:B(\Ha)\to B(\Ka)$ be a completely positive unital map with Kraus
representation $\Psi(\cdot)=\sum_i K_i^* (\cdot) K_i$. Then the multiplicative domain of
$\Psi$ has the form
\[
\Me_\Psi=\{K_iK_j^*,\ i,j\}',
\]
(here $C'$ denotes the commutant of a subset $C\subseteq B(\Ha)$).
\end{lemma}

This implies the following result. Assume that $\Phi: B(\Ha)\to B(\Ka)$ has the form
$\Phi(\cdot)=\sum_{i=1}^n L_i^*(\cdot)L_i$, for some $L_i:\Ka\to \Ha$ such that
$\sum_iL_iL_i^*=I_\Ha$. Then the equality \eqref{eq:eq} holds if and only if $d$ commutes
with all elements of the form 
\[
\sigma^{1/2}L_i\Phi(\sigma)^{-1}L_j^*\sigma^{1/2},\qquad i,j=1,\dots,n.
\]

Let us apply this in the special case when $\rho=\rho_{ABC}\in B(\Ha_{ABC})^+$, $\sigma=\rho_{AB}\otimes
\tau_C$ and $\Phi=\Tr_A$, here $\tau_C=\dim(\Ha_C)^{-1}I_C$ is the maximally mixed state.
The condition then becomes that $d$ must commute with all elements of the form
\[
\rho_{AB}^{1/2}(|i\>\<j|_A\otimes \rho_B^{-1})\rho_{AB}^{1/2}\otimes I_C,\qquad
i,j=1,\dots\dim(\Ha_A).
\]
We may clearly replace $d$ by $\tilde d=(\rho_{AB}^{-1/2}\otimes
I_C)\rho_{ABC}(\rho_{AB}^{-1/2}\otimes I_C)$. The condition is then 
\[
\tilde d\in \Rr\otimes B(\Ha_C),
\]
where $\Rr= \Gamma(B(\Ha_A))'$, with $\Gamma: B(\Ha_A)\to B(\Ha_{AB})$ is a cpmpletely
positive map given as 
\[
\Gamma(X_A)=V^*(X_A\otimes I_B)V,\qquad X_A\in B(\Ha_A),
\]
here $V=(I_A\otimes \rho_B^{-1/2})\rho_{AB}^{1/2}$. Assume that $\rho_{AB}$ is invertible. By Arveson's commutant lifting
theorem \cite[Thm. 1.3.1]{arveson1969subalgebras}, for every $T\in \Rr$ there is a unique
$T_1\in B(\Ha_B)$ such that $(I_A\otimes T_1)V=VT$ and the map $T\mapsto T_1$ is a
*-isomorphism of $\Rr$ onto $(I_A\otimes B(\Ha_B))\cap \{VV^*\}'$.

Note that $VV^*=(I_A\otimes \rho_B^{-1/2})\rho_{AB}(I_A\otimes \rho_B^{-1/2})$, so that
the map $T\mapsto T_1$ is given as
\[
\Tr_A [VTV^*]=\Tr_A[(I_A\otimes T_1)VV^*]=T_1\Tr_A [VV^*]=T_1.
\]
The inverse map $T_1\mapsto T$ is obtained from the polar decomposition of $V$: $V=HW$,
where $H=(VV^*)^{1/2}$ and $W$ is a unitary. Then $T_1$ commutes with $H$ and we have
\[
VW^*(I_A\otimes T_1)W=H(I_A\otimes T_1)W=(I_A\otimes T_1)HW=(I_A\otimes T_1)V,
\]
so that $T=W^*(I_A\otimes T_1)W$. 

This leads to the following construction of states $\rho_{ABC}$ such that 
\[
\hat D(\rho_{ABC}\|\rho_{AB}\otimes \tau_C)=\hat D(\rho_{BC}\|\rho_B\otimes  \tau_C).
\]
Let $M\in B(\Ha_{AB})^{++}$ be any element such that $\Tr_A[M]=I_B$.  Let $\Te\subseteq
B(\Ha_B)$ be such that $I_A\otimes \Te=\{M\}'\cap I_A\otimes B(\Ha_B)$. Take any state
$\rho_B\in B(\Ha_B)^{++}$ and put
\[
\rho_{AB}=(I_A\otimes \rho_B^{1/2})M(I_A\otimes \rho_B^{1/2}),
\]
then clearly $\rho_{AB}$ is a state and $\Tr_A \rho_{AB}=\rho_B$. Let $W$ be the unitary such that 
\[
(I_A\otimes \rho_B)^{1/2}M^{1/2}=\rho_{AB}^{1/2}W^*.
\]
Now choose any positive  element $\tilde d\in W^*(I_A\otimes \Te)W\otimes B(\Ha_C)$ such
that $\Tr_C \tilde d=I_{AB}$ and put 
\[
\rho_{ABC}=(\rho_{AB}^{1/2}\otimes I_C)\tilde d (\rho_{AB}^{1/2}\otimes I_C).
\]
Then we have $\Tr_C \rho_{ABC}=\rho_{AB}$, $\Tr_{AC}\rho_{ABC}=\rho_{AB}$.

Since $\Te$ is a subalgebra in $B(\Ha_B)$, there is a unitary $U_B$ and a decomposition
\[
\Te=U_B\left(\bigoplus_n I_{B^L_n}\otimes B(\Ha_{B^R_n})\right) U_B^*,
\]
so that 
\[
(W\otimes I_C)\tilde d(W^*\otimes I_C)=(I_A\otimes U_B\otimes I_C)\left(\oplus_n I_{AB^L_n}\otimes
d_{B^R_nC}\right)(I_A\otimes U_B^*\otimes I_C)
\]
for some positive elements $d_{B^R_nC}\in B(\Ha_{B^R_n}\otimes \Ha_C)^+$.
Moreover, from  $M\subseteq (I\otimes \Te)'$, we get
\[
M=(I_A\otimes U_B)\left(\oplus_n M_{AB^L_n}\otimes I_{B^R_n}\right)(I_A\otimes U_B^*)
\]
for $M_{AB^L_n}\in B(\Ha_A\otimes \Ha_{B^L_n})^+$. 
Putting all together, we get
\begin{align*}
\rho_{ABC}&=(\rho_{AB}^{1/2}W^*W\otimes I_C)\tilde d (W^*W\rho_{AB}^{1/2}\otimes
I_C)\\ &=((I_A\otimes \rho_B^{1/2})M^{1/2}\otimes I_C)(I_A\otimes U_B\otimes I_C)\left(\oplus_n I_{AB^L_n}\otimes
d_{B^R_nC}\right)(I_A\otimes U_B^*\otimes I_C)(M^{1/2}(I_A\otimes \rho_B^{1/2})\otimes
I_C)\\
&= (I_A\otimes \rho_B^{1/2}U_B\otimes I_C)\left(\oplus_n M_{AB^L_n}\otimes
d_{B^R_nC}\right)(I_A\otimes U_B^*\rho_B^{1/2}\otimes I_C). 
\end{align*}


\section{The structure of $\rho_{ABC}$}


\begin{prop} Let $\rho_{ABC}$ be a state (such that $\rho_{AB}$ is invertible). The equality
\begin{equation}\label{eq:eqm}
\hat D(\rho_{ABC}\|\rho_{AB}\otimes \tau_C)=\hat D(\rho_{BC}\|\rho_B\otimes  \tau_C)
\end{equation}
holds if and only if there are:
\begin{enumerate}
\item [(i)]Hilbert spaces $\Ha_{B^L_n}$, $\Ha_{B^R_n}$ such that 
$\Ha_B\simeq \oplus_n(\Ha_{B^L_n}\otimes \Ha_{B^R_n})$,
\item[(ii)] positive (invertible) elements $M_n\in B(\Ha_A\otimes \Ha_{B^L_n})$ such that
$\Tr_A M_n=I_{B^L_n}$,
\item [(iii)] positive elements $d_n\in B(\Ha_{B^R_n}\otimes \Ha_C)$ such that $\Tr_C
d_n=I_{B^R_n}$,
\item[(iv)] an (invertible) operator $S_B: \oplus_n (\Ha_{B^L_n}\otimes \Ha_{B^R_n})\to \Ha_B$ such
that $\Tr[S_BS_B^*]=1$
\end{enumerate}
such that
\[
\rho_{ABC}=(I_A\otimes S_B\otimes I_C)\left(\oplus_n M_n\otimes d_n \right)(I_A\otimes
S_B^*\otimes I_C)
\]

\end{prop}

\begin{proof} Assume that $\rho_{ABC}$ has this form. Then 
\[
\rho_{AB}=\Tr_C \rho_{ABC}= (I_A\otimes S_B)\left(\oplus_n M_n\otimes
I_{B^R_n}\right)(I_A\otimes S_B^*),\qquad \rho_B=S_BS_B^*.
\]
Let us denote $M:= \oplus_n M_n\otimes I_{B^R_n}$, $d:=\oplus_n I_{B^L_n}\otimes d_n$. 
Using polar decompositions,  there is some unitary $W\in B(\Ha_{AB})$ such that 
\[
(I_A\otimes S_B)M^{1/2}W^*=\rho_{AB}^{1/2}=WM^{1/2}(I_A\otimes S_B^*).
\]
It follows that 
\[
(\rho_{AB}^{-1/2}\otimes I_C)\rho_{ABC}(\rho_{AB}^{-1/2}\otimes I_C)=(W\otimes
I_C)(I_A\otimes d)(W^*\otimes I_C)
\]
and 
\[
\rho_{AB}=WM^{1/2}(I_A\otimes S_B^*S_B)M^{1/2}W^*.
\]
We may clearly replace $\tau_C$ by $I_C$ in the equality \eqref{eq:eqm}, since this only
adds a constant to both sides. We get
\begin{align*}
\hat D(\rho_{ABC}\|\rho_{AB}\otimes I_C)&=\Tr (\rho_{AB}\otimes I_C)f((W\otimes
I_C)(I_A\otimes d)(W^*\otimes I_C))\\
&=\Tr [(M^{1/2}(I_A\otimes S_B^*S_B)M^{1/2}\otimes I_C) f(I_A\otimes d)]\\
&=\Tr [(M(I_A\otimes S_B^*S_B)\otimes I_C)f(I_A\otimes d)]=\Tr [(S_B^*S_B\otimes I_C) f(d)],
\end{align*}
here $f(t)=t\log t$ and we have used the fact that $M\otimes I_C$ commutes with
$I_A\otimes d$. 

We also have
\[
\rho_{BC}=(S_B\otimes I_C)d(S_B^*\otimes I_C)
\]
and with the polar decomposition $S_B=\rho_B^{1/2}U_B$, we get 
\[
(\rho_B^{-1/2}\otimes I_C)\rho_{BC}(\rho_B^{-1/2}\otimes I_C)=(U_B\otimes
I_C)d(U_B^*\otimes I_C).
\]
It follows that
\[
\hat D(\rho_{BC}\|\rho_B\otimes I_C)=\Tr [(\rho_B\otimes I_C)f((U_B\otimes
I_C)d(U_B^*\otimes I_C))]=\Tr [(S_B^*S_B\otimes I_C)f(d)]=\hat
D(\rho_{ABC}\|\rho_{AB}\otimes I_C).
\]

For the converse, assume that \eqref{eq:eqm} holds. Put $R:=(\rho_{AB}^{-1/2}\otimes
I_C)\rho_{ABC}(\rho_{AB}^{-1/2}\otimes I_C)$, so that $R\ge 0$ and $\Tr_C[R]=I_{AB}$.
Moreover,  $R$ must be in the multiplicative
domain of the map
\[
\Phi_\sigma(X_{ABC})=(\rho_B^{-1/2}\otimes
I_C)\Tr_A[(\rho_{AB}^{1/2}\otimes I_C)X(\rho_{AB}^{1/2}\otimes I_C)](\rho_B^{-1/2}\otimes
I_C)=\sum_i L_i^*XL_i,
\]
where the Kraus operators have the form
\[
L_i=(\rho_{AB}^{1/2}(|i\>_A\otimes I_B)\rho_B^{-1/2})\otimes I_C.
\]
By Lemma \ref{lemma:multiplicative}, the operator $R$ must commute with all elements of
the form 
\[
\rho_{AB}^{1/2}(|i\>\<j|_A\otimes \rho_B^{-1})\rho_{AB}^{1/2}\otimes I_C,\qquad
i,j=1,\dots\dim(\Ha_A).
\]
This means that  
\[
R\in \Rr\otimes B(\Ha_C),
\]
where $\Rr= \Gamma(B(\Ha_A))'$, with $\Gamma: B(\Ha_A)\to B(\Ha_{AB})$ is a completely
positive map given as 
\[
\Gamma(X_A)=V^*(X_A\otimes I_B)V,\qquad X_A\in B(\Ha_A),
\]
with $V:=(I_A\otimes \rho_B^{-1/2})\rho_{AB}^{1/2}$. Since  $\rho_{AB}$ is invertible by
the assumption,  Arveson's commutant lifting
theorem \cite[Thm. 1.3.1]{arveson1969subalgebras} says that  for every $T\in \Rr$ there is a unique
$T_1\in B(\Ha_B)$ such that $(I_A\otimes T_1)V=VT$ and the map $T\mapsto T_1$ is a
*-isomorphism of $\Rr$ onto the subalgebra $\Rr_1\subseteq B(\Ha_B)$ given by
\[
(I_A\otimes B(\Ha_B))\cap \{VV^*\}'=I_A\otimes \Rr_1.
\]
Note that $M:=VV^*=(I_A\otimes \rho_B^{-1/2})\rho_{AB}(I_A\otimes \rho_B^{-1/2})$
satisfies $\Tr_A[M]=I_B$, so that this *-isomorphism is defined by
\[
\Tr_A [VTV^*]=\Tr_A[(I_A\otimes T_1)VV^*]=T_1\Tr_A [VV^*]=T_1.
\]
The inverse map $\Rr_1\to \Rr$ is obtained from the polar decomposition  $V=M^{1/2}W$,
where  $W$ is a unitary. For any $T_1\in \Rr_1$, $I_A\otimes T_1$ commutes with $M^{1/2}$ and we have
\[
VW^*(I_A\otimes T_1)W=M^{1/2}(I_A\otimes T_1)W=(I_A\otimes T_1)M^{1/2}W=(I_A\otimes T_1)V,
\]
so that $T=W^*(I_A\otimes T_1)W$. It follows that $\Rr=W^*(I_A\otimes \Rr_1)W$ and hence
\[
R\in \Rr\otimes B(\Ha_C)=(W^*\otimes I_C)(I_A\otimes \Rr_1\otimes B(\Ha_C))(W\otimes I_C). 
\]
It follows that there is some positive element $N\in \Rr_1\otimes B(\Ha_C)$ such that 
\begin{equation}\label{eq:R}
R=(W^*\otimes I_C)(I_A\otimes N)(W\otimes I_C).
\end{equation}
Moreover, since $\Tr_C[R]=I_{AB}$, we must have
$\Tr_C[N]=I_B$. 
 Note also that 
\[
 M\otimes I_C\in (I_A\otimes \Rr_1)'\otimes I_C=(I_A\otimes \Rr_1\otimes B(\Ha_C))',
\]
so that $M\otimes I_C$ commutes with $I_A\otimes N$. To finish the proof, we write
\[
\rho_{ABC}=(\rho_{AB}\otimes I_C)R(\rho_{AB}\otimes I_C)
\]
and 
\[
\rho_{AB}=(I_A\otimes \rho_B^{1/2})V=(I_A\otimes \rho_B^{1/2})M^{1/2}W.
\]
Combining this with \eqref{eq:R}, we obtain
\[
\rho_{ABC}=(I_A\otimes \rho_B^{1/2})(M\otimes I_C)(I_A\otimes N)(I_A\otimes \rho_B^{1/2}).
\]
Since $\Rr_1\subseteq B(\Ha_B)$ is a subalgebra, there are Hilbert spaces as in (i) and a
unitary $U_B:\Ha_B\to \oplus_n \Ha_{B^L_n}\otimes \Ha_{B^R_n}$ such that
\[
\Rr_1=U^*_B\left(\bigoplus_n I_{B^L_n}\otimes B(\Ha_{B^R_n})\right) U_B,
\]
Using this decomposition, we see that there are elements $M_n$ as in (ii) such that 
$M=(I_A\otimes U^*_B)(\oplus_n M_n\otimes I_{B^R_n})(I_A\otimes U_B)$ and similarly, there
are elements $N_n$ as in (iii) such that $N=(U_B^*\otimes I_C)(\oplus_n I_{B^L_n}\otimes
N_n)(U_B\otimes I_C)$. Now we see that $\rho_{ABC}$ has the required form, with
$S_B=\rho^{1/2}_BU^*_B$.


\end{proof}



\end{document}


\begin{thebibliography}{99}

\bibitem{jencova2021renyi} A. Jen\v cov\'a, Rényi relative entropies and noncommutative
$L_p$-spaces II, Ann. Henri Poincaré 22, 3235–3254 (2021)
\bibitem{kato2023onrenyi} Shinya Kato, On $\alpha-z$-R\'enyi divergence in the von Neumann
algebra setting, arXiv:2311.01748.

\end{thebibliography}



