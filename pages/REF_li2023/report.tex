\documentclass[12pt]{article}
\usepackage{geometry}
\usepackage{amsfonts}
\geometry{total={210mm,290mm},
 left=23mm,right=23mm,%
 bindingoffset=0mm, top=20mm,bottom=20mm}





\begin{document}
\begin{center}
{\large  Yuan Li, Shuhui Gao, Cong Zhao, Nan Ma: On spectra of some completely positive maps }

\end{center}
\medskip

\centerline{Referee report}

\bigskip


This paper concerns normal completely positive maps $\Phi$ on $\mathcal B(\mathcal H)$ such that
the Kraus operators  $A_1,A_2,\dots$ satisfy both  $\sum_i A_i^*A_i$ and $\sum_i
A_iA_i^*$. It is proved that such a map $\Phi$  preserves the subspace $\mathcal
K(\mathcal H)$ of compact operators and the restriction to this subspace has the same
spectrum as the original map $\Phi$.  Further, peripheral eigenvalues and the
corresponding eigenvectors for the restricted map are characterized. This is based on the
previous results obtained in references [7] and [8] on the fixed points of the restricted
map. Furthermore, the spectra of two examples of such maps related to the unilateral shift
are described.



\bigskip
\noindent
\textbf{Overall assesment}

\medskip
The results of this paper are mathematically correct and sound, the proofs are clearly
written and readable. The characterization of spectra of normal completely
positive maps is and important problem with many applications. However, the results of the
present paper do not seem to be strong enough for publication. The proofs follow by
standard manipulations, basically from those in Reference  [7].  It is
also not clarified why the properties of the restriction to compact operators are
important or interesting. 


\bigskip
\noindent
\textbf{Some further comments}
\begin{enumerate}
\item p. 2, line 6 from below: ''is the commutants'' (typo)
\item p. 2 ''...the Poisson boundary (the set of all fixed points of compact operators)''
it seems that this should be ''the set of compact operators in the Poisson boundary...''
\item p. 11 lines 7 and 8 from below: why is the dagger here?
\item The notation $\mathbb T$ in Example 1 is a bit confusing, since this notation is
commonly used for the unit circle group: $\mathbb T=\{z\in \mathbb C\ :\ |z|=1\}$, which
is somewhat complementary to the present notation.
\item What is the meaning or purpose of the two final Remarks 2 and 3? Remark 2 concerns
unitary operations and I doubt it is new. As for Remark 3, it also does not seem to be
new, see E. Thorp, Projections onto the subspace of compact operators, Pacific J. Math. 10
(1960), 693-696 (the proof there even seems somewhat similar).





\end{enumerate}











\end{document}

