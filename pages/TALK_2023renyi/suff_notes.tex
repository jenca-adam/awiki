\documentclass[12pt]{article}


\usepackage{hyperref}
\usepackage{amsmath, amssymb, amsthm, mathtools}
\usepackage[sort&compress,numbers]{natbib}
\usepackage{doi}
\usepackage[margin=0.8in]{geometry}
%\textheight23cm \topmargin-20mm  
%\textwidth175mm  
%\oddsidemargin=0mm
%\evensidemargin=0mm
%


\newtheorem{lemma}{Lemma}
\newtheorem{theorem}{Theorem}
\newtheorem{coro}{Corollary}

\newtheorem{prop}{Proposition}

\theoremstyle{definition}
\newtheorem{defi}{Definition}


\theoremstyle{remark}
\newtheorem{remark}{Remark}
\newtheorem{exm}{Example}


\def\supp{\mathrm{supp}}
\def\Tr{\mathrm{Tr}\,}
\def\Me{\mathcal M}

\def\Le{\mathcal L}
\def\Ne{\mathcal N}

\def\Se{\mathcal S}

\title{Some notes on sufficient channels  2}

\begin{document}

\maketitle
Let $\Me,\Ne$ be von Neumann algebras and let 
$\Se$ be a set of normal states. We will assume that $\Se$ is convex and contains a
faithful state $\omega\in \Se$.

Let $\Phi:\Ne\to \Me$ be a unital normal completely positive (or 2-positive) map. Such a
map, or its predual $\Phi_*:\Me_*\to \Ne_*$, will be called a quantum channel.  We will
also assume that $\tilde\omega:=\Phi_*(\omega)$ is faithful as well. 

We say that $\Phi$ is sufficient with respect to $\Se$ if there exists a recovery channel
$\Psi:\Me\to \Ne$ such that 
\[
\Psi_*\circ\Phi_*(\rho)=\rho,\qquad \forall \rho\in \Se.
\]
If $\Ne\subseteq \Me$ is a subalgebra and $\Phi$ is the inclusion $\Ne\hookrightarrow
\Me$, then we say that $\Ne$ is a sufficient subalgebra with respect to $\Se$.

\section{Minimal sufficient subalgebra}

Let us denote a set of quantum channels
\[
\mathcal L:=\{\Theta:\Me\to \Me,\ \Psi_*(\rho)=\rho,\ \forall \rho\in \Se\}.
\]
Then $\mathcal L$ is a semigroup (closed under composition), which is convex and closed in
the  point - $\sigma$-weak topology. By the mean ergodic theorem, there exists an element
$E\in \Le$ such that 
\[
E\circ \Theta= \Theta\circ E=E,\qquad \forall \Theta\in \Le.
\]
Then 
\begin{enumerate}
\item $E$ is clearly idempotent ($E^2=E$) and preserves $\omega$. Hence $E$ is a
conditional expectation and its range is a subalgebra in $\Me$, invariant under the
modular group $\sigma_t^\omega$.

\item A channel $\Theta:\Me\to\Me$ is in $\Le$ if and only if $\Theta\circ E=E$.

\item The range of $E$ is 
\[
\mathcal R(E)=\{x\in \Me, \ \Theta(x)=x,\ \forall \Theta\in
\Le\}.
\]

\end{enumerate}

We will use the notation $E_\Se:=E$ and $\Me_\Se:=\mathcal R(E)$.

\begin{prop}\label{prop:minsuf}  $\Me_\Se$ is a minimal sufficient subalgebra with respect to
$\Se$.

\end{prop}

\begin{proof} It is easily seen that $E_\Se: \Me\to \Me_\Se$ is a recovery channel. Let
$\Ne\subseteq \Me$ be any sufficient subalgebra with respect to $\Se$ and let $\Psi:
\Me\to \Ne\subseteq \Me$ be the recovery channel, then $\Psi\in \Le$ and we have for any
$a\in \Me_\Se$:
\[
a=E(a)=\Psi\circ E(a)=\Psi(a)\in \Ne,
\]
so that $\Me_\Se\subseteq \Ne$.


\end{proof}

\begin{prop}\label{prop:minsuf_generators}
\begin{enumerate}
\item $\Me_\Se$ is generated by the set of Connes cocycles 
\[
\{[D\rho:D\omega]_t, \ \rho\in \Se,\
t\in \mathbb R\}
\]
\item If the commutant Radon-Nikodym derivatives $d(\rho,\omega)$ exist for $\rho\in \Se$,
then $\Me_\Se$ is generated by $\{\sigma_t^\omega(d(\rho,\omega)),\ \rho\in \Se,\ t\in
\mathbb R\}$.
\end{enumerate}

\end{prop}

\begin{proof}

\end{proof}

\begin{prop}\label{prop:koashi_imoto} If $\Me=B(\mathcal H)$, then there is a
decomposition...

Any channel preserving $\Se$ must satisfy....

\end{prop}

\begin{exm}\label{exm:broadcasting}

\end{exm}


\section{Sufficient channels}

A channel $\Phi:\Ne\to \Me$ is sufficient with respect to $\Se$ if and only if there is
some channel $\Psi:\Me\to \Ne$ such that $\Phi\circ \Psi=E_\Se$. In this case, 
\begin{enumerate}
\item $\Psi|_{\Me_\Se}$ is an isomorphism onto the subalgebra $\tilde \Me=\Psi(\Me_\Se)$.
\item $\Phi|_{\tilde \Me}$ is an isomorphism onto $\Me_\Se$ and $\Psi|_{\Me_\Se}$ is its
inverse.
\item $\tilde E:=\Psi\circ E_\Se\circ \Phi$ is a faithful conditional expectation onto
$\tilde \Me$ that preserves $\tilde \omega$.
\item Let $\tilde \Se=\{\Phi_*(\rho),\ \rho\in \Se\}$, then $\tilde E=E_{\tilde\Se}$.
\end{enumerate}


\begin{proof}
\begin{enumerate}
\item For $a\in \Me_\Se$, we have, since $\tilde \omega\circ\Psi=\omega\circ E_\Se$, 
\[
\omega(a^*a)=\tilde\omega(\Psi(a^*a))\ge \tilde \omega (\Psi(a)^*\Psi(a))\ge
\omega(\Phi(\Psi(a))^*\Phi(\Psi(a)))=\omega(a^*a).
\]
It follows in particular that the first inequality must be an equality. Since
$\Psi(a^*a)\ge \Psi(a)^*\Psi(a)$ and $\omega$ is faithful, it follows that
$\Psi(a^*a)=\Psi(a)^*\Psi(a)$. Hence $\Psi|_{\Me_\Se}$ is a homomorphism. Since
$\tilde\omega\circ \Psi=\omega\circ E_\Se=\omega$ is faithful, $\Psi$ must be an
isomorphism.
\item For any $a\in \Me_\Se$, we have $\Phi(\Psi(a))=E_\Se(a)=a$, so that $\Phi|_{\tilde
\Me}$ is the inverse of $\Psi|_{\Me_\Se}$, this clearly proves the point 2.
\item It is easily seen that $\tilde E^2=\tilde E$ and that $\tilde \omega\circ \tilde
E=\tilde \omega$, so that $\tilde E$ is a faithful conditional expectation with range
$\tilde \Me$.
\item It is easily seen that $\tilde E$ preserves $\rho\circ \Phi$ for all $\rho\in \Se$,
so that we have $\tilde E\circ E_{\tilde \Se}=E_{\tilde \Se}\circ \tilde E=E_{\tilde \Se}$. On the other hand, we have
\[
\rho\circ (\Phi\circ E_{\tilde \Se}\circ \Psi)=(\rho\circ\Phi)\circ E_{\tilde
\Se}\circ\Psi=\rho\circ \Phi\circ\Psi=\rho,
\]
so that $(\Phi\circ E_{\tilde \Se}\circ \Psi)\circ E_\Se=E_\Se$. It follows that
\begin{align*}
E_{\tilde S}&=\tilde E\circ E_{\tilde \Se}\circ \tilde E=(\Psi\circ E_\Se\circ \Phi)\circ
E_{\tilde \Se}\circ (\Psi\circ E_\Se\circ \Phi) =\Psi\circ E_\Se\circ (\Phi\circ E_{\tilde
\Se}\circ \Psi\circ
E_{\Se})\circ \Phi\\
&= \Psi\circ E_\Se\circ \Phi=\tilde E.
\end{align*}


\end{enumerate}


\end{proof}





\section{R\'enyi relative entropies and sufficiency}


For $\alpha\in (0,\infty)\setminus \{1\}$, the sandwiched (minimal) R\'enyi relative entropy is defined as
\begin{align*}
\tilde D_\alpha(\rho\|\sigma)&:=\frac1{\alpha-1}\log\frac{\tilde
Q_\alpha(\rho\|\sigma)}{\Tr\rho},\\[0.5em]
\tilde
Q_\alpha(\rho\|\sigma)&:=\begin{dcases}
\Tr\left(\sigma^{\frac{1-\alpha}{2\alpha}}\rho\sigma^{\frac{1-\alpha}{2\alpha}}\right)^\alpha
& \text{if } \alpha\in (0,1)\text{ or } \supp(\rho)\le \supp(\sigma)\\
\infty & \text{otherwise}.
\end{dcases}
\end{align*}

Let us assume $\alpha\in [1/2,1)$. We have the following variational expression \cite{frank2013monotonicity,hiai2021quantum}

\begin{equation}
\label{eq:sandw_var}\tilde Q_\alpha(\rho\|\sigma)=\inf_{x\in \Me^{++}} \alpha \Tr \rho x +(1-\alpha)\Tr
\left(\sigma^{\frac{1-\alpha}{2\alpha}}x^{-1}\sigma^{\frac{1-\alpha}{2\alpha}}\right)^{\frac{\alpha}{1-\alpha}}
\end{equation}

\begin{prop}\label{prop:var_alph}
Assume that $\alpha>1/2$ and put $\gamma:=\frac{\alpha}{1-\alpha}$, then $\gamma>1$ and we have
\[
\tilde Q_\alpha(\rho\|\sigma)=\inf_{x\in \Me^{++}} \alpha \Tr \rho x +(1-\alpha)\tilde
Q_\gamma(\sigma^{1/2}x^{-1}\sigma^{1/2}\|\sigma).
\]
If $\supp(\rho)=\supp(\sigma)$, then the infimum is attained at a unique $\bar x\in \Me^{++}$
such that 
\[
\sigma^{1/2}\bar x^{-1}\sigma^{1/2}=\sigma^{\frac{\gamma-1}{2\gamma}}\left(\sigma^{\frac{1}{2\gamma}}\rho\sigma^{\frac{1}{2\gamma}}\right)^{1-\alpha}\sigma^{\frac{\gamma-1}{2\gamma}}
\]
\end{prop}

\begin{proof} 
By the properties of the $L_p$-norms, the infimum can be attained at a unique element in
$\Me^{++}$. Let $\bar x$ be such that the above equality holds, then  
\[
\bar
x=\sigma^{\frac{1}{2\gamma}}(\sigma^{\frac{1}{2\gamma}}\rho\sigma^{\frac{1}{2\gamma}})^{\alpha-1}\sigma^{\frac{1}{2\gamma}}.
\]
It is easily checked that
\[
\tilde Q_\gamma(\sigma^{1/2}\bar x^{-1}\sigma^{1/2}\|\sigma)=\tilde
Q_\alpha(\rho\|\sigma)=\Tr \bar x\rho,
\]
so that the infimum is attained.

\end{proof}


\begin{theorem}\label{thm:dpi}
Let $\Phi$ be a 2-positive trace preseving map. Then for $\alpha\in [1/2,1)$,
$\tilde D_\alpha(\Phi(\rho)\|\Phi(\sigma))\le \tilde D_\alpha(\rho\|\sigma)$.
\end{theorem}

\begin{proof} Assume first that $\alpha=1/2$. Then we obtain from \eqref{eq:sandw_var}
that for any $y\in \Ne^{++}$,
\begin{align*}
\tilde Q_{1/2}(\rho\|\sigma)&=\frac12\inf_{x\in \Me^{++}} \Tr \rho
x+\Tr\sigma x^{-1}\le \frac12(\Tr\rho\Phi^*(y) +\Tr\sigma \Phi^*(y)^{-1})\\
&\le \frac12(\Tr\Phi(\rho)x+ \Tr\Phi(\sigma)y^{-1}),
\end{align*}
the second inequality follows by the  Choi inequality $\Phi^*(y)^{-1}\le \Phi^*(y^{-1})$
for a unital positive map. Taking the infimum over all $y\in \Ne^{++}$ implies the result.

Let us now assume that $\alpha\in (1/2,1)$. Then, similarly as above, for any $y\in \Ne^{++}$ we have

\begin{align*}
\tilde Q_\alpha(\rho\|\sigma)&\le \alpha \Tr \rho\Phi^*(y)+(1-\alpha)\tilde
Q_\gamma(\sigma^{1/2}\Phi^*(y)^{-1}\sigma^{1/2}\|\sigma)\\
&\le \alpha\Tr\Phi(\rho)y+(1-\alpha)\tilde
Q_\gamma(\sigma^{1/2}\Phi^*(y^{-1})\sigma^{1/2}\|\sigma)\\
&= \alpha\Tr\Phi(\rho)y+(1-\alpha)\tilde
Q_\gamma(\Phi_\sigma(\Phi(\sigma)^{1/2}y^{-1}\Phi(\sigma)^{1/2})\|\sigma)\\
&\le \alpha\Tr\Phi(\rho)y+(1-\alpha)\tilde
Q_\gamma(\Phi(\sigma)^{1/2}y^{-1}\Phi(\sigma)^{1/2}\|\Phi(\sigma))
\end{align*}
Here the second inequality follows by the Choi inequality and the fact that $\tilde
Q_\gamma$ is nondecreasing in the first variable,  the next equality is by definition of
the Petz recovery map. The third inequality is by the fact that
$\Phi_\sigma(\Phi(\sigma))=\sigma$ and monotonicity of $\tilde Q_\gamma$ under positive
maps. Taking infimum over $y\in \Ne^{++}$ we get the result.



\end{proof}

\begin{theorem}\label{thm:suff_equality} Let $\supp (\rho)\le\supp(\sigma)$ and let $\alpha\in
(1/2,1)$. Then
\[
\tilde Q_\alpha(\rho\|\sigma)=\tilde Q_\alpha(\Phi(\rho)\|\Phi(\sigma))\iff \Phi\text{ is
sufficient with respect to }\{\rho,\sigma\}.
\]
\end{theorem}

\begin{proof} We will prove this for the case when $\supp(\rho)=\supp(\sigma)$. Then also
$\supp(\Phi(\rho))=\supp(\Phi(\sigma))$, so we may assume that all the states are
faithful. The infima in the variational expressions are
attained at some (unique) $\bar x\in \Me^{++}$ and  $\bar y\in \Ne^{++}$. 
\[
\tilde Q_\alpha(\Phi(\rho)\|\Phi(\sigma))=\alpha\Tr \rho\Phi^*(\bar y)+(1-\alpha)\tilde
Q_\gamma(\Phi(\sigma)^{1/2}\bar y^{-1}\Phi(\sigma)^{1/2}\|\Phi(\sigma)).
\]
Inserting $\bar y$ into the chain of (in)equalities in the proof of Theorem \ref{thm:dpi},
we obtain that all the inequalities must be equalities. This implies that the infimum for
in the variational expression for $\tilde Q(\rho\|\sigma)$ is attained at $\Phi^*(\bar
y)$, so that we must have $\bar x=\Phi^*(\bar y)$. We also hav the following chain of
equalities:
\begin{align}
\tilde
Q_\gamma(\sigma^{1/2}\Phi^*(\bar y)^{-1}\sigma^{1/2}\|\sigma)&=\tilde
Q_\gamma(\sigma^{1/2}\Phi^*(\bar y^{-1})\sigma^{1/2}\|\sigma)=\tilde
Q_\gamma(\Phi_\sigma(\Phi(\sigma)^{1/2}\bar y^{-1}\Phi(\sigma)^{1/2})\|\sigma)\notag \\&=\tilde
Q_\gamma(\Phi(\sigma)^{1/2}\bar
y^{-1}\Phi(\sigma)^{1/2}\|\Phi(\sigma))\label{eq:equalities}
\end{align}
From $\bar x=\Phi^*(\bar y)$ and the first equality, we get (by the properties of the $L_p$-norms (Fack-Kosaki
inequality), that
\[
\mu:= \sigma^{1/2}\bar x^{-1}\sigma^{1/2}=\sigma^{1/2}\Phi^*(\bar
y^{-1})\sigma^{1/2}=\Phi_\sigma(\nu),
\]
where $\nu:=\Phi(\sigma)^{1/2}\bar y^{-1}\Phi(\sigma)^{1/2}$. But we also have
\[
\tilde Q_\gamma(\nu\|\Phi(\sigma))=\tilde Q_\alpha(\Phi(\rho)\|\Phi(\sigma))=\tilde
Q_\alpha(\rho\|\sigma)=\tilde Q_\gamma(\mu\|\sigma)=\tilde
Q_\gamma(\Phi_\sigma(\nu)\|\Phi_\sigma(\Phi(\sigma))).
\]
Since $\gamma>1$, this implies that $\Phi_\sigma$ is sufficient with respect to the pair
$\{\nu,\Phi(\sigma)\}$. Now note that the Petz dual of $\Phi_\sigma$ with respect to
$\Phi(\sigma)$ is $\Phi$ itself, we obtain that $\Phi(\Phi_\sigma(\nu))=\nu$. Hence
\[
\Phi_\sigma\circ \Phi(\mu))=\Phi_\sigma\circ \Phi\circ
\Phi_\sigma(\nu)=\Phi_\sigma(\nu)=\mu,
\]
so that $\Phi$ is sufficient with respect to $\{\mu,\sigma\}$. 
\end{proof}



















\end{document}



