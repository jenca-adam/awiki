\documentclass[12pt]{article}

\usepackage{amsfonts,amsmath}

\begin{document}
\noindent
\textbf{Problem 6.18:} Let $M$ be a von Neumann algebra and $M_0\subseteq M$ a subalgebra. Let 
 $\varphi_1,\varphi_2$ be  normal semifinite weights, we assume that  $\varphi_2$ is faithful.
Let $E: M\to M_0$ a faithful normal conditional expectation onto $M_0$.
We show that the equality
\begin{equation}
[D(\varphi_1\circ E):D(\varphi_2\circ E)]_t=[D\varphi_1:D\varphi_2]_t,\qquad \forall t\in \mathbb R\tag{6.36}
\end{equation}
holds. 

It is known that (6.36) always holds if $\varphi_1$ is faithful. In general, let $e=s(\varphi_1)(\in M_0)$ and let 
$\varphi_0$ be a semifinite weight on $M_0$ such that $s(\varphi_0)=1-e$. Then $\varphi=\varphi_0+\varphi_1$
is a faithful normal semifinite weight on $M_0$, such that $\varphi_1=\varphi(e\cdot)=\varphi(\cdot e)$ and we have
\[
[D\varphi_1:D\varphi_2]_t= e[D\varphi:D\varphi_2]_t\qquad t\in \mathbb R.
\]
We next show that we also have $s(\varphi_1\circ E)=e$. Indeed, let $q=1-s(\varphi_1\circ E)$, then $q$ is the largest
projection in $M$ such that $\varphi_1\circ E(q)=0$. Since 
$\varphi_1\circ E(1-e)=\varphi_1(1-e)=0$, we obtain $1-e\le q$. On the other hand, since $e$ is the support projection
of $\varphi_1$ and $\varphi_1(E(q))=0$, we obtain $E(q)\le 1-e\le q$.  Since $E$ is  
faithful, this implies $q=E(q)=1-e$, so that also   $s(\varphi_1\circ E)=e$.

The last equality implies that $\varphi_1\circ E=\varphi\circ E(e\cdot)=\varphi\circ E(\cdot e)$, so that  we have for
all $t\in \mathbb R$:
\[
[D(\varphi_1\circ E):D(\varphi_2\circ E)]_t=e[D(\varphi\circ E):D(\varphi_2\circ
E)]_t=e[D\varphi:D\varphi_2]_t=[D\varphi_1:D\varphi_2]_t
\] 
\end{document}

