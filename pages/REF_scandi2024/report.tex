\documentclass[12pt]{article}
\usepackage{geometry}
\usepackage{amsfonts}
\geometry{total={210mm,290mm},
 left=23mm,right=23mm,%
 bindingoffset=0mm, top=20mm,bottom=20mm}





\begin{document}
\begin{center}
{\large  M. Scandi, P. Abiuso, D. De Santis, and J. Surace, Physicality of evolution and statistical contractivity are equivalent notions of maps

}

\end{center}
\medskip

\centerline{Referee report}

\bigskip

The Fisher information is an important information measure, that can be seen as a local
distinguishability measure on probability distributions/quantum states. Both classical and
quantum versions have a wide range of applications, though in the quantum case, only few
of the large family  of QFI's are used. It is therefore of importance to understand the structure and
properties of QFI. 

The family of QFI is selected among all  Riemannian metrics on the manifold of
density matrices as those that are contractive under physical maps, that is, quantum
channels. The present paper contributes to the study of QFI by proving that also
conversely,  any metric in
the family will select the physical maps among all ''reasonable'' linear maps, as those
that cannot increase the metric. 

This is an interesting observation, though it is based on a very simple idea, namely that
while any QFI is bounded on any closed line segment contained in the interior of the
positive cone, it diverges to infinity when approaching the boundary. Basically the same  result is also
contained in a recent longer preprint [14], where the proof is also given. 
The authors do not give any further applications of their observation, though there are
some lines in the concluding section. In [14], the result is used in the study of
Markovianity of evolutions. 




\medskip

\noindent
\textbf{Overall evaluation}

I am not entirely convinced that this paper contains enough significant new esults to be published,
though the observations are of potential interest in studying quantum evolutions or in
quantum foundations. It may also be redundant, since it does not go beyond the results
contained in a recent preprint by the same authors. It is also not well written, in any
case it should be revised before publication. See the remarks below.

\medskip
 

\medskip

\noindent
\textbf{Some specific comments and suggestions}

\begin{enumerate}
\item The family of quantum Fisher informations was characterized by Petz in [23], but the
fact that contrary to the classical case, a monotone metric is not unique in the
quantum case was observed  by Chentsov and Morozova (Chentsov, N. N. and Morozova, E. A.
(1990). Markov invariant geometry on state manifolds, Itogi Nauki i Tekhniki 36, 69–102).
In fact, to my knowledge this was the first paper where the approach to QFI as a monotone
Riemannian metric was considered. It may be appropriate to cite this paper. 

\item Corollary 3.2: the condition  Eq.(13) should be required for states such that
$\Phi(\rho),\Phi(\sigma)\in \mathcal S^\circ_d$. (Otherwise the condition already
presupposes that $\Phi$ is positive, since $H_g$ is defined only on positive matrices).

\item Proof of Theorem 3 is not well written and confusing, especially the part after
''Then, choose a
perturbation...'' is hard to understand and needs to be revised. As far as I can see from Eq. (18), the only
thing that is needed is to find some matrix $\delta\rho_\eta$ for each $\eta$ such that
$|\langle\psi_\eta|\Phi(\delta\rho_\eta)|\psi_\eta\rangle|> \alpha$ for some $\alpha>0$ (the
requirement that it is ''positive and finite'' is rather puzzling for me). Then why not
take $\delta\rho_\eta=\pi$ for all $\eta$? Since by the assumption $\Phi(\pi)$ is positive
definite, $\langle \psi|\Phi(\pi)|\psi\rangle$ is bounded from below by the smallest
eigenvalue of $\Phi(\pi)$, for any unit vector $|\psi\rangle$. 

\item Below Eq.(19), the sentence ''...the assumption that $\Phi$ contracts the
Fisher metric for any two points in $\mathcal P^\circ_d$'' is strange, since the
requirement is for any point in $\mathcal P^\circ_d$ that is mapped to $\mathcal
P^\circ_d$, and any tangent vector (that means for any hermitian matrix). So why ''any two
points''?



\end{enumerate}








\end{document}

