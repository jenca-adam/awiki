\documentclass[12pt]{article}
\usepackage{geometry}
\usepackage{amsfonts}
\geometry{total={210mm,290mm},
 left=23mm,right=23mm,%
 bindingoffset=0mm, top=20mm,bottom=20mm}





\begin{document}
\begin{center}
{\large A. {\L}uczak: Strong subadditivity of Segal’s entropy}

\end{center}
\medskip

\centerline{Referee report}

\bigskip

This paper deals with trace preserving conditional expectations on a semifinite von
Neumann algebra, forming a commuting square. Such commuting squares are characterized
using the Segal entropy of normal states. This extends a result of Ref. [1], where this
characterization was proved in the finite dimensional case. One of the consequences of
this characterization is the strong subadditivity of quantum entropy.

The proofs in the present paper closely follow those in [1] and the techniques necessary
in infinite dimensions are quite standard manipulations, based on the previous
results of the author concerning Segal's entropy.  On the other hand, various versions of 
SSA related to entropic uncertainty relations and their extensions were recently intensely studied in quantum
information theory, but most of the results are restricted to finite dimensions. An extension to arbitrary semifinite
von Neumann algebras is therefore timely and valuable.


\medskip


\noindent
\textbf{Some comments}


\begin{enumerate}
\item The title is somewhat misleading. The SSA is obtained as a consequence in the case
of finite von Neumann algebras and the reader is directed to Ref. [7] for the general semifinite
case.
\item page 1, last line: ''normal states'' - better stress that the ''states'' here are
not assumed to be normalized.

\item page 3, lines 5-6: ''...nonnegative for a normalised state and finite trace''. In
fact, in finite dimensions, the usual trace is finite and the expression for the entropy
as given in line 3 is nonpositive for normalized states. It seems that the author means
that the trace is also normalized, hence a tracial state.

\item page 4, Lemma 2: it would be better to add that  x is affiliated with M.

\item page 7, line 4 from below: Ref. [2] refers to an unpublished preprint by the present
author, with no further information. The preprint should be made available, or another
reference should be given.

\item Thm. 7: Ref. [6] gives another equivalent condition for equality in monotonicity of
the relative entropy, namely that there is a recovery map, given by the Petz dual
(generalized conditional expectation). A form specific for SSA was given in 
(P. Hayden et al., Structure of states which satisfy strong subadditivity of quantum
entropy with equality, Commun. Math. Phys., 246(2):359-374, 2004). 
It might be useful to consider this condition in case of commuting squares.

\item Thm. 10: Better write the specific form of Eq. (8) for independent subalgebras.


\end{enumerate}








\end{document}

