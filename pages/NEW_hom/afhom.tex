\documentclass[12pt]{article}

\usepackage{hyperref}
\usepackage{amsmath, amssymb, amsthm, bm}
\usepackage[sort&compress,numbers]{natbib}
\usepackage{doi}
\usepackage[margin=0.8in]{geometry}
%\textheight23cm \topmargin-20mm  
%\textwidth175mm  
%\oddsidemargin=0mm
%\evensidemargin=0mm
%

\usepackage{amsmath, amssymb, amsthm, mathtools}

\newtheorem{lemma}{Lemma}
\newtheorem{theorem}{Theorem}
\newtheorem{coro}{Corollary}
\newtheorem{prop}{Proposition}


\theoremstyle{definition}
\newtheorem{defi}{Definition}


\theoremstyle{remark}
\newtheorem{remark}{Remark}

\def\type{\mathrm {setting}}
\def\ii{\bm{\emptyset}}
\def\tt{\bm{0}}
\def\Ce{\mathcal C}
\def\Ne{\mathcal N}
\def \Tr{\mathrm{Tr}\,}
\def\Se {\mathcal S}
\def\supp{\mathrm{supp}}
\def\<{\langle\,}
\def\>{\,\rangle}
\def \BS{\mathrm{BS}}
\def \Afh{\mathrm{AfH}}
\def \Af{\mathrm{Af}}
\def \FV{\mathrm{FinVect}}
\def\bV{\mathbf V}
\def\bW{\mathbf W}
\def\bE{ E}
\def\bI{I}

\def\bX{ X}
\def\bQ{\mathbf Q}
\def\bY{ Y}
\def\bZ{Z}

\title{On the category of affine subspaces}
\author{Anna Jen\v cov\'a}

\begin{document}

\maketitle

We present some important categories.

\subsection{The category $\FV$}

Let  $\FV$ be the category of finite dimensional real vector spaces with linear maps. Then
$(\FV,\otimes, \mathbb R)$ is a symmetric monoidal category, with the usual tensor product
of vector spaces. With the usual duality $(-)^*: V\mapsto V^*$  of vector spaces, $\FV$ is
compact closed.  Put
\[
e_U: U\otimes U^*\to \mathbb R,\qquad e_U(u\otimes u^*)=\<u^*,u\>,
\]
then $e_U$ is the cap (counit) for the duality of $U$ and $U^*$. The unit (cup) of the
duality is 
\[
\eta_U: \mathbb R\to U^*\otimes U, \qquad \eta_U(1)=\sum_i e_i^*\otimes e_i,
\]
where $\{e_i\}$ is a basis of $U$ and $\{e_i^*\}$ the dual basis of $U^*$, determined by
$\<e_i^*,e_j\>=\delta_{ij}$. It is easily verified that $\eta_U$ does not depend on the
choice of the basis $\{e_i\}$. We will identify $\eta_U:=\eta_U(1)$. Note that $\eta_U$ is
just $e_U$ seen as an element in $(U\otimes U^*)^*=U^*\otimes U$.

By compactness, the internal hom is  $[U,V]=U^*\otimes V$. Note that we have the usual identification 
\begin{equation}\label{eq:fvhoms}
\<f(u),v^*\>=\<w,u\otimes v^*\>,\qquad u\in U,\
v^*\in V^*
\end{equation}
between maps $f:U\to V$ and elements $w\in U^*\otimes V=[U,V]$, so the internal hom is
identified with  the set of all morphisms $U\to V$.  The above relation of $f$ and $w$ can be also
written as
\[
f(u)=(e_U\otimes V)(u\otimes w),\qquad 
w=(f^*\otimes V)(\eta_V),
\]
(we write $V$ for the identity map  $id_V$). Here $f^*$ is the usual adjoint map
$f^*:V^*\to U^*$.  

The composition $\circ$ defines a map
\[
[U,V]\otimes [V,W]\to [U,W],
\]
using the identification $[U,V]=U^*\otimes V$, $\circ$ is identified with 
\[
U^*\otimes V\otimes V^*\otimes W\xrightarrow{U^*\otimes e_V\otimes W} U^*\otimes W.
\]
Using symmetry and $e_V$, we can also define partial compositions through $V$ in an obvious way.
This can be depicted graphically in a nice way.


\subsection{Affine subspaces}

A subset $A\subseteq V$ of a finite dimensional vector space $V$ is an affine subspace if 
$\sum_i\alpha_i a_i\in A$ whenever all $a_i\in A$ and $\sum_i\alpha_i=1$. We say that $A$
is proper if $0\ne A$ and $A\ne \emptyset$. We will always
mean  that an affine subspace is proper (if not explicitly stated otherwise).

An affine
subspace can be determined in two ways:
\begin{enumerate}
\item[(i)] Let $L\subseteq V$ be a linear subspace and $a_0\ne L$. Then 
\[
A=a_0+L
\]
is a proper affine subspace.  Note that $a_0\in A$ and $A\cap L=\emptyset$.
Conversely, any proper affine subspace $A$ can be given in this way, with $a_0$ an arbitrary element in $A$ and
\[
L=Lin(A):=\{a_1-a_2,\ a_1,a_2\in A\}=\{a-a_0,\ a\in A\}.
\]
\item[(ii)] Let $S\subseteq V$ be a linear subspace and $a_0^*\in V^*\setminus S^\perp$. Then
\[
A=\{a\in S, \<a_0^*,a\>=1\}
\]
is a proper affine subspace. Conversely, any proper affine subspace $A$ is given in this way, with
$S=span(A)$ and $a_0^*$ an arbitrary element in 
\[
\tilde A=\{a^*\in V^*,\ \<a^*,a\>=1,\ \forall a\in A\}.
\]
\end{enumerate}

For an affine subspace $A$, $\tilde A$ is an affine subspace as well. If $A$ is proper,
then $\tilde A$ is proper and we have
$\tilde{\tilde A}=A$. More generally, if $\emptyset\ne C\subseteq A$ is any  subset of a
proper  affine
subspace $A$, then $\tilde C$ is a proper affine subspace 
and $\tilde{\tilde C}$ is the smallest
affine subspace containing $C$, that is,
\[
\tilde{\tilde C}=\{\sum_i \alpha_i c_i,\ c_i\in C,\ \sum_i\alpha_i=1\}.
\]
In this case, we may write $\tilde{\tilde C}$ as
\[
\tilde{\tilde C}=c_0+Lin(C)=c_0+span(\{c_1-c_2,\ c_1,c_2\in C\})
\]
with an arbitrary element $c_0\in C$, or as
\[
\tilde{\tilde C}=\{c\in span(C),\ \<a_0^*,c\>=1\}
\]
for an arbitrary element $a_0^*\in \tilde A$. We clearly have
\[
Lin(\tilde C)=C^\perp=span(C)^\perp,\qquad Lin(C)=Lin(\tilde{\tilde C})=\tilde
C^\perp=span(\tilde C)^\perp
\]
and by duality also
\[
span(C)=C^{\perp\perp} =Lin(\tilde C)^\perp,\qquad span(\tilde C)=Lin(C)^\perp.
\]





\subsection{The category $\Af$}


The objects of $\Af$ are of the form $X=(V_X,A_X,a_X,\tilde a_X)$, where $V_X$ is in
$\FV$, $A_X\subseteq V_X$ an affine subspace, $a_X\in A_X$ and $\tilde a_X\in \tilde A_X$
are some elements. Morphisms $X\to Y$ are linear maps $f:V_X\to V_Y$ such that
$f(A_X)\subseteq A_Y$. Note that by definition $A_X$ is proper for any object $X$. We may
also add
two special objects: the initial object $\emptyset:=(\{0\}, \emptyset, -, 0)$ and the terminal
object $0:=(\{0\},\{0\},0,-)$, here the affine subspaces are obviously not proper.

For any object $X$, we also put 
\[
L_X:=Lin(A_X)\qquad S_X:=span(A_X),\qquad  d_X:=\dim(L_X),\qquad D_X:=\dim(V_X).
\]
Note that $X$ is
uniquely determined also when $A_X$ is replaced by $L_X$ or $S_X$. 


\subsubsection{Limits and colimits}

Limits and colimits should be obtained from those in $\FV$, we have to spectify the other
structures and check  whether the corresponding arrows are in $\Af$. 

First, note that $\{0\}$ is both initial and terminal in $\FV$. In $\Af$, it is easily
seen that $\emptyset$ is \textbf{initial} and $0$ is \textbf{terminal} in $\Af$. 

Let $\bX$, $\bY$ be two objects in $\Af$. Assume first that both are proper. We define
their \textbf{product} as
\[
\bX\times \bY:=(V_\bX\times V_\bY, A_\bX\times A_\bY, (a_X,a_Y), \frac12(\tilde a_X,\tilde
a_Y)),
\]
where 
\[
A_\bX\times A_\bY:=\{(x,y)\in V_\bX\times V_\bY,\ x\in A_\bX, y\in A_\bY\}
\]
is the direct product of $A_\bX$ and $A_\bY$. It is easily verified that this is indeed an
affine subspace and the usual projections $\pi_\bX:V_\bX\times V_\bY\to V_\bX$ and $\pi_\bY:V_\bX\times
V_\bY\to V_\bY$ are in $\Af$. Moreover, for $f:\bZ\to \bX$  and $g:\bZ\to \bY$, the map
$f\times g(z)=(f(z),g(z))$ is also clearly a morphism $\bZ\to \bX\times\bY$  in $\Af$. 
We have
\[
L_{X\times Y}=L_X\times L_Y,\qquad S_{X\times Y}= (S_X\times S_Y)\wedge \{(\tilde
a_X,-\tilde a_Y)\}^\perp.
\]
We put $X\times \emptyset=\emptyset$, with the unique morphisms $\pi_X:\emptyset \to X$
and $\pi_{\emptyset}:\emptyset \to \emptyset$. If $Y\xrightarrow{f}X$ and
$Y\xrightarrow{g} \emptyset$, then it is clear that $Y=\emptyset$, this shows that this is
indeed the product. Further, put  $X\times 0=X$, with $\pi_X=id_X$  and
$\pi_0: X\xrightarrow{!}0$. It is also readily verified that this is the product.


The \textbf{coproduct} for proper objects $X$, $Y$ is defined as 
\[
\bX\oplus \bY=(V_\bX\times V_\bY, A_\bX\oplus A_\bY,\frac12(a_X,a_Y), (\tilde a_X,\tilde
a_Y)),
\]
where 
\[
A_\bX\oplus A_\bY:=\{(tx,(1-t)y),\ x\in A_{\bX}, y\in A_\bY,\ t\in \mathbb R\}
\]
is the direct sum. To check that this is an affine subspace, let $x_i\in A_\bX$, $y_i\in A_\bY$, $s_i\in
\mathbb R$ and let $\sum_i\alpha_i=1$, then 
\[
\sum_i\alpha_i(s_ix_i,(1-s_i)y_i)=(\sum_is_i\alpha_ix_i,\sum_i(1-s_i)\alpha_iy_i)=(sx,(1-s)y)\in
A_\bX\oplus A_\bY,
\]
where $s=\sum_is_i\alpha_i$, $x=s^{-1}\sum_is_i\alpha_ix_i$ if $s\ne 0$ and is arbitrary
in $A_\bX$ otherwise, similarly $y=(1-s)^{-1}\sum_i(1-s_i)\alpha_iy_i$ if $s\ne 1$ and is
arbitrary otherwise. The usual embeddings  $p_\bX:V_\bX\to V_\bX\times V_\bY$ and $p_\bY:
V_{\bY}\to V_\bX\times V_\bY$ are easily seen to be morphsims in $\Af$.   

Let  $f:\bX\to \bZ$, $g:\bY\to \bZ$ be any morphisms in $\Af$ and consider the map
$V_\bX\times V_\bY\to \bZ$ given as
$f\oplus g(u,v)=f(u)+g(v)$. We need to show that it preserves the affine subspaces. So let 
$x\in A_\bX$, $y\in  A_\bY$, then since $f(x),g(y)\in A_\bZ$, we have for any $s\in \mathbb R$,
\[
f\oplus g(sx,(1-s)y)=sf(x)+(1-s)g(y)\in A_\bZ.
\]
We also have
\[
L_{X\oplus Y}= (L_X\times L_Y)\vee \mathbb R\{(a_X,-a_Y)\},\qquad S_{X\oplus Y}=S_X\times
S_Y.
\]
Similarly as in the case of products, it is verified that $X\oplus \emptyset=X$ and
$X\oplus 0=0$. (All the statements for coproducts can be obtained from duality defined
below).


One can also discuss equalizers and coequalizers, here we only note that these may be
trivial even for proper objects. We next consider pullbacks and pushouts, in some special
cases that will be needed below.


Let $X,Y$ be proper objects. We say that $X\xrightarrow{f} Y$ is an inclusion if the
corresponding arrow in $\FV$ is an isomorphism and $f(a_X)=a_Y$, $f^*(\tilde a_Y)=\tilde
a_X$.  Assume that $X,Y,Z_0,Z_1$ are proper
objects such that there are inclusions
\[
Z_0\xrightarrow{f_0} X\xrightarrow{f_1} Z_1,\qquad Z_0\xrightarrow{g_0} Y\xrightarrow{g_1}
Z_1.
\]
Note that in particular $\psi_0:=f_0\circ g_0^{-1}$ is an isomorphism of $V_Y$ onto $V_X$ and
$\psi_0(a_Y)=a_X$, $\psi_0^*(\tilde a_X)=\tilde a_Y$. 
The \textbf{pushout} of $f_0,g_0$ is the object $X\sqcup_{f_0,g_0} Y=(V_X,
A_{X\sqcup_{f_0,g_0} Y}, a_X,
\tilde a_X)$, with 
\[
A_{X\sqcup_{f_0,g_0} Y}=\{sa+(1-s)\psi_0(b),\ a\in A_X,\ b\in A_Y,\ s\in \mathbb R\},
\]
together with the inclusions given by the linear maps $id:V_X\to V_X$ and $\psi_0:V_Y\to
V_X$. Indeed, these are clearly morphisms in $\Af$, and we have 
\[
id\circ f_0=f_0=\psi_0\circ g_0.
\]
Also, if $X\xrightarrow{i} Z$ and $Y\xrightarrow{j} Z$ are such that $i\circ f_0=j\circ
g_0$, then $i=i\circ id$, $j=i\circ \psi_0$, so $i$ is a morphism $X\sqcup_{f_0,g_0} Y\to Z$,
obviously unique, with the required properties. We have
\[
L_{X\sqcup_{f_0,g_0}Y}=L_X\vee \psi_0(L_Y),\qquad S_{X\sqcup_{f_0,g_0}Y}=S_X\vee
\psi_0(S_Y).
\]
Similarly, it can be shown that the \textbf{pullback} of $f_1, g_1$ is
$X\sqcap_{f_1,g_1}Y=(V_X, A_X\cap \phi_1(A_Y),a_X,\tilde a_X)$ where $\phi_1=f_1^{-1}g_1$,
and the inclusions given by $id_X$ and $\phi_1^{-1}$.




%Let us turn to equalizers. So let $f,g:\bX\to \bY$ and let 
%\[
%V_E=\{v\in V_\bX,\ f(v)=g(v)\}. 
%\]
%Let $h: \bZ\to \bX$ equalize $f,g$, then $h(V_\bZ)\subseteq V_E$ and $h(A_\bZ)\subseteq
%A_\bX\cap V_E$, so that $A_\bX\cap V_E$ must be nonempty. In this case, 
%\[
%\bE=(V_E, A_E:=V_E\cap A_\bX, a_E,\tilde a_E:=\tilde a_X)
%\]
%with the inclusion map $V_E\hookrightarrow V_\bX$ is an
%equalizer of $f,g$ for any choice of $a_E\in A_E$ (note that choosing another $a_E$ gives
%us an isomorphic object in $\Af$). If the
%intersection $V_E\cap A_\bX$ is empty, then the only equalizing arrow 
%for $f$ and $g$ is $\emptyset\to \bX$, which is
%then the equalizer.
%
%For the coequalizer, let $V_Q$ be the quotient space $V_Q:=V_\bY|_{Im(f-g)}$ and let
%$q:V_\bY\to V_Q$
%be the quotient map. If  some $h: \bY\to \bZ$ coequalizes $f$ and $g$, then $h$ maps
%$Im(f-g)$ to 0, so that $Im(f-g)\cap A_\bY=\emptyset$, unless $\bZ$ is the terminal
%object. It is easily checked that if $Im(f-g)\cap A_\bY=\emptyset$, then
%\[
%Q=(V_Q,A_Q:=q(A_\bY),a_Q:=q(a_Y),\tilde a_Q )
%\]
%together with the quotient map  $q$ is the
%coequalizer of $f$ and $g$ for any choice of $\tilde a_Q\in \tilde A_Q$.
%If the intersection is nonempty,
%then the unique coeqalizing arrow is $\bY\to 0$, which is then the coequalizer.
%
%Let us mention pullbacks and pushouts. Since pullbacks can be obtained from products and
%equalizers, we see that we have a similar situation: if a pullback is ''well defined'',
%then it coincides with the pullback in $\FV$, otherwise it is trivial. More precisely, if 
%$f:\bX\to \bZ$ and $g:\bY\to \bZ$, then we put 
%\[
%V_P:=\{(x,y)\in V_\bX\times V_\bY, f(x)=g(y)\}.
%\]
%If $V_P\cap A_\bX\times A_\bY\ne \emptyset$, that is, there are some $x\in A_\bX$ and $y\in
%A_\bY$ such that $f(x)=g(y)$, then 
%\[
%(V_P,A_P:=(A_\bX\times A_\bY)\cap V_P, a_P, \frac 12(\tilde a_X,\tilde a_Y))
%\]
%with the two projections
%is a pullback of $f$ and $g$ for any choice of $a_P\in A_P$, otherwise the pullback is
%just the initial object $\emptyset$.
%
%Similarly, let $f:\bZ\to \bX$, $g:\bZ\to \bY$, then let $V_Q$ be the quotient of
%$V_\bX\times V_\bY$ by the subspace
%\[
%\{(f(z),-g(z)),\ x\in V_\bZ\}.
%\]
%If this subspace does not contain any element of $A_\bX\oplus A_\bY$, that is, there is no
%$z\in V_\bZ$ such that for some $t\in \mathbb R$,
%\[
%f(tz)\in A_\bX,\qquad g((t-1)z)\in A_\bY,
%\]
%then 
%\[
%Q=(V_Q, A_Q:=q(A_\bX\oplus A_\bY), \frac12 q(a_X,a_Y), \tilde a_Q)
%\]
%with maps $x\mapsto q(x,0)$ and $y\mapsto q(0,y)$ is the
%pushout of $f$ and $g$. Otherwise the pushout is just $0$.
%
\subsubsection{Tensor products}

Let $X$, $Y$ be objects in $\Af$. Let us define
\[
A_{X\otimes Y}:=\{x\otimes y, x\in A_X, y\in A_Y\}^{\approx}.
\]
In other words, $A_{X\otimes Y}$ is the smallest affine subspace in $V_X\otimes V_Y$ containing
$A_X\otimes A_Y$. 
For proper objects, we  have
\begin{align}
L_{X\otimes Y}&=Lin(A_X\otimes A_Y)=span(\{x\otimes y-a_X\otimes a_Y,\ x\in A_X,\ y\in
A_Y\})\notag\\
&= (a_X\otimes L_Y)+ (L_X\otimes a_Y)+ (L_X\otimes L_Y)\label{eq:lxy}
\end{align}
(here $+$ denotes the direct sum of subspaces). We also have
\[
S_{X\otimes Y}=S_X\otimes S_Y.
\]

\begin{proof} Let $x\in A_X$, $y\in A_Y$, then
\[
x\otimes y-a_X\otimes a_Y=a_X\otimes (y-a_Y)+(x-a_X)\otimes a_Y+(x-a_X)\otimes (y-a_Y),
\]
so that $L_{X\otimes Y}=Lin(A_X\otimes A_Y)$ is contained in the subspace on the RHS of \eqref{eq:lxy}.
Let $d$ be the dimension of this subspace, then clearly
\[
d_{X\otimes Y}\le d\le d_X+d_Y+d_Xd_Y.
\]
On the other hand, any element of $S_X$ has the form $tx$ for some $t\in \mathbb R$ and
$x\in A_X$, so that it is easily seen that $S_X\otimes S_Y=S_{X\otimes Y}$. 
Hence 
\begin{align*}
d_{X\otimes Y}&=\dim(L_{X\otimes Y})=\dim(S_{X\otimes
X})-1=\dim(S_X)\dim(S_Y)-1=(d_X+1)(d_Y+1)-1\\
&=d_X+d_Y+d_Xd_Y.
\end{align*}
This completes the proof.

\end{proof}



For $X,Y$ in $\Af$, put 
\[
X\otimes Y:=(V_X\otimes V_Y,A_{X\otimes Y},a_X\otimes a_Y, \tilde
a_X\otimes \tilde a_Y).
\]
Note that by this definition, $X\otimes \emptyset =\emptyset$ and
$X\otimes 0=0$ unless $X=\emptyset$, in which case $\emptyset\otimes 0=\emptyset$. 


Also let $I:=(\mathbb R, \{1\},\{1\},\{1\})$. Then $(\Af,\otimes, I)$ is a symmetric
monoidal category. We only have to check that the associators, unitors and symmetries from
$\FV$ are morphisms in $\Af$. Indeed, let $\alpha_{X,Y,Z}:V_X\otimes (V_Y\otimes V_Z)\to
(V_X\otimes V_Y)\otimes V_Z$ be the associator in $\FV$. We need to check that
$\alpha_{X,Y,Z}(A_{X\otimes(Y\otimes Z)})\subseteq A_{(X\otimes Y)\otimes Z}$. It is easily
checked that $A_{X\otimes(Y\otimes Z)}$ is the affine span of elements of the form
$x\otimes (y\otimes z)$, $z\in A_X$, $y\in A_Y$ and $z\in A_Z$, and we have
\[
\alpha_{X,Y,Z}(x\otimes (y\otimes z))=(x\otimes y)\otimes z\in A_{(X\otimes Y)\otimes Z}
\]
for all such elements. The desired inclusion follows by linearity.
The proof of the other inclusions is similar.




\subsubsection{Duality}

We define $X^*:=(V_X^*,\tilde A_X,\tilde a_X,a_X)$. Note that we have
\[
L_{X^*}=S_X^\perp,\qquad S_{X^*}=L_X^\perp,'\qquad d_{X^*}=D_X-d_X-1.
\]
It is easily seen  that $(-)^*$ defines a full and faithful functor $\Af^{op}\to \Af$,
moreover, $X^{**}=X$ (if we us the canonical identification of  any $V$ in $\FV$ with its second dual). 


\begin{theorem} $(\Af,\otimes,I)$ is a *-autonomous category, with duality $(-)^*$, such
that $I^*=I$.

\end{theorem}


\begin{proof} We only need to check the natural isomorphisms 
\[
\Af(X\otimes Y,Z^*)\simeq \Af(X,(Y\otimes Z)^*).
\]
Since $\FV$ is compact, we have the natural isomorphisms
\[
\FV(V_X\otimes V_Y,V^*_Z)\simeq \FV(V_X,V_Y^*\otimes V_Z^*),
\]
determined by the equalities
\[
\<f(x\otimes y),z\>=\<h(x),y\otimes z\>,\qquad x\in V_X,\ y\in V_Y,\ z\in V_Z,
\]
for $f\in \FV(V_X\otimes V_Y,V_Z^*)$ and $h\in \FV(V_Z,V_Y^*\otimes V_Z^*)$. Since
$A_{X\otimes Y}$ is an affine span of $A_X$ and $A_Y$, we see that
$f$ is in $\Af$ if and only if $f(x\otimes y)\in \tilde A_Z$, that is, 
\[
1=\<f(x\otimes y),z\>=\<h(x),y\otimes z\>\qquad \forall x\in A_X,\
y\in A_Y,\ z\in A_Z.
\]
But this is equivalent to
\[
h(x)\in (A_Y\otimes A_Z)^\sim=\tilde A_{Y\otimes Z},\qquad \forall x\in A_X,
\]
which means that $h\in \Af$.

\end{proof}

\subsubsection{The dual tensor product}

Let us define the dual tensor product by $\odot$, that is
\[
X\odot Y=(X^*\otimes Y^*)^*.
\]
We then have
\begin{align*}
L_{X\odot Y}&=S^\perp_{X^*\otimes Y^*}=(S_{X^*}\otimes S_{Y^*})^\perp=(L_X^\perp\otimes
L_Y^\perp)^\perp\\
S_{X\odot Y}&= L_{X^*\otimes Y^*}^\perp=(\tilde a_X\otimes
S_Y^\perp)^\perp\wedge(S_X^\perp\otimes \tilde a_Y)^\perp\wedge (S_X^\perp\otimes
S_Y^\perp)^\perp
\end{align*}
In particular,
\[
d_{X\odot Y}=D_Xd_Y+d_XD_Y-d_Xd_Y.
\]
\begin{lemma}\label{lemma:tensors} Let $X$, $Y$ be nontrivial. Then $X\otimes Y=X\odot Y$
if and only if $D_X=d_X+1$ and $D_Y=d_Y+1$.

\end{lemma}

\begin{proof} It is easy to see that (when identifying $X=X^{**}$), we have $A_X\otimes
A_Y\subseteq \tilde A_{X^*\otimes  Y^*}$, hence $A_{X\otimes Y}\subseteq A_{X\odot Y}$. We see from the above computatons that
\[
d_{X\odot Y}-d_{X\otimes Y}=d_X(D_Y-1)+(D_X-1)d_Y-2d_Xd_Y\ge 0,
\]
with equality if and only if the conditions of the lemma hold.

\end{proof}


\subsubsection{Internal hom}

The internal hom has the form
\begin{equation}\label{eq:ihom}
[X,Y]=(X\otimes Y^*)^*=X^*\odot Y.
\end{equation}
We then have 
\[
L_{[X,Y]}=(S_X\otimes L_Y^\perp)^\perp,\qquad S_{[X,Y]}=(\tilde a_X\otimes
S_Y^\perp)^\perp\wedge(L_X\otimes \tilde a_Y)^\perp\wedge (L_X\otimes
S_Y^\perp)^\perp
\]
and
\[
d_{[X,Y]}=D_XD_Y-(d_X+1)(D_Y-d_Y).
\]


As we have seen in $\FV$, the space $V_{[X,Y]}=V_X^*\otimes V_Y$ is identified with the
space of all linear maps $V_X\to V_Y$, by \eqref{eq:fvhoms}. We will show that $A_{[X,Y]}$ corresponds to the
affine subspace of maps mapping $A_X$ into $A_Y$, that is, morphisms in $\Af$. Indeed, we see from \eqref{eq:fvhoms}
that $f$ is in $\Af$ if and only if 
\[
\<f(x),y^*\>=\<w,x\otimes y^*\>=1, \qquad x\in A_X,\ y^*\in \tilde A_Y,
\]
which is equivalent to $w\in (A_X\otimes \tilde A_Y)^\sim=\tilde A_{X\otimes Y^*}$. 

\subsubsection{Dualizable (nuclear) objects}

An object in $\Af$ is nuclear if the natural map $X^*\otimes X\to [X,X]$ is an isomorphism
(santocanale). That is, the inclusion $X^*\otimes X\subseteq X^*\odot X$ that comes from the embedding 
\[
\tilde A_X\otimes A_X\subseteq (A_X\otimes \tilde A_X)^\sim
\]
becomes an equality. As we have seen in Lemma \ref{lemma:tensors}, for proper objects we
have $X^*\otimes X=X^*\odot X$ if and only if
\[
d_X+1=D_X=D_{X^*}=d_{X^*}+1=D_X-d_X.
\]
It follows that $d_X=0$ and $D_X=1$, so that $X\simeq I$. Hence the tensor unit is the
unique dualizable (or nuclear) object in $\Af$.


\subsection{The category $\Afh$} 


The category $\Afh$ will be constructed as a subcategory in $\Af$. 

\subsubsection{First order objects}

It is easily seen that the following are equivalent:
\begin{enumerate}
\item $D_X=d_X+1$;
\item $S_X=V_X$;
\item $L_X=\{\tilde a_X\}^\perp$;
\item $S_{X^*}=\mathbb R \tilde a_X$;
\item $L_{X^*}=\{0\}$.
\end{enumerate}
We say that an object $X$ is first order if any of these conditions is fulfilled. We have
seen that for proper objects, $X\otimes Y=X\odot Y$ if and only if both $X$ and $Y$ are
first order. We also have

\begin{lemma}\label{lemma:firstorder} $X$ is first order if and only if  $X\otimes
X=X\odot X$.

\end{lemma}
`
\begin{lemma}\label{lemma:1ordertensor} Let $X$, $Y$ be first order, then $X\otimes Y$ is
first order.

\end{lemma}

\begin{proof} We have
\[
S_{X\otimes Y}=S_X\otimes S_Y=V_X\otimes V_Y=V_{X\otimes Y}.
\]

\end{proof}


\subsubsection{Channels}
A channel is an object $[X,Y]$ where  $X$ and $Y$ are first order. As we have seen,
\[
X^*\otimes Y\subseteq X^*\odot Y=[X,Y].
\]
If $X$ is first order, $\tilde A_X=\{\tilde a_X\}$ and the elements of $A_{X^*\otimes
Y}=\tilde a_X\otimes A_Y$ are identified with channels of the form
\[
f(x)=\<\tilde a_X,x\>y,\qquad x\in V_X,
\]
for some $y\in A_Y$. Such maps will be called replacement channels.

\begin{lemma}\label{lemma:channels} Let $X$, $Y$ be first order and let  $w\in V_X^*\otimes V_Y$. Then $w
\in A_{[X,Y]}$ if and only if
\[
\circ_Y :w_{X^*Y}\otimes \tilde a_Y\mapsto \tilde a_X.
\]
\end{lemma}

\begin{proof} Let $f:V_X\to V_Y$ be the map corresponding to $w$, then 
\[
\circ_Y(w\otimes \tilde a_Y)=(V_X^*\otimes e_{V_Y})(w\otimes \tilde a_Y)=\tilde a_Y\circ
f,
\]
where $\tilde a_Y\in V_Y^*$ is seen as a map $V_y\to \mathbb R$. So $\tilde a_Y\circ f:
V_X\to \mathbb R$ is an element in $V_X^*$. We know that $w\in A_{[X,Y]}$ iff
$f(A_X)\subseteq A_Y$, which is equivalent to  $\tilde a_Y\circ f(x)=1$ for all $x\in A_X$, so
that $\tilde a_Y\circ f\in \tilde A_X=\{\tilde a_X\}$, since $X$ is first order.  
\end{proof}

\begin{lemma}\label{lemma:mapsdual} Let $Y$ be first order and $w\in V_X^*\otimes V_Y$.
Then $w\in A_{[X,Y]}$ if and only if
\[
\circ_Y(w_{X^*Y}\otimes \tilde a_Y)\in \tilde A_X.
\]
Moreover, 
\[
\tilde A_{[X,Y]}=A_X\otimes \{\tilde a_Y\}.
\]


\end{lemma}

\begin{proof} Since $Y$ is first order, we have $A_{Y^*}=\tilde A_Y=\{\tilde a_Y\}$ and by
\eqref{eq:ihom}
\[
\tilde A_{[X,Y]}=A_{X\otimes Y^*}=A_X\otimes \{\tilde a_Y\}.
\]
As in the above proof, let $f:V_X\to V_Y$ be the map corresponding to $w$. Then $\tilde
a_Y\circ f\in V_X^*$ and $w\in A_{[X,Y]}$ iff $f(A_X)\subseteq A_Y$. This means that
\[
\tilde a_Y\circ f(x)=1,\qquad \forall x\in A_X,
\]
which means that $\tilde a_Y\circ f\in \tilde A_X$.

\end{proof}

\subsubsection{$\Afh$}

The category $\Afh$ is the full subcategory in $\Af$ created from first order objects by
taking tensor products and duals. We will add more later. We will use the notation
$V_{XY^*}$ for $V_X\otimes V^*_Y$, etc. 

Any object $X$ in $\Afh$ is created from first order objects $X_1,\dots, X_k$, so that
$V_X=\tilde V_{X_1}\otimes\dots\otimes \tilde V_{X_k}$, where $\tilde V_{X_i}$ is either
$V_{X_i}$ or $V^*_{X_i}$, $i=1,\dots,k$. We will next show that any object is a set of
channels that contains all replacement channels. 



\begin{prop}\label{prop:afh} Let $X$ be an object in $\Afh$. Then there are  first order
objects $Y_I$ and $Y_O$ and inclusions $f$, $g$ such that
\begin{equation}\label{eq:inclusions}
Y_I^*\otimes Y_O \xrightarrow{f} X\xrightarrow{g} [Y_I,Y_O].
\end{equation}


\end{prop}

\begin{proof} Let $X$ be first order, then since $I$ is first order,
\[
I^*\otimes X=I\otimes X \xrightarrow{\lambda_X} X\xrightarrow{\lambda^{-1}_X} I\otimes
X=I\odot X=[I,X].
\]
Clearly, $f=\lambda_X$ and $g=\lambda^{-1}_X$ are inclusions. Now assume that $Z$
satisfies \eqref{eq:inclusions} and let $X=Z^*$. Taking duals and composing with
symmetries, we get
\[
Y_O^*\otimes Y_I\xrightarrow{\sigma_{V_{Y,O}^*,V_{Y_I}}} Y_I\otimes Y_0^*=[Y_I\otimes
Y_O]^*\xrightarrow{g^*} X\xrightarrow{f^*} (Y_I^*\otimes
Y_O)^*\xrightarrow{\sigma_{V_{Y_I},V_{X_O}^*}} (Y_O\otimes Y_I)^*=[Y_O,Y_I].
\]
Since the compositions of $f^*$ and $g^*$ with symmetries are inclusions, we see that $X$
satisfies \eqref{eq:inclusions}.

Next, let $X_1$ and $X_2$ satisfy \eqref{eq:inclusions}  with some first order objects $Y_I^i$, $Y_O^i$ and
inclusions $f^i,g^i$, $i=1,2$, and let $X=X_1\otimes X_2$. We then have, using the
appropriate symmetries
\[
Y_I^1Y_I^2\otimes (Y_O^1Y_O^2)^*\xrightarrow{\sigma_{Y_I^2,Y_O^1}} Y_I^1\otimes
(Y_O^1)^*\otimes Y_I^2\otimes (Y_O^1)^*\xrightarrow{f^1\otimes f^2} X
\xrightarrow{g^1\otimes g^2} [Y_I^1,Y_O^1]\otimes
[Y_I^2,Y_O^2]\xrightarrow{\sigma_{Y_O^1,Y_O^2}}[Y_I^1Y_I^2,I_O^1Y_O^2].
\]
Perhaps the last arrow needs some checking, so lt us do it properly. We need to show that
for $w\in A_{[Y_I^1,Y_O^1]\otimes
[Y_I^2,Y_O^2]}$, we have $\sigma_{Y_O^1,Y_O^2}(w)\in A_{[Y_I^1Y_I^2,I_O^1Y_O^2]}$, but
this is clear using Lemma \ref{lemma:channels}.

\end{proof}

The pair $(Y_I,Y_O)$ for an object $X$ will be called the $\type$ of $X$. For objects of
the same $\type$ we may take
pullbacks and pushouts of the  corresponding inclusions.




Pullbacks are intersections, pushouts the affine mixture.

Channels into (from) products and coproducts

We define $\Afh$ as the full subcategory of $\Af$ containing all first order objects and
closed under (finite products,) duals and  tensor products . 



\end{document}

