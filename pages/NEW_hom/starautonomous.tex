\documentclass[12pt]{article}

\usepackage{hyperref}
\usepackage{amsmath, amssymb, amsthm}
\usepackage[sort&compress,numbers]{natbib}
\usepackage{doi}
\usepackage[margin=0.8in]{geometry}
%\textheight23cm \topmargin-20mm  
%\textwidth175mm  
%\oddsidemargin=0mm
%\evensidemargin=0mm
%

\usepackage{amsmath, amssymb, amsthm, mathtools}

\newtheorem{lemma}{Lemma}
\newtheorem{theorem}{Theorem}
\newtheorem{coro}{Corollary}
\newtheorem{prop}{Proposition}


\theoremstyle{definition}
\newtheorem{defi}{Definition}


\theoremstyle{remark}
\newtheorem{remark}{Remark}

\def\Ce{\mathcal C}
\def\Ne{\mathcal N}
\def \Tr{\mathrm{Tr}\,}
\def\Se {\mathcal S}
\def\supp{\mathrm{supp}}
\def\<{\langle\.}
\def\>{\.\rangle}

\title{Notes on *-autonomous categories}
\author{Anna Jen\v cov\'a}

\begin{document}

\maketitle


\section{Definitions of a *-autonomous category}


(nLab)
Let $(\Ce,\otimes,I)$ be a symmetric monoidal category.

\subsection{Definition 1}

\begin{enumerate}
\item[(i)] $(\Ce,\otimes, I)$ is closed; (the functor $-\otimes B$ has a right adjoint $[B,-]$. In fact, this defines a
functor  $[-,-]: C^{op}\times C\to C$ (internal hom) and the  isomorphism
\[
\Ce(A\otimes B,C)\simeq \Ce(A,[B,C])
\]
is natural in all 3 variables $A,B,C$. This follows by  Yoneda (nLab). The corresponding
map $\Ce(A\otimes B,C)\ni f\mapsto \hat f\in \Ce(A,[B,C])$ is the transpose of $f$.)

\item[(ii)] there is a dualizing object $0\in \Ce$, such that for every $A$ the transpose of $ev_{A,0}$ is an iso:
\[
\Ce([A,0]\otimes A,0)\simeq \Ce(A, [[A,0],0]),\qquad ev_{A,0}\mapsto \hat{ev}_{A,0}\equiv j_A
\]
Here  $ev_{A,B}: [A,B]\otimes A\to B$ is  the counit of the
adjunction in 2. Note that we used symmetry.

\end{enumerate}

\subsection{Definition 2} 

\begin{enumerate}
\item[(1)] A symmetric monoidal category $(\Ce,\otimes,I)$;
\item[(2)] $(-)^*: \Ce^{op}\to \Ce$ full and faithful functor such that 
\[
\Ce(A\otimes B,C)\simeq \Ce(A,(B\otimes C^*)^*) \qquad \text{(natural in } A,B,C),
\]
that is, $[B,C]\simeq (B\otimes C^*)^*$.

\end{enumerate}


\subsection{Equivalence}

{$1\to 2$:} \enspace
Define the contravariant endofunctor: $A^*=[A,0]$, then $j_A$ in (ii) is a natural
isomorphism $id_\Ce \to (-)^{**}$(?), so $(-)^*$ is full and faithful, all else is clear.

\medskip

\noindent 
$2\to 1$: \enspace
Define internal hom as $[A,B]:=(A\otimes B^*)^*$, then (i) holds. Define the dualizing object as $0=I^*$

\subsection{Compact closed categories}


An object $A\in \Ce$ is dualizable if there is some $A^*\in \Ce$ (dual object) and
morphisms $\cup_A: I\to A^*\otimes A$ and $\cap_A: A\otimes A^*\to I$ such that the snake
identities hold:
\[
(\cap_A\otimes A)\circ (A\otimes \cup_A)=A,\qquad (A^*\otimes
\cap_A)\circ (\cup_A\otimes A^*)=A^*.
\]

$\Ce$ is compact closed if every object is dualizable.
Equivalently, $\Ce$ is *-autonomous and there is an iso 
\[
(A\otimes B)^*\simeq A^*\otimes B^*,\qquad \text{natural in } A,B.
\]

\begin{lemma} Let $\Ce$ be *-autonomous and $I=I^*$. Let  $A$ be an object in $\Ce$. Then $A$ is dualizable
iff there is an iso
\[
(A\otimes B)^*\simeq A^*\otimes B^*,\qquad \text{natural in } A, B.
\]
\end{lemma}

\begin{proof} Assume $A$ is such that there is the iso. Then using the *-autonomous
structure, we have
\[
\Ce(A,A)\simeq \Ce(I\otimes A,A)\simeq \Ce(I,(A\otimes A^*)^*)\simeq \Ce(I,A^*\otimes A)
\]
so we may put $\cup_A\in \Ce(I,A^*\otimes A^*)$ as the morphism corresponding to the
identity $A$. Similarly, we use
\[
\Ce(A,A)\simeq \Ce(A,A\otimes I)\simeq \Ce(A,(A^*\otimes I)^*)\simeq\Ce(A\otimes A^*,I)
\]
to define $\cap_A$. In fact, take the subcategory consisting of  $I,A,A^*$ and all their
tensor products as objects and all morphisms, then we should get a compact closed
category, so that indeed $A$ is dualizable. Conversely, assume that $A$ is dualizable




\end{proof}

/////


\section{Symmetric monoidal categories (SMC)}

 \textbf{Monoidal category:} A category $C$ equipped with
\begin{itemize}
\item A functor $\otimes: C\times C\to C$;
\item unit object $I\in C$;
\item associator: natural iso $(a\otimes b)\otimes c \xrightarrow{\alpha_{a,b,c}} a\otimes
(b\otimes c)$;
\item left unitor: natural iso $I\otimes a \xrightarrow{\lambda_a} a$;


\item right unitor: natural iso $a\otimes I \xrightarrow{\rho_a} a$

\item \textbf{symmetric} if there is a symmetry: natural iso $a\otimes
b\xrightarrow{\sigma_{a,b}} b\otimes a$ such that $\sigma_{b,a}=\sigma_{a,b}^{-1}$, 

\end{itemize}
satisfying triangle, pentagon (+ symmetry) diagrams.

\medskip

We will always assume that $C$ is a SMC.


\subsection{Closed SMC}

A SMC $C$ is \textbf{closed} if:

\medskip %\vskip 3mm
for every $b\in C$, the endofunctor $-\otimes b$ has a
right adjoint $[b,-]$ (internal hom). 

\medskip
\noindent
What does this mean?

\begin{enumerate}
\item[(1)] For all $a,c\in C$, $C(a\otimes b,c)\simeq C(a,[b,c])$, naturally in $a,c$.
\item[(2)] unit $\eta^b_a: a\to [b,a\otimes b]$, counit: $\epsilon^b_a: [b,a]\otimes
b\to a$, natural transformations, triangle identities
\end{enumerate}
`
Relation of the two: 
\begin{itemize}
\item Let $i$ be the iso of (1): 
\begin{align*}
\eta^b_a\in C(a,[b,a\otimes b])&\simeq C(a\otimes b,a\otimes b),\qquad
\eta^b_a=i(id_{a\otimes b})\\
\epsilon^b_a\in C([b,a]\otimes b,a)&\simeq C([b,a],[b,a]),\qquad \epsilon^b_a=i^{-1}(id_{[b,a]}).
\end{align*}
\item Conversely, from $\eta^b$, $\epsilon^b$ of (2), we define $i$ as 
\[
g\in C(a\otimes b,c),\qquad i(g)=[b,g]\circ \eta^b_a
\]
with inverse
\[
h\in C(a,[b,c]),\quad i^{-1}(h)=\epsilon^b_c\circ (h\otimes b)
\]

\end{itemize}

\medskip
\noindent
Equivalently: a SMC $C$ is closed if and only if for all $b,c\in C$, there is an object
$[b,c]$ and an \textbf{evaluation map} $eval_{b,c}: [b,c]\otimes b\to c$ that has the
following \textbf{universal property}: for all $a\in C$ and $f:a\otimes b\to c$ there is a
unique $h:a\to [b,c]$ such that
\[
f=eval_{b,c}\circ(h\otimes b).
\]
The evaluation map is the counit $eval_{b,c}=\epsilon^b_c$ above.

\medskip
\noindent
Internal hom is a functor $[-,-]: C^{op}\times C\to C$ and the isomorphism in
(1) is natural in all 3 variables $a,b,c$. This follows by  Yoneda (nLab).


\subsection{Compact SMC}

A SMC is \textbf{compact} if each object $a\in C$ has a dual $a^*\in C$ such that there
are maps $\cup_a: I\to a^*\otimes a$ and $\cap_a: a\otimes a^*\to I$ satisfying the snake
identities
\[
(\cap_a\otimes id_a)\circ (id_a\otimes \cup_a)=id_a,\qquad (id_{a^*}\otimes
\cap_a)\circ (\cup_a\otimes id_{a^*})=id_{a^*}.
\]

The following are easily seen by the snake identities (pictures):
\begin{itemize}
\item[(1)] $a^*$ is determined up to iso;
\item[(1)] $I^*\simeq I$, by the isomorphisms 
\[
\rho_{I^*}\circ \cup_I: I\to I^*,\qquad \cap_I\circ \lambda^{-1}_{I^*}: I^*\to I;
\]
\item[(2)] $a^{**}\simeq a$, indeed, we may define $\cup_{a^*}: I\to a\otimes a^*$ and
$\cap_{a^*}: a^*\otimes a$ as
\[
\cup_{a^*}=\sigma_{a^*,a}\circ \cup_a,\qquad \cap_{a^*}=\cap_a\circ \sigma_{a^*,a},
\]
so that $a$ is dual to $a^*$, and use (1);
\item[(3)] if we fix $a^*$ and $\cup_a$ ($\cap_a$), then  $\cap_a$ ($\cup_a$) is uniquely
determined;
\item[(4)] any assignment $a\mapsto a^*$ defines a functor $C\to C^{op}$ (if $f:a\to b$,
we can use $\cup_a$ and $\cap_b$ to ''bend the wires'' to obtain a map $b^*\to a^*$, this
is obviously functorial);
\item[(5)] $(a\otimes b)^*\simeq a^*\otimes b^*$, we can clearly put (using symmetry)
\[
\cup_{a\otimes b}=\cup_a\otimes \cup_b,\qquad \cap_{a\otimes b}=\cap_a\otimes \cap_b
\]
\item[(5)] $C$ is closed, with $[b,c]=b^*\otimes c$: the iso $i: C(a\otimes b,c)\simeq
C(a,b^*\otimes c)$ can be obtain (roughly) as
\[
i(g)=g\circ \cup_b,\qquad i^{-1}(h)=\cap_b\circ h,
\]
the maps are inverses by the snake identities, naturality is obvious, since $i$ does
nothing on $a$ or $c$. The unit and counit of the adjunction are given as
\[
\eta^b_a=a\otimes \cup_b : a\to b^*\otimes a\otimes b,\qquad \epsilon^b_a=\cap_b\otimes a:
b^*\otimes a\otimes b\to a.
\]
\item[(6)] Can we state a theorem like:  $C$ is compact if and only if for each $b\in C$
there is some $b^*\in C$ such that $b^*\otimes -$ is the right adjoint of $-\otimes b$ and
...? What should be the additional conditions?

\end{itemize}


\section{Kleisli categories and monoidal monads} 

A \textbf{monad} on $C$ is a triple $(P,\eta,\mu)$, where:
\begin{itemize}
\item $P: C\to C$ is an endofunctor;

\item $\eta: Id_C\to P$, $\mu: P^2\to P$ are natural transformations satisfying some
triangles and squares.
\end{itemize}

\subsection{Kleisli categories}

The \textbf{Kleisli category} $C_P$ has the same objects as $C$, with morphisms:
\[
C_p(a,b)=C(a,P(b)),
\]
the identity $id_a=\eta_a$ and for $f\in C_p(a,b)$, $g\in C_p(b,c)$, the composition is defined as
\[
g\circ f:=\mu_c\circ P(g)\circ f.
\]
We have the following adjunction:
\begin{itemize}
\item the \textbf{left adjoint functor} $F_P: C\to C_P$ is defined as $a\mapsto a$ and for $f:a\to b$, we
put $F_P(f)\in C_P(a,b)=C(a,P(b))$ as $\eta_b\circ f$;

\item the \textbf{right adjoint functor} $G_P:C_P\to C$ is
given as $a\mapsto P(a)$ and for $f\in C_P(a,b)=C(a,P(b))$ we put $G_P(f)\in C(P(a),P(b))$
as $G_P(f)=\mu_b\circ P(f)$. 


\end{itemize}
This is indeed an adjunction, where the unit is given by $\eta$ and the counit is
determined as $\epsilon_a=id_{P(a)}\in C_P(P(a),a)$.


\subsection{Monoidal monads}

 A monad is \textbf{monoidal} \cite{seal2013tensors} if there are maps
 \[
\kappa_{a,b}:Pa\otimes Pb\to P(a\otimes b),\qquad a,b\in C,
 \]
 natural in $a,b$ and such that 
 \begin{itemize}
\item $(P,\eta, \kappa)$ is a \textbf{monoidal functor}, that is, some diagrams involving
$P$, $\alpha$, $\lambda$, $\rho$, $\kappa$ and $\eta$ commute;
\item additional diagrams containing $\mu$ commutes;
\item\textbf{symmetric}: additionally a diagram with $\sigma$ commutes. 

 \end{itemize}
A monoidal functor is \textbf{strict} if $\kappa$ is iso.

\begin{prop}\cite[Prop. 1.2.2]{seal2013tensors}  There is a bijective correspondence
between:
\begin{enumerate}
\item[(i)] families of morphisms $\{\kappa_{a,b}\}$ such that $(P,\eta,\mu,\kappa)$ is a
(symmetric)
monoidal monad;
\item[(ii)] (symmetric) monoidal structures on $C_P$ such that the left adjoint functor $F_P:C\to C_P$
is strict monoidal.

\end{enumerate}

\end{prop}

If $(P,\eta,\mu,\kappa)$ is a symmetric monoidal monad, we define the monoidal structure
on  $C_P$ as follows. The functor
\[
\otimes_P: C_P\times C_P\to C_P
\]
is given as as $a\otimes_P b=a\otimes b$ on objects, and for $f\in C_P(a,c)=C(a,P(c))$ and $g\in
C_P(b,d)=C(b,p(d))$, we define $f\otimes_P g\in C_P(a\otimes_P b, c\otimes_P d)=C(a\otimes
b, P(c\otimes d))$ as
\[
f\otimes_P g:= \kappa_{b,d}\circ (f\otimes g).
\]
The associator and unitors  and symmetry in $C_P$ can be defined from those in $C$ by composition with
$\eta$. 


\section{Compact Kleisli categories}

Assume that $C$ is a symmetric monoidal closed category. Assume that $(P,\eta,\mu,\kappa)$ is a
monoidal monad, such that the category $C_p$ with corresponding monoidal structure is
compact. We will study some consequences of this. 

\medskip

\subsection{First consequences}
It follows that each object $a\in C_P$ has a dual object $a^*$, such that there is an
isomorphism
\[
i:C_P(a\otimes_P b^*,c)\simeq C_P(a,b\otimes_P c),
\]
which is natural (in $C_P$) in $a$ and $c$. This means that for any arrows
$a\xrightarrow{f} a'$ and $a'\otimes b^*\xrightarrow{h'} c$ in $C_P$, we have
\[
i(h'\circ_P (f\otimes_P id_{b^*}^P))=i(h')\circ_P f,
\]
and similarly, gor $c\xrightarrow{g} c'$ and $a\otimes b^*\xrightarrow{h} c$ in $C_P$, we get
\[
i(g\circ_P h)=(id_b^P\otimes_P g)\circ_P i(h).
\]
By definition of the Kleisli category, $i$ is an isomorphism (that is, a bijection of
sets)
\[
i: C(a\otimes b^*,P(c))\simeq C(a,P(b\otimes c)).
\]
We would like to show that $i$ is natural in $a$ and $c$ also in the category $C$.

So let $f\in C(a,a')$ and let $\tilde f:= \eta_{a'}\circ f\in C(a, P(a'))=C_P(a,a')$. 
Let $h'\in C(a'\otimes b^*, P(c))$, then 
\begin{align*}
h'\circ_P (\tilde f\otimes_P b^*)&=\mu_c\circ P(h')\circ (\tilde f\otimes_P \eta_{b^*})=\mu_c\circ P(h')\circ \kappa_{a',b^*}\circ
(\eta_{a'}\otimes \eta_{b^*})\circ (f\otimes b^*)\\
&= \mu_c\circ P(h')\circ \eta_{a'\otimes b^*}\circ (f\otimes b^*)=\mu_c\circ
\eta_{P(c)}\circ h'\circ (f\otimes b^*)\\
&= h'\circ(f\otimes b^*),
\end{align*}
where we used that $\kappa_{a',b^*}\circ
(\eta_{a'}\otimes \eta_{b^*})=\eta_{a'\otimes b^*}$, naturality of $\eta$ and the
triangle identity. 
Similarly, we get for any $\bar h'\in C(a, P(b\otimes c))$,
\[
\bar h'\circ_Pf=\mu_{b\otimes c}\circ P(\bar h')\circ \eta_{a'}\circ f=\bar h'\circ f,
\]
in particular, putting these together, this implies 
\[
i(h'\circ(f\otimes b^*))= i(h'\circ_P(\tilde f\otimes_P id_{b^*}))=i(h')\circ_P\tilde
f=i(h')\circ f.
\]
Naturality in $c$ is proved similarly.

\medskip
It follows that there is an isomorphism 
\[
C(a,P(b\otimes c))\simeq C(a\otimes b^*, P(c))\simeq C(a,[b^*,P(c)]),
\]
natural in $a$ and $c$. By Yoneda, we get the isomorphism
\[
P(b\otimes c)\simeq [b^*,P(c)],
\]
natural in $c$. Putting $c=I$, we obtain
\[
P(b)\simeq P(b\otimes I)\simeq [b^*,P(I)].
\]

\subsection{A construction of a monoidal monad}

Fix an object  $p\in C$ and assume that 
\begin{itemize}
\item there is a bijective map $a\mapsto a^*$ on objects;
\item for each $a\in C$, there is a morphism $\theta_a\in C(a,[a^*,p])$;
\item for each morphism $f\in C(a,[b^*,p])$ there is some $\hat f\in C([a^*,p],[b^*,p])$
\end{itemize}
such that 
\begin{enumerate}
\item[(i)] $\hat \theta_a=id_{[a^*,p]}$;
\item[(ii)] for $f\in C(a,[b^*,p])$, 
$\hat f\circ \theta_a=f$;
\item[(iii)] for $f\in C(a,[b^*,p])$ and $g\in C(b,[c^*,p])$,
\begin{equation}\label{eq:hatf}
(\hat g\circ f)^{\wedge}=\hat g\circ \hat f.
\end{equation}

\end{enumerate}


From this data, we may define a monad $(P_p,\theta,\nu)$. Here the functor $P_p$ acts as $a\mapsto [a^*,p]$ on objects and 
for $f\in C(a,b)$, we define $P_p(f)\in C([a^*,p],[b^*,p])$ by
\[
P_p(f):=(\theta_b\circ f)^{\wedge}.
\]
Moreover, $\nu$ is defined as $\nu_a=\hat{id}_{[a^*,p]}$. The fact that this is a monad 
 follows easily from the properties (i)-(iii). 

To make it monoidal, we add  family of maps 
\[
\kappa_{a,b}:[a^*,p]\otimes [b^*,p]\to [(a\otimes b)^*,p],
\]
such that
\begin{enumerate}
\item[(iv)] for all $a,b\in C$, 
\[
\theta_{a\otimes b}=\kappa\circ (\theta_a\otimes \theta_b);
\]
\item[(v)] for $f\in C(a,[b^*,p])$ and $g\in C(c,[d^*,p])$,
\[
\kappa_{b,d}\circ(\hat f\otimes \hat g)=(\kappa_{b,d}\circ(f\otimes g))^\wedge\circ
\kappa_{a,c}.
\]

\end{enumerate}

Then one can check that $\kappa_{a,b}$ are natural in $a,b$ and that 
\[
\nu_{a\otimes b}\circ P_p(\kappa_{a,b})\circ
\kappa_{[a^*,p],[b^*,p]}=\kappa_{a,b}\circ(\nu_a\otimes \nu_b).
\]
We also need some properties with respect to $\alpha,\lambda,\rho$ and $\sigma$:
\begin{enumerate}
\item[(vi)] for all $a,b,c$,
\[
\kappa_{a,b\otimes c}\circ
(id_{[a^*,p]}\otimes\kappa_{b,c})\circ\alpha_{[a^*,p],[b^*,p],[c^*,p]}=P_p(\alpha_{a,b,c})\circ\kappa_{a\otimes
b,c}\circ(\kappa_{a,b}\otimes id_{[c^*,p]})
\]
\item[(vii)] for all $a$,
\begin{align*}
(\theta_a\circ \lambda_a)^\wedge\circ \kappa_{I,a}\circ (\theta_I\otimes
id_{[a^*,p]})&=\lambda_{[a^*,p]}\\
(\theta_a\circ \rho_a)^\wedge\circ \kappa_{I,a}\circ (id_{[a^*,p]}\otimes
\theta_I)&=\rho_{[a^*,p]};
\end{align*}

\item[(viii)] for all $a,b$,
\[
(\theta_{b\otimes a}\circ \sigma_{a,b})^\wedge\circ \kappa_{a,b}=\kappa_{b,a}\circ
\sigma_{[a^*,p],[b^*,p]}.
\]
\end{enumerate}
Then $(P_p,\theta,\nu,\kappa)$ is a monoidal monad, \cite{seal2013tensors}. 

\subsection{The Kleisli category  $C_p$}

The Kleisli category $C_p:=C_{P_p}$ has the same objects as $C$, with morphisms
$C_p(a,b)=C(a, [b^*,p])$, the identity is $id^p_a=\theta_a$ and for $f\in C_p(a,b)$, $g\in
C_p(b,c)$, the composition is fiven as 
\[
g\circ_p f= \hat g\circ f.
\]
\begin{remark}
Let $j$ be the natural iso (in $C$):
\[
j: C(a\otimes b,c)\simeq C(a,[b,c])
\]
Note that  $C_p(a,b)$ can be identified with $C(a\otimes b^*,p)$, with composition given by
\[
j^{-1}(j(\psi)^\wedge\circ j(\varphi)),\qquad \varphi\in C(a\otimes b^*,p),\ \psi\in
C(b\otimes c^*,p).
\]
\end{remark}

We equip $C_p$ with the tensor product  $\otimes_p$ defined by $a\otimes_p b=a\otimes
b$ on objects and $f\otimes_p g=\kappa\circ(f\otimes g)$ on morphisms. 
Then  $(C_p,\otimes_p,I)$ is a symmetric monoidal category, with  the natural isomorphisms
$\alpha,\lambda,\rho,\sigma$ extended by $\theta$, that is,
$\alpha^p:=\theta\circ \alpha$, $\lambda^p:=\theta\circ \lambda$, $\rho^p:=\theta\circ
\rho$, $\sigma^p:=\theta\circ \sigma$.

\subsection{When is $C_p$ closed?} 

We need to define the internal hom $b\overset{p}{\multimap} c$, such that
$b\overset{p}\multimap -$ is the right adjoint of $b\otimes_p -$ in $C_p$. In fact, it is
enough to specify $b\overset{p}\multimap c$ on objects and to find an iso
\[
C_p(a\otimes_p b,c)\simeq C_p(a,b\overset{p}\multimap c)
\]
natural in $a$. As for the isomorphism, we must have
\[
C(a\otimes b,[c^*,p])\simeq C(a, [(b\overset{p}\multimap c)^*,p])
\]
Since $C$ is SMC, we have
\[
C(a\otimes b, [c^*,p])\simeq C((a\otimes b)\otimes c^*,p)\simeq C(a\otimes (b\otimes
c^*),p)\simeq C(a, [b\otimes c^*,p]) 
\]
and the isomorphisms are natural (in $C$) in all variables. This suggests to define $b
\overset{p}\multimap c$ as the object such that $(b\overset{p}\multimap c)^*=b\otimes
c^*$. Since $(-)^*$ is bijective, such an object exists and is unique. 

As for naturality of the isomorphism, let us denote by $i$ the resulting isomorphism
\[
i: C(a\otimes b, [c^*,p])\simeq C(a, [b\otimes c^*,p])
\]
Let $f\in C_p(a',a)=C(a',[a^*,p])$. Then naturality means that we require
\[
i(h\circ_p(f\otimes_p id^p_b))=i(h)\circ_p f=i(h)^\wedge \circ f.
\]
on the left hand side  we obtain 
\[
h\circ_p(f\otimes_p id^p_b)=\hat h\circ \kappa_{a,b}\circ (f\otimes \theta_b)=\hat
h\circ s_{a,b}\circ (f\otimes b),
\]
where $\hat h\circ s_{a,b}\in C([a^*,p]\otimes b, [c^*,p])$. By naturality in $C$, we see
that 
\[
i(\hat h\circ s_{a,b}\circ (f\otimes b))=i(\hat h \circ s_{a,b})\circ f,
\]
where $i(\hat h\circ s_{a,b})\in C([a^*,p],[b\otimes c^*,p])$.
It follows that we need to have
\begin{equation}\label{eq:cpclosed}
i(\hat h\circ s_{a,b})=i(h)^\wedge.
\end{equation}






 \end{document}

