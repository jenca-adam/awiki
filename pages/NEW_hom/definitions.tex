\documentclass[12pt]{article}

\usepackage{hyperref}
\usepackage{amsmath, amssymb, amsthm}
\usepackage[sort&compress,numbers]{natbib}
\usepackage{doi}
\usepackage[margin=0.8in]{geometry}
%\textheight23cm \topmargin-20mm  
%\textwidth175mm  
%\oddsidemargin=0mm
%\evensidemargin=0mm
%

\usepackage{amsmath, amssymb, amsthm, mathtools}

\newtheorem{lemma}{Lemma}
\newtheorem{theorem}{Theorem}
\newtheorem{coro}{Corollary}
\newtheorem{prop}{Proposition}


\theoremstyle{definition}
\newtheorem{defi}{Definition}


\theoremstyle{remark}
\newtheorem{remark}{Remark}

\def\Me{\mathcal M}
\def\Ne{\mathcal N}
\def \Tr{\mathrm{Tr}\,}
\def\Se {\mathcal S}
\def\supp{\mathrm{supp}}
\def\<{\langle\.}
\def\>{\.\rangle}

\title{Various definitions}
\author{Anna Jen\v cov\'a}

\begin{document}

\maketitle

\section{Symmetric monoidal categories (SMC)}

 \textbf{Monoidal category:} A category $C$ equipped with
\begin{itemize}
\item A functor $\otimes: C\times C\to C$;
\item unit object $I\in C$;
\item associator: natural iso $(A\otimes B)\otimes C \xrightarrow{\alpha_{A,B,C}} A\otimes
(B\otimes C)$;
\item left unitor: natural iso $I\otimes A \xrightarrow{\lambda_A} A$;


\item right unitor: natural iso $A\otimes I \xrightarrow{\rho_A} A$

\item \textbf{symmetric} if there is a symmetry: natural iso $A\otimes
B\xrightarrow{\sigma_{A,B}} B\otimes A$ such that $\sigma_{B,A}=\sigma_{A,B}^{-1}$, 

\end{itemize}
satisfying triangle, pentagon (+ symmetry) diagrams.

\medskip

We will always assume that $C$ is a SMC.


\subsection{Closed SMC}

A SMC $C$ is \textbf{closed} if:

\medskip %\vskip 3mm
for every $b\in C$, the endofunctor $-\otimes b$ has a
right adjoint $[b,-]$ (internal hom). 

\medskip
\noindent
What does this mean?

\begin{enumerate}
\item[(1)] For all $a,c\in C$, $C(a\otimes b,c)\simeq C(a,[b,c])$, naturally in $a,c$.
\item[(2)] unit $\eta^b_a: a\to [b,a\otimes b]$, counit: $\epsilon^b_a: [b,a]\otimes
b\to a$, natural transformations, triangle identities
\end{enumerate}
`
Relation of the two: 
\begin{itemize}
\item Let $i$ be the iso of (1): 
\begin{align*}
\eta^b_a\in C(a,[b,a\otimes b])&\simeq C(a\otimes b,a\otimes b),\qquad
\eta^b_a=i(id_{a\otimes b})\\
\epsilon^b_a\in C([b,a]\otimes b,a)&\simeq C([b,a],[b,a]),\qquad \epsilon^b_a=i^{-1}(id_{[b,a]}).
\end{align*}
\item Conversely, from $\eta^b$, $\epsilon^b$ of (2), we define $i$ as 
\[
g\in C(a\otimes b,c),\qquad i(g)=[b,g]\circ \eta^b_a
\]
with inverse
\[
h\in C(a,[b,c]),\quad i^{-1}(h)=\epsilon^b_c\circ (h\otimes b)
\]

\end{itemize}

Informally, we may interpret $\eta^b_a$ as 'labeling of $b$ by $a$' and $\epsilon^b_a$ as 'evaluation of $[b,a]$'.

\subsection{Compact SMC}

A SMC is \textbf{compact} if each object $a\in C$ has a dual $a^*\in C$ such that there
are maps $\cup_a: I\to a^*\otimes a$ and $\cap_a: a\otimes a^*\to I$ satisfying the snake
identities
\[
(\cap_a\otimes id_a)\circ (id_a\otimes \cup_a)=id_a,\qquad (id_{a^*}\otimes
\cap_a)\circ (\cup_a\otimes id_{a^*})=id_{a^*}.
\]

The following are easily seen by the snake identities (pictures):
\begin{itemize}
\item[(1)] $a^*$ is determined up to iso;
\item[(2)] $a^{**}\simeq a$, indeed, we may define $\cup_{a^*}: I\to a\otimes a^*$ and
$\cap_{a^*}: a^*\otimes a$ as
\[
\cup_{a^*}=\sigma_{a^*,a}\circ \cup_a,\qquad \cap_{a^*}=\cap_a\circ \sigma_{a^*,a},
\]
so that $a$ is dual to $a^*$, and use (1);
\item[(3)] if we fix $a^*$ and $\cup_a$ ($\cap_a$), then  $\cap_a$ ($\cup_a$) is uniquely
determined;
\item[(4)] any assignment $a\mapsto a^*$ defines a functor $C\to C^{op}$ (if $f:a\to b$,
we can use $\cup_a$ and $\cap_b$ to ''bend the wires'' to obtain a map $b^*\to a^*$, this
is obviously functorial);
\item[(5)] $(a\otimes b)^*\simeq a^*\otimes b^*$, we can clearly put (using symmetry)
\[
\cup_{a\otimes b}=\cup_a\otimes \cup_b,\qquad \cap_{a\otimes b}=\cap_a\otimes \cap_b
\]
\item[(5)] $C$ is closed, with $[b,c]=b^*\otimes c$: the iso $i: C(a\otimes b,c)\simeq
C(a,b^*\otimes c)$ can be obtain (roughly) as
\[
i(g)=g\circ \cup_b,\qquad i^{-1}(h)=\cap_b\circ h,
\]
the maps are inverses by the snake identities, naturality is obvious, since $i$ does
nothing on $a$ or $c$. The unit and counit of the adjunction are given as
\[
\eta^b_a=a\otimes \cup_b : a\to b^*\otimes a\otimes b,\qquad \epsilon^b_a=\cap_b\otimes a:
b^*\otimes a\otimes b\to a.
\]
\item[(6)] Can we state a theorem like:  $C$ is compact if and only if for each $b\in C$
there is some $b^*\in C$ such that $b^*\otimes -$ is the right adjoint of $-\otimes b$ and
...? What should be the additional conditions?

\end{itemize}


\section{Kleisli categories and monoidal monads} 

A \textbf{monad} on $C$ is a triple $(P,\eta,\mu)$, where:
\begin{itemize}
\item $P: C\to C$ is an endofunctor;

\item $\eta: Id_C\to P$, $\mu: P^2\to P$ are natural transformations satisfying some
triangles and squares.
\end{itemize}

The \textbf{Kleisli category} $C_P$ has the same objects as $C$, with morphisms:
\[
C_p(a,b)=C(a,P(b)),
\]
and for $f\in C_p(a,b)$, $g\in C_p(b,c)$, the composition is defined as
\[
g\circ f:=\mu_c\circ P(g)\circ f.
\]

 A monad is \textbf{monoidal} \cite{seal2013tensors} if there are maps
 \[
\kappa_{a,b}:Pa\otimes Pb\to P(a\otimes b),\qquad a,b\in C,
 \]
 natural in $a,b$ and such that 
 \begin{itemize}
\item $(P,\eta, \kappa)$ is a \textbf{monoidal functor}, that is, some diagrams involving
$P$, $\alpha$, $\lambda$, $\rho$, $\kappa$ and $\eta$ commute;
\item additional diagrams containing $\mu$ commutes;
\item\textbf{symmetric}: additionally a diagram with $\sigma$ commutes. 

 \end{itemize}

\begin{prop}\cite[Prop. 1.2.2]{seal2013tensors}  There is a bijective correspondence
between:
\begin{enumerate}
\item[(i)] families of morphisms $\{\kappa_{a,b}\}$ such that $(P,\eta,\mu,\kappa)$ is a
(symmetric)
monoidal monad;
\item[(ii)] (symmetric) monoidal structures on $C_P$ such that the left adjoint functor $F_P:C\to C_P$
is strict monoidal.

\end{enumerate}

\end{prop}





\end{document}

$\cup_a$, $\cap_a$
\end{document}

