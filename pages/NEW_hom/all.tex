\documentclass[12pt]{article}


\usepackage{showkeys}

\usepackage{hyperref}
\usepackage{amsmath, amssymb, amsthm, bm}
\usepackage[sort&compress,numbers]{natbib}
\usepackage{doi}
\usepackage[margin=0.8in]{geometry}
\usepackage[all]{xy}
\usepackage{mathrsfs} 
%\textheight23cm \topmargin-20mm  
%\textwidth175mm  
%\oddsidemargin=0mm
%\evensidemargin=0mm
%

\usepackage{amsmath, amssymb, amsthm, mathtools}

\newtheorem{lemma}{Lemma}
\newtheorem{theorem}{Theorem}
\newtheorem{coro}{Corollary}
\newtheorem{prop}{Proposition}


\theoremstyle{definition}
\newtheorem{defi}{Definition}


\theoremstyle{remark}
\newtheorem{remark}{Remark}
\newtheorem{exm}{Example}

\def\aff{\operatorname{Aff}}
\def\lin{\operatorname{Lin}}
\def\Span{\operatorname{Span}}
\def\type{\mathrm {setting}}
\def\ii{\bm{\emptyset}}
\def\tt{\bm{0}}
\def\Ce{\mathcal C}
\def\Ae{\mathcal A}
\def\Ne{\mathcal N}
\def\Le{\mathcal L}
\def\Te{\mathcal T}
\def\Fe{\mathcal F}
\def\Pe{\mathcal P}
\def \Tr{\mathrm{Tr}\,}
\def\Se {\mathcal S}
\def\supp{\mathrm{supp}}
\def\permut{\mathscr{S}}

\def\<{\langle\,}
\def\>{\,\rangle}
%\def \BS{\mathrm{BS}}
\def\vtl{\vartriangleleft}
\def\vtr{\vartriangleright}
\def \Afh{\mathrm{AfH}}
\def \Af{\mathrm{Af}}
\def \FV{\mathrm{FinVect}}
\def\bV{\mathbf V}
\def\bW{\mathbf W}
\def\bE{ E}
\def\bI{I}

\def\Cl{\mathrm{Clas}}
\def\Quant{\mathrm{Quant}}
\def\bX{ X}
\def\bQ{\mathbf Q}
\def\bY{ Y}
\def\bZ{Z}

\title{On the structure of  higher order quantum maps}
\author{Anna Jen\v cov\'a}

\begin{document}

\maketitle

\section{Preliminaries}

\subsection{The category $\FV$} \label{sec:fv}

Let  $\FV$ be the category of finite dimensional real vector spaces with linear maps. 
We will denote the usual tensor product by $\otimes$, then  $(\FV,\otimes, I=\mathbb R)$
is a symmetric monoidal category, with the associators, unitors and symmetries given by
the obvious isomorphisms 
\begin{align*}
\alpha_{U,V,W}&:(U\otimes V)\otimes W\simeq U\otimes (V\otimes W), \\
\lambda_V&: I\otimes
V\simeq
V, \qquad \rho_V: V\otimes I\simeq V,\\
\sigma_{U,V}&: U\otimes V\simeq V\otimes U.
\end{align*}




Let  $(-)^*: V\mapsto V^*$ be the usual vector space dual, with duality denoted by
$\<\cdot,\cdot\>: V^*\times V\to \mathbb R$. We will use the canonical identification
$V^{**}=V$ and $(V_1\otimes V_2)^*=V_1^*\otimes V_2^*$. With this duality, $\FV$ is
compact closed. This means that for each object $V$, there are maps $\eta_V: I\to V^*\otimes
V$ (the ''cup'') and $\epsilon_V: V\otimes V^*\to I$ (the ''cap'') such that the following snake
identities hold:
\begin{equation}\label{eq:snake}
(\epsilon_V\otimes V)\circ (V\otimes \eta_V)=V,\qquad (V^*\otimes \epsilon_V)\circ
(\eta_V\otimes V^*)=V^*,
\end{equation}
here we denote the identity map on the object $V$ by $V$. Indeed, $\eta_V$ can be
identified with an element $\eta_V(1)\in V^*\otimes V$ and   $\epsilon_V\in (V\otimes
V^*)^*=V^*\otimes V$ is again an element of the same space.  Choose a basis
$\{e_i\}$ of $V$, let $\{e_i^*\}$ be the dual basis of $V^*$, that is,
$\<e_i^*,e_j\>=\delta_{i,j}$. Let us then define
\[
\eta_V(1)=\epsilon_V:=\sum_i e_i^*\otimes e_i.
\]
It is easy to see that this definition does not depend on the choice of the basis, indeed,
$\epsilon_V$ is the linear functional on $V\otimes V^*$ defined by
\[
\<\epsilon_V, x\otimes x^*\>=\<x^*,x\>,\qquad x\in V, \ x^*\in V^*.
\]
It is also easily checked that the snake identities \eqref{eq:snake} hold.

For two objects $V$ and $W$ in $\FV$, we will denote the set of all morphisms (i.e. linear
maps) $V\to
W$ $L(V,W)$  by $\FV(V,W)$. Then $\FV(V,W)$ is itself a real linear space and  we have the well-known identification 
$\FV(V,W)\simeq V^*\otimes W$. This can be given as follows: for each $f\in FV(V,W)$, we have 
$C_f:=(V^*\otimes f)(\epsilon_V)=\sum_i e_i^*\otimes f(e_i)\in V^*\otimes W$. Conversely,
since $\{e_i^*\}$ is a basis of $V^*$, 
any element $w\in V^*\otimes W$ can be uniquely written as $w=\sum_i e_i^*\otimes w_i$ for
$w_i\in W$, and since $\{e_i\}$ is a basis of $V$, the assignment $f(e_i):=w_i$ determines a
unique map $f:V\to W$. The relations between $f\in \FV(V,W)$ and $C_f\in V^*\otimes W$ can
be also written as
\[
\<C_f,x\otimes y^*\>=\<\epsilon_V,x\otimes f^*(y^*)\>=\<f^*(y^*),x\>=\<y^*,f(x)\>,\qquad x\in
V,\ y^*\in W^*,
\]
here $f^*:W^*\to V^*$ is the adjoint of $f$.
Note that by compactness, the internal hom in $\FV$ satisfies $[V,W]\simeq V^*\otimes W$,
so that  in the case of $\FV$, the object $[V,W]$ can be identified with the space of linear
maps $\FV(V,W)$. 


\begin{exm}\label{exm:classical} Let $V=\mathbb R^N$. In this case, we fix the canonical basis $\{|i\>,\
i=1,\dots,N\}$. We will identify $(\mathbb R^N)^*=\mathbb R^N$, with duality
$\<x,y\>=\sum_i x_iy_i$, in particular, we identify $I=I^*$. We then have
$\epsilon_V=\sum_i |i\>\otimes |i\>$ and if $f:\mathbb R^N\to \mathbb R^M$ is given by the
matrix $A$ in the two canonical bases, then  $C_f=\sum_i |i\>\otimes A|i\>$ is the
vectorization of $A$.

\end{exm}


\begin{exm}\label{exm:quantum} Let $V=M_n^h$ be the space of $n\times n$ complex hermitian matrices. We again
identify $(M_n^h)^*=M_n^h$, with duality $\<A,B\>=\Tr A^TB$, where $A^T$ is the usual
transpose of the matrix $A$. Let us choose the basis in $M_n^h$, given as
\[
\left\{|j\>\<k|+|k\>\<j|,\ j\le k,\ i\biggl(|j\>\<k|-|k\>\<j|\biggr),\ j<k\right\}.
\]
Then one can check that
\[
\left\{\frac12\biggl(|j\>\<k|+|k\>\<j|\biggl),\ j\le k,\
\frac{i}{2}\biggl(|k\>\<j|-|j\>\<k|\biggr),\ j<k\right\}
\]
is the dual basis and we have
\[
\epsilon_V=\sum_{j,k} |j\>\<k|\otimes |j\>\<k|.
\]
For any $f:M_n^h\to M_m^h$, 
\[
C_f=\sum_{j,k} |j\>\<k|\otimes f(|j\>\<k|)
\]
is the Choi matrix of $f$.

\end{exm}


\subsection{Boolean functions}


We will work with certain boolean functions, which are defined as  functions from binary
strings $\{0,1\}^*$ to $\{0,1\}$. We now list some basic notations used below.

For $s\in \{0,1\}$, we denote $\bar s:=1-s$.
For binary strings of fixed length $n$, that is, elements of
$\{0,1\}^n$, we will denote by $0_n$ or just $0$ the string $00\dots0$ and 
\[
e^i=\delta_{i,1}\dots\delta_{i,n}.
\]
For $m,n\in \mathbb N$, we may identify $ \{0,1\}^m\times \{0,1\}^n\simeq \{0,1\}^{m+n}$
by concatenation: 
\[
\{0,1\}^m\times \{0,1\}^n\ni(s^1,s^2)\mapsto s^1s^2=s^1_1\dots s^1_ms^2_1\dots s^2_n\in \{0,1\}^{m+n}.
\]
With this identification, we will sometimes use the identification of $[m+n]$ as the
disjoint union of indices  $[m]\dot{\cup} [n]$, so the index set becomes
$\{(1,1),\dots, (1,m),(2,1),\dots,(2,n)\}$, where $(1,i)=i$ and $(2,j)=m+j$, $i\in [m]$,
$j\in [n]$, and for $s\in \{0,1\}^{m+n}$, we put $s_{(l,j)}=s^l_j$. We will use similar
notations for any $n=n_1+\dots+n_k$. 





%For a string $x\in \{0,1\}^n$ and any set of indices $\{i_1,\dots,i_k\}\subseteq [n]$, we
%will denote by $x^{i_1\dots i_k}$ the string in $\{0,1\}^{n-k}$ obtained from $x$ by removing
%$x_{i_1},\dots, x_{i_k}$. 

For any permutation $\sigma\in \permut_n$, we will denote by the same symbol the action
of $\sigma$ on $\{0,1\}^n$ given as 
\[
\sigma(s_1\dots s_n)=s_{\sigma^{-1}(1)}\dots s_{\sigma^{-1}(n)}.
\]
Note that then for $\rho,\sigma\in \permut_n$, we have
$\rho(\sigma(s))=(\rho\circ\sigma)(s)$. 

Let us introduce the subset of boolean functions 
\[
\Fe_n:=\{f:\{0,1\}^n\to \{0,1\},\ f(0_n)=1\}.
\]
With the poitwise ordering, $\Fe_n$ is  a (finite) distributive lattice, with top element
 the constant 1 function and the bottom element $p_n:=\chi_{0_n}$, the characteristic
 function of the zero string. Moreover, we have for $f,g\in \Fe_n$,
\begin{equation}\label{eq:wedgevee_fun}
f\vee g= f+g-fg,\qquad f\wedge g=fg
\end{equation}
(with pointwise addition and multiplication). We also define complementation  in $\Fe_n$ as
\[
f^*:=1-f+p_n.
\]
It can be easily checked that with these structures $\Fe_n$ is a boolean algebra.

We now introduce some more operations in $\Fe_n$. For $f\in \Fe_n$ and any permutation
$\sigma\in \permut_n$, we see that $f\circ \sigma\in \Fe_n$.
For $f\in \Fe_{m}$ and $g\in \Fe_n$, we define
the function $f\otimes g\in \Fe_{m+n}$ as
\[
(f\otimes g)(s^1s^2)=f(s^1)g(s^2),\qquad s^1\in \{0,1\}^m,\ s^2\in \{0,1\}^n
\]
As it is, this tensor product is not symmetric, but there is a permutation $\sigma\in
\permut_{m+n}$  such that $\sigma(s^1s^2)=s^2s^1$ and  
$(g\otimes f)=(f\otimes g)\circ \sigma$ for any $f\in \Fe_m$ and $g\in \Fe_n$.

\begin{lemma}\label{lemma:fproduct} For $f\in \Fe_m$, $g,h\in \Fe_n$, we have
\begin{enumerate}
\item[(i)] $f\otimes g\le (f^*\otimes g^*)^*$, with equality if and only if $f$ and $g$ are either 
both the  top or both the bottom elements in
$\Fe_m$ resp. $\Fe_n$
\item[(ii)] $f\otimes (g\vee h)= (f\otimes g)\vee (f\otimes h)$, $f\otimes (g\wedge h)=
(f\otimes g)\wedge (f\otimes h)$.
\end{enumerate}

\end{lemma}

\begin{proof} The inequality in (i) is easily  checked, since $(f\otimes g)(s^1s^2)$ can be 1 only if
$f(s^1)=g(s^2)=1$. If both $s^1$ and $s^2$ are the zero strings, then $s^1s^2=0_{m+n}$ and both sides
are equal to 1. Otherwise, the condition $f(s^1)=g(s^2)=1$ implies that $(f^*\otimes
g^*)(s^1s^2)=0$, so that the right hand side must be 1. If $f$ and $g$ are both
constant 1, then $(1\otimes 1)^*=1^*=p_{n+m}=1^*\otimes 1^*$, in the case when both $f$
and $g$ are the minimal elements equality  follows by
duality. Finally, asume the equality holds and that $f\ne 1$, so that there is some $s^1$ such that 
$f(s^1)=0$. But then $s^1\ne 0_m$ and for any $s^2$,
\[
0=(f\otimes g)(s^1s^2)=(f^*\otimes
g^*)^*(s^1s^2)=1-f^*(s^1)g^*(s^2)+p_{m+n}(s^1s^2)=1-g^*(s^2),
\]
which implies that $g(s^2)=0$ for all $s^2\ne0_n$, that is, $g=p_n$. By the same argument,
$f=p_m$ if $g\ne 1$, which implies that either $f=1$ and $g=1$, or $f=p_m$ and $g=p_n$.

To prove (ii), notice that for $s^1\in \{0,1\}^n$ and $s^2\in \{0,1\}^m$,  we have by
\eqref{eq:wedgevee_fun}
\begin{align*}
f(s^1)(g\vee h)(s^2)&=
f(s^1)\left(g(s^2)+h(s^2)-g(s^2)h(s^2)\right)\\&=f(s^1)g(s^2)+f(s^1)h(s^2)-f(s^1)^2g(s^2)h(s^2)=f(s^1)g(s^2)\vee
f(s^1)h(s^2),
\end{align*}
note that $f(s^1)^2=f(s^1)$. The statement for $\wedge$ is also clear from
\eqref{eq:wedgevee_fun}.

\end{proof}



We now describe  an important example. 


\begin{exm}\label{ex:ps} Let $S\subseteq [n]$ be any subset. We will denote
\[
p_S(s):=\prod_{i\in S}\bar s_i.
\]
It is clear that $p_S\in \Fe_n$ and  $p_{\emptyset}=1$, $p_{[n]}=p_n$. The following
properties are also easy
to see:
\begin{enumerate}
\item[(i)] if $S\subseteq T\subseteq [n]$, then $p_T\le p_S$,
\item[(ii)]  $p_S\wedge p_T=p_Sp_T=p_{S\cup T}$, for $S,T\subseteq [n]$,
\item[(iii)]  $p_S\vee p_T=p_S+p_T-p_{S\cup T}$, for $S,T\subseteq [n]$,
\item[(iv)] $p_S\circ\sigma=p_{\sigma^{-1}(S)}$, $S\subseteq [n]$, $\sigma\in \permut_n$,
\item[(iv)] $p_S\otimes p_T=p_{S\dot{\cup} T}$, for  $S\subseteq [m]$ and $T\subseteq
[n]$.

\end{enumerate}


\end{exm}




We will use the above functions to introduce  a convenient  parametrization to $\Fe_n$,
closely related to the Fourier transform. 


\begin{theorem}\label{thm:basis} Any $f:\{0,1\}^n\to \{0,1\}$ can be written in the form 
\[
f=\sum_{S\subseteq [n]} \hat f_Sp_S
\]
in a unique way, with  $\hat f_S\in \mathbb R$ obtained as
\[
\hat f_S=\sum_{\substack{s\in \{0,1\}^n\\ s_i=1, \forall  i\in S^c}} (-1)^{\sum_{i\in
S}s_i}f(s).
\]

\end{theorem}

\begin{proof} Let $\Le_n$ be the boolean algebra of all subsets of $[n]$. We may interpret
 $f$ as the function $\Le_n\to \mathbb R$, given by
\[
\varphi: S\mapsto f(s),\qquad s\in \{0,1\}^n,\ s_i=0 \iff i\in S.
\]
By the M\"obius inversion formula (see [Stanley, Sec. 3.7] for details), a  function $g:
\Le_n\to \mathbb R$ satisfies  
\[
f(s)=\varphi(S)=\sum_{T\subseteq S} g(T),\qquad S\in \Le_n
\]
if and only if 
\[
g(S)=\sum_{T\subseteq S}(-1)^{|S\setminus T|} \varphi(T)=\hat f_S.
\]
Plugging this into the first equality, we get for any $s\in \{0,1\}^n$,
\[
f(s)=\sum_{T: s_i=0, \forall i\in T} \hat f_T=\sum_{T: p_T(s)=1}\hat f_T=\sum_T \hat
f_Tp_T(s).
\]
To show uniqueness, assume that $f=\sum_{T\subseteq [n]} c_T p_T$ for some
$c_T\in \mathbb R$, $T\in \Le_n$,  and let $s$ and $S$ be connected as above. Then 
\[
\varphi(S)=f(s)= \sum_{T\subseteq [n]} c_Tp_T(s)=\sum_{T\subseteq S} c_T,
\]
so that we must have $c_T=\hat f_T$.

\end{proof}



%For this, we first include $\Fe_n$ into a larger set
%\[
%\Fe_n\subseteq \{f:\{0,1\}^n\to \mathbb R\}=:\mathcal V,
%\]
%which is a $2^n$-dimensional real vector space. It becomes a Hilbert space with respect to
%the inner product
%\[
%\<f,g\>=\sum_{s\in \{0,1\}^n} f(s)g(s). 
%\]
%
%\begin{lemma}\label{lemma:Vbasis} The set $\{p_S,\ S\subseteq [n]\}$ is a basis of
%$\mathcal V$.
%Any $f\in \mathcal V$ can be uniquely written as
%\[
%f=\sum_{S\subseteq [n]} \hat f_Sp_S,
%\]
%where the coefficients $\hat f_S\in \mathbb R$ are obtained as
%\end{lemma}
%
%
%
%\begin{proof} For $T\subseteq [n]$, let us define the function $p_T^\perp$ as
%\[
%p_T^\perp(x):= (-1)^{\sum_{i\in T}x_i}\prod_{i\in T^c}x_i. 
%\]
%We prove that for $S,T\subseteq [n]$,
%\[
%\<p_S,p^\perp_T\>=\delta_{S,T},
%\]
%which shows that $\{p_S,\ S\subseteq [n]\}$ is a basis and 
%$\{p_T^\perp, \ T\subseteq [n]\}$  is the dual basis. We compute
%\[
%\<p_S,p^\perp_T\>=\sum_x p_S(x)p^\perp_T(x)=\sum_x (-1)^{\sum_{i\in T} x_i}\prod_{i\in S}
%\bar x_i\prod_{j\in T^c}x_j.
%\]
%This expression can be nonzero only if $S\cap T^c=\emptyset$, that is, $S\subseteq T$. In
%this case, the last sum  is equal to
%\[
%\sum_{\substack{x\in \{0,1\}^n\\ x_i=0,\forall i\in S\\x_i=1,\forall i\in T^c}}
%(-1)^{\sum_{j\in T\setminus S}x_j}=\begin{dcases} 0 & \text{if } S\subsetneq T\\
%  1 &\text{if } S=T
%  \end{dcases}
%\]
%It is now clear that the coefficients
%\[
%\hat f_S=\<f,p^\perp_S\>
%\]
%have the given form.
%
%
%\end{proof}
%
%\begin{remark} This can be also obtained using M\"obius inversion formula, see [Stanley,
%Sec. 3.7].
%
%\end{remark}
%

 We can  visualise $\mathcal L_n$ as a hypercube, and the coefficients $\hat f_S$ of  $f$ as labels for its vertices.
The fact that the function $f$ takes  values in $\{0,1\}$ means that we must have 
\[
f(s)=\varphi(S)=\sum_{T\subseteq S} \hat f_T\in \{0,1\},
\]
so that  the sum of labels $\hat f_S$ over any face of the hypercube $\mathcal L_n$ containing the vertex 
$\emptyset$ must
be 0 or 1. In particular, $\hat f_\emptyset=f(11\dots 1)\in \{0,1\}$. Clearly,  $f\in
\Fe_n$ if and only if,  in addition,
\[
f(0)=\sum_{S\subseteq [n]}\hat f_S=1.
\]


\begin{prop}\label{prop:mobius} 

\begin{enumerate}
\item[(i)] For $f\in \Fe_n$ and  $\sigma\in \permut_n$, 
\[
\widehat{(f\circ \sigma)}_S=\hat
f_{\sigma(S)}, \qquad S\subseteq [n].
\]
\item[(ii)] For $f\in \Fe_n$, 
\[
\widehat{f^*}_S=\begin{dcases} 1-\hat f_S & S=\emptyset\text{ or } S=[n],\\
-\hat f_S & \text{otherwise}.
\end{dcases}
\]
\item[(iii)] For $f\in \Fe_m$, $g\in \Fe_n$, we have 
\[
\widehat{(f\otimes g)}_{S\dot{\cup}T}=\hat f_S\hat g_T,\qquad S\subseteq [m],\ T\subseteq
[n].
\]
\item[(iv)] For $f,g\in \Fe_n$, we have
\[
\widehat{(f\wedge g)}_S=\sum_{T\subseteq S} \hat f_T\hat g_{S\setminus T}.
\]
\item[(v)] For $f,g\in \Fe_n$, we have
\[
\widehat{(f\vee g)}_S=\hat f_S+\hat g_S-\sum_{T\subseteq S} \hat f_T\hat g_{S\setminus T}.
\]

\end{enumerate}


\end{prop}

\begin{proof} All statements follow easily from the corresponding expressions and the uniqueness part in Theorem
\ref{thm:basis}. 


\end{proof}


\section{The category of affine subspaces}


\subsection{The category $\Af$}

We now introduce the category $\Af$, whose objects  are of the form $X=(V_X,A_X)$, where
$V_X$ is an object in $\FV$  and $A_X\subseteq V_X$ a proper affine subspace, which means
that $0\notin A_X\ne \emptyset$. Morphisms $X\xrightarrow{f} Y$ in $\Af$ are linear maps $f:V_X\to V_Y$  such that
$f(A_X)\subseteq A_Y$. For any object $X$, we put
\begin{align*}
L_X=\lin(A_X):=\{a_1-a_2,\ a_1,a_2\in A_X\}=\{a-a_X,\ a\in A_X\}, \qquad S_X:=\Span(A_X).
\end{align*}
Here $a_X$ is any element in $A_X$ and $L_X$ does not depend on this choice.
Then $L_X$ and $S_X$ are linear subspaces such that $d_X:=\dim(L_X)=\dim(S_X)-1$. We will
also denote $D_X=\dim(V_X)$. For
any element $a_X\in A_X$, the affine subspace is determined as
\[
A_X=a_X+L_X.
\]

Let us now define the duality of affine subspaces as follows. Let $V$ be an object in
$\FV$ and let $C\subseteq  V$ be any subset. Let
\[
\tilde C:=\{v^*\in V^*, \ \<v^*,c\>=1\}.
\]
The following lemma collects some properties that are easily proven.

\begin{lemma}\label{lemma:dual}
\begin{enumerate}
\item[(i)] $\tilde C$ is an affine subspace.
\item[(ii)] $0\in \tilde C$ if and only if $C= \emptyset$ and   $\tilde C=\emptyset$ if and only if $0\in \aff(C)$.
\item[(iii)] Let  $0\notin \aff(C)$, then $\aff(C)=\tilde{\tilde C}$ and we have
\begin{align*}
\lin(C)&=\lin(\tilde{\tilde C})=\tilde
C^\perp=\Span(\tilde C)^\perp,&  \lin(\tilde C)&=C^\perp=\Span(C)^\perp \\
\Span(C)&=C^{\perp\perp} =\lin(\tilde C)^\perp,&  \Span(\tilde C)&=\lin(C)^\perp.
\end{align*}

\end{enumerate}


\end{lemma}



For any  $\tilde a_X\in \tilde A_X$, the subspace $A_X$ is determined as
\[
A_X=S_X\cap\{\tilde a_X\}^\sim.
\]
The relation between the subspaces $L_X$ and $S_X$ is given as
\[
S_X=L_X\oplus \mathbb R a_X,\qquad L_X=S_X\cap \{\tilde a_X\}^\perp.
\]
By Lemma \ref{lemma:dual} above,  $\tilde A_X$ is a proper affine subspace in $V_X^*$, so that
$X^*:=(V_X^*,\tilde A_X)$ is an object in $\Af$. We have $X^{**}=X$ and the corresponding subspaces
are related as
\begin{equation}\label{eq:duality}
L_{X^*}=S_X^\perp,\qquad S_{X^*}=L_X^\perp.
\end{equation}
Note also that for $X\xrightarrow{f} Y$, the adjoint map satisfies $f^*(\tilde
A_Y)\subseteq \tilde A_X$, so that $Y^*\xrightarrow{f^*} X^*$ and the duality $(-)^*$ is a
full and faithful functor 
$\Af^{op}\to \Af$.

We will next introduce a monoidal structure $\otimes$ as follows. For two objects $X$ and
$Y$, we put  $V_{X\otimes Y}=V_X\otimes V_Y$ and construct the affine subspace
$A_{X\otimes Y}$ as the affine span of 
\[
A_X\otimes A_Y=\{a\otimes b,\ a\in A_X,\ b\in A_Y\}.
\]
Fix any $\tilde a_X\in \tilde A_X$ and $\tilde a_Y\in \tilde A_Y$. 
Since $A_X\otimes A_Y\subseteq \{\tilde a_X\otimes \tilde a_Y\}^\sim$, the affine span of
$A_X\otimes A_Y$ is a proper affine subspace and we have by Lemma \ref{lemma:dual}
\[
A_{X\otimes Y}:=\aff(A_X\otimes A_Y)=\{A_X\otimes A_Y\}^{\approx}.
\]
\begin{lemma}\label{lemma:tensor_spaces}
For any $a_X\in A_X$, $a_Y\in A_Y$, we  have
\begin{align}
L_{X\otimes Y}&=\lin(A_X\otimes A_Y)=\Span(\{x\otimes y-a_X\otimes a_Y,\ x\in A_X,\ y\in
A_Y\})\label{eq:lxy1}\\
&= (a_X\otimes L_Y)+ (L_X\otimes a_Y)+ (L_X\otimes L_Y)\label{eq:lxy}
%\\ &= S_X\otimes L_Y+L_X\otimes a_Y=a_X\otimes L_Y+L_X\otimes S_Y
\end{align}
(here $+$ denotes the direct sum of subspaces). We also have
\[
S_{X\otimes Y}=S_X\otimes S_Y.
\]
\end{lemma}

\begin{proof} The equality \eqref{eq:lxy1} follows from Lemma \ref{lemma:dual}. For any $x\in A_X$, $y\in A_Y$
 we have
\[
x\otimes y-a_X\otimes a_Y=a_X\otimes (y-a_Y)+(x-a_X)\otimes a_Y+(x-a_X)\otimes (y-a_Y),
\]
so that $L_{X\otimes Y}=\lin(A_X\otimes A_Y)$ is contained in the subspace on the RHS of \eqref{eq:lxy}.
Let $d$ be the dimension of this subspace, then clearly
\[
d_{X\otimes Y}\le d\le d_X+d_Y+d_Xd_Y.
\]
On the other hand, any element of $S_X$ has the form $tx$ for some $t\in \mathbb R$ and
$x\in A_X$, so that it is easily seen that $S_X\otimes S_Y=S_{X\otimes Y}$. 
Hence 
\begin{align*}
d_{X\otimes Y}&=\dim(L_{X\otimes Y})=\dim(S_{X\otimes
Y})-1=\dim(S_X)\dim(S_Y)-1=(d_X+1)(d_Y+1)-1\\
&=d_X+d_Y+d_Xd_Y.
\end{align*}
This completes the proof.

\end{proof}




\begin{lemma}\label{lemma:monoidal} Let $I=(\mathbb R,\{1\})$. Then 
$(\Af,\otimes, I)$ is a symmetric monoidal category.
\end{lemma}

\begin{proof} Note that this structure is inherited from the symmetric monoidal structure
in $\FV$. To show that $\otimes$ is a functor, we have to check that for $X_1\xrightarrow{f} Y_1$ and $X_2\xrightarrow{g} Y_2$ in
$\Af$, we have $X_1\otimes Y_1\xrightarrow{f\otimes g} X_2\otimes Y_2$ which amounts to
showing that 
\[
(f\otimes g)(A_{X_1\otimes Y_1})\subseteq A_{X_2\otimes Y_2}.
\]
Let $x\in A_{X_1}$, $y\in A_{Y_1}$, then $f(x)\otimes g(y)\in A_{X_2}\otimes
A_{Y_2}\subseteq A_{X_2\otimes Y_2}$. Since  $A_{X_1\otimes Y_1}$ is the affine subspace
generated by $A(X_1)\otimes A(Y_1)$, the above inclusion follows by linearity of $f\otimes
g$. 

It only remains to prove that the associators, unitors and symmetries from
$\FV$ are morphisms in $\Af$. We will prove this for the associators $\alpha_{X,Y,Z}:V_X\otimes (V_Y\otimes V_Z)\to
(V_X\otimes V_Y)\otimes V_Z$, the other proofs are similar. We need to check that
$\alpha_{X,Y,Z}(A_{X\otimes(Y\otimes Z)})\subseteq A_{(X\otimes Y)\otimes Z}$. It is easily
checked that $A_{X\otimes(Y\otimes Z)}$ is the affine span of elements of the form
$x\otimes (y\otimes z)$, $x\in A_X$, $y\in A_Y$ and $z\in A_Z$, and we have
\[
\alpha_{X,Y,Z}(x\otimes (y\otimes z))=(x\otimes y)\otimes z\in A_{(X\otimes Y)\otimes Z}
\]
for all such elements. The desired inclusion follows by linearity.

\end{proof}


\begin{theorem} $(\Af,\otimes,I)$ is a *-autonomous category, with duality $(-)^*$, such
that $I^*=I$.

\end{theorem}


\begin{proof} By Lemma \ref{lemma:monoidal}, we have that $(\Af,\otimes I)$ is a symmetric
monoidal category. We have also seen that the duality $(-)^*$ is a full and faithful
contravariant functor. We only need to check the natural isomorphisms 
\[
\Af(X\otimes Y,Z^*)\simeq \Af(X,(Y\otimes Z)^*).
\]
Since $\FV$ is compact, we have the natural isomorphisms
\[
\FV(V_X\otimes V_Y,V^*_Z)\simeq \FV(V_X,V_Y^*\otimes V_Z^*),
\]
determined by the equalities
\[
\<f(x\otimes y),z\>=\<h(x),y\otimes z\>,\qquad x\in V_X,\ y\in V_Y,\ z\in V_Z,
\]
for $f\in \FV(V_X\otimes V_Y,V_Z^*)$ and $h\in \FV(V_Z,V_Y^*\otimes V_Z^*)$. Since
$A_{X\otimes Y}$ is an affine span of $A_X\otimes A_Y$, we see that
$f$ is in $\Af(X\otimes Y, Z^*)$ if and only if $f(x\otimes y)\in \tilde A_Z$ for all $x\in A_X$, $y\in
A_Y$, that is, 
\[
1=\<f(x\otimes y),z\>=\<h(x),y\otimes z\>\qquad \forall x\in A_X,\
\forall y\in A_Y,\ \forall z\in A_Z.
\]
But this is equivalent to
\[
h(x)\in (A_Y\otimes A_Z)^\sim=\tilde A_{Y\otimes Z},\qquad \forall x\in A_X,
\]
which means that $h\in \Af(X, (Y\otimes Z)^*)$.

\end{proof}
A *-autonomous category is compact closed if it satisfies $(X\otimes Y)^*=X^*\otimes
Y^*$. 
In general, $X\odot Y=(X^*\otimes Y^*)^*$ defines a dual symmetric monoidal
structure that is different from $\otimes$. 
We next show that $\Af$ is not compact.

\begin{prop}\label{prop:noncompact} For objects in $\Af$, we have $(X\otimes
Y)^*=X^*\otimes Y^*$ exactly in one of the following situations:
\begin{enumerate}
\item[(i)] $X\simeq I$ or $Y\simeq I$,
\item[(ii)] $d_X=d_Y=0$,
\item[(iii)] $d_{X^*}=d_{Y^*}=0$.
\end{enumerate}



\end{prop}

\begin{proof} It is easily seen by definition that $A_{X^*}\otimes A_{Y^*}=\tilde A_X\otimes \tilde
A_Y\subseteq \tilde A_{X\otimes Y}=A_{(X\otimes Y)^*}$. Hence the equality holds if and
only if $d_{X^*\otimes Y^*}=d_{(X\otimes Y)^*}$. From Lemma
\ref{lemma:tensor_spaces}, we see that
\[
d_{X^*\otimes Y^*}=d_{X^*}+d_{Y^*}+d_{X^*}d_{Y^*}.
\]
On the other hand, we have using \eqref{eq:duality} that $L_{(X\otimes Y)^*}=S_{X\otimes
Y}^\perp=(S_X\otimes S_Y)^\perp$, so that
\[
d_{(X\otimes Y)^*}=D_XD_Y-\dim(S_X)\dim(S_Y)=D_XD_Y-(d_X+1)(d_Y+1).
\]
Taking into account that by \eqref{eq:duality} we have $d_{X^*}=D_X-d_{X}-1$, similarly
for $d_{Y^*}$, we obtain

\[
d_{(X\otimes Y)^*}-d_{X^*\otimes Y^*}=d_Xd_{Y^*}+d_{X^*}d_Y.
\]
This is equal to 0 iff $d_Xd_{Y^*}=d_Yd_{X^*}=0$, which amounts to the conditions in the
lemma.

\end{proof}



In a *-autonomous category, the internal hom can be identified as $[X,Y]=(X\otimes
Y^*)^*$. The underlying vector space is $V_{[X,Y]}=(V_X\otimes V_Y^*)^*=V_X^*\otimes V_Y$
and we have seen in Section \ref{sec:fv} that we may identify this space with
$\FV(V_X,V_Y)$, by $f \leftrightarrow C_f$. This property is extended to $\Af$, in the
following sense.

\begin{prop}\label{prop:ihom_morphisms} For any objects $X,Y$ in $\Af$, the map $f\mapsto C_f$ is a bijection
of $\Af(X,Y)$ onto $A_{[X,Y]}$. 

\end{prop}

\begin{proof} Let $f\in \FV(V_X,V_Y)$. Since by definition $A_{[X,Y]}=\tilde A_{X\otimes Y^*}$ and $A_{X\otimes
Y^*}$ is an affine span of $A_X\otimes A_Y^*$, we see that $C_f\in A_{[X,Y]}$ if and only
if for all $x\in A_X$ and $y^*\in \tilde A_Y$, we have
\[
1=\<C_f, x\otimes y^*\>=\<y^*,f(x)\>.
\]
This latter statement is clearly equivalent to $f(A_X)\subseteq A_Y$, so that $f\in
\Af(X,Y)$. 
\end{proof}

In the next result, we restrict the objects to spaces of hermitian matrices, as in Example
\ref{exm:quantum} and morphisms to completely positive maps. We show that this restriction
amounts to taking an intersection of $A_{[X,Y]}$ with the cone of positive semidefinite
matrices. This, and subsequent examples,  shows that for characterization of sets  of quantum
objects such as states, channels, combs and transformations between them, it is enough to
work with the category $\Af$. 

An object $X$ of $\Af$ will be called quantum if $V_X=M_n^h$ for some $n$ and $A_X$ is an
affine subspace such that both $A_X$ and $\tilde A_X$ have nonempty intersection with the
interior of the positive cone 
$int(M_n^+)$ (recall that we identify $(M_n^h)^*=M_n^h$). 




\begin{prop}\label{prop:ihom_quantum} Let $X$, $Y$ be quantum objects in $\Af$. Then 
\begin{enumerate}
\item[(i)] $X^*$ and $X\otimes Y$ are quantum objects as well
\item[(ii)] Let $V_X=M_n^h$, $V_Y=M_m^h$. Then for any $f\in \FV(M_n^h,M_m^h)$, we have
$C_f\in A_{[X,Y]}\cap M_{mn}^+$ if and only if $f$ is completely positive and
\[
f(A_X\cap M_n^+)\subseteq A_Y\cap M_m^+.
\]
\end{enumerate}


\end{prop}

\begin{proof} The statement (i) is easily seen from  $A_X\otimes A_Y
\subseteq  A_{X\otimes Y}$ and $\tilde A_X\otimes \tilde A_Y\subseteq \tilde A_{X\otimes
Y}$, together with the fact that $int(M_n^+)\otimes int(M_m^+)\subseteq int(M_{mn}^+)$. 
To show (ii), let $C_f\in   A_{[X,Y]}\cap M_{mn}^+$. By the properties of the Choi
isomorphism $f$ is completely positive and by Proposition \ref{prop:ihom_morphisms},
$f(A_X)\subseteq A_Y$, this proves one implication. For the converse, note that we only
need to prove that under the given assumptions, $f(A_X)\subseteq A_Y$, for which it is enough
to show that $A_X\subseteq \aff(A_X\cap M_n^+)$. To see this, pick some  $a_X\in  A_X\cap
int(M_n^+)$. Any element in $A_X$ can be written in the form $a_X+v$ for some $v\in L_X$.
Since $a_X\in int(M_n^+)$, there is some $s>0$ such that $a_\pm:=a_X\pm sv\in M_n^+$, and
since $\pm sv\in L_X$, we see that $a_\pm \in A_X\cap M_n^+$. It is now easily checked
that
\[
a_X+v=\frac{1+s}{2s}a_++\frac{s-1}{2s}a_-\in \aff(A_X\cap M_n^+). 
\]


\end{proof}

We can define classical obejcts in $\Af$ in a similar way, replacing $M_n^h$ by $\mathbb
R^N$ and the positive cone by $\mathbb R_+^N$. A similar statement holds in this case,
with complete positivity replaced by positivity. We can similarly treat
classical-to-quantum and quantum-to-classical maps as morphisms between these types of
objects, satisfying appropriate positivity assumptions.

\begin{exm}\label{exm:quantum_maps} States, channels, combs, nonsignaling, etb, dual, process
matrices

\end{exm}

\begin{exm}\label{exm:qccq} POVMs, instruments, multimeters.


\end{exm}


\subsection{First order and higher order objects}


We say that an object $X$ in $\Af$ is first order if $d_X=D_X-1$, equivalently, $S_X=V_X$.
Another equivalent condition is $d_{X^*}=0$, which means that $A_X$ is determined by a
single element $\tilde a_X\in V_X^*$ as 
\[
A_X=\{\tilde a_X\}^\sim,\qquad \tilde A_X=\{\tilde a_X\}.
\]
In the case of first order quantum objects we additionally require that $\tilde a_X\in
int(M_n^+)$, similarly for classical first order objects.
Note that first order objects, resp. their duals, are exactly those satisfying
condition (iii), resp. condition (ii), in Proposition \ref{prop:noncompact}, in
particular, $(X\otimes Y)^*=X^*\otimes Y^*$ for first order objects $X$ and $Y$.

For a first order object $X=(V_X, \{\tilde a_X\}^\sim)$, let us pick an element $a_X\in
A_X$, then we have a direct sum decomposition
\[
V_X=L_{X,0}\oplus L_{X,1},
\]
where $L_{X,0}:= \mathbb R\{a_X\}$, $L_{X,1}:=\{\tilde a_X\}^\perp=L_X$.
We also define the {\em conjugate object} $\tilde X=(V_X^*,\{a_X\}^\sim)$, note that we always
have $\tilde a_X\in A_{\tilde X}$ and with the choice $a_{\tilde X}=\tilde a_X$, we have
$\tilde{\tilde X}=X$ and 
\begin{equation}\label{eq:complement}
L_{\tilde X,u}=L_{X,\bar u}^\perp,\qquad u\in \{0,1\}.
\end{equation}

These  definitions depend on the choice of $a_X$, but we will assume below that this
choice is fixed and that we choose $a_{\tilde X}=\tilde a_X$. For quantum objects we will
assume that $a_X\in int(M_n^+)$.


\begin{exm} Example: quantum states, multiple of identity...

\end{exm}


Higher order objects are those obtained from a finite set $\{X_1,\dots,X_n\}$ of first order objects by
taking tensor products and duals, and applying any permuations of the spaces. The above is indeed a set, so that all the objects are
different (though they may be isomorphic) and the ordering is not essential. We will also
assume that the tensor unit is not contained in this set. Of course, any first order
object is also higher order with $n=1$. Note that we cannot say that
such an object is automatically ''of order $n$'', as the following lemma shows. 

\begin{lemma}\label{lemma:1ordertensor} Let $X$, $Y$ be first order, then $X\otimes Y$ is
first order as well.

\end{lemma}

\begin{proof} We have
\[
S_{X\otimes Y}=S_X\otimes S_Y=V_X\otimes V_Y=V_{X\otimes Y}.
\]

\end{proof}




\begin{exm} (states quantum first order, channels,  supermaps - quantum higher order)

\end{exm}






\begin{exm} replacement $X^*\otimes Y$, quantum
\end{exm}

\subsection{Description of higher order objects}


We start by noticing that there are certain objects in $\Af$ that can be constructed from
a set of  first order objects and functions in $\Fe_n$. 

%{\color{blue} Define $\tilde X_i=(V_i^*, \{a_i\}^\sim)$!  }
Let $X_1,\dots,X_n$ be first order objects in $\Af$. Let $a_{X_i}\in A_{X_i}$ be fixed and
let $\tilde X_i$ be the conjugate first order objects. Let us denote $V_i=V_{X_i}$ and 
\[
L_{i,u}:= L_{X_i,u},\qquad  \tilde L_{i,u}:= L_{\tilde X_i,u} \qquad u\in \{0,1\},\ i\in [n].
\]
For $s\in \{0,1\}^n$, we define
\[
L_s:=L_{1,s_1}\otimes\dots \otimes L_{n,s_n}, \qquad \tilde L_s:=\tilde
L_{1,s_1}\otimes\dots \otimes \tilde L_{n,s_n},
\]
then we have the direct sum decompositions 
\[
V:=V_1\otimes \dots \otimes V_n=\bigoplus_{s\in \{0,1\}^n} L_s,\qquad V^*=V_1^*\otimes
\dots\otimes V_n^*=\bigoplus_{s\in \{0,1\}^n} \tilde L_s.
\]


\begin{lemma}\label{lemma:Lperp}   For any $s\in \{0,1\}^n$, we have 
\[
L_s^\perp=
\bigoplus_{t\in\{0,1\}^n} \bar{\chi}_s(t)\tilde L_t,\qquad \tilde L_s^\perp=
\bigoplus_{t\in\{0,1\}^n} \bar{\chi}_s(t)L_t.
\]
Here $\chi_s:\{0,1\}^n\to\{0,1\}$ is the characteristic function of $s$,
$\bar{\chi}_s=1-\chi_s$. 

\end{lemma}

\begin{proof} Using \eqref{eq:complement} and the direct sum decomposition of $V_i^*$, we get
\begin{align*}
\left(L_{1,s_{1}}\otimes \dots\otimes L_{n,s_{n}}\right)^\perp&= \bigvee_j\left(
V_{1}^*\otimes
\dots \otimes V_{j-1}^*\otimes \tilde L_{j,\bar s_{j}}\otimes V_{j+1}^*\otimes\dots \otimes
V_{n}^*\right)\\
&= \bigvee_j \left( \bigoplus_{\substack{t\in \{0,1\}^n\\ t_{j}\ne s_{j}}} \tilde
L_{1,t_{1}}\otimes\dots \otimes \tilde
L_{n,t_{n}}\right)\\
&= \bigoplus_{\substack{t\in \{0,1\}^n\\ t\ne s}} \left( \tilde L_{1,t_{1}}\otimes\dots \otimes \tilde
L_{n,t_{n}}\right).
\end{align*}
The proof of the other equality is the same.

\end{proof}



\begin{lemma}\label{lemma:Xf} Put $a:= a_1\otimes\dots \otimes  a_n$, $\tilde a:= \tilde
a_1\otimes\dots\otimes  \tilde a_n$.
For  $f\in \Fe_n$ define 
\[
S_f:=\bigoplus_{s\in \{0,1\}^n} f(s)L_s,\qquad A_f=A_f(X_1,\dots,X_n):=S_f\cap \{\tilde a\}^\sim.
\]
Then $A_f$ is a proper affine subspace in $V$ containing $a$. Moreover,
\[
L_{A_f}=\bigoplus_{s\in\{0,1\}^n\setminus\{0\}} f(s)L_s,\qquad S_{A_f}=S_f
\]
and the dual affine subspace satisfies  
\[
\tilde A_f(X_1,\dots,X_n)=A_{f^*}(\tilde X_1,\dots, \tilde X_n)=\bigoplus_{s\in \{0,1\}^n}
f^*(s)\tilde L_s\cap\{a\}^\sim .
\]
 

%determined by 
%\[
%\tilde A_f=S_{\tilde A_f}\cap \{a\}^\sim,\qquad S_{\tilde A_f}= \bigoplus_{s\in \{0,1\}^n} f^*(s)\tilde L_s.
%\]

\end{lemma}

\begin{proof} It is clear from definition that $A_f$ is an affine subspace. Since
$f(0)=1$, the space $S_f$ always contains the subspace $L_0=L_{1,0}\otimes\dots\otimes
l_{n,0}=\mathbb R\{a\}$ and it is clear that $L_s\subseteq \{\tilde a\}^\perp$ for any
$s\ne 0$. It follows that $a\in A_f$, so that $A_f\ne \emptyset$, and since $A_f\subseteq
\{\tilde a\}^\sim$, we see that  $A_f$ is proper and $\tilde
a\in \tilde A_f$.  The expressions for $L_{A_f}$ and $S_{A_f}$ are immediate from the definition and
$L_{A_f}=S_{A_f}\cap\{\tilde a\}^\sim$. To obtain the dual affine subspace, we compute
using Lemma \ref{lemma:Lperp} and the fact that the subspaces form an independent
decomposition,
\begin{align*}
S_{\tilde A_f}&=L_{A_f}^\perp=\left(\bigoplus_{s\in\{0,1\}^n\setminus\{0\}}
f(s)L_s\right)^\perp=
\bigwedge_{\substack{s\in\{0,1\}^n\\ s\ne 0, f(s)=1}}L_s^\perp=
\bigwedge_{\substack{s\in\{0,1\}^n\\ s\ne 0,
f(s)=1}}\left(\bigoplus_{t\in\{0,1\}^n} \bar{\chi}_s(t)\tilde L_t\right)\\
&=\bigoplus_{t\in\{0,1\}^n} \left(\bigwedge_{\substack{s\in \{0,1\}^n\\ s\ne 0, f(s)=1}}
\bar{\chi}_s(t)\tilde L_t\right)=\bigoplus_{t\in \{0,1\}^n} f^*(t) \tilde L_t.
\end{align*}
To see the last equality, note that
\[
\bigwedge_{\substack{s\in \{0,1\}^n\\ s\ne 0, f(s)=1}}
\bar{\chi}_s(t)=\begin{dcases} 1 & \text{if } t=0\\ 1-f(t) & \text{if } t\ne 0
\end{dcases} \ = f^*(t).
\]


\end{proof}

Since $L_s$, $s\in \{0,1\}$ is an independent decomposition, the map $f\mapsto S_f$, and
hence also $f\mapsto A_f$, is injective. This map  has the following further properties, which are easily checked:
\begin{enumerate}
\item[(i)] For the bottom and top elements in $\Fe_n$ we have
\[
A_{p_n}=\{a\},\qquad A_{1_n}=\{\tilde a\}^\sim,
\]
\item[(ii)] We have $f\le g$ if and only if $A_f\subseteq A_g$,
\item[(iii)] $A_{f\wedge g}=A_f\cap A_g$,
\item[(iv)] $A_{f\vee g}= A_f\vee A_g:=\aff(A_f\cup A_g)$.
\end{enumerate}
It follows that the set $\{A_f,\ f\in
\Fe_n\}$ is a distributive lattice, with respect to the lattice operations $\cap$ and
$\vee$.

Since all the affince subspaces  $A_f\subseteq V$ are proper, there are objects $X_f:=(V,A_f)$ in $\Af$. The
above relations can be rephrased as follows:
\begin{enumerate}
\item[(i)] $X_{1_n}=(V,\{\tilde a\}^\sim)$ is a first order object,
$X_{p_n}=(V^*,\{a\}^\sim)^\sim$ is a dual first order object.
\item[(ii)] We have $f\le g$ if and only if $id_V$ is a morphism $X_f\to X_g$ in $\Af$,
\item[(iii)] Let $f,g\le h$. The following is a pullback diagram:
\[
%\xymatrixcolsep{5pc}\xymatrixrowsep{3pc}
\xymatrix{
X_{f\wedge g}\ar[r]^{id_V}
\ar[d]_{id_V} & X_f\ar[d]^{id_V} \\
X_{g} \ar[r]_{id_V}& X_{h} 
}
\]

\item[(iv)] Let $h\le f,g$. The following is a pushout diagram:
\[
\xymatrix{
X_{h}\ar[r]^{id_V}
\ar[d]_{id_V} & X_f\ar[d]^{id_V} \\
X_{g} \ar[r]_{id_V}& X_{f\vee g} 
}
\]


\end{enumerate}
In particular, it follows that $\{X_f,\ f\in \Fe_n\}$, is a distributive lattice, with
pullbacks and pushouts as lattice operations. Furthermore, using the conjugate objects, we
may construct 
\[
\tilde X_f:=(V^*, A_f(\tilde X_1,\dots,\tilde X_n))
\]
and we see from Lemma
\ref{lemma:Xf} that 
\begin{equation}\label{eq:dualityXf}
X_f^*=\tilde X_{f^*},\qquad f\in \Fe_n.
\end{equation}



We next observe that the higher order objects are of the form $X_f$, for some choice of the
first order objects $X_1,\dots, X_n$ and a function $f$ that belongs to a special subclass of $\Fe_n$.  
So assume that  $Y$ is a higher order object constructed from a set of distinct first
order objects $Y_1,\dots, Y_n$, $Y_i=(V_{Y_i},\{\tilde a_{Y_i}\}^\sim)$, we will write
$Y\sim\{Y_1,\dots,Y_n\}$ in this case. Let us fix elements $a_{Y_i}\in A_{Y_i}$ and construct the objects $\tilde Y_i$. 
 
 By compactness of $\FV$, we may assume (relabeling the objects if necessary) that the vector space of $Y$ has the form
\[
V_Y=V:=V_{1}\otimes \dots\otimes V_{n},
\]
where  $V_i$ is either $V_{Y_i}$ or $V_{Y_i}^*$, according to whether $Y_i$ was subjected
to taking duals an even or odd number of times. The indices such that the first
case is true will be called the outputs and the subset of outputs in $[n]$ will be denoted
by $O$, or $O_Y$, when we need to specify the object. The set $I=I_Y:=[n]\setminus O_Y$ is
the set of inputs. The reason for this terminology will become clear later. 

\begin{prop}\label{prop:boolean} For $i\in [n]$, let 
$X_i=Y_i$ if $i\in O_Y$ and $X_i=\tilde Y_i$ for $i\in I_Y$. 
There is a unique function $f\in \Fe_n$ such that 
\[
Y= X_f=(V, A_f(X_1,\dots,X_n)).
\]

\end{prop}

\begin{proof} Since the map $f\mapsto X_f$ is injective, uniqueness is clear.  To show existence of this
function, we will proceed by induction on $n$. For $n=1$, the assertion is easily seen
to be true, since in this case, we we have either $Y=Y_1$ or $Y=Y_1^*$. In the first case, $O=[1]$,
$X_1=Y_1$ and 
\[
S_Y=V_Y=V_1=L_{1,0}\oplus L_{1,1}=1(0)L_{1,0}\oplus 1(1)L_{1,1},
\]
so in this case $f$ is the constant 1. If $Y=Y_1^*$, we have $O=\emptyset$, $X_1=\tilde
Y_1$, and then
\[
S_Y=\mathbb R\{\tilde a_{Y_1}\}=L_{1,0}=1^*(0)L_{1,0}\oplus 1^*(1)L_{1,1},
\]
so that $f=1^*$. 

Assume now that the assertion is true for
all $m<n$. By construction, $Y$ is either the tensor
product $Y=Z_1\otimes Z_2$, with
\[
Z_1\sim \{Y_{1},\dots, Y_{m}\},\qquad Z_2\sim\{Y_{{m+1}},\dots, Y_{n}\},
\]
 or $Y$ is the dual of such a product. Let us assume the first case. It is clear that
 $O_{Z_1}\cup O_{Z_2}=O_Y$, and similarly for $I$, so that the corresponding objects
 $X_1,\dots, X_m$ and $X_{m+1},\dots,X_n$  remain the same. By the induction 
assumption, there are functions $f_1\in \Fe_m$ and $f_2\in \Fe_{n-m}$ such that
\[
S_Y=S_{Z_1}\otimes S_{Z_2}=\bigoplus_{\substack{s\in\{0,1\}^{m}\\ t\in
\{0,1\}^{n-m}}}
f_1(s)f_2(t)L_{1,s_{1}}\otimes\dots \otimes L_{m,s_{m}}\otimes
L_{{m+1},t_{1}}\otimes\dots\otimes L_{n,t_{n-m}}.
\]
This implies the assertion, with $f=f_1\otimes f_2$. 

To finish the proof, it is now enough to observe that if the assertion holds for $Y$ then
it also  holds for $Y^*$. So assume that $Y=X_f=(V, A_f(X_1,\dots,X_n))$ for some $f\in \Fe_n$,
then $Y^*=X_f^*=\tilde X_{f^*}=(V^*,A_{f^*}(\tilde X_1,\dots,\tilde X_n)$. 
In is now enough to notice that $\tilde X_i=\tilde{\tilde Y}_i=Y_i$
if $i\in I_{Y}$ and $\tilde X_i=\tilde {Y}_i$ if $i\in O_Y$. Since by definition
$O_{Y^*}=I_Y$, this proves the statement.

\end{proof}

Let us stress that in general, the objects $X_f$ depend on the choice of the elements
$a_{X_i}$. From the above proof, it is clear that  the  description in Proposition
\ref{prop:boolean} does not depend on the choice of the elements $a_{Y_i}\in A_{Y_i}$.


%\begin{exm} Quantum higher order objects from state spaces, \cite{bisio2019theoretical}.

%\end{exm}



%%

\subsection{Type funtions and higher order objects}


Let $\Te_n\subseteq \Fe_n$ be defined as the subset generated from the constant 
function $1$ on $\{0,1\}$ by taking duals and tensor products. For example, we have
\[
\Te_1=\Fe_1=\{1,1^*\},\quad
\Te_2=\{1\otimes 1, (1\otimes 1)^*, 1\otimes 1^*,1^*\otimes 1, (1^*\otimes 1)^*,
(1\otimes 1^*)^*\},
\]
etc.  Elements of $\Te_n$ will be called {\em type functions}. Similarly as for the higher order
objects, the indexes in $[n]$ such that the corresponding
component was subjected to taking the dual an even number of times will be called the
outputs (of $f$) and denoted by $O=O_f$, indexes in $I=I_f:=[n]\setminus O_f$ will be
called inputs. From the proof of Proposition \ref{prop:boolean}, it is easily seen that a
higher order object is of the form $Y=X_f$ for a function $f\in \Te_n$ with the same outputs (and of
course also inputs) as $Y$. We next show that the converse is true. 


\begin{prop}\label{prop:type_hom} Let $\{X_1,\dots, X_n\}$  be first order objects  and let
$f\in \Te_n$. Then $Y=X_f$ is a higher order object with $O_Y=O_f$ and  $Y\sim
\{Y_1,\dots, Y_n\}$, where $Y_i=X_i$ for $i\in O_f$ and $Y_i=\tilde X_i$ for
$i\in I_f$.  

\end{prop}

\begin{proof} As before, we will proceed by induction on $n$. For $n=1$, we only have the
possibilities $f=1$ or $f=1^*$. In the first case, $O=[1]$ and we get
\[
S_f=1L_{1,0}\oplus 1L_{1,1}= V_{1},
\]
so that $X_f=(V_1,\{\tilde a_1\}^\sim)=X_1$. In the second case, $O=\emptyset$ and 
\[
S_f=1L_{1,0}=\mathbb R\{a_1\},
\]
so that $X_f=(V_1,\{a_1\})=\tilde X_1^*$. Assume next that the statement
is true for all $m<n$ and assume that $f=f_1\otimes f_2$ for some $f_1\in \Fe_m$, $f_2\in
\Fe_{n-m}$, then it is easily seen that $Y=Z_1\otimes Z_2$ for $Z_1=X_{f_1}$ and
$Z_2=X_{f_2}$, constructed from $\{X_1,\dots,X_m\}$ resp. $\{X_{m+1},\dots, X_n\}$. 
By the induction assumption, $Z_1$ and $Z_2$ are higher order objects, with
$O_{Z_i}=O_{f_i}$, it follows that $Y$ is a higher order object with $O_Y=O_{Z_1}\cup
O_{Z_2}=O_{f_1}\cup O_{f_2}=O_f$.

Finally, assume that the statement is true for $f\in \Fe_n$, we will show that it holds
for $f^*$. From \eqref{eq:dualityXf}, we see that $X_{f^*}=\tilde X_f$, which shows that 
$X_{f^*}\sim\{\tilde Y_1,\dots, \tilde Y_n\}$. Since taking duals will switch inputs and
outputs, this finishes the proof.




\end{proof}


Let $\{Y_1,\dots, Y_n\}$ be first order objects.
The above results show that any of higher order object  $Y\sim \{Y_1,\dots,Y_n\}$ with
fixed set of outputs $O_Y=O$ satisfies $Y\simeq X_f$ for a unique type function $f\in
\Te_n$, $O_f=O$,  and a fixed set of objects $\{X_1,\dots,X_n\}$, where the isomorphism is given
by the action of some 
permutation in $S_n$ on the space  $V_1\otimes\dots\otimes  V_n$. Conversely, any object
of this form has the above properties. A basic example of such a function is (see Section
\ref{sec:boolean})
\[
p_I(s)= \Pi_{i\in I} \bar s_i=\otimes_{i\in I} 1^*(s_i),\qquad s\in \{0,1\}^n,
\]
where $I=[n]\setminus O$. Clearly, $p_I\in \Te_n$ and the set of outputs of $p_I$ is $O$. It is easy to see that $X_{p_I}\simeq  Y_I^*\otimes Y_O$,
where $Y_I=\otimes_{i\in I} Y_i$ and $Y_O= \otimes_{i\in O} Y_i$ are first order objects, so that the corresponding
higher order object can be identified with the set of replacement channels $Y_I\to Y_O$.
Similarly, the function $p_O^*\in \Te_n$ has output set $O$ and $X_{p_O^*}\simeq
(Y_I\otimes Y_O^*)^*=[Y_I,Y_O]$, the set of all channels $Y_I\to Y_O$. 

\begin{lemma}\label{lemma:fh_setting} Let $f\in\Te_n$ and let $O_f=O$,  $I=I_f$. Then
\[
p_I\le f\le p_O^*.
\]

\end{lemma}

\begin{proof} This is obviously true for $n=1$. Indeed, in this case,
$\Te_1=\Fe_1=\{1=p_\emptyset,1^*=p_{[1]}\}$. If $f=1$, then $O=[1]$, $I=\emptyset$ and 
\[
p_I=p_{\emptyset}=1=p_O^*,
\]
the case  $f=1^*$ is obtained by taking complements. Assume that the assertion holds for
$m<n$. Let $f\in \Te_n$ and assume that  $f=g\otimes h$ for some  $g\in
\Te_m$, $h\in \Te_{n-m}$.  By the assumption,
\[
p_{I_g}\otimes p_{I_h}\le g\otimes h\le p^*_{O_g}\otimes p^*_{O_h}\le (p_{O_g}\otimes
p_{O_f})^*,
\]
the last inequality follows from Lemma \ref{lemma:fproduct}. We have  
$O_f=O_g\cup (m+O_h)$, $I_f=I_g\cup (m+I_h)$, so that $p_{O_f}=p_{O_g}\otimes p_{O_h}$ and
similarly for $p_{I_f}$. Now notice that any $f\in \Te_n$ is either of the form $(f\otimes
g)\circ \sigma$ or of the form $(f\otimes g)^*\circ \sigma$, for some permutation
$\sigma$. Since the inequality is easily seen to be preserved by
permutations, and reversed by duality which also swiches the input and output sets, the
assertion is proved.

\end{proof}


Combining this with the remarks below Lemma \ref{lemma:Xf}, we get the following result
(cf. cite). Recall that a bimorphism in a category is a morphism $X\xrightarrow{\varphi} Y$ which is
both mono and epi, that is, such that
for any pairs of arrows (with appropriate sources and targets) we have $\psi\circ
f=\psi\circ g \iff f=g$ and $k\circ\psi=l\circ\psi \iff k=l$.  It can be shown that the
bimorphisms in $\Af$ are precisely those morphisms that are given by isomorphisms in
$\FV$.


\begin{theorem}\label{thm:setting} Let $Y\sim \{Y_1,\dots, Y_n\}$ be such that $O_Y=O$,
$I_Y=I$. Then there exist bimorphisms 
\[
Y_I^*\otimes Y_O\xrightarrow{\varphi} Y
\xrightarrow{\psi} [Y_I, Y_O].
\]
The bimorphisms are given by permutations. 


\end{theorem}


We will see below that $\Te_n$ is not a lattice for $n\ge 2$, so that for $f_1,f_2\in
\Te_n$, neither of $f_1\wedge f_2$ or $f_1\vee f_2$ has to be a type function.
Nevertheless, we have by the above results that if $O_{f_1}=O_{f_2}$, 
\[
Y_I^*\otimes Y_0 \xrightarrow{\varphi} X_{f_1\wedge f_2}\xrightarrow{id_V} X_{f_1\vee f_2}
\xrightarrow{\psi} [Y_I,Y_O]
\]
for some suitable bimorphisms $\varphi$, $\psi$, moreover, the objects $X_{f_1\wedge f_2}$
and $X_{f_1\vee f_2}$ are obtained as a pullback resp. pushout. It follows that although
these objects may  not be  higher order objects themselves, they are included in some higher order object (e.g.
 $[Y_I,Y_O]$) with the same sets of inputs and outputs. 



We finish this section by showing a simple  way to obtain the output set of a type
function.

\begin{prop}\label{prop:fh_outputs} For $f\in \Te_n$, $i\in O_f$ if and only if $f(e^i)=1$.


\end{prop}


\begin{proof} Let $i\in O_f$, then by Lemma \ref{lemma:fh_setting}, $p_{I_f}(e^i)=1\le
f(e^i)$, so that $f(e^i)=1$. Conversely, if $f(e^i)=1$, then by the other inequality in
lemma \ref{lemma:fh_setting}, $p_{O_f}(e^i)=0$, whence $i\in O_f$.


\end{proof}


\section{Characterizations of type  functions}


We have the following  description of the sets of type functions.

\begin{prop}\label{prop:type_min} The set $\Te_n$ is the smallest subset in $\Fe_n$ such
that:
\begin{enumerate}

\item $\Te_n$  is invariant under permutations: if $f\in \Te_n$, then $f\circ \sigma\in
\Te_n$ for any permutation $\sigma\in S_n$,
\item $\Te_n$  is invariant under taking duals: if $f\in \Te_n$ then $f^*\in \Te_n$,
\item $\Te_m\otimes \Te_n\subseteq \Te_{m+n}$,

\item $\Te_1=\{1,p_1\}=\Fe_1$.


\end{enumerate}

\end{prop}


\begin{proof} It is clear by construction that any system of subsets $\{\Se_n\}_n$ with
these properties must contain the type functions and that $\{\Te_n\}_n$ itself has these
properties.

\end{proof}

Our goal is to find some characterization of the type functions.  We start by looking at some examples and non-examples.

\begin{exm}\label{exm:T2} The type functions for $n=2$ are given as 
\[
s\quad \mapsto   \quad 1,\quad \bar s_1\bar s_2,\quad \bar s_1, \quad 1-\bar s_1+\bar
s_1\bar s_2,
\]
and functions  obtained from these by exchanging $s_1\leftrightarrow s_2$, which gives 6
elements.
It can be seen that $\Fe_n$ has $2^{2^n-1}$ elements, so that $\Fe_2$ has 8
elements in total. The two of them that are not type functions are
\[
g(s)=1-\bar s_1-\bar s_2+2\bar s_1s_2,\qquad g^*(s)=\bar s_1+\bar s_2-\bar s_1\bar s_2.
\]
This can be checked directly from Propositions \ref{lemma:fh_setting} and
\ref{prop:fh_outputs}. Indeed, if $g\in \Te_2$, we would have $O_g=\emptyset$, so that
$p_2\le g\le p_\emptyset^*=p_2$, which is obviously not the case. Clearly, also the
complement $g^*\notin \Te_2$. Notice also that $g^*=p_{\{1\}}\vee p_{\{2\}}$, so that
$\Te_2$ is not a lattice. Since $\Fe_2$ can be identified as a sublattice in $\Fe_n$ for
all $n\ge 2$ as $\Fe_2\ni f\mapsto f\otimes 1_{n-2}\in \Fe_n$, we see that $\Te_n$, $n\ge 2$ is a  subposet in the distributive lattice 
$\Fe_n$ but itself not a lattice. 


\end{exm}


\subsection{The poset $\Pe_f$}


It will be convenient to use the representation 
\[
f=\sum_{S\subseteq [n]} \hat f_Sp_S,
\]
of a boolean function $f$ obtained in  Theorem  \ref{thm:basis}. Let $\mathcal P_f$ be the subposet in $\mathcal L_n$ of elements such that
$\hat f_S\ne 0$. We will show that if $f$ is a type function, it is fully determined by $\Pe_f$. %Can we characterize such subposets? (we cannot so far.) Can we extract
%some information about some 'causal structure' on the indices? 
 

 We say that a poset $\Pe$  is graded of rank
$k$ if every maximal chain in $\Pe$ has the same length equal to $k$ (recall that the length
of a chain is defined as number of its elements -1). Equivalently,  there is
a unique rank function $\rho: \Pe\to \{0,1,\dots,k\}$ such that $\rho(S)=0$ if $S$ is a
minimal element of $\Pe$ and $\rho(T)=\rho(S)+1$ if $T$ covers $S$, that is, $S\le T$ and for any $R$ such that
$S\le R\le T$ we have $R=T$ or $R=S$. 

\begin{prop}\label{prop:graded} Let $f\in \Te_n$, then $\mathcal P_f$ is a graded poset
with even rank $k\le n$. If $\rho$ is the rank function, then we have
\[
f=\sum_{S\in \mathcal P_f}(-1)^{\rho(S)}p_S.
\]
See [Stanley] for details.

\end{prop}

Then rank of $\Pe_f$ will be denoted by $r(f)$ and called the rank of $f$. Note that the
assertion means that for $f\in \Te_n$, 
\[
\hat f_S=\begin{dcases} (-1)^{\rho(S)},& \text{if } S\in \Pe_f\\
0, & \text{otherwise}.
\end{dcases}
\]


\begin{proof} 
We first note that the property in the statement is invariant under permutations and
complements. Assume  the statement holds for $f$ and let us take any $\sigma\in \permut_n$.
From Proposition \ref{prop:mobius} that 
$\widehat{f\circ \sigma}_S=\hat f_{\sigma(S)}$ so that $S\mapsto \sigma(S)$ is an
isomprphism  of  
$\mathcal P_{f\circ \sigma}$ onto $\mathcal P_{f}$. Hence if $\Pe_f$ is graded with rank
function $\rho$, then $\Pe_{f\circ\sigma}$ is graded with the same rank and has rank function
$\rho\circ \sigma$. By the assumption we have 
\[
f\circ\sigma=\sum_{S\in \Pe_f}(-1)^{\rho(S)}p_S\circ\sigma=\sum_{S\in
\Pe_f}(-1)^{\rho(S)}p_{\sigma^{-1}(S)}=\sum_{S\in
\Pe_{f\circ\sigma}}(-1)^{\rho\circ \sigma(S)}p_{S}.
\]
For the complement, we have from the assumption and Proposition \ref{prop:mobius}(ii) that
\begin{equation}\label{eq:dual_rank}
f^*=(1-\hat f_\emptyset)1 -\sum_{\substack{S\in \mathcal P_f\\ \emptyset \ne S,
[n]\ne S}}
(-1)^{\rho(S)}p_S+(1- \hat f_{[n]})p_n.
\end{equation}
If $\emptyset \in \Pe_f$, then $\emptyset$ is the least element of $\Pe_f$, so that 
$\rho(\emptyset)=0$ and therefore $\hat f_\emptyset =
(-1)^0=1$. Similarly, if $[n]\in \Pe_f$, then $[n]$ is the largest element in $\Pe_f$,
hence it is the last element in any maximal chain. It follows that $\rho([n])=k$ and hence
$\hat f_{[n]}=(-1)^k=1$ (since $k$ is even). 
Therefore the equality \eqref{eq:dual_rank} implies that $\mathcal P_{f^*}$ differs from $\mathcal P_f$ only in the bottom  and
top elements:  $\emptyset \in \mathcal P_f$ iff  $\emptyset \notin \mathcal P_{f^*}$
and $p_n \in \mathcal P_f$ iff  $p_n \notin \mathcal P_{f^*}$. It follows that $\mathcal
P_{f^*}$ is graded as well, with rank  equal to $k-2$, $k$ or $k+2$, which in any case
is even. Furthermore, let $\rho^*$ be the rank function of $f^*$, then this also implies 
that for all $S\in \Pe_f$, $S\notin \{\emptyset, [n]\}$, we
have  $\rho^*(S)=\rho(S)\pm 1$, according to whether $\emptyset$ was added or removed. The
statement now follows from \eqref{eq:dual_rank}. 

We now proceed by induction on $n$. For $n=1$, we have $\mathcal L_1=\{\emptyset, [1]\}$ and
$\Te_1=\{1,1^*\}$. For $f=1$, $\mathcal P_f=\{\emptyset\}$ is a singleton, which 
is clearly a graded poset, with rank $k=0$ and trivial rank function $\rho$.  We have
\[
f = 1=p_\emptyset=(-1)^{\rho(\emptyset)}p_\emptyset.
\]
The proof for $f=1^*$ is similar, replacing $\emptyset$ by $[1]$.

To finish the proof, assume that the statement is true for $m<n$ and let $f\in \Te_n$.
Then $f$ is either a permutation of a product of some $f_1\in \Fe_{m}$ and $f_2\in \Te_{n-m}$, or a
dual  of such an element. By the first part of the proof,  we only need to prove
that the statement holds for $f=f_1\otimes f_2$. But in this case, by the induction
assumption, $\Pe_{f_i}$ is graded with even rank $k_i$ and rank function $\rho_i$. We also
have
\[
f=f_1\otimes f_2=\sum_{S\subseteq [m], T\subseteq [n-m]} (\widehat {f_1})_S(\widehat {f_2})_T p_Sp_T=
\sum_{S\subseteq [m], T\subseteq [n-m]}(-1)^{\rho_1(S)+\rho_2(T)}p_{S\dot{\cup}T}.
\]
It follows that $\Pe_f=\Pe_{f_1}\times \Pe_{f_2}$ is the product of the two posets, which
is a graded poset with rank $k=k_1+k_2$ and rank function $\rho=\rho_1+\rho_2$. This
proves the statement. 

\end{proof}


We next show that the input and output sets of $f\in \Te_n$ can be obtained from $\Pe_f$. For an
index $i\in [n]$, let  $M_{f,i}$ be the set of minimal elements of the subposet $\{S\in \Pe_f,\ i\in S\}$.
Note that $M_{f,i}$ may be empty.

\begin{prop}\label{prop:pfinput} Let $f\in \Te_n$ and $i\in [n]$. Then
\begin{enumerate}
\item If $M_{f,i}\ne
\emptyset$, then all elements in $M_{f,i}$  have the same rank,  which will be
denoted by $r_f(i)$. If $M_{f,i}=\emptyset$, we put $r_f(i):=r(f)+1$. 
\item $i\in O_f$ if and only if $r_f(i)$ is odd.

\end{enumerate}
\end{prop}



\begin{proof} Since $\Pe_f\simeq \Pe_{f\circ\sigma}$, it is
quite clear that the two properties are preserved by permutations. We will show that they
are preserved by complementation. Observe first that $M_{f,i}=\emptyset$ if and only if
$M_{f^*,i}=\{[n]\}$, since $\Pe_{f^*}$ differs from $\Pe_f$ only up to adding/removing the
least element $\emptyset$ and the greatest element $[n]$. If $M_{f,i}$ is empty, then $p_S(e^i)=1$ for all $S\in
\Pe_f$, so that $f(e^i)=f(0)=1$ and $i\in O_f$, we also see that $r_f(i)=r(f)+1$ is odd.
If $\Pe_{f,i}=[n]$, then $r_f(i)=\rho_f([n])=r(f)$ by definition of the rank, hence
$r_{f}(i)$ is even. As we have seen, $i\in O_{f^*}=I_f$. 

Let us assume that $M_{f,i}$ is not equal to $\emptyset$ or $\{[n]\}$. Then we must have
$M_{f,i}=M_{f^*,i}$ and by the
proof of Proposition \ref{prop:graded} we have  
$\rho_{f^*}(S)=\rho_f(S)\pm 1$ for any $S$, depending only on the fact whether $\emptyset
\in \Pe_f$. This implies that the properties are preserved by complementation.  


We will now proceed by induction on $n$ as before. Both  assertions are quite trivial for $n=1$,
so assume the statements hold for $m<n$. It is enough to assume that
$f=g\otimes h$ for some $g\in \Te_m$ and $h\in \Te_{n-m}$. 
Suppose without loss of generality that $i\in [m]$, then all elements of $M_{f,i}$ have
the form $S\dot{\cup} T$, with $S\in M_{g,i}$ and  $T$ a minimal element in $\Pe_h$. 
 Since $\rho_h(T)=0$ for any minimal element $T\in \Pe_h$, we have
by the induction assumption
\[
\rho_f(S\dot{\cup} T)=\rho_g(S)+\rho_h(T)=\rho_g(S)= r_g(i).
\]
The statement (ii) follows from the fact that $i\in O_f$ if and only if $i\in O_g$.


\end{proof}

\begin{coro}\label{coro:free} We have $\cap{\Pe_f}\in I_f$, $[n]\setminus\cup{\Pe_f}\in O_f$.

\end{coro}

\begin{proof} If $i\in \cap\Pe_f$, then clearly $M_{f,i}$ is the set of minimal elements
in $\Pe_f$, so that $r_f(i)=0$ and $i\in I_f$ by Proposition \ref{prop:pfinput}. If
$i\notin S$ for any $S\in \Pe_f$, then $M_{f,i}=\emptyset$ and $r_f(i)=r(f)+1$ is odd.
Hence $i\in O_f$. 

\end{proof}

Let us denote $F_{f,in}:= \cap\Pe_f$, $F_{f,out}:=[n]\setminus\cup{\Pe_f}$.
Elements of these sets will be called free inputs resp. outputs. It is easily seen that we
have
\[
f=p_{F_{f,in}}\otimes g \otimes 1_{F_{f,out}}
\]
for some type function $g$ with no free inputs or outputs.


Examples/NOnexamples?

Obrazky?



\subsection{Chains and combs}

 Let $\Pe=\{S_1\subsetneq S_2\subsetneq \dots \subsetneq
S_N\}$ be a chain in $\mathcal L_n$. Then  $\Pe$ is graded with rank $N-1$
and rank function $\rho(S_i)=i-1$. 

\begin{prop}\label{prop:chains} For a chain   $\Pe=\{S_1\subsetneq S_2\subsetneq \dots \subsetneq
S_N\}$, the function  
\[
f=f_\Pe:=\sum_i (-1)^{i-1} p_{S_i}
\]
is a type function if and only if $N$ odd.

\end{prop}

\begin{proof}
By Proposition \ref{prop:graded}, if $f\in \Te_n$, then the rank of $f$ must be even, so
that $N$ must be odd. 
We will show that the converse is true. We proceed by induction on $N$. For $N=1$, we have
$f=p_{S_1}\in \Te_n$. Assume that the statement holds for all odd numbers $M<N$ and let
$\Pe$ be a chain as above. Then we have 
\[
f=p_{S_1}\otimes g\otimes 1_{[n]\setminus S_N}
\]
where $g$ is the function for the chain $\emptyset=S'_1\subsetneq S'_2\subsetneq \dots
\subsetneq S'_N$, with $S'_i:=S_i\setminus S_1$. Since $f$ is a type function if $g$ is, this shows that we may assume that 
the chain contains $\emptyset$ and $[n]$.  But then 
\[
f=1+\sum_{j=2}^{N-1} (-1)^{j-1}p_{S_j}+ p_n 
\]
and
\[
f^*=1-f+p_n=\sum_{j=1}^{N-2} (-1)^{j-1}p_{T_j},
\]
where $T_j:=S_{j+1}$. By the induction assumption $f^*\in \Te_n$, hence also $f=f^{**}\in
\Te_n$.
\end{proof}


As we can see from Example \ref{exm:T2}, all elements in $\Te_2$ are chains. This is
also true for $n=3$. Indeed, up to a permutation that does not change the chain structure, 
any $f\in \Te_3$ is either a product of two elements $g\in \Te_2$ and $h\in \Te_1$,
or the dual of such a  product. Since $g$ must be a chain and $|\Pe_h|=1$, their product
must be a chain as well. Taking the dual of a chain only adds/removes the least/largest
elements, so the dual of a chain must be a chain as well.

We now show that chains correspond to important higher order objects. Let $k$ be odd and let $\Pe=\{\emptyset\subsetneq S_1\subsetneq \dots
\subsetneq S_k\subsetneq [n]\}$. Let $f=f_\Pe$, then $f$ is a type function. By
Proposition \ref{prop:type_hom}, for any first order objects $X_1,\dots,X_n$, $Y=X_f$ is a
higher order object such that $Y\sim\{Y_1,\dots,Y_n\}$. 

\begin{prop}\label{prop:chains_combs}  Let $T_1=S_1$, $T_i=S_i\setminus S_{i-1}$ for
$i=1,\dots, k$ and $T_{k+1}=S_k^c$. For $S\subseteq [n]$, we denote  $Y_S:=\otimes_{i\in
S}Y_i$. Then for  $k=1$, $Y\simeq [Y_{T_2},Y_{T_1}]$ and for any odd $
k>1$, 
 \[
Y\simeq [Y_{T_{k+1}},[[Y_{T_k},[[....],Y_{T_2}]],Y_{T_1}]].
 \]
\end{prop}

Remark (quantum) comb (examples)

\begin{proof} It is easily checked by Proposition \ref{prop:fh_outputs} that $T_i\subseteq
O_f$ if  $i$ is odd and $T_i\subseteq I_f$ otherwise. We therefore have 
\[
Y_{T_i}=\begin{dcases} \otimes_{i\in T_i}X_i,  & \text{ if } i\text{ is odd},\\
\otimes_{i\in T_i}\tilde X_i,& \text{ if }i\text{ is even}.
\end{dcases}
\]
Let $k=1$, then $f=1-p_{T_1}+p_n$, and $f^*=p_{T_1}=p_{T_1}\otimes 1_{T_2}$. Then
$X_f=\tilde X_{f^*}^*$, and we see that $\tilde X_{f^*}= X_{T_1}^*\otimes \tilde
X_{T_2}=Y^*_{T_1}\otimes Y_{T_2}$. It
follows that $X_f= (Y_{T_1}^*\otimes Y_{T_2})^*\simeq [Y_{T_2}, Y_{T_1}]$, where the
isomorphism is given by swapping the spaces $V_{T_1}^*$ and $V_{T_2}$. Assume the assertion is
true for $k-2$. 
As in the proof of Proposition \ref{prop:chains}, we see that 
\[
f^*=\sum_{i=1}^k (-1)^{i-1}p_{S_i}=p_{T_1}\otimes g\otimes 1_{T_{k+1}}
\]
where $g=1-\sum_{i=2}^{k-1}(-1)^{i-1}p_{S'_i}+ p_{S_k'}$ is the function for the chain
$\Pe'=\{\emptyset\subsetneq S_2'\subsetneq \dots\subsetneq S'_k\}$ in $S'_k\simeq [n']$ for
 $n'=|S'_k|$, $S_i'=S_i\setminus S_1=\cup_{j=2}^i T_j$. We have 
\[
X_f=\tilde X^*_{f^*}=(X_{Y_1}^*\otimes \tilde X_g\otimes \tilde X_{T_{k+1}})^*\simeq (\tilde
X_{T_{k+1}}\otimes [\tilde X_g,X_{T_1}]^*)^*=[\tilde X_{T_{k+1}},[\tilde X_g,
X_{T_1}]].
\]
Here we have used the fact that $\tilde X_{f^*}$ is constructed from $\tilde
X_1,\dots,\tilde X_n$. By induction assumption, we get
\[
[\tilde X_{T_{k+1}},[\tilde X_g,X_{T_1}]]= [\tilde X_{T_{k+1}},[[X_{T_k},[[....],\tilde
X_{T_2}]],X_{T_1}],
\]
which is as required.



\end{proof}

Combs, picture, without $\emptyset$ or $[n]$? Free inputs and outputs!


\subsection{Connecting chains: the causal product}




We will introduce further operations of boolean functions. For
$f_1:\{0,1\}^m\to \{0,1\}$,
$f_2:\{0,1\}^n\to \{0,1\}$, we define 
\[
f_1\vartriangleleft f_2:=f_1\otimes 1_n+p_m\otimes (f_2-1),\qquad  f_1\vartriangleright
f_2:=1_m\otimes f_2+(f_1-1)\otimes p_n.
\]

\begin{lemma}\label{lemma:causal_product}
Let $f_1\in \Fe_m$, $f_2\in \Fe_n$. Then $f_1\vartriangleleft f_2\in \Fe_{n+m}$ and we
have
\begin{enumerate}
\item[(i)] $f_1\vtl (f_2 \vtl f_3)=(f_1\vtl f_2)\vtl f_3$,
\item[(ii)] $(f_1\vtl f_2)^*=f_1^*\vtl f_2^*$,
\item[(iii)] $(f_1\wedge f_2)\vtl f_3=(f_1\vtl f_3)\wedge ( f_2\vtl f_3)$ and  $(f_1\vee
f_2)\vtl f_3=(f_1\vtl f_3)\vee ( f_2\vtl f_3)$,
\item[(iv)] $f_1\vtl (f_2\wedge f_3)=(f_1\vtl f_2)\wedge ( f_1\vtl f_3)$ and  $f_1\vtl
(f_2\vee
f_3)=(f_1\vtl f_2)\vee ( f_1\vtl f_3)$.

\end{enumerate}
Similar properties hold for $\vtr$. Moreover,
\begin{enumerate}
\item[(v)] $f_1\vtr f_2=(f_2\vtl f_1)\circ \pi$, where $\pi$ is the permutation that
acts on $\{0,1\}^{m}\times \{0,1\}^n$ as $\pi(s^1s^2)=s^2s^1$.
\item[(vi)] $f_1\otimes f_2=(f_1\vtl f_2)\wedge (f_1\vtr f_2)=(f_1\vtl f_2)\wedge (f_2\vtl
f_1)\circ \pi$.
\end{enumerate}


\end{lemma}

\begin{proof} The first assertion follows easily from
\[
f_1\vtl f_2(s^1s^2)=\begin{dcases} f_1(s^1), & \text{ if } s^1\ne 0_m,\\
   f_2(s^2), & \text{ if } s^1=0_m,
   \end{dcases}
\]
similarly for $\vtr$. 
The statements (i)--(v) follow by straightforward calculations and the equality above.  To prove (vi), let
$s^1\in \{0,1\}^m$, $s^2\in \{0,1\}^n$ and compute
\begin{align*}
(f_1\vtl f_2)\wedge (f_1\vtr
f_2)(s^1s^2)&=\left(f_1(s^1)+p_{m}(s^1)(f_2(s^2)-1)\right)\left(f_2(s^2)+p_{n}(s^2)(f_1(s^1)-1)\right)\\
=f_1(s^1)f_2(s^2),
\end{align*}
the last equality follows from the fact that $f_i(s^i)(1-f_i(s^i))=0$ (since $f_i(s^i)\in
\{0,1\}$) and the fact that $p_m$ is the least element in $\Fe_m$, so that
$p_m(s^1)(f_1(s^1)-1)=0$. 

\end{proof}

It is not clear that if $f_1$ and $f_2$ are type functions, then $f_1\vtl f_2$ or $f_1\vtr
f_2$ are type functions as well. Nevertheless, we next show that this is true for chains.



\begin{prop}\label{prop:append_chains} Let $\Pe_1=\{S_1\subsetneq \dots\subsetneq S_M\}$ be a chain in  $[m]$ and
$\Pe_2=\{T_1\subsetneq \dots \subsetneq T_N\}$ a chain in $[n]$, 
 with corresponding functions $\beta_1$ and $\beta_2$. Assume that both $M$ and $N$ are
 odd, so that $\beta_1$ and $\beta_2$ are type functions. Then 
$\beta=\beta_1\vtl \beta_2$ is a type function corresponding to a chain of $M+N\pm 1$
elements in $[m+n]$, with $O_\beta=O_{\beta_1}\dot{\cup}O_{\beta_2}$,
$I_\beta=I_{\beta_1}\dot{\cup} I_{\beta_2}$. A similar statement
holds for $\vtr$.

\end{prop}

\begin{proof} We have
\[
\beta_1=\sum_{j=1}^M(-1)^{j-1}p_{S_j},\qquad \beta_2=\sum_{k=1}^N(-1)^{k-1}p_{T_k},
\]
so that
\[
\beta=\sum_{j=1}^{M-1}(-1)^{j-1}p_{S_j}+(p_{S_M}-p_m+p_{[m]\dot{\cup}
T_1})+\sum_{k=2}^N(-1)^{k-1}p_{[m]\dot{\cup} T_k}.
\]
The resulting funnction depends on whether  $S_M=[m]$ and $T_1=\emptyset$. If at least one
of the equalities is true, then the expression in brackets is equal to $p_{[m]}$,
$p_{S_M}$ or $p_{[m]\dot{\cup} T_1}$  and $\beta$ corresponds to
a chain of $M+N-1$ elements. If both $S_M\ne [m]$ and $T_1\ne \emptyset$, then
$p_{S_M}\ne p_m\ne p_{[m]\dot{\cup} T_1}$ and $\beta$ corresponds to a chain of
$M+N+1$ elements.

For any $i\in [m]\dot{\cup} [n]$, we have  $e^i_{m+n}=e^j_m0_n$ or $e^i_{m+n}=0_me^k_n$
for some $j\in [m]$, $k\in [n]$. Then   
\[
\beta(e^i)=\beta_1(e^j_m)\ \text{ or } \beta(e^i)=\beta_2(e^k_n).
\]
The statement on input/output indices  follow from Lemma \ref{lemma:fh_setting}. 

\end{proof}


\begin{theorem}\label{thm:structure}
For any $f\in \Te_n$, there are chains $\beta_1\in \Te_{n_1}$,\dots, $\beta_k\in
\Te_{n_k}$, $n=n_1+\dots+n_k$, such that 
\[
f=\bigvee_{a\in A}\bigwedge_{b\in B} (\beta_{\pi^{-1}_{a,b}(1)}\vtl \dots \vtl
\beta_{\pi^{-1}_{a,b}(k)})\circ {\rho_{a,b}}
\]
for some finite index sets $A$ and $B$, where $\pi_{a,b}\in \permut_k$ and $\rho_{a,b}=\rho_{\pi_{a,b}}\circ \sigma$ for some $\sigma\in\permut_n$ and 
$\rho_{\pi_{a,b}}(s^1\dots
s^k)=s^{\pi^{-1}_{a,b}(1)}\dots s^{\pi^{-1}_{a,b}(k)}$, $a\in A$, $b\in B$.
\end{theorem}


\begin{proof} It is quite clear that the condition is invariant under permutations. Assume
$f$ can be written in the given form, then
\[
f^*=\bigwedge _{a\in A}\bigvee_{b\in B} (\beta^*_{\pi^{-1}_{a,b}(1)}\vtl \dots \vtl
\beta^*_{\pi^{-1}_{a,b}(k)})\circ {\rho_{a,b}}.
\]
Since $\Fe_n$ is a distributive lattice, this can be rewritten as
\[
f^*=\bigvee_{a^*\in B^{|A|}}\bigwedge_{b^*\in A}(\beta^*_{\pi^{-1}_{a^*,b^*}(1)}\vtl \dots \vtl
\beta^*_{\pi^{-1}_{a^*,b^*}(k)})\circ {\rho_{a^*,b^*}},
\]
where for $a^*=(b_a)_{a\in A}$ and $b^*=a$, we have $\pi_{a^*,b^*}=\pi_{a,b_a}$, and
$\rho_{a^*,b^*}=\rho_{\pi_{a^*,b^*}}\circ\sigma$. Since $\beta^*_j$ are chains in $[n_j]$,
the assertion is true also for $f$.

Since for $n=1$,  $f\in \Te_n$ is itself a (trivial) chain, the assertion is true in this
case (note that it is also trivially true for $n=2$ and $n=3$, since all elements in
$\Te_2$ and $\Te_3$ are chains).  
Proceeding by induction, it is now enough to show this form for  $f=f_1\otimes f_2$, where 
$f_1\in \Te_m$, $f_2\in \Te_{n-m}$ satisfy the conditions, so that
\[
f_1=\bigvee_{a\in A}\bigwedge_{b\in B} (\beta^1_{\pi^{-1}_{a,b}(1)}\vtl \dots \vtl
\beta^1_{\pi^{-1}_{a,b}(k_1)})\circ \rho_{\pi_{a,b}}\circ\sigma_1,\qquad f_2=\bigvee_{c\in
C}\bigwedge_{d\in D} (\beta^2_{\pi^{-1}_{c,d}(1)}\vtl \dots \vtl
\beta^2_{\pi^{-1}_{c,d}(k_2)})\circ \rho_{\pi_{c,d}}\circ\sigma_2.
\]
for some chains $\beta^1_j\in \Te_{m_j}$, $\sum_jm_j=m$, and $\beta^2_j\in \Te_{l_j}$,
$\sum_jl_j=n-m$ and permutations $\pi_{a,b}\in \permut_{k_1}$, $\pi_{c,d}\in
\permut_{k_2}$, $\sigma_1\in \permut_m$, $\sigma_2\in \permut_{n-m}$.
Using  properties of the tensor product, we get
\begin{align*}
f_1\otimes f_2=\bigvee_{a\in A,c\in C} \bigwedge_{b\in B,d\in D}(\beta^1_{\pi^{-1}_{a,b}(1)}\vtl \dots \vtl
\beta^1_{\pi^{-1}_{a,b}(k_1)})\otimes (\beta^2_{\pi^{-1}_{c,d}(1)}\vtl \dots \vtl
\beta^2_{\pi^{-1}_{c,d}(k_2)}) \circ (\rho_{\pi^1_{a,b}}\otimes \rho_{\pi^2_{c,d}})\circ
(\sigma_1\otimes \sigma_2).
\end{align*}
Let 
\[
\beta^{a,b}_1:=\beta^1_{\pi^{-1}_{a,b}(1)}\vtl \dots \vtl
\beta^1_{\pi^{-1}_{a,b}(k_1)},\qquad \beta^{c,d}_2:= \beta^2_{\pi^{-1}_{c,d}(1)}\vtl \dots \vtl
\beta^2_{\pi^{-1}_{c,d}(k_2)}.
\]
Then $\beta^{a,b}_1\in \Te_{m}$ and $\beta^{c,d}_2\in \Te_{n-m}$ are chains and we have
\[
\beta_1^{a,b}\otimes \beta_2^{c,d}=(\beta_1^{a,b}\vtl \beta_2^{c,d}) \wedge
(\beta_{\xi^{-1}(1)}^{c,d}\vtl \beta_{\xi^{-1}(2)}^{a,b})\circ \rho_{\xi},
\]
where $\xi$ is the swap $1\leftrightarrow 2$ and $\rho_\xi$ acts on $[n]=[m]\dot{\cup}
[n]$ as $\rho_\xi(s^1s^2)=s^2s^1$. Taking into account the decompositions $[m]=\dot{\cup}_j
[m_j]$ and $[n-m]=\dot{\cup}_j [l_j]$, we see that $\rho_\xi=\rho_\varpi$ for a 
permutation $\varpi\in \permut_{k_1+k_2}$ that swaps the two blocks $[k_1]$ and
$[k_2]$. 

Put  $\beta_j:=\beta^1_j$, $j=1,\dots, k_1$ and $\beta_{k_1+j}:=\beta^2_j$, $j=1,\dots
k_2$. Then $\beta_j$, $j=1,\dots,k:= k_1+k_2$ are chains, $\beta_j\in \Te_{n_j}$, where
$n_j:= m_j$, $j=1,\dots,k_1$, $n_{k_1+j}:=l_j$, $j=1,\dots,k_2$ and $\sum_{j=1}^k n_j=n$. 

To get the permutation sets, let $A'=A\times C$, $B'=\{0,1\}\times B\times D$. 
Let $\pi_{(a,c), (0,b,d)}=\pi_{a,b}\dot{\cup} \pi_{c,d}$ and $\pi_{(a,c),
(1,b,d)}=\varpi\circ (\pi_{a,b}\dot{\cup} \pi_{c,d})$. Put also $\sigma=\sigma_1\otimes
\sigma_2$. Then   
\[
f=\bigvee_{a'\in A'} \bigwedge_{b'\in B'}(\beta_{\pi^{-1}_{a',b'}(1)}\vtl \dots \vtl
\beta_{\pi^{-1}_{a',b'}(k)}) \circ \rho_{\pi_{a',b'}}\circ
(\sigma_1\otimes \sigma_2).
\]
The proof is complete.



\end{proof}

Examples??

Let $f$ be a function of the form as in Theorem \ref{thm:structure}. Since all the chains
$\beta_{a,b}:=\beta_{\pi^{-1}_{a,b}(1)}\vtl \dots \vtl \beta_{\pi^{-1}_{a,b}(k)}\circ
\rho_{a,b}$ in the
decomposition have the same input and  output indices, they must satsfy the inequality
$p_{I}\le \beta_{a,b} \le p_{O}^*$. But then the same is true for $f$.
It follows that although we do not know whether $f$ is a type function, the corresponding
object $X_f$ is always included in a set of channels.

In particular, in the quantum case, each $X_{\beta_{a,b}}$ describes quantum combs  obtained by
connecting combs described by  $X_{\beta_1},\dots, X_{\beta_k}$, in different orders
according to $\pi_{a,b}$. ... NOT QUITE LIKE THIS!!!










-- Quantum combs






\end{document}


---




Napisat toto dolu lepsie,  definicie do prelim, ako vyzera f?

\begin{prop}\label{prop:isom} Let $\ell :[m]\to [n]$ be a map. Let $f\in \Te_n$ and let
\[
g=\sum_{S\subseteq [n]} \hat f_S p_{\ell^{-1}(S)}.
\]
Then $g\in \Te_m$.

\end{prop}

Note that $S\mapsto \ell^{-1}(S)$ defines a homomorphism of the Boolean algebras $\mathcal
L_n$ and $\mathcal L_m$, in particular, it preseves unions and intersections, and
$\ell^{-1}(\emptyset)=\emptyset$, $\ell^{-1}([m])=[n]$. 


\begin{proof} Let us denote $g=\ell(f)$. For any permutation $\sigma\in S_m$, we have
\[
g\circ\sigma=\sum_{S\subseteq [n]} \hat f_S p_{\ell^{-1}(S)}\circ\sigma=\sum_{S\subseteq [n]}
\hat f_S p_{\sigma^{-1}(\ell^{-1}(S))}=(\ell\circ\sigma)(f),
\]
so that $\ell(f)\circ \sigma=(\ell\circ\sigma)(f)$. We can similarly show that for any
$\tau\in S_n$, $\ell(f\circ\tau)=(\tau\circ\ell)(f)$.  We may therefore assume
that either $f=f_1\otimes f_2$ for some $f_1\in \Te_k$, $f_2\in \Te_{n-k}$ or $f$ is the
dual of such a product. We then have
\[
\ell(f)=\sum_{\substack{S\subseteq [k]\\T\subseteq [n-k]}} (\hat f_1)_S (\hat
f_2)_Tp_{\ell^{-1}(S\cup (k+T))}=\sum_{\substack{S\subseteq [k]\\ T\subseteq [n-k]}} (\hat f_1)_S (\hat
f_2)_Tp_{\ell^{-1}(S)\cup \ell^{-1}(k+T)}
\]
Composing by a suitable permutation if
necessary, we may assume that there are some maps $\ell_1:[k_1]\to [k]$ and
$\ell_2:[k_2]\to [n-k]$ such that $\ell^{-1}(S)=\ell^{-1}_1(S)$
for $S\subseteq [k]$ and $\ell^{-1}(k+T)=k_1+\ell^{-1}_2(T)$, $T\subseteq [n-k]$. It follows that 
\[
\ell(f)=\ell_1(f_1)\otimes \ell_2(f_2).
\]
Since $\ell^{-1}(\emptyset)=\emptyset$ and $\ell^{-1}([n])=[m]$, we can infer from
\eqref{eq:dual_rank} that $\ell(f^*)=\ell(f)^*$. Since clearly $\ell(1)=1$, we can prove
the statement  by induction on $n$ as before.


\end{proof}


\subsection{Separating type functions}


We say that a function $f\in \Fe_n$ separates $i,j\in [n]$ if $f\neq f^{ij}$, where
$f^{ij}\in \Fe_n$ is such that for any $s\in \{0,1\}^n$,  $f^{ij}(s)=f(\hat s^{ij})$,
here $\hat s^{ij}$ is the string obtained from $s$ by replacing both $s_i$ and $s_j$ by
$s_i\vee s_j$. Equivalently, there exists some $S\in \Pe_f$ such that $S\cap\{i,j\}$ is a
singleton. 

To see this equivalence, note that 
$p_S(s)=p_S(\hat s^{ij})$ whenever $\{i,j\}\subseteq S$ or $\{i,j\}\cap S=\emptyset$. 
Hence, if $f$ separates $i$ and $j$, there must be at least one $S\in \Pe_f$ that contains one of
the indices but not the other. Conversely, let $S$ be 
minimal with the property that, say, $i\in S$ and $j\notin S$. Let $s$ be the string such
that $s_k=0$ if and only if $k\in S$, in particular, $s_i=0$, $s_j=1$ and $s_i\vee s_j=1$. Then
\[
f(s)=\sum_{T\subseteq S} \hat f_Tp_T(s)=\sum_{T\subseteq S\setminus\{i\}}\hat
f_Tp_T(s)+\hat f_Sp_S(s),
\]
the second equality follows from the fact that $S$ is minimal set with $\hat f_S\ne 0$
containing $i$ but not $j$. On the other hand, 
\[
f(\hat s^{ij})=\sum_{T\subseteq S\setminus\{i\}} \hat f_T(s),
\]
so that $f(s)-f^{ij}(s)=\hat f_S\ne 0$.



\begin{lemma}\label{lemma:separateij} Let $f\in \Fe_n$ and assume that $f$ does not separate $i$ and $j$. Then
there is some $g\in \Fe_{n-1}$ such that $f(s)=g(s^{ij}(s_i\vee s_j))$. We have
$\Pe_f\simeq \Pe_g$ (as posets) and $f\in \Te_n$ if and only if $g\in \Te_{n-1}$.




\end{lemma}


\begin{proof} We may assume $i=n-1$, $j=n$. Put $g(s_1\dots s_{n-1}):= f(s_1\dots
s_{n-1}s_{n-1})$. Then clearly $g\in \Fe_{n-1}$ and for $s\in \{0,1\}^n$, 
\[
g(s_1\dots s_{n-2}(s_{n-1}\vee s_n))=f(\hat s^{n(n-1)})=f(s).
\]
Since any $S\in \Pe_f$ either contains none of $n-1,n$ or both of them, we have
\begin{align*}
g(s_1\dots s_{n-1})&=f(s_1\dots s_{n-1}s_{n-1})=\sum_{S\subseteq [n]} \hat f_S p_S(s_1
\dots s_{n-1}s_{n-1})\\
&= \sum_{S\subseteq [n-2]}\hat f_Sp_S(s_1\dots s_{n-1})+\sum_{\substack{S\subseteq [n]\\
\{n-1,n\}\subseteq S}} \hat f_Sp_S(s_1\dots s_{n-1}s_{n-1})\\
&= \sum_{S\subseteq [n]} \hat f_Sp_{S\setminus\{n\}}(s_1\dots s_{n-1}).
\end{align*}
Note that for $S\subseteq [n]$, we have $S\setminus \{n\}=\ell^{-1}(S)$, where $\ell$ is
the inclusion map $[n-1]\hookrightarrow [n]$. It follows by Proposition \ref{prop:isom}
that if $f\in \Te_n$ then $g\in \Te_{n-1}$. Moreover, $\ell^{-1}$ is not injective on
$\mathcal L_n$, since $\ell$ is not surjective, but its restriction to $\Pe_f$ is, so that 
$\Pe_f\simeq \Pe_g$. 

Assume $g\in\Te_{n-1}$, then since $\overline {s_{n-1}\vee s_n}=\bar s_{n-1}\bar s_n$, we
get
\begin{align*}
f(s)&=\sum_{T\subseteq [n-1]}\hat g_Tp_T(s_1\dots s_{n-2}(s_{n-1}\vee s_n))\\
&= \sum_{T\subseteq [n-2]}\hat g_Tp_T(s)+ \sum_{T\subseteq [n-1], n-1\in T}\hat
g_Tp_{T\cup \{n\}}(s)=\sum_{T\subseteq [n-1]} \hat g_Tp_{\ell_1^{-1}(T)}(s),
\end{align*}
were $\ell_1: [n]\to [n-1]$ is the map such that $\ell_1(i)=i$ for $i\ne n$ and
$\ell_1(n)=n-1$. Again, using Proposition \ref{prop:isom}, this implies that $f\in \Te_n$.



\end{proof}

We say that $f$ is separating, or separates the poits of $[n]$, if it separates all pairs $i,j\in [n]$.
Notice that in this case both $\cap \Pe_f$ and $(\cup\Pe_f)^c$ are at most singletons.


\begin{theorem}\label{thm:separates} For any $f\in \Te_n$ there is some $g\in \Te_k$ such
that $g$ is separating, $\Pe_f\simeq \Pe_g$ and there are subsets $S_1,\dots, S_l$ in
$[n]$ such that $f(s)=g(s^{\cup_j S_j}(\vee_{j\in I_1} s_j)\dots (\vee_{j\in I_l} s_j))$.

\end{theorem}

\begin{proof} We can construct the function $g$ by repeated application of Lemma
\ref{lemma:separateij}, glueing together nonseparable points until, after a finite number
of steps, none are left, so that the resulting function $g$ separates the points of $m$.

\end{proof}

Let us remark that  the higher order object constructed from $f$ and a set of first order
objects $X_1,\dots, X_n$ is isomorphic to the object constructed from $g$ and a sequence
$Z_1,\dots, Z_k$, which is obtained by removing the first order objects $X_i$, $i\in S_j$
and replacing them by their tensor product
$\otimes_{i\in S_j} X_i$, for any set of nonseparable indices $S_j$.


\subsubsection{Separating chains}


We will describe all separating chains, up to permutations. 
Clearly, a chain $\Pe=\{S_1\subsetneq \dots \subsetneq
S_N\}$ in $\mathcal L_n$  is
separating if and only if  $S_i\setminus S_{i-1}$ is a singleton for $i=2,\dots, N$ and
both $S_1$ and $S_N^c$ are at most singletons (that is, these might be also empty). 
We have seen that the correspnding function is in $\Te_n$ if and only if $N=r(\Pe)+1$ is
odd. By the separating property, we must have $n-1\le N\le n+1$, depending on whether
$S_1$ or $S_N^C$ are empty or not.
 Since $N$ must be  odd, we see that the only possibilities are
\begin{align*}
N&=n-1 \text{ or } N=n+1  & \text{ if $n$ is even}\\
N&=n & \text{ if $n$ is odd}.
\end{align*}
For even $n$, put
\[
\gamma_n(s):=\sum_{l=0}^n (-1)^lp_{[l]}(s)=1-\bar s_1+\bar s_1\bar s_2-\dots +\bar
s_1\dots \bar s_n.
\]
Then $\gamma_n$  is a separating chain in $\Te_n$ and the corresponding sets of input and output
indices  are
\[
I=\{2j, j=1,\dots n/2\},\qquad O=\{2j-1,\ j=1,\dots n/2\}.
\]
Up to a permutation, any separating chain in $\Te_n$ is of the form 
\[
\gamma_{0,n}:=\gamma_n \text{ or }\   \gamma_{1,n}:=p_1\otimes\gamma_{n-2}\otimes 1_1.
\]
These two chains  are easily seen to be each others complement.
Similarly, if $n$ is odd, then (up to permutation) any separating chain in $\Te_n$  must either of the two
comlementary  forms
\[
\gamma_{0,n}:=p_1\otimes \gamma_{n-1}  \text{ or }\  \gamma_{1,n}=\gamma_{n-1}\otimes 1_1.
\]
 From the remarks below Proposition \ref{prop:chains}, we see that for $n=2$ or 3, all separating type
 functions (up to permutations)  are of this form.



Let $\Pe_1=\{S_1\subsetneq \dots\subsetneq S_N\}$ be a chain in  $[n]$ and
$\Pe_2=\{T_1\subsetneq \dots \subsetneq T_M\}$ a chain in $[m]$. Then we can append the
two chains into  a single  chain $\Pe$ in $[m+n]$, in an obvious way. Namely,  assuming that
$T_1=\emptyset$, then we may put $\Pe_2$ after $\Pe_1$ as $\Pe=\{S_1\subsetneq \dots \subsetneq S_N\subsetneq S_N\cup
T_2\subsetneq \dots \subsetneq S_N\cup T_M\}$. If the corresponding functions are $f_1$
and $f_2$, then the function related to $\Pe$ is
\[
f=f_1\otimes 1_m+p_{S_N}\otimes (f_2-1_m).
\]
Conversely, any chain can be subdivided into two or more chains, as above. Let now $n$ be
even and let $n=n_1+\dots+ n_k$ be any subdivision into even numbers $n_i$. Then 
\begin{equation}\label{eq:gamma_decomp}
\gamma_n=\gamma_{n_1}+p_{n_1}\otimes (\gamma_{n_2}-1)+p_{n_1}\otimes p_{n_2}\otimes
(\gamma_{n_3}-1)+\dots+ (\otimes_{i=1}^{k-1}p_{n_i})\otimes (\gamma_{n_k}-1) 
\end{equation}
(we skip the obvious tensorings with the constant $1$). We can append the strings
$\gamma_{n_1},\dots, \gamma_{n_k}$ in any order.  

We introduce the following notations. Let $N_j:=\sum_{i\le j} n_i$, $j=0,1,\dots,k$. 
Let $\Lambda_1,\dots,\Lambda_k$ be the subdivision of $\{1,\dots, n\}$ corresponding to
$n=n_1+\dots +n_k$, so that  $\Lambda_i=\{N_{i-1}+1,\dots,
N_i\}$, $i=1,\dots,k$. Let  $s=s^1\dots s^k$ be the corresponding subdivision of the
string $s\in \{0,1\}^n$, so $s^j=\Pi_{i\in \Lambda_j}s_i$.
For $\pi\in \permut_k$, let $\sigma_\pi$ be the permutation of
$\{1,\dots,n\}$ obtained by permuting the $k$ blocks $\Lambda_1,\dots, \Lambda_k$ by
$\pi$,  so that 
\[
\sigma_\pi(s)=s^{\pi(1)}\dots s^{\pi(k)}.
\]

\begin{remark} Combs and causal order! Different ordering - composing  with a permutation.
Also the causal tensor product of...

\end{remark}



\begin{lemma} 
We have $\gamma_{n_1}\otimes \dots\otimes \gamma_{n_k}=\bigwedge_{\pi\in \permut_k}
\gamma_n\circ \sigma_\pi$.
\end{lemma}

\begin{proof} For  a  string $s\in \{0,1\}^n$,  let us    denote
$c_i:=\gamma_{n_i}(s^{i})$. For any $\pi\in \permut_k$,  we get using
\eqref{eq:gamma_decomp} for $n=n_{\pi(1)}+\dots +n_{\pi(k)}$, 
\[
\gamma_n\circ\sigma_\pi(s)=c_{\pi(1)}+p_{n_{\pi(1)}}(s^{\pi(1)})(c_{\pi(2)}-1)+\dots
+ \Pi_{i=1}^{k-1}p_{n_{\pi(i)}}(s^{\pi(i)})(c_{\pi(k)}-1),
\]
while on  the left hand side of the equality in the lemma, we have
\[
\gamma_{n_1}\otimes \dots\otimes \gamma_{n_k}(s)=c_1\dots c_k.
\]
Since $c_i\in \{0,1\}$, we have $c_i^2=c_i$ so that  $c_i(c_i-1)=0$ and since the infimum
of two elements in $\Fe_n$ is their product, we get
\[
(\gamma_{n_1}\otimes \dots\otimes \gamma_{n_k})\wedge (\gamma_n\circ\sigma_\pi)(s)=
c_1\dots c_k=(\gamma_{n_1}\otimes \dots\otimes \gamma_{n_k})(s),
\]
so that $\gamma_{n_1}\otimes \dots\otimes \gamma_{n_k}\le \gamma_n\circ \sigma_\pi$. 

We now peoceed by induction on $k$. The statement is trivial for $k=1$. Assume that
it holds for $k$. Let $\pi\in \permut_k$ and let $\pi'\in \permut_{k+1}$ be a permutation
that fixes $k+1$ and permutes the other elements by $\pi$. We then have
\[
\gamma_n=\gamma_{N_{k}}+p_{N_k}\otimes (\gamma_{n_{k+1}}-1),\qquad
\gamma_n\circ\sigma_{\pi'}=\gamma_{N_k}\circ \sigma_\pi+p_{N_k}\otimes
(\gamma_{n_{k+1}}-1),
\]
since $p_{N_k}$ is invariant under permutations. Let $\pi''\in \permut_{k+1}$ be given by
postcomposition of $\pi'$ by the right shift permutation $(23\dots1)$, that is,
$\{1,\dots k+1\} \mapsto \{k+1, \pi(1),\dots, \pi(k)\}$. Then 
\[
\gamma_n=\gamma_{n_{k+1}}+p_{n_{k+1}}\otimes (\gamma_{N_k}-1),\qquad
\gamma_n\circ\sigma_{\pi''}=\gamma_{n_{k+1}}+p_{n_{k+1}}\otimes
(\gamma_{N_k}\circ\sigma_\pi-1).
\]
Applying to any string $s$ and multiplying, we obtain that 
\[
\gamma_n\circ\sigma_{\pi'}\wedge
\gamma_n\circ\sigma_{\pi''}=\gamma_{N_k}\circ\sigma_\pi\otimes \gamma_{n_{k+1}},
\]
for this we note that since $p_{n_{k+1}}$ is the least element in $\Fe_{n_k+1}$, we have
$p_{n_{k+1}}(t)\gamma_{n_{k+1}}(t)=p_{n_{k+1}}(t)$ for any string $t\in
\{0,1\}^{n_{k+1}}$. Using this and the first part of the proof,
\begin{align*}
\gamma_{n_1}\otimes \dots\otimes \gamma_{n_k+1}&\le \bigwedge_{\pi'\in \permut_{k+1}}
\gamma_n\circ \sigma_{\pi'}\le \bigwedge_{\pi\in
\permut_k}(\gamma_{N_k}\circ\sigma_\pi\otimes \gamma_{n_{k+1}})=(\bigwedge_{\pi\in
\permut_k}\gamma_{N_k}\circ\sigma_\pi)\otimes \gamma_{n_{k+1}}\\
&=\gamma_{n_1}\otimes \dots\otimes \gamma_{n_k+1}.
\end{align*}

\end{proof}

Let us now describe products of the form 
\[
\gamma_{u_1,n_1}\otimes\dots\otimes \gamma_{u_k,n_k},
\]
with $u_i\in \{0,1\}$, $n=n_1+\dots+n_k$. Let $U$ be the set of $i$ such that $n_i$ is odd. Notice that such a product 
is separating if and only if one of the following is true:
\begin{enumerate}
\item $U=\emptyset$ and at most one of the $u_i$ is 1,
\item $U=\{i\}$ and $u_j=0$ for $j\ne i$,
\item $U=\{i,j\}$, $u_i\ne u_j$ and $u_k=0$ for $k\notin U$. 
\end{enumerate}
In any case, up to a permutation, the above product is of the form
\begin{align*}
\gamma_{u_1,n_1}\otimes\dots\otimes \gamma_{u_k,n_k}&=p_{n_0}\otimes (\gamma_{n_1}\otimes
\dots\otimes \gamma_{n_k})\otimes 1_{n_{n+1}}=p_{n_0}\otimes (\bigwedge_{\pi\in \permut_k}
\gamma_n\circ \sigma_\pi)\otimes 1_{n_{k+1}}\\ &=\bigwedge_{\pi\in \permut_k}p_{n_0}\otimes
(\gamma_n\circ \sigma_\pi)\otimes 1_{n_{k+1}}.
\end{align*}
In the separating case, $n_0,n_{k+1}\le 1$, so that the
product is an infimum over separating chains obtained from $\gamma_{n_k,n+n_0+n_k}$ by
permutations.

Let   $K\subset [n]$ and let $m:=n-|K|$. For $s\in \{0,1\}^m$, we define a
string $s_K\in \Fe_n$ as follows. Let $K=\{i_1<\dots<i_l\}$ and put $i_0=0$,
$i_{l+1}=n+1$, then
\[
(s_K)_i=\begin{dcases} 0, & \text{if } i=i_j,\ j=1,\dots,l\\
s_{i-j}, & \text{if } i_j<i<i_{j+1},\ j=0,\dots, l.
\end{dcases}
\]
For example, if $n=5$, $K=\{2,4\}$, then $s_K=s_10s_20s_3$. For a function $f\in \Fe_n$,
we define a function $f_K\in \Fe_m$ as
\begin{equation}\label{eq:restricted_chain}
f_K(s)=f(s_K)=\sum_{S\subseteq [n]} \hat f_Sp_{S\setminus K}(s).
\end{equation}
We also have
\[
(f_K)^*(s)=1(s)-f_K(s)+p_m(s)=1(s_K)-f(s_K)+p_n(s_K)=f^*(s_K)=f^*_K(s)
\]
and
\[
f_K\otimes g_L(st)=f\otimes g(s_Kt_L)=f\otimes g((st)_{K\cup n+L}).
\]
It is not clear that $f_K\in \Te_m$ if $f\in \Te_n$, but it is easily seen to be true for
chains. Indeed, one can see from \eqref{eq:restricted_chain} that if $f$ is a chain of odd length, then so is also $f_K$. One can also check that $X_{f_K}$ can be obtained from $X_f$ by replacing the objects
corresponding to $i\in K$ by the trivial first order object $I$.

\begin{prop}\label{prop:supinf} For any $f\in \Te_m$ there is some $n\ge m$, a separating
$g\in \Te_n$ and $K\subset [n]$,  such that $f=g_K$ and
\[
g=\bigvee_{\sigma\in G}\bigwedge_{\tau\in H} \gamma_n\circ (\tau\circ\sigma)
\]
for some sets of permutations $G,H\subseteq \permut_n$. 
\end{prop}

\begin{proof} The condition is stable under permutations. If it holds for $f$, then 
$f^*=(g_K)^*=g^*_K$ and
\[
g^*=\bigwedge_{\sigma\in G}\bigvee_{\tau\in H}\gamma_n^* \circ (\tau\circ\sigma).
\]
Note that we have
\[
\gamma^*_n=p_1\otimes \gamma_{n-2}\otimes 1_1=(\gamma_{n+2})_{1,n+2}
\]
and for any $\rho\in \permut_n$,
\[
\gamma^*_n\circ \rho=(\gamma_{n+2}\circ (id_1\otimes
\rho\otimes id_1))_{1,n+2}.
\]
Since the infima and suprema are pointwise, it is clear that 
\[
g^*=\bigwedge_{\sigma\in G}\bigvee_{\tau\in H}\gamma_n^* \circ (\tau\circ\sigma)=
(\bigwedge_{\sigma\in G}\bigvee_{\tau\in H} \gamma_{n+2}\circ (id_1\otimes
(\tau\circ\sigma)\otimes id_1))_{1,n+2}.
\]
Put
\[
g':=\bigwedge_{\sigma\in G}\bigvee_{\tau\in H} \gamma_{n+2}\circ (id_1\otimes
(\tau\circ\sigma)\otimes id_1),
\]
then 
\[
f^*=g^*_K=g'_{(K+1)\cup\{1,n+2\}}.
\]
Assume that $f_1=(g_1)_{K_1}$, $f_2=(g_2)_{K_2}$, and $g_1\in \Te_{n_1}$, $g_2\in
\Te_{n_2}$,
\[
g_i=\bigvee_{\sigma_i\in G_i} \bigwedge_{\tau_i\in H_i} \gamma_{n_i}\circ
(\sigma_i\circ\tau_i),\qquad i=1,2,
\]
$G_i, H_i \subseteq \permut_{n_i}$. Then 
\[
f_1\otimes f_2=(g_1\otimes g_2)_{K_1\cup n_1+K_2}
\]
and
\begin{align*}
g_1\otimes g_2&=\bigvee_{\sigma_1\in G_1,\sigma_2\in G_2}\bigwedge_{\tau_1\in H_1,\tau_2\in
H_2} \gamma_{n_1}\otimes\gamma_{n_2}\circ ((\sigma_1\otimes \sigma_2)\circ (\tau_1\otimes
\tau_2))\\
&=\bigvee_{\sigma_1\in G_1,\sigma_2\in G_2}\bigwedge_{\tau_1\in H_1,\tau_2\in
H_2} (\gamma_{n_1+n_2}\wedge \gamma_{n_1+n_2}\circ\pi)\circ ((\sigma_1\otimes \sigma_2)\circ (\tau_1\otimes
\tau_2)),
\end{align*}
where $\pi$ swaps the two blocks of length $n_1$ and $n_2$. Since $\rho$ commutes with
both $\sigma_1\otimes \sigma_2$ and $\tau_1\otimes \tau_2$, the statement follows, with
\[
G=\{\sigma_1\otimes \sigma_2, \sigma_i\in G_i\},\quad H=\{(\tau_1\otimes \tau_2)\circ\rho,
\tau_i\in H_i, \rho\in \{id, \pi\}\}.
\]

\end{proof}







\begin{prop}\label{prop:supinf+} Let $f\in \Te_n$ be separating. Then there is some
separating chain $\gamma\in \Fe_n$  and subgroups $G,H\subseteq \permut_n$, such that $G\subseteq H'$ (all elements
in $G$ commute with all elements in $H$), such that
\[
f=\bigvee_{\sigma\in G} \bigwedge_{\tau\in H} \gamma\circ\sigma\circ\tau.
\]

\end{prop}


\begin{proof} Assume the above is true for $f\in \Te_n$, then
\[
f^*=\bigwedge_{\sigma\in G} \bigvee_{\tau\in H} \gamma^*\circ\sigma\circ\tau=
\bigwedge_{\sigma\in G} \bigvee_{\tau\in H} \gamma^*\circ\tau\circ\sigma.
\]
Since $\gamma^*$ is a separating chain if $\gamma$ is, the statement holds also for $f^*$.
Further, let $\rho\in \permut_n$ be any permutation, then
\[
f\circ \rho = \bigvee_{\sigma\in G} \bigwedge_{\tau\in H}
\gamma\circ\sigma\circ\tau\circ\rho=
\bigvee_{\sigma'\in \rho^{-1}G\rho} \bigwedge_{\tau'\in \rho^{-1}H\rho} (\gamma\circ\rho)\circ\sigma'\circ\tau'.
\]
Since $\gamma\circ\rho$ is a separating chain, we see that the statement is also invariant
under permutations. It also follows that we may always assume that $\gamma=\gamma_{u,n}$
for some $u\in \{0,1\}$.

We now proceed by induction. The assertion is trivial for $n=1$. Assume it holds for all
$m<n$ and let $f\in \Te_n$. It is enough to prove the statement if $f=f_1\otimes f_2$, for
$f_1\in \Te_{m}$ and $f_2\in \Te_{n-m}$. By the induction assumption, there are some
separating chains $\gamma^1\in \Te_{m}$ and $\gamma^2\in \Te_{n-m}$, and subgroups
$G_1,H_1\subseteq \permut_m$, $G_1\subseteq H_1'$ and $G_2,H_2\subseteq \permut_{n-m}$,
$G_2\subseteq H_2'$ such that  
\[
f_1=\bigvee_{\sigma_1\in G_1}\bigwedge_{\tau_1\in H_1} \gamma^1\circ\sigma_1\circ\tau_1,\qquad 
f_2=\bigvee_{\sigma_2\in G_2}\bigwedge_{\tau_2\in H_2} \gamma^2\circ\sigma_2\circ\tau_2.
\]
We then have
\begin{align*}
f&=f_1\otimes f_2=
\bigvee_{\sigma_1\in G_1,\sigma_2\in G_2}\bigwedge_{\tau_1\in H_1,
\tau_2\in H_2} (\gamma^1\circ\sigma_1\circ\tau_1)\otimes
(\gamma^2\circ\sigma_2\circ\tau_2)\\
&=\bigvee_{\sigma_1\in G_1,\sigma_2\in G_2}\bigwedge_{\tau_1\in H_1,\tau_2\in
H_2}(\gamma^1\otimes \gamma^2)\circ(\sigma_1\otimes\sigma_2)\circ(\tau_1\otimes \tau_2).
\end{align*}
As noted above, we may assume that $\gamma^1=\gamma_{u_1,m}$ and
$\gamma^2=\gamma_{u_2,n-m}$ for some $u_1,u_2\in \{0,1\}$. Since $f$ is separating,
$\gamma^1\otimes \gamma^2$



\end{proof}
\end{document}
\end{document}

