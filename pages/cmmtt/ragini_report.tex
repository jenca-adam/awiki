\documentclass[12pt]{article}
\usepackage{geometry}
\usepackage{amsfonts, amssymb, mathtools}
\geometry{total={210mm,290mm},
 left=23mm,right=23mm,%
 bindingoffset=0mm, top=20mm,bottom=20mm}





\begin{document}
\begin{center}
{\large Report on the PhD thesis by Nidhin Sudarsanan Ragini
: \\  Labeling outcomes of Quantum Devices and Higher-order distinguishability     
 }

\end{center}
\medskip

\bigskip

\bigskip

In the thesis, two problems concerning higher order quantum testers are studied. In the
first part, the task of labeling of a quantum measurement (also called an observable or a  POVM) is discussed. This can be
cast as a measurement discrimination problem, where the ensemble consists of measurements that
are obtained from all possible labelings (permutations) of a given set of effects summing up to identity, with equal
probabilities. In this setting, perfect discrimination, minimum error and unambiguous
discrimination is considered in various regimes: one shot or multiple-shot, using simple,
entanglement-assisted or adaptive schemes. In the second part, the role of incompatibility
of quantum testers as a resource in discrimination tasks for quantum combs is demonstrated
by a generalization of the main result of the paper
[S\v SC19] by Skrzypczyk, \v Supi\'c and Cavalcanti from POVMs to quantum testers. 

The problem studied in the first part is quite natural and although it is rather simple,
the obtained results are interesting and nontrivial. The work in this part well
demonstrates the capabilities of the student for working with the formalism of quantum combs and testers. 
The results in the single
shot case were already published  in Phys. Rev. A, which is a highly respected journal in
the field of quantum information theory. The multiple-shot case is published as a preprint in
arXiv. The second part follows the ideas of the original paper [S\v CS19] very closely,
almost step-by-step, adding necessary modifications. Nevertheless, the obtained
generalization is valuable as it brings a much wider applicability to many different
settings. The results of this part are published in an arXiv preprint as well. 

At this point, it could be said that the thesis contains enough results to be accepted for
awarding the PhD title, though only one paper has been  published so far. But there are
some issues that have to be addressed here. 


The first is the thesis presentation. What one
notices very quickly is the somewhat strange writing style, with long, complicated sentences,
often grammatically incorrect, that are difficult to decipher. It is also very repetitive,
explaining basic concepts over and over again. A little less would be much more here. On
the other hand, there are some places where it would be
appropriate to write more  calculations explicitly, e.g. between Eq.(5.25) and Eq.(5.27)
and the text below them, or large parts of Chap. 6 (see also below).

Another thing is a rather large number of typos and the use of semicolon (;) instead of fullstop (.) at the end of
sentences. This could have been easily avoided by using spellcheck and simple
proofreading. 

There are further instances where the thesis could be improved.  Just a few examples:
\begin{itemize}
\item The first part of Proposition 2.1.1 is obviously false (just put $X=Y=\begin{pmatrix} 0 & 1\\ 0 &
0\end{pmatrix}$)

\item partial transpose is missing in the link product Eq.(2.33)

\item Corollaries 3.2.1 and 3.2.2 are given as immediate corollaries of Proposition
3.2.1. But Prop. 3.2.1 is formulated (and proved) as a necessary condition, whereas Cor.
3.2.1 is a necessary and sufficient condition, Cor. 3.2.2 gives a sufficient condition.

\item Theorem 4.2.2:  the maximal absolute eigenvalue of $X$ is the usual operator norm
$\|X\|$, I do not understand why it is called, and denoted, as the operator 2-norm 

\item There is a mistake somewhere in Lemma 4.3.1, it works only if $\mathrm{Tr}\,
\Pi_k=1$

\item Prop. 4.3.1: In fact, we have $\alpha_{\mathsf M}=\max_j \|M_j\|$

\item Eq.(5.2) and the sentence below: strictly speaking, the summands are not positive
semidefinite (but Eq.(5.3) holds, since the whole sum is block-diagonal, so each block
must be 0) Similar wrong argumentation is used repeatedly.

\item Theorem 4.2.1 is immediate, without use of Prop. 3.2.1, since if both $M$ and $I-M$
are full rank, then both Choi matrices are full rank as well.

\end{itemize}


But the most serious problems are the following two:

\begin{enumerate}
\item It is claimed repeatedly that in the binary multiple-shot scenario, the error
probability is increasing with the number of shots (e.g. at the end of Sec. 5.1.2,
beginning of Sec. 5.2 or in the concluding section). But the multiple-shot scenario naturally
contains single shot: we can use one shot for discrimination and discard all the other. So
the error probability certainly cannot increase with the number of shots. Something must
be wrong here, either in Eq.(5.27), which was derived here in detail only for $n=2$, or in the
subsequent paragraph, where it is simply stated that: "...we find that the error is
increasing with more number of shots", without any further explanation or argument.

\item Chapter 6 is not readable. There are notations that are
not introduced anywhere (e.g. $\mathsf{Q}_{\vec{a}}$, $d_{\vec{a}}$,
$\tilde{\mathsf{Q}}_{\vec{a}}$, $\tilde \Theta$, etc.) 
The text is taken from the arXiv preprint arXiv:2405.20080, with some shortenings that
removed some important parts. But the preprint itself is not well written. In particular,
the paragraph between Eq.(6.35) and Eq.(6.36) makes little sense, moreover, I would say
that the dual SDP in Eq.(6.36) is wrong. I still believe that the main result, Theorem
6.2.1, is true, but the proof should be improved substantially.



\end{enumerate}


In conclusion, based on the present thesis, the PhD title can be recommended provided the
issues described in the points 1. and 2. above are sufficiently addressed in the defense.



\vfill

12. 8. 2024            \hfill                                              Anna Jen\v
cov\'a

\hfill Mathematical Institute

\hfill Slovak Academy of Sciences

\hfill Bratislava


 \end{document}
