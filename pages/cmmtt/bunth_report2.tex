\documentclass[12pt]{article}
\usepackage{geometry}
\usepackage{amsfonts, amssymb, mathtools}
\geometry{total={210mm,290mm},
 left=23mm,right=23mm,%
 bindingoffset=0mm, top=20mm,bottom=20mm}





\begin{document}
\begin{center}
{\large Report on the PhD thesis by Gergely Bunth: \\
On quantum R\'enyi divergences }

\end{center}
\medskip


\bigskip

The topic of the thesis is centered around quantum extensions of R\'enyi divergences, which can be
defined  as certain quantities on pairs (or larger sets) of quantum states that coincide with the
classical (multivariate) R\'enyi divergence in mutually commuting cases. Of course, such extensions cannot
be arbitrary and the quantities must satisfy certain properties to be potentially useful.
To construct such extensions, essentially two different approaches are taken in the
present thesis. 

The first approach, described in Section 3, is inspired by a variational formula that relates the classical R\'enyi divergence to the weighted (left) radius of two probability functions with respect to the relative entropy
(Kullback-Leibler divergence). This formula can be extended to multivariate
cases and coincides with the Hellinger transform if the weights are given by a probability
distribution. 
In this part of the thesis, an analogous formula is used to define barycentric R\'enyi
divergences for pairs or families of quantum states (or positive semidefinite operators). 
Since there exist many versions of the quantum relative entropy,  a different version  may be used, even for each state,
so that the obtained class of quantities is quite large.  These quantities inherit some useful properties of the defining relative entropies, such as the scaling property or the data processing inequality. It is also proved that 
 the previously known quantum versions of the R\'enyi divergences, that is,  the
$\alpha-z$-divergences, measured or maximal R\'enyi divergences, are not contained in this
class. 

Another approach, described in Section 4, is axiomatic, based on the fact that the
elements in the spectrum of certain preordered semirings constructed from relative
(sub)majorization of families of probability vectors (or vectors of nonnegative numbers)
are obtained from R\'enyi divergences.  The importance of the spectrum lies in the fact
that its elements characterize asymptotic or catalytic majorization in preordered semirings. 
This idea is applied to ordered semirings consisting of  pairs of continuous functions 
$\rho:X\to B(\mathcal H)_{>0}$ and $\sigma: Y\to B(\mathcal H)_{>0}$ from compact Hausdorff spaces to
positive definite operators on a finite dimensional Hilbert space and with preorder given by
relative submajorization with respect to completely positive trace nonincreasing maps. 
In the case when all elements in the range of $\sigma$ commute, the spectrum is fully
characterized in terms of sandwiched R\'enyi divergences with $\alpha>1$, which leads to a
characterization of asymptotic submajorization and a sufficient condition for catalytic
submajorization. In the general
case, some further elements of the spectrum are constructed by composing the sandwiched
R\'enyi divergence with a quantum version of the geometric mean, thus obtaining a new set of
necessary conditions.

Applications discussed in the thesis include characterization of the error exponents
in the strong converse regime in the composite hypothesis testing, equivariant relative
submajorization and a description of asymptotic transformations of by thermal
processes, hypothesis testing with group symmetry or with reference frames and  approximate
joint transformations. 

The thesis is based on  three papers published in to quality journals. Two of the papers
are written with the coauthor P\'eter Vrana, the  third paper including also the thesis supervisor is currently  in
press. 
Given the established importance of the quantum R\'enyi divergences in quantum information
theory, including quite recent results on the use of related quantities in the
characterization of (asymptotic, approximate or catalytic) convertibility between sets of
classical or quantum states, the topic of the thesis is timely and important. The results of the thesis
are based on original ideas and are rather strong, especially in Sec. 4. The approach of the first part
 gives a very reasonable extension to the case of more than two variables (hence a quantum
 version of the classical Hellinger transform) and even in the two-variable case, the proposed barycentric R\'enyi
divergences introduce a potentially  rich supply of promising  quantities to be explored
further. In any case, the thesis brings an important contribution to the field, with many possible
further applications and follow up research directions. Putting all together, the thesis
clearly demonstrates the capability of Gergely Bunth for research in his field and I am
happy to recommend him to be awarded the PhD title.







\end{document}

