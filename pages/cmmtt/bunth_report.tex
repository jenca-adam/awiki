\documentclass[12pt]{article}
\usepackage{geometry}
\usepackage{amsfonts, amssymb, mathtools}
\geometry{total={210mm,290mm},
 left=23mm,right=23mm,%
 bindingoffset=0mm, top=20mm,bottom=20mm}





\begin{document}
\begin{center}
{\large Gergely Bunth: On quantum R\'enyi divergences }

\end{center}
\medskip


\bigskip

Topic - quantum extensions of R\'enyi divergences from several perspectives, 
applications in asymptotics of hypothesis testing, convertibility - asymptotic, catalytic, relative
(sub)majorization, even in classical case....

Part I - Sec. 3: a variational formula that relates the classical R\'enyi divergence to the
weighted (left) radius of two probability functions with respect to the relative entropy
(Kullback-Leibler divergence). This formula can be easily extended to multivariate
cases and coincides with the Hellinger transform if the weights are given by a probability
distribution.
In this part of the thesis, this formula is extended for families of quantum states
(or positive semidefinite operators) in a natural way. Since there exist many versions of the quantum
relative entropy, the multivariate formula may contain different quantum relative
entropies for each state. It is proved that if the defining quantum relative entropies are
monotone under pinchings, the corresponding divergence is a quantum extension of the
classical Hellinger transform. In particular, in the two-variable case, this definition gives a quantum extension of the 
R\'enyi $\alpha$-divergence if $\alpha\in (0,1)$. Moreover, these quantities inherit the
monotonicity and scaling properties of the involved quantum relative entropies.
It is proved that none of the previously known quantu versions of $D_\alpha$, that is the
$\alpha-z$-divergences, measured or maximal R\'enyi divergences, are obtained in this way.


Part II -Sec. 4: This section is inspired by the fact  that comparison of the  elements in the
spectrum characterizes asymptotic or catalytic majorization in preordered semirings. 
The spectrum consists of homomorphisms into certain fixed ordered semirings (reals with
the usual structure and reals with the tropical structure (max instead of +). 
This is applied to ordered semirings consisting of  pairs of continuous functions $\rho:X\to B(\mathcal H)_>$ and
$\sigma:Y\to B(\mathcal H)_>$, $\mathcal H$ a finite dimensional Hilbert space (can be
different in each pair) with $X$ and $Y$ compact Hausdorff spaces and with preorder given by
relative submajorization with respect to completely positive trace nonincreasing maps. 

In the classical  case, (that is if of both $\sigma$ and $\rho$ range in $\sum_i
a_i|i\rangle\langle i|$), all elements of the spectrum are obtained from R\'enyi
divergences with $\alpha>1$. It is shown that in the semiclassical case, that is, if all
the elements in the range of $\sigma$ commute, all the morphisms in the spectrum are
obtained from sandwiched R\'enyi $\alpha$ divergences with $\alpha>1$, so that the
asymptotic or catalytic majorization can be characterized in this case. 
In general, some further elements of the spectrum are obtained by composing the sandwiched
R\'enyi divergence with a quantum version of the geometric mean, which leads to a family
of necessary consitions.

This is applied to several important cases, such as characterization the error exponents
in the strong converse regime in the composite hypotesis testing, equivariant relative
subjaoriztion, which leads to description of asymptotic transformations of by thermal
processes, hypothesis testing with group symmetry or with reference frames, approximate
joint transformations. 

The ideas in this section also yield a new two-parameter family of quantum R\'enyi
divergences - the $\gamma$-weighted sandwiched R\'enyi divergence.






Elements of the spectrum are closely related to quantum versions of R\'enyi
$\alpha$-divergences with $\alpha>1$, in particular if all operators in the range of
$\sigma$ commute, then the spectrum is fully characterized by sandwiched R\'enyi
divergences. It is proved that if the operators in the range of $\sigma$ commute, the 
asymptotic majorization is characterized by sandwiched R\'enyi divergences for $\alpha>1$,
and a sufficient condition is obtained for catalytic majorization. 
In the general case, a set of necessary conditions is given. 




The topic of the thesis is timely and important, given the importance of R\'enyi
divergences in quantum information theory, resoure theories, convertibility, hypothesis
testing. The obtained results are based on novel ideas and are rather strong, especially
in the second part, builds on and extends known and recently obtained results on the
importance of R\'enyi divergences in information theory and resource theories...
blablabla...



Nevertheless, the thesis not written veryu well:
\begin{itemize}
\item p. 6, l. 6 from below: the sentence starting with ''In a resource theory...'' is
rather strange. Also ''sates'' -> states
\item p. 7: ''...target operators are only bounds'' ??
\item $\mathcal P_f$ is used to denote two very different things: $\mathcal P_f(\mathcal
I)$ in p. 10 and  the  operator perspective function  on p. 13. Though it should not cause
confusion, a different notation would be better.

\item p. 13, Theorem 2.2.1: ''...so that the following...'' ->  ''...such that the
following...'' would be better

\item It seems that a definition of a quantum divergence  is missing here.
In Definition 2.2.6, the quantum R\'enyi divergence is defined merely by its
values on jointly diagonalizable elements, where it should coincide with the classical
R\'enyi divergence. However, unitary equivalence is repeatedly used in proofs. Also Lemma
3.2.1 is quite mysterious.  In the paper [MBV22] a quantum divergence is defined first, as a unitarily
invariant quantity. This seems to have been left out in the thesis.

\item p.17, first displayed equation: $D_{\alpha,+\infty}...$ in the second line,
$(1-\alpha)$ is missing

\item p. 17, paragraph above Ex. 2.2.11: I think that strict positivity of $D_{\alpha,z}$
is treated in [Mos23], Corollary III.28 only in the case $\alpha>1$

\item p. 20: ''realation''

\item p. 24, displayed equation under Eq. (2.40): I would say that here should be
$\subseteq$ (instead of $\not\subseteq$)


\item p. 25, Remark 2.2.25: Actually, it is the Hellinger transform of a set of
probability distributions ($P$ is a parameter)


\item Properties of the quantum R\'enyi divergences are often refered to before their
definition in Section 2.2.4. This is very  confusing, especially when it concerns some
commonly used expressions  like ''nonnegative'', which is quite difficult to gues  that
this is actually  some special property to be defined later.  I strongly suggest to rearrange this part. 

\item p. 28, last paragraph: $\mathcal D_{\mathcal H}(D_P)$ etc, does not seem to be defined before. 



\item p. 29, displayed equations on the top of the page: $V$ and $V^*$ seem missing in
line 3 and 4 of the equations

\item p. 29: ''isomeric''

\item p. 29, definition of ''regular'' $D_\alpha$: here $\kappa_{\rho,\sigma}>0$ has no
meaning

\item p. 34, Theorem 2.3.36, part (ii)(c): $x\succcurlyeq_c y$ (instead of
$ax\succcurlyeq_c ay$)

\item p. 36., Remark 3.1.3: I think $P(x)\le 0$, $x\in \mathcal X$ is not possible, since
$\sum_xP(x)=1$

\item p. 36., Remark 3.1.7, equality for $R_{D^q,left}$: $\omega$ (instead of $\tau$)

\item p. 37, Lemma 3.1.9: Reference to Eq.(3.12) is used repeatedly, but I think it should
be Eq. (3.3)

\item I do not understand Remark 3.1.10

\item Remark 3.1.13, also p. 40, first paragraph of Sec. 3.2): $\gamma$-weighted versions of $D^{\text{meas}}$, $D^{\text{Um}}$ are
mentioned here, but these were not 
defined

\item p. 41, line 6 from below: $S^0_+$ should be $S_+$

\item p. 42, last line: ''Lemma 3.2.1...'' perhaps should be ''Lemma 3.2.2''

\item p. 43, Remark 3.2.10: $Q_P^{G_P^{D^q}}$ very strange notation, it is not clear what
it means. I think it comes from some notations in Sec. IV in [MBV22] that was not
introduced here (as well as the references to $\gamma$-weighted versions...)

\item p. 47, paragraph below the proof of Prop. 3.4.2: ''...all barycentric R\'enyi
divergences...'' only those with $D^q$ monotone under CPTP maps

\item p. 56, Example 4.1.4: relative majorization should be defined (also proof of Prop.
4.2.3). Also its relation to relative submajorization for pairs of states should be clarified, or a reference to [BV21]
(prop. 3.3) should be given. Also DPI (in proof of Prop. 4.2.3).

\item p. 57, line above Def. 4.1.8: ''to subsemiring'' -> two subsemirings (?)

\item top of p. 60, end of the proof of Lemma 4.2.4: better write out that $k=1$ in the
real case and $k=0$  in the tropical case.

\item p. 64, line 2: ''realative''. Also line 3 from below: in one case $\tilde
f(c_1,d_1)$ should be $\tilde f(c_2,d_2)$.

\item p. 65, Eq.(4.8) and the displayed equation above it: $\rho$ should be $\rho(x)$ in
some places

\item p. 66, Prop 4.3.4: $\mathrm{Tr}\, \sigma(y')$ -> $\mathrm{Tr}\, \sigma(y')=1$

\item p. 67, Def. 4.3.5: ''...a collection of maps $M:\bigtimes_{\mathcal H} C(Y,\mathcal
B(\mathcal H)_{>0}\to\bigtimes_{\mathcal H} C(Y',\mathcal
B(\mathcal H)_{>0} $... this makes no sense. I think what you mean is $M=\{M_{\mathcal
H}\}$, where for each $\mathcal H$, $M_{\mathcal H}: C(Y,\mathcal
B(\mathcal H)_{>0}\to C(Y',\mathcal
B(\mathcal H)_{>0}$ (but this is not a map between the products...).

\item p. 69, displayed equations on the top: the inequality sign should be opposite

\item p. 84, Corollary 4.4.21: $\alpha\ge 0$ -> $\alpha\ge 1$  (?)

\item p. 85, Prop. 4.4.23 (i): ''trace-nonincreasing'' - according to the proof, maybe
''trace-preserving''?

\item p. 88: Example 4.4.27: ''Lemma 4.3.16'' is actually a Proposition

\item p. 88 Example 4.4.28 ''...arbitrary an suppose..,'' -> ''and''

\item Ref. [JV18], [FF20] I think these were already published

\item Incomplete references: [MBV22], [MH23a], [FFHT23], [HT23]

\item some conclusion, possible future directions, further questions, ideas? 

\end{itemize}






\medskip

\noindent
\textbf{Overall evaluation}

\medskip
 

\medskip

\noindent
\textbf{Some specific comments}


\







\end{document}

