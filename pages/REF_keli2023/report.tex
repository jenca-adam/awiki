\documentclass[12pt]{article}
\usepackage{geometry}
\usepackage{amsfonts}
\geometry{total={210mm,290mm},
 left=23mm,right=23mm,%
 bindingoffset=0mm, top=20mm,bottom=20mm}





\begin{document}
\begin{center}
{\large  Ke Li, Yongsheng Yao: Reliability Function of Quantum Information Decoupling via
the Sandwiched R\'enyi Divergence}

\end{center}
\medskip

\centerline{Referee report}

\bigskip

The authors investigate the reliability function for the task of catalytic quantum information
decoupling, describing the exponent of the asymptotic decay of the error. An upper and
lower bound for the reliability functions are obtained, in terms of the sandwiched R\'enyi
mutual information. If the decoupling cost is under a certain critical value, these bounds
coincide, giving an exact formula. 

This gives a significant improvement over previous works, where only achievability
bounds were obtained. Besides, the catalytic setting and the description of the error
in  terms of the purified distance enables the authors to connect three different types
of decoupling operations, as well as to relate their scenario to the task of quantum state merging,
which gives a broader applicability of the findings. It is also remarkable that the results imply an
operational interpretation of the sandwiched R\'enyi divergence in the exact domain, in
contrast with the operational interpretations established  before. 

The proofs of the main results  are based on a newly obtained bound in the convex split lemma  and
on  the formula for the smoothing quantity for the max information and conditional min entropy, which are
given in terms of the sandwiched R\'enyi divergence. These results are  of independent interest.

The paper is quite well written and I have only a few minor remarks listed below.

\noindent


\begin{enumerate}
\item  p. 2: ''There is no discuss''
\item p. 3: ''the notation'' -> The notation
\item p. 5: ''a sate'' -> state
\item p. 14: ''the optimal state that makes ... achieves the minimum'' -> e.g.: ...the
optimal state at which ... achieves the minimum
\item p. 7, Sec. III. B: better specify also here where the proofs of Prop. 6, Thms. 7 and
8 can be found. 
\item Some of the proofs especially in Sec. IV.B refer to statements that are stated and
proved only later (such as Lemmas 17 and 18 in the proof of Prop. 16). This makes the reading
less smooth. It might be better to slightly reorganize this part, if possible.




\end{enumerate}








\end{document}

